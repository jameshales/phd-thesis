\chapter{Conclusion}\label{conclusion}

In this work we considered a two logics for quantifying over epistemic updates: refinement modal logic, which quantifies over refinements, and action model logic, which quantifies over action models.
Compared to previous dynamic epistemic logics such as the public announcement logic of Plaza~\cite{plaza:1989} and Gerbrandy and Groenveld~\cite{gerbrandy:1997}, and the action model logic of Baltag, Moss and Solecki~\cite{baltag:1998,baltag:2004}, which reason about the results of specific epistemic updates, the logics we consider reason about the results of arbitrary epistemic updates, allowing us to ask questions such as ``Is there an epistemic update that results in the desired change in knowledge?'', and ``What is a specific epistemic update that results in the desired change in knowledge?''.
Compared to previous dynamic epistemic logics such as the arbitrary public announcement logic of Balbiani, et al.~\cite{balbiani:2007} and the group announcement logic (GAL) of {\AA}gotnes, et al.~\cite{agotnes:2008,agotnes:2010}, which quantify over relatively restricted forms of epistemic updates, the logics we consider quantify over much more general forms of epistemic updates.
Logics for quantifying over epistemic updates allow us to reason about the existence or non-existence of epistemic updates that result in desired epistemic goals.
A closely related problem we have also considered is that of synthesising specific epistemic updates that achieve desired knowledge based goals.
Such tools could see applications in the development of network protocols, the verification of secure computer systems, games, or in artificial intelligence.

\section{Contributions}

In Chapter~\ref{rml} we recalled the refinement modal logic of van Ditmarsch and French~\cite{vanditmarsch:2009} and generalised the semantics to other modal settings.
Compared to previous treatments of \logicRml{}~\cite{vanditmarsch:2009,vanditmarsch:2010,bozzelli:2014a} we considered only multi-agent variants of \logicRml{}. 
We also used a multi-agent notion of $\agentsB$-refinement, which we believe to be more elegant in a multi-agent setting than the single-agent notion of $\agentA$-refinements used previously~\cite{vanditmarsch:2010}.
We generalised many known results about refinements to $\agentsB$-refinements.
A significant new result is a better partial correspondence between refinements and positive formulas.
We showed that a modally saturated Kripke model is a refinement of another modally saturated Kripke model if the former satisfies all of the positive formulas satisfied by the latter.
This is similar to the Hennessy-Milner property that shows the partial correspondence between bisimilarity and modal equivalence.
This correspondence, along with the partial correspondence of refinements with the results of executing action models, provides a strong justification for our interpretation of refinements as a very general form of epistemic update.
We showed a number of validities in \logicRml{} that either apply to all modal settings, such as validities corresponding to modal axioms {\bf K}, {\bf T}, {\bf 4}, and the modal rule {\bf NecK}, or apply to all or most of the modal settings we considered in this work, such as the Church-Rosser, McKinsey, and finality properties.
We showed that many variants of \logicRml{} are not closed under uniform substitution, and that in any variant of \logicRml{} that is, the refinement quantifiers carry no meaning.
We also showed that the no distinct pair of variants of \logicRml{} that we consider in this work are sublogics of one another, thus for example, many results specific to \logicRmlK{} do not trivially generalise to other variants of \logicRml{}.

In Chapter~\ref{rml-k}, Chapter~\ref{rml-kd45}, and Chapter~\ref{rml-s5} we provided results specific to the logics \logicRmlK{}, \logicRmlKFF{} and \logicRmlKD{}, and \logicRmlS{}, respectively.
For each logic we presented a sound and complete axiomatisation.
Each axiomatisation formed a set of reduction axioms, admitting a provably correct translation from \langRml{} to the underlying modal language \langMl{}.
We used these provably correct translations to show the completeness of the corresponding axiomatisations, to show that each variant of \logicRml{} is expressively equivalent to its underlying modal logic, and to show that each variant of \logicRml{} is compact and decidable.

In Chapter~\ref{rml-k4} we provided expressivity results specific to the logic \logicRmlKF{}.
We showed that \logicRmlKF{} is strictly more expressive than \logicKF{} and strictly less expressive than \logicMuKF{} and \logicBqmlKF{}.
We showed that \logicRmlKF{} is non-strictly less expressive than \logicBqmlKF{} by demonstrating a translation from \langRml{} to \langBqml{}.
A corollary of this translation is that \logicRmlKF{} is decidable, via the decidability of \logicBqmlKF{}.

In Chapter~\ref{aaml} we introduced the arbitrary action model logic and provided results specific to the logics \logicAamlK{}, \logicAamlKFF{}, and \logicAamlS{}.
\logicAaml{} extends the action model logic of Baltag, Moss and Solecki~\cite{baltag:1998,baltag:2004} with action model quantifiers, and was proposed by Balbiani, et al.~\cite{balbiani:2007} as a possible generalisation for \logicApal{}.
For each logic we showed that the action model quantifiers of \logicAaml{} are equivalent to the refinement quantifiers of \logicRml{}.
As a consequence, most of the results for \logicRml{} from the previous chapters also hold in \logicAaml{} in these settings.
We showed the equivalence by showing that if there exists a refinement that where a given formula is satisfied then we can construct a finite action model that results in that formula being satisfied.
This forms a synthesis procedure for the epistemic planning problem for action model logic.
This equivalence also further justifies our interpretation of refinement quantifiers as quantifiers for epistemic updates.

\section{Related work}

The refinement modal logic is somewhat related to the bisimulation quantified logic of Ghilardi and Zawadowski~\cite{ghilardi:2002}, and Visser~\cite{visser:1996}.
Refinements are essentially generalisations of bisimulations, as a refinement is a bisimulation where the condition {\bf forth-$\agentA$} is relaxed for some agents.
Bozzelli, et al.~\cite{bozzelli:2014a} partially characterised refinements as bisimulations followed by restrictions of the accessibility relation, and refinement quantifiers in \logicRmlK{} as bisimulation quantifiers in \logicBqmlK{} along with a syntactic notion of relativisation that essentially corresponds to a restrictions of the accessibility relation.
In Chapter~\ref{rml-k4} we adapted these results to the settings of \logicRmlKF{} and \logicBqmlKF{}.
In principle these results could be adapted to other variants of \logicRml{} and \logicBqml{}.

The arbitrary action model logic introduced in this work solves many of the same problems as the DEL-sequents of Aucher~\cite{aucher:2011,aucher:2012}.
The DEL-sequents provide a sequent calculus for reasoning about arbitrary action models.
In contrast to the arbitrary action model logic, the system of DEL-sequents does not extend the syntax or semantics of action model logic with quantifiers.
Rather, all reasoning about arbitrary action models is performed at the meta-logical level.
The particular case of epistemic planning with DEL-sequents gives a method to determine, given a formula describing an initial knowledge situation, and a formula describing a desired knowledge situation, a formula describing an action model that takes us from the initial situation to the desired situation.
If the formula describing the action model is satisfiable then we can produce a specific action model that takes us from the initial situation to the desired situation.
Otherwise if the formula describing the action model is unsatisfiable then we know that no such action model exists.
This essentially corresponds to having a single action model quantifier at the meta-logical level, that can only quantify over quantifier-free formulas.
With \logicAaml{} we are able to directly express and reason about complex statements involving action model quantifiers at the logical level.
For example, we can nest quantifiers to reason about the existence of an action model that results in a situation where all subsequent action model executions will preserve a given property.
Having action model quantifiers in the logic also allowed us to show that action model quantifiers are equivalent to refinement quantifiers, and that action model quantifiers add no expressivity to modal logic in the settings we considered.

\section{Future work}

There are several immediate avenues for future work based on the results presented in this work.

We have not shown complexity or succinctness results for the multi-agent variants of \logicRml{} and \logicAaml{}.
Bozzelli, van Ditmarsch and Pinchinat~\cite{bozzelli:2014a} gave complexity bounds for the decision problem and succinctness results for the single-agent variant of \logicRmlK{}, and Achilleos and Lampis~\cite{achilleos:2013} provided complexity results for the model-checking problem in addition to tighter complexity bounds for the decision problem, both for the single-agent variant of \logicRmlK{}.
Hales, French and Davies~\cite{hales:2011b} described a decision procedure for the single-agent variants of \logicRmlKD{} and \logicRmlS{}.
Decision procedures for the multi-agent logics \logicRmlK{}, \logicRmlKFF{}, \logicRmlKD{}, and \logicRmlS{}, can be formed by combining the provably correct translation for each with a decision procedure for the respective underlying modal logic.
However the provably correct translations result in a non-elementary increase in formula size, so such decision procedures will have a non-elementary complexity, which is less than ideal.

Although we provided decidability and expressivity results for \logicRmlKF{} we do not yet have a sound and complete axiomatisation.
As \logicRmlKF{} is strictly more expressive than \logicKF{} a provably correct translation from \langRml{} to modal logic is not possible, so a different strategy for proving the completeness of a candidate axiomatisation is required.
A candidate axiomatisation must also greatly differ from the axiomatisations of \logicRmlK{}, \logicRmlKFF{}, \logicRmlKD{}, and \logicRmlS{}, as these axiomatisations admit provably correct translations, which would be unsound in \logicRmlKF{}.

As this work has focussed on generalising \logicRml{} to different modal settings, a natural avenue for future work would be generalisation to even more modal settings.
Bozzelli, et al.~\cite{bozzelli:2014b} suggest that refinement quantifiers may be applicable to any modal logic, not just epistemic modal logics.
A strong candidate for future consideration is \logicRmlSF{}, in the setting of reflexive and transitive Kripke frames.
We have briefly considered expressivity and decidability results for \logicRmlSF{}, and we conjecture that the expressivity results of \logicRmlKF{} can be adapted to \logicRmlSF{}.
We conjecture that the expressivity of \logicRmlSF{} lies strictly between that of the modal logic \logicSF{} and the modal $\mu$-calculus \logicMuSF{}, similar to the expressivity of \logicRmlKF{}.
Work on these results is on-going.

We have not yet considered the addition of common knowledge operators to \logicRml{} and \logicAaml{}.
Many natural questions in epistemic logics and dynamic epistemic logics are about common knowledge, so it seems natural that we would want to consider common knowledge in connection with quantifiers over epistemic updates.
In principle this would allow us to answer questions such as whether or not desired common knowledge is attainable, or how desired common knowledge can be achieved through a specific epistemic update.

Although we provided a synthesis procedure for \logicAaml{} we have not yet considered efficient or optimal synthesis procedures for \logicAaml{}.
The provided synthesis procedure relies on the expressive equivalence of \logicAaml{} with the corresponding underlying modal logic.
Expressive equivalence of \logicAaml{} and modal logic is shown via a provably correct translation that results in a non-elementary increase in formula size, so a synthesis procedure that uses this provably correct translation will have a non-elementary complexity.
In addition to an efficient synthesis procedure it would be desirable to have a synthesis procedure that results in ``optimal'' action models, such as by minimising the overall size of the action model as measured by the number of states, the size of the preconditions, and so on.

Finally, we have not yet considered in detail the relationship between \logicAaml{} and the DEL-sequents of Aucher~\cite{aucher:2011,aucher:2012}.
In contrast to \logicAaml{}, which introduces syntactic quantifiers over action models, the system of DEL-sequents does not extend the syntax or semantics of action model logic with quantifiers.
However the DEL-sequents can answer similar questions to those considered by \logicAaml{} by performing reasoning at the meta-logical level.
We showed most of our results in \logicAaml{} by showing that the action model quantifiers of \logicAaml{} are equivalent to the refinement quantifiers of \logicRml{}, however it may be possible to show similar results by relating the action model quantifiers of \logicAaml{} to the meta-logical reasoning that can be performed with the DEL-sequents.
As we have shown that the action model quantifiers of \logicAaml{} are equivalent to the refinement quantifiers of \logicRml{} this also suggests that results in \logicAaml{} derived from the DEL-sequents would also be valid in \logicRml{}.
