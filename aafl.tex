\chapter{Arbitrary action formula logic}\label{aafl}

\begin{example}\label{grant-example}
James, Ed and Tim submit a research grant proposal, and eagerly await the outcome.
Is there a series of actions that will result in: 

\begin{enumerate}
  \item Ed knowing the grant application was successful; 
  \item James not knowing whether the grant application was successful, but knowing that either Ed or Tim does know;
  \item Tim does not know whether the grant application was successful, but knows that if the grant application was unsuccessful, then James knows that it was unsuccessful.
\end{enumerate}

Such an epistemic state may be achieved by a series of messages: Ed is sent a message congratulating him on a successful application, James is sent a message informing him that at least one applicant on each grant has been informed of the outcome, and Tim is sent a message informing him that the first investigator of all unsuccessful grants has been notified.
\end{example}

\section{Syntax and semantics}\label{aafl-semantics}

\begin{definition}[Language of arbitrary action formula logic]
The language \langAafl{} of arbitrary action formula logic is inductively defined as:
$$
    \phi ::= \atomP \mid 
           \neg \phi \mid
           (\phi \land \phi) \mid
           \necessaryA \phi \mid
           \actionA{\alpha} \phi \mid
           \allrefs \phi
$$
where $\atomP \in \atoms$, $\agentA \in \agents$ and $\alpha \in \langAaflAct{}$, and where the language \langAaflAct{} of arbitrary action formulas is inductively as:
$$
    \alpha ::= \test{\phi} \mid
           \alpha \choice \alpha \mid
           \alpha \compose \alpha \mid
           \learnsBs (\alpha, \alpha)
$$
where $\phi \in \langAafl{}$ and $\emptyset \subset \agentsB \subseteq \agents$.
\end{definition}

We use all of the standard abbreviations for arbitrary action model logic, in addition to the abbreviations $\learnsBs \alpha ::= \learnsBs (\alpha, \alpha)$ and $\learnsA (\alpha, \beta) ::= \learns[{\{\agentA\}}] (\alpha, \beta)$.

We denote non-deterministic choice ($\choice$) over a finite set of action formula $\Delta \subseteq \langAaflAct$ by $\bigchoice \Delta$ and we denote sequential execution ($\compose$) of a finite, non-empty sequence of action formulas $(\alpha_i)_{i=0}^{n} \in \mathbb{N}^\langAaflAct$ by $\bigcompose (\alpha_i)_{i=0}^{n}$ and define them in the obvious way.

We refer to the languages \langAfl{} of action formula logic and \langAflAct{} of action formulas, which are \langAafl{} and \langAaflAct{} respectively, both without the $\allrefs$ operator, 

As in the action model logic~\cite{baltag:2004}, the intended meaning of the operator $\actionA{\alpha} \phi$ is that ``$\phi$ is true in the result of any successful execution of the action $\alpha$''.
In the following section we define the semantics of the action formula logic in terms of action model execution.
For each setting of \classK{}, \classKFF{} and \classS{} we provide a function $\tau_\classC : \langAflAct \to \classAmK$ of translating action formulas from \langAflAct{} into action models.
The result of executing an action $\alpha \in \langAflAct{}$ is determined by translating $\alpha$ into an action model $\tau_\classC(\alpha) \in \classAmC$, and then executing the action model in the usual way.

In each setting we have attempted to define the translation from action formulas into action models in such a way that the action formulas carry an intuitive description of the action that is performed by the corresponding action model.
We call the $\test{}$ operator the test operator, and describe the action $\test{\phi}$ as a test for $\phi$.
A test is intended to restrict the states in which an action can successfully execute to states where the condition $\phi$ is true initially, but otherwise leaves the state unchanged. 
We call the $\choice$ operator the non-deterministic choice operator, and describe the action $\alpha \choice \beta$ as a non-deterministic choice between $\alpha$ and $\beta$.
We call the $\compose$ operator the sequential execution operator, and describe the action $\alpha \compose \beta$ as an execution of $\alpha$ followed by $\beta$. 
Finally we call $\learnsBs$ the learning operator, and describe the action $\learnsBs (\alpha, \beta)$ as the agents in $\agentsB$ learning that the actions $\alpha$ or $\beta$ occurred.
This action is intended to result in the agents $\agentsB$ knowing or believing what would be true if $\alpha$ or $\beta$ were executed.
For example, if a consequence of executing $\alpha$ is that $\phi$ is true in the result, then the intention is that a consequence of executing $\learnsA (\alpha, \alpha)$ is that $\knows_\agentA \phi$ is true in the result.
As we will see, this property is generally true in \logicAflK{}, however due to the extra frame conditions of \classKFF{} and \classS{} it is only true for some formulas $\phi$ in \logicAflKFF{} and \logicAflS{}.

\begin{example}\label{grant-example-formula}
If $\atomP$ stands for the proposition ``the grant application was successful'' then the action described in Example~\ref{grant-example} might be written in the form of an action formula as:
\begin{align*}
    \alpha = &\learns[Ed] (\test{\atomP}) \compose\\
    &\learns[James] (\learns[Ed] \test{\atomP} \choice \learns[Ed] \test{\neg \atomP} \choice \learns[Tim] \test{\atomP} \choice \learns[Tim] \test{\neg \atomP}) \compose\\
    &\learns[Tim] ((\test{\neg \atomP} \compose \learns[James] \test{\neg \atomP}) \choice \test{\top})
\end{align*}
\end{example}

We now define the semantics of arbitrary action formula logic.
As mentioned earlier, the semantics are defined by translating action formulas into action models.
The translation used varies in each class of \classK{}, \classKFF{} and \classS{} that we work in, according to the frame conditions in each class.
Therefore our semantics are parameterised by a function $\tau_\classC: \langAflAct \to \classAmK$ that will vary according to the class of Kripke models.

\begin{definition}[Semantics of arbitrary action formula logic]
Let \classC{} be a class of Kripke models, let $\tau : \langAflAct \to \classAmK$ be a function from action formulas to multi-pointed action models, and let $\kModel = \kModelTuple \in \classC$ be a Kripke model.

Then the interpretation of $\phi \in \langAafl$ in the logic $\logicAaflC$ is the same as its interpretation in modal logic given in Definition~\ref{ml-semantics}, with the additional inductive cases:
\begin{eqnarray*}
    \kPModel{\kStateS} \entails \allrefs \phi &\text{ iff }& \text{for every } \kPModelP{\kStateSP} \in \classC \text{ such that } \kPModelP{\kStateSP} \refines \kPModel{\kStateS}: \kPModelP{\kStateSP} \entails \phi
\end{eqnarray*}
where the refinement relation is defined in Definition~\ref{refinements}.
\end{definition}

\section{Synthesis}\label{synthesis}

In the following subsections we give a computational method for synthesising action formulas to achieve epistemic goals, whenever those goals are achievable.
We note that the notion of when an epistemic goal is achievable is captured by the refinement quantifiers of refinement modal logic~\cite{vanditmarsch:2009,bozzelli:2014b}, which are also included in the arbitrary action formula logic, and so in this section we will refer to the full arbitrary action formula logic, keeping in mind the correspondence with arbitrary action model logic mentioned in Section~\ref{aafl-semantics}.

\subsection{\classK{}}

\begin{proposition}\label{afl-k-synthesis}
For every $\phi \in \langAfl$ there exists $\alpha \in \langAflAct$ such that $\proves \actionA{\alpha} \phi$ and $\proves \somerefs \phi \implies \actionE{\alpha} \phi$.
\end{proposition}

\begin{proof}
Without loss of generality we assume that $\phi$ is in disjunctive normal form.
We proceed by induction on the structure of $\phi$.

Suppose that $\phi = \psi \lor \chi$.
By the induction hypothesis there exists $\alpha^\psi, \alpha^\chi \in \langAflAct$ such that $\proves \actionA{\alpha^\psi} \psi$, $\proves \somerefs \psi \implies \actionE{\alpha^\psi} \psi$, $\proves \actionA{\alpha^\chi} \chi$ and $\proves \somerefs \chi \implies \actionE{\alpha^\chi} \chi$.
Let $\alpha = \alpha^\psi \choice \alpha^\chi$.
Then:
\begin{eqnarray}
    &\proves& \actionA{\alpha^\psi} (\psi \lor \chi) \land \actionA{\alpha^\chi} (\psi \lor \chi)\label{afl-k-synthesis-or-1}\\
    &\proves& \actionA{\alpha^\psi \choice \alpha^\chi} (\psi \lor \chi)\label{afl-k-synthesis-or-2}
\end{eqnarray}
(\ref{afl-k-synthesis-or-1}) follows from the induction hypothesis and
(\ref{afl-k-synthesis-or-2}) follows from {\bf LU}.

Further:
\begin{eqnarray}
    &\proves& (\somerefs \psi \lor \somerefs \chi) \implies (\actionE{\alpha^\psi} (\psi \lor \chi) \lor \actionE{\alpha^\chi} (\psi \lor \chi))\label{afl-k-synthesis-or-3}\\
    &\proves& (\somerefs \psi \lor \somerefs \chi) \implies \actionE{\alpha^\psi \choice \alpha^\chi} (\psi \lor \chi)\label{afl-k-synthesis-or-4}\\
    &\proves& \somerefs (\psi \lor \chi) \implies \actionE{\alpha^\psi \choice \alpha^\chi} (\psi \lor \chi)\label{afl-k-synthesis-or-5}
\end{eqnarray}
(\ref{afl-k-synthesis-or-3}) follows from the induction hypothesis,
(\ref{afl-k-synthesis-or-4}) follows from {\bf LU} and
(\ref{afl-k-synthesis-or-5}) follows from {\bf R}.

Suppose that $\phi = \pi \land \bigwedge_{\agentB \in \agentsB \subseteq \agents} \coversB \Gamma_\agentB$.
By the induction hypothesis for every $\agentB \in \agentsB$, $\gamma \in \Gamma_\agentB$ there exists $\alpha^\gamma \in \langAflAct$ such that $\proves \actionA{\alpha^\gamma} \gamma$ and $\proves \somerefs \gamma \implies \actionE{\alpha^\gamma} \gamma$.
Let $\alpha = \test{\somerefs \phi} \compose \bigcompose_{\agentB \in \agentsB} \learnsB (\bigchoice_{\gamma \in \Gamma_\agentB} \alpha^\gamma)$.

Then for every $\agentB \in \agentsB$: 
\begin{eqnarray}
    &\proves& \actionA{\bigchoice_{\gamma \in \Gamma_\agentB} \alpha^\gamma} \bigvee_{\gamma \in \Gamma} \gamma\label{afl-k-synthesis-covers-1}\\
    &\proves& \necessary_\agentB \actionA{\bigchoice_{\gamma \in \Gamma_\agentB} \alpha^\gamma} \bigvee_{\gamma \in \Gamma} \gamma\label{afl-k-synthesis-covers-2}\\
    &\proves& \actionA{\learnsB (\bigchoice_{\gamma \in \Gamma_\agentB} \alpha^\gamma)} \necessary_\agentB \bigvee_{\gamma \in \Gamma} \gamma\label{afl-k-synthesis-covers-3}\\
    &\proves& \actionA{\bigcompose_{\agentC \in \agentsB} \learnsC (\bigchoice_{\gamma \in \Gamma_\agentC} \alpha^\gamma)} \necessary_\agentB \bigvee_{\gamma \in \Gamma} \gamma\label{afl-k-synthesis-covers-4}\\
    &\proves& \actionA{\test{\somerefs \phi}} \actionA{\bigcompose_{\agentC \in \agentsB} \learnsC (\bigchoice_{\gamma \in \Gamma_\agentC} \alpha^\gamma)} \necessary_\agentB \bigvee_{\gamma \in \Gamma} \gamma\label{afl-k-synthesis-covers-5}\\
    &\proves& \actionA{\test{\somerefs \phi} \compose \bigcompose_{\agentC \in \agentsB} \learnsC (\bigchoice_{\gamma \in \Gamma_\agentC} \alpha^\gamma)} \necessary_\agentB \bigvee_{\gamma \in \Gamma} \gamma\label{afl-k-synthesis-covers-6}
\end{eqnarray}
(\ref{afl-k-synthesis-covers-1}) follows from the induction hypothesis and {\bf LU},
(\ref{afl-k-synthesis-covers-2}) follows from {\bf NecK},
(\ref{afl-k-synthesis-covers-3}) follows from {\bf LK1},
(\ref{afl-k-synthesis-covers-4}) follows from {\bf LK2} and {\bf LS},
(\ref{afl-k-synthesis-covers-5}) follows from {\bf NecL} and
(\ref{afl-k-synthesis-covers-6}) follows from {\bf LS}.

Further:
\begin{eqnarray}
    &\proves& \somerefs \phi \implies \bigwedge_{\agentB \in \agentsB, \gamma \in \Gamma_\agentB} \possible_\agentB \somerefs \gamma\label{afl-k-synthesis-covers-7}\\
    &\proves& \somerefs \phi \implies \bigwedge_{\agentB \in \agentsB, \gamma \in \Gamma_\agentB} \possible_\agentB \actionE{\alpha^{\gamma}} \gamma\label{afl-k-synthesis-covers-8}\\
    &\proves& \somerefs \phi \implies \bigwedge_{\agentB \in \agentsB, \gamma \in \Gamma_\agentB} \possible_\agentB \actionE{\bigchoice_{\gamma' \in \Gamma_\agentB} \alpha^{\gamma'}} \gamma\label{afl-k-synthesis-covers-9}\\
    &\proves& \somerefs \phi \implies \actionE{\bigcompose_{\agentC \in \agentsB} \learnsC (\bigchoice_{\gamma \in \Gamma_\agentC} \alpha^\gamma)} \bigwedge_{\agentB \in \agentsB, \gamma \in \Gamma_\agentB} \possible_\agentB \gamma\label{afl-k-synthesis-covers-10}\\
    &\proves& \somerefs \phi \implies \actionE{\test{\somerefs \phi} \compose \bigcompose_{\agentC \in \agentsB} \learnsC (\bigchoice_{\gamma \in \Gamma_\agentC} \alpha^\gamma)} \bigwedge_{\agentB \in \agentsB, \gamma \in \Gamma_\agentB} \possible_\agentB \gamma\label{afl-k-synthesis-covers-11}\\
    &\proves& \actionA{\test{\somerefs \phi} \compose \bigcompose_{\agentC \in \agentsB} \learnsC (\bigchoice_{\gamma \in \Gamma_\agentC} \alpha^\gamma)} \bigwedge_{\agentB \in \agentsB, \gamma \in \Gamma_\agentB} \possible_\agentB \gamma\label{afl-k-synthesis-covers-12}\\
    &\proves& \actionA{\test{\somerefs \phi} \compose \bigcompose_{\agentC \in \agentsB} \learnsC (\bigchoice_{\gamma \in \Gamma_\agentC} \alpha^\gamma)} (\pi \land \bigwedge_{\agentB \in \agentsB} \coversB \Gamma_\agentB)\label{afl-k-synthesis-covers-13}
\end{eqnarray}
(\ref{afl-k-synthesis-covers-7}) follows from {\bf RK},
(\ref{afl-k-synthesis-covers-8}) follows from the induction hypothesis,
(\ref{afl-k-synthesis-covers-9}) follows from {\bf LU},
(\ref{afl-k-synthesis-covers-10}) follows from {\bf LK1}, {\bf LK2} and {\bf LS},
(\ref{afl-k-synthesis-covers-11}) and (\ref{afl-k-synthesis-covers-12}) follow from {\bf LT}, and
(\ref{afl-k-synthesis-covers-13}) follows from (\ref{afl-k-synthesis-covers-6}), {\bf RP} {\bf LC} and the definition of the cover operator.

Therefore $\proves \actionA{\alpha} \phi$.

Finally:
\begin{eqnarray}
&\proves& \actionE{\bigcompose_{\agentC \in \agentsB} \learnsC (\bigchoice_{\gamma \in \Gamma_\agentC} \alpha^\gamma)} \top \iff \top\label{afl-k-synthesis-covers-14}\\
&\proves& \actionE{\test{\somerefs \phi} \compose \bigcompose_{\agentC \in \agentsB} \learnsC (\bigchoice_{\gamma \in \Gamma_\agentC} \alpha^\gamma)} \top \iff \somerefs \phi\label{afl-k-synthesis-covers-15}\\
&\proves& \somerefs \phi \implies \actionE{\alpha} \top\label{afl-k-synthesis-covers-16}\\
&\proves& \somerefs \phi \implies \actionE{\alpha} \phi\label{afl-k-synthesis-covers-17}
\end{eqnarray}
(\ref{afl-k-synthesis-covers-14}) follows from {\bf LS} and {\bf LP},
(\ref{afl-k-synthesis-covers-15}) follows from {\bf LS} and {\bf LT},
(\ref{afl-k-synthesis-covers-16}) follows from (\ref{afl-k-synthesis-covers-15}),
(\ref{afl-k-synthesis-covers-17}) follows from (\ref{afl-k-synthesis-covers-13}) and (\ref{afl-k-synthesis-covers-16}),

Therefore $\proves \somerefs \phi \implies \actionE{\alpha} \phi$.
\end{proof}

\begin{corollary}
For every $\kPModel{\kStateS} \in \classK$ and $\phi \in \langAaml$: $\kPModel{\kStateS} \entails \somerefs \phi$ if and only if there exists $\aPModel{\aStateS} \in \classAmK$ such that $\kPModel{\kStateS} \entails \actionE{\aPModel{\aStateS}} \phi$.
\end{corollary}

\subsection{\classKFF{}}

\begin{proposition}\label{afl-kff-synthesis}
For every $\phi \in \langAfl$ there exists $\alpha \in \langAflAct$ such that $\proves \actionA{\alpha} \phi$ and $\proves \somerefs \phi \implies \actionE{\alpha} \phi$.
\end{proposition} 

\begin{proof}
Without loss of generality we assume that $\phi$ is in alternating disjunctive normal form.
We use the same reasoning as in the proof of Proposition~\ref{afl-k-synthesis}, substituting \axiomAflKFF{} axioms for the corresponding \axiomAflK{} axioms, noting that the alternating disjunctive normal form gives the $(\agents \setminus \{\agentA\})$-restricted properties required for {\bf LK1} and the \axiomRmlKFF{} axioms {\bf RK45}, {\bf RComm} and  {\bf RDist} to be applicable.
\end{proof}

\begin{corollary}
For every $\kPModel{\kStateS} \in \classKFF$ and $\phi \in \langAaml$: $\kPModel{\kStateS} \entails \somerefs \phi$ if and only if there exists $\aPModel{\aStateS} \in \classAmKFF$ such that $\kPModel{\kStateS} \entails \actionE{\aPModel{\aStateS}} \phi$.
\end{corollary}

\subsection{\classS{}}

\begin{proposition}\label{afl-s-synthesis}
For every $\phi \in \langAfl$ there exists $\alpha \in \langAflAct$ such that $\proves \actionA{\alpha} \phi$ and $\proves \somerefs \phi \implies \actionE{\alpha} \phi$.
\end{proposition}

\begin{proof}
Without loss of generality, assume that $\phi$ is a disjunction of explicit formulas.
We proceed by induction on the structure of $\phi$.

Suppose that $\phi = \psi \lor \chi$. We use the same reasoning as in the proof of Proposition~\ref{afl-k-synthesis}.

Suppose that $\phi = \pi \land \gamma^0 \land \bigwedge_{\agentA \in \agents} \coversA \Gamma_\agentA$ is an explicit formula.
By the induction hypothesis for every $\agentA \in \agents$, $\gamma \in \Gamma_\agentA$ there exists $\alpha^{\agentA,\gamma} \in \langAflAct$ such that $\proves \actionA{\alpha^{\agentA,\gamma}} \gamma$ and $\proves \somerefs \gamma \implies \actionE{\alpha^{\agentA,\gamma}} \gamma$, where $\tau(\alpha^{\agentA,\gamma}) = \aPModel[\agentA,\gamma]{\aStateS[\agentA,\gamma]} = \aPModelTuple[\agentA,\gamma]{\aStateS[\agentA,\gamma]}$.

Let $\alpha = \test{\somerefs \gamma^0} \compose \bigcompose_{\agentA \in \agents} \learnsA (\test{\top}, \bigchoice_{\gamma \in \Gamma_\agentA} \alpha^{\agentA,\gamma})$.
Then from Lemmas~\ref{afl-s-construction-test} and~\ref{afl-s-construction-learning}: $\tau(\alpha) \bisimilar \aPModel{\aStateS} = \aPModelTuple{\aStateS}$ where:
\begin{eqnarray*}
    \aStates &=& \bigcup_{\agentA \in \agents, \gamma \in \Gamma_\agentA} \aStates[\agentA,\gamma] \cup \{\aPStateS[\agentA,\gamma] \mid \agentA \in \agents, \gamma \in \Gamma_\agentA\} \cup \{\aStateS\}\\
    \aAccessibility{\agentA} &=& \bigcup_{\agentB \in \agents, \gamma \in \Gamma_\agentB} \aAccessibility[\agentB,\gamma]{\agentA} \cup (\{\aStateS\} \cup \{\aPStateS[\agentA,\gamma] \mid \gamma \in \Gamma_\agentA\})^2 \cup \bigcup_{\agentB \in \agents \setminus \{\agentA\}, \gamma \in \Gamma_\agentB} (\{\aPStateS[\agentB,\gamma]\} \cup \aStateS[\agentB,\gamma] \aAccessibility[\agentB,\gamma]{\agentA})^2 \text{ for } \agentA \in \agents\\
    \aPrecondition &=& \bigcup_{\agentA \in \agents, \gamma \in \Gamma_\agentA} \aPrecondition[\agentA,\gamma] \cup \{(\aPStateS[\agentA,\gamma], \aPrecondition[\agentA,\gamma](\aStateS[\agentA<,\gamma])) \mid \agentA \in \agents, \gamma \in \gamma_\agentA\} \cup \{(\aStateS, \somerefs \gamma^0)\}
\end{eqnarray*}

Let $\Psi = \{\psi \leq \gamma \mid \agentA \in \agents, \gamma \in \Gamma_\agentA\}$. We need to show for every $\psi \in \Psi$:

\begin{enumerate}
    \item For every $\agentA \in \agents$: $\proves \actionA{\aPModel{\aStateS}} \psi \iff \actionA{\aPModel{\aStateS[\agentA,\gamma^0]}} \psi$.
    \item For every $\agentA \in \agents$, $\gamma \in \Gamma_\agentA$: $\proves \actionA{\aPModel{\aPStateS[\agentA,\gamma]}} \psi \iff \actionA{\aPModel{\aStateS[\agentA,\gamma]}} \psi$.
    \item For every $\agentA \in \agents$, $\gamma \in \Gamma_\agentA$, $\aStateU \in \aStates[\agentA,\gamma]$: $\proves \actionA{\aPModel{\aStateU}} \psi \iff \actionA{\aPModel[\agentA,\gamma]{\aStateU}} \psi$.
\end{enumerate}

We proceed by induction on $\psi$.

\begin{enumerate}
    \item For every $\agentA \in \agents$: $\proves \actionA{\aPModel{\aStateS}} \psi \iff \actionA{\aPModel{\aStateS[\agentA,\gamma^0]}} \psi$.

        Suppose that $\psi = \atomP$ where $\atomP \in \atoms$. 
        This follows trivially from {\bf AP}.

        Suppose that $\psi = \neg \chi$ or that $\psi = \chi_1 \land \chi_2$.
        These cases follow trivially from the induction hypothesis.

        Suppose that $\psi = \necessaryA \chi$.
        By construction $\aStateS \aAccessibility{\agentA} = \aPStateS[\agentA,\gamma^0] \aAccessibility{\agentA}$ and $\aPrecondition(\aStateS) = \aPrecondition(\aPStateS[\agentA,\gamma^0])$ and so $\proves \actionA{\aPModel{\aStateS}} \necessaryA \chi \iff \actionA{\aPModel{\aPStateS[\agentA,\gamma^0]}} \necessaryA \chi$ follows from {\bf AK} trivially.

        Suppose that $\psi = \necessaryB \chi$ where $\agentB \neq \agentA$. 
        By construction $\aStateS \aAccessibility{\agentB} = \{\aStateS\} \cup \aStateS[\agentB,\gamma^0] \aAccessibility{\agentB}$ and $\aPStateS[\agentA,\gamma^0] \aAccessibility{\agentB} = \{\aPStateS[\agentA,\gamma^0]\} \cup \aStateS[\agentA,\gamma^0] \aAccessibility[\agentA,\gamma^0]{\agentB}$.
        As $\phi$ is an explicit formula and $\necessaryB \chi \in \Psi$ then either $\proves \gamma^0 \implies \necessaryB \chi$ or $\proves \gamma^0 \implies \neg \necessaryB \chi$.
        Suppose that $\proves \gamma^0 \implies \necessaryB \chi$.
        Then for every $\gamma \in \Gamma_\agentB$ we have $\proves \gamma \implies \necessaryB \chi$.
        By the outer induction hypothesis $\proves \actionA{\aPModel[\agentB,\gamma]{\aStateS[\agentB,\gamma]}} \gamma$ and so $\proves \actionA{\aPModel[\agentB,\gamma]{\aStateS[\agentB,\gamma]}} \chi$.
        By the inner induction hypothesis $\proves \actionA{\aPModel{\aStateS[\agentB,\gamma]}} \chi$.
        As $\gamma^0 \in \Gamma_\agentB$ then $\proves \actionA{\aPModel{\aStateS[\agentB,\gamma^0]}} \chi$ and so by the inner induction hypothesis $\proves \actionA{\aPModel{\aStateS}} \chi$.
        So $\proves \actionA{\aPModel{\aStateS \aAccessibility{\agentB}}} \chi$ and therefore $\proves \actionA{\aPModel{\aStateS}} \necessaryB \chi$ follows from {\bf AK}.
        By the outer induction hypothesis $\proves \actionA{\aPModel[\agentA,\gamma^0]{\aStateS[\agentA,\gamma^0]}} \gamma^0$ and so $\proves \actionA{\aPModel[\agentA,\gamma^0]{\aStateS[\agentA,\gamma^0]}} \necessaryB \chi$.
        From {\bf AK} we have $\proves \somerefs{\gamma^0} \implies \necessaryB \actionA{\aPModel[\agentA,\gamma^0]{\aStateS[\agentA,\gamma^0] \aAccessibility[\agentA,\gamma^0]{\agentB}}} \chi$.
        By the inner induction hypothesis $\proves \actionA{\aPModel[\agentA,\gamma^0]{\aStateS[\agentA,\gamma^0] \aAccessibility[\agentA,\gamma^0]{\agentB}}} \chi \iff \actionA{\aPModel{\aStateS[\agentA,\gamma^0] \aAccessibility[\agentA,\gamma^0]{\agentB}}} \chi$ and as $\proves \actionA{\aPModel{\aStateS}} \chi$ then $\proves \actionA{\aPModel{\aPStateS[\agentA,\gamma^0]}} \chi$.
        So we have $\proves \actionA{\aPModel[\agentA,\gamma^0]{\aStateS[\agentA,\gamma^0] \aAccessibility[\agentA,\gamma^0]{\agentB}}} \chi \iff \actionA{\aPModel{\aPStateS[\agentA,\gamma^0] \aAccessibility{\agentB}}} \chi$ and $\proves \somerefs{\gamma^0} \implies \necessaryB \actionA{\aPModel{\aPStateS[\agentA,\gamma^0] \aAccessibility{\agentB}}} \chi$ and so $\proves \actionA{\aPModel{\aPStateS[\agentA,\gamma^0]}} \necessaryB$ follows from {\bf AK}.
        Therefore $\proves \actionA{\aPModel{\aStateS}} \necessaryB \chi \iff \actionA{\aPModel{\aPStateS[\agentA,\gamma^0]}} \necessaryB \chi$.
        Suppose that $\proves \gamma^0 \implies \neg \necessaryB \chi$.
        A dual argument can be used to show that $\proves \neg \actionA{\aPModel{\aStateS}} \necessaryB \chi$ and $\proves \neg \actionA{\aPModel{\aPStateS[\agentA,\gamma^0]}} \necessaryB \chi$ and therefore $\proves \actionA{\aPModel{\aStateS}} \necessaryB \chi \iff \actionA{\aPModel{\aPStateS[\agentA,\gamma^0]}} \necessaryB \chi$.

    \item For every $\agentA \in \agents$, $\gamma \in \Gamma_\agentA$: $\proves \actionA{\aPModel{\aPStateS[\agentA,\gamma]}} \psi \iff \actionA{\aPModel{\aStateS[\agentA,\gamma]}} \psi$.

        Suppose that $\psi = \atomP$ where $\atomP \in \atoms$. 
        This follows trivially from {\bf AP}.

        Suppose that $\psi = \neg \chi$ or that $\psi = \chi_1 \land \chi_2$. These cases follow trivially from the induction hypothesis.

        Suppose that $\psi = \necessaryA \chi$.
        By construction $\aPStateS[\agentA,\gamma] \aAccessibility{\agentA} = \{\aStateS\} \cup \{\aPStateS[\agentA,\gamma] \mid \delta \in \Gamma_\agentA\}$ and $\aStateS[\agentA,\gamma] \aAccessibility{\agentA} = \aStateS[\agentA,\gamma] \aAccessibility[\agentA,\gamma]{\agentA}$.
        As $\phi$ is an explicit formula and $\necessaryA \chi \in \Psi$ then either $\proves \gamma \implies \necessaryA \chi$ or $\proves \gamma \implies \neg \necessaryA \chi$.
        Suppose that $\proves \gamma \implies \necessaryA \chi$.
        Then for every $\delta \in \Gamma_\agentA$ we have $\proves \delta \implies \necessaryA \chi$.
        By the outer induction hypothesis $\proves \actionA{\aPModel[\agentA,\delta]{\aStateS[\agentA,\delta]}} \delta$ and so $\proves \actionA{\aPModel[\agentA,\delta]{\aStateS[\agentA,\delta]}} \chi$.
        By the inner induction hypothesis $\proves \actionA{\aPModel{\aStateS[\agentA,\delta]}} \chi$ and $\proves \actionA{\aPModel{\aPStateS[\agentA,\delta]}} \chi$.
        As $\gamma^0 \in \Gamma_\agentA$ then $\proves \actionA{\aPModel{\aPStateS[\agentA,\gamma^0]}} \chi$ and by the inner induction hypothesis $\proves \actionA{\aPModel{\aStateS}} \chi$.
        So $\proves \actionA{\aPModel{\aPStateS[\agentA,\gamma] \aAccessibility{\agentA}}} \chi$ and therefore $\proves \actionA{\aPModel{\aPStateS[\agentA,\gamma]}} \necessaryA \chi$ follows from {\bf AK}.
        By the outer induction hypothesis $\proves \actionA{\aPModel[\agentA,\gamma]{\aStateS[\agentA,\gamma]}} \gamma$ and so $\proves \actionA{\aPModel[\agentA,\gamma]{\aStateS[\agentA,\gamma]}} \necessaryA \chi$.
        From {\bf AK} we have $\proves \somerefs \gamma \implies \necessaryA \actionA{\aPModel[\agentA,\gamma]{\aStateS[\agentA,\gamma] \aAccessibility[\agentA,\gamma]{\agentA}}} \chi$.
        By the inner induction hypothesis $\proves \actionA{\aPModel[\agentA,\gamma]{\aStateS[\agentA,\gamma] \aAccessibility[\agentA,\gamma]{\agentA}}} \chi \iff \actionA{\aPModel{\aStateS[\agentA,\gamma] \aAccessibility[\agentA,\gamma]{\agentA}}} \chi$ so $\proves \somerefs \gamma \implies \necessaryA \actionA{\aPModel{\aStateS[\agentA,\gamma] \aAccessibility{\agentA}}} \chi$ and so $\proves \actionA{\aPModel{\aStateS[\agentA,\gamma]}} \necessaryA \chi$ follows from {\bf AK}.

        Suppose that $\proves \gamma \implies \neg \necessaryB \chi$ where $\agentB \neq \agentA$.
        Therefore $\proves \actionA{\aPModel{\aPStateS[\agentA,\gamma]}} \necessaryA \chi \iff \actionA{\aPModel{\aStateS[\agentA,\gamma]}} \necessaryA \chi$.
        A dual argument can be used to show that $\proves \neg \actionA{\aPModel{\aPStateS[\agentA,\gamma]}} \necessaryA \chi$ and $\proves \neg \actionA{\aPModel{\aStateS[\agentA,\gamma]}} \necessaryA \chi$ and therefore $\proves \actionA{\aPModel{\aPStateS[\agentA,\gamma]}} \necessaryA \chi \iff \actionA{\aPModel{\aStateS[\agentA,\gamma]}} \necessaryA \chi$.

        Suppose that $\psi = \necessaryB \chi$ where $\agentB \neq \agentA$.
        By construction $\aPStateS[\agentA,\gamma] \aAccessibility{\agentB} = \aStateS[\agentA,\gamma] \aAccessibility{\agentB}$ and $\aPrecondition(\aPStateS[\agentA,\gamma]) = \aPrecondition(\aStateS[\agentA,\gamma])$ and so $\proves \actionA{\aPModel{\aPStateS[\agentA,\gamma]}} \necessaryB \chi \iff \actionA{\aPModel{\aStateS[\agentA,\gamma]}} \necessaryB \chi$ follows from {\bf AK} trivially.

    \item For every $\agentA \in \agents$, $\gamma \in \Gamma_\agentA$, $\aStateU \in \aStates[\agentA,\gamma]$: $\proves \actionA{\aPModel{\aStateU}} \psi \iff \actionA{\aPModel[\agentA,\gamma]{\aStateU}} \psi$.

        Suppose that $\psi = \atomP$ where $\atomP \in \atoms$. 
        This follows trivially from {\bf AP}.

        Suppose that $\psi = \neg \chi$ or that $\psi = \chi_1 \land \chi_2$. These cases follow trivially from the induction hypothesis.

        Suppose that $\psi = \necessaryA \chi$.
        By construction $\aStateU \aAccessibility{\agentA} = \aStateU \aAccessibility[\agentA,\gamma]{\agentA}$ and $\aPrecondition(\aStateU) = \aPrecondition[\agentA,\gamma](\aStateU)$ and so $\proves \actionA{\aPModel{\aStateU}} \necessaryA \chi \iff \actionA{\aPModel[\agentA,\gamma]{\aStateU}} \necessaryA \chi$ follows from {\bf AK} and the induction hypothesis trivially.

        Suppose that $\psi = \necessaryB \chi$ where $\agentB \neq \agentA$.
        By construction $\aStateU \aAccessibility{\agentA} = \aStateU \aAccessibility[\agentA,\gamma]{\agentA}$ or $\aStateU \aAccessibility{\agentA} = \{\aPStateS[\agentA,\gamma]\} \cup \aStateU \aAccessibility[\agentA,\gamma]{\agentA}$ and $\aPrecondition(\aStateU) = \aPrecondition[\agentA,\gamma](\aStateU)$ and so $\proves \actionA{\aPModel{\aStateU}} \necessaryB \chi \iff \actionA{\aPModel[\agentA,\gamma]{\aStateU}} \necessaryB \chi$ follows from {\bf AK} and the induction hypothesis trivially.
\end{enumerate}

Therefore for every $\agentA \in \agents$, $\gamma \in \Gamma_\agentA$ we have that $\proves \actionA{\aPModel{\aStateS[\agentA,\gamma]}} \gamma$ and $\proves \actionA{\aPModel{\aStateS}} \gamma^0$.
Therefore for every $\agentA \in \agents$ we have $\proves \actionA{\aPModel{\aStateS \aAccessibility{\agentA}}} \bigvee_{\gamma \in \Gamma_\agentA} \gamma$ and so from {\bf AK} we have that $\proves \actionA{\aPModel{\aStateS}} \necessaryA \bigvee_{\gamma \in \Gamma_\agentA} \gamma$.

As $\phi$ is an explicit formula, from {\bf RDist}, {\bf RS5} and {\bf RComm} we have that $\somerefs \phi \implies \pi \land \bigwedge_{\agentA \in \agents, \gamma \in \Gamma_\agentA} \possibleA \somerefs \gamma$.
By construction for every $\agentA \in \agents$, $\gamma \in \Gamma_\agentA$ we have $\aPrecondition(\aPStateS[\agentA,\gamma]) = \somerefs \gamma$ and from above we have $\proves \actionA{\aPModel{\aStateS[\agentA,\gamma]}} \gamma$ therefore $\proves \somerefs \phi \implies \pi \land \bigwedge_{\agentA \in \agents, \gamma \in \Gamma_\agentA} \possibleA \actionE{\aPModel{\aPStateS[\agentA,\gamma]}} \gamma$.
Therefore by {\bf AK} we have $\proves \somerefs \phi \implies \actionE{\aPModel{\aStateS}} (\pi \land \bigwedge_{\agentA \in \agents, \gamma \in \Gamma_\agentA} \possibleA \gamma)$.
From above we have $\proves \actionA{\aPModel{\aStateS}} \necessaryA \bigvee_{\gamma \in \Gamma_\agentA} \gamma$ and therefore $\proves \somerefs \phi \implies \actionA{\aPModel{\aStateS}} \phi$.
As $\proves \phi \implies \gamma^0$ then $\proves \somerefs \phi \implies \somerefs \gamma^0$ and so $\proves \somerefs \phi \implies \actionE{\aPModel{\aStateS}} \phi$.

Let $\alpha' = \test{\somerefs \phi} \compose \alpha$.
By {\bf LS} we have $\proves \actionA{\alpha'} \phi \iff \actionA{\test{\somerefs \phi}} \actionA{\alpha} \phi$.
By {\bf LT} we have $\proves \actionA{\alpha'} \phi \iff (\somerefs \phi \implies \actionA{\alpha} \phi)$.
From above we have $\proves \somerefs \phi \implies \actionA{\alpha} \phi$ and therefore $\proves \actionA{\alpha'} \phi$.
By {\bf LS} we have $\proves \actionE{\alpha'} \phi \iff \actionE{\test{\somerefs \phi}} \actionE{\alpha} \phi$.
By {\bf LT} we have $\proves \actionE{\alpha'} \phi \iff (\somerefs \phi \land \actionE{\alpha} \phi)$.
From above we have $\proves \somerefs \phi \implies \actionE{\alpha} \phi$ and therefore $\proves \somerefs \phi \implies \actionE{\alpha'} \phi$.
\end{proof}

\begin{corollary}
For every $\kPModel{\kStateS} \in \classS$ and $\phi \in \langAaml$: $\kPModel{\kStateS} \entails \somerefs \phi$ if and only if there exists $\aPModel{\aStateS} \in \classAmS$ such that $\kPModel{\kStateS} \entails \actionE{\aPModel{\aStateS}} \phi$.
\end{corollary}

\section{Related work}\label{related-work}

Several other papers have addressed the problem of describing and reasoning about epistemic actions.
One of the most important works in this area is the work of Baltag, Moss and Solecki~\cite{baltag:1998} which introduced the notion of action model logic, building on the earlier work of Gerbrandy and Groeneveld~\cite{gerbrandy:1997}.
In later work Baltag and Moss extended action model logic to consider epistemic programs~\cite{baltag:2004} which are expressions built from action models using such operators as sequential composition, non-deterministic choice and iteration.
The atoms of these programs are action models, so the approach is still inherently semantic in nature.
The logic is unable to decompose the program beyond the level of the atoms, which themselves may be complex semantic objects.

The relational actions of van Ditmarsch~\cite{vanditmarsch:2000} provides a syntactic mechanism for describing an epistemic action, and provides the foundation for a lot of the work presented in this paper.
The relational actions are constructed using essentially the same operators as in the language of action formulas.
While the language is very similar, the semantics given are quite different~\cite{vanditmarsch:2007}.
In the logic of epistemic actions the semantics are given in such a way that worlds in a model are specified with respect to subsets of agents, so that the model is restricted to agents for whom the epistemic action was applied.
The semantics were also specific to \classS{}, and non-trivial to generalise to other epistemic logics.
A version of relational actions with concurrency is able to describe any \classS{} action model, although it is unknown whether the expressivity of concurrent relational actions is greater than that of action models~\cite{baltag:2006}.
Here we have generalised the approach and provided a correspondence theorem for action model logic.
This has allowed us to retain the more familiar semantics of epistemic logic, generalise the logic to \classK{} and \classKFF{} as well as access existing synthesis results for dynamic epistemic logic~\cite{hales:2013}.

The synthesis result presented here is built on the work of Hales~\cite{hales:2013} which gave a method to build an action model to satisfy a given epistemic goal.
This construction inspired the syntactic description of epistemic actions and approach that we have used in this paper.

Related synthesis results have been given by Aucher, et al.~\cite{aucher:2011,aucher:2012,aucher:2013} which presents an event model language and uses it to give a thorough exploration of the relationship between epistemic models, action models and epistemic goals.
Aucher defines a logic for action models and provides calculi to describe epistemic progression (what is true after executing a given action in a given model) epistemic regression (what is the most general precondition for an epistemic action given an epistemic goal) and epistemic planning (what action is sufficient to achieve an epistemic goal given some precondition).
In future work we hope to extend the correspondence between action formula logic and action models to include Aucher's event model language.
