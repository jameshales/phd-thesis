\section{Modal logic}\label{ml}

We recall standard definitions and results from modal logic.
For an introductory text on modal logic we direct the reader to the books by Blackburn, de~Rijke and Venema~\cite{blackburn:2001}, and Hughes and Cresswell~\cite{hughes:1996}.
Many simple or well-known propositions are given without proof and left as an exercise for the curious reader.

\subsection{Syntax and semantics}

Let $\atoms$ be a non-empty, countable set of propositional atoms, and
let $\agents$ be a non-empty, finite set of agents.

\begin{definition}[Language of modal logic]
The {\em language of modal logic}, \langMl{}, is inductively defined as:
$$
\phi ::= 
    \atomP \mid
    \lnot \phi \mid
    \phi \land \phi \mid
    \necessaryA \phi
$$
where $\atomP \in \atoms$ and $\agentA \in \agents$.
\end{definition}

We use all of the standard abbreviations from propositional logic:
$\bot ::= \atomP \land \lnot \atomP$;
$\top ::= \lnot \bot$;
$\phi \lor \psi ::= \lnot (\lnot \phi \land \lnot \psi)$;
$\phi \implies \psi ::= \lnot \phi \lor \psi$; and
$\phi \iff \psi ::= (\phi \implies \psi) \land (\psi \implies \phi)$.
We also use the abbreviation $\possibleA \phi ::= \lnot \necessaryA \lnot \phi$.
When we are working in a single-agent setting we will write $\necessary$ and $\possible$ instead of $\necessaryA$ and $\possibleA$.
We write $\psi \leq \phi$ to denote that $\psi$ is a subformula or is equal to $\phi$.

The formula $\necessaryA \phi$ may be read as ``agent $\agentA$ {\em knows} that $\phi$ is true'', or ``agent $\agentA$ {\em believes} that $\phi$ is true'', depending on which terminology is appropriate for the setting we are working in.
The formula $\possibleA \phi$ may be read as ``agent $\agentA$ considers it possible that $\phi$ is true''.
This reading can be understood with respect to the definition of the $\possibleA$ operator as the dual of the $\necessaryA$ operator: if an agent doesn't know or believe that a statement is false, then the agent must be open to the possibility that the statement is true (and vice versa).

\begin{example}\label{example-ml-formula}
Suppose that Alice (agent $\agentA$) has flipped a coin and, being careful that Bob (agent $\agentB$) can't see it, she looks at the coin and sees that it has landed heads up (atom $\atomP$).
Then the formula $\atomP$ may be read as ``the coin landed heads up'',
the formula $\knowsA \atomP$ may be read as ``Alice knows that the coin landed heads up'',
the formula $\lnot \knowsB \atomP$ may be read as ``Bob doesn't know that the coin landed heads up'',
and the formula $\knowsA \lnot \knowsB \atomP$ may be read as ``Alice knows that Bob doesn't know that the coin landed heads up''.
If Alice was not as careful when she looked at the coin we might write the formula $\possibleA \knowsB \atomP$, which may be read as ``Alice considers it possible that Bob knows that the coin landed heads up'', 
or equivalently $\lnot \knowsA \lnot \knowsB \atomP$, which may be read as ``Alice doesn't know that Bob doesn't know that the coin landed heads up''.
\end{example}

Under the standard Kripke semantics for modal logics, modal formulas are interpreted over relational structures known as Kripke models~\cite{tarski:1941,kripke:1959,hintikka:1962}.
In epistemic and doxastic logics, Kripke models are thought of as abstract models of the knowledge or beliefs of a set of agents.

We first define Kripke frames, the relational components of Kripke models.

\begin{definition}[Kripke frames]
A {\em Kripke frame}, $\kFrameAndTuple$ consists of:
a {\em domain} $\kStates$, which is a non-empty set of states; and
an indexed set of {\em accessibility relations} $\kAccessibilityRel$, indexed on $\agents$, where for every $\agentA \in \agents$, $\kAccessibilityA \subseteq \kStates \times \kStates$ is a binary relation on states.
\end{definition}

We write $\kStateS \kAccessibilityA \kStateT$ to denote that $(\kStateS, \kStateT) \in \kAccessibilityA$.
We write $\kSuccessorsA{\kStateS}$ to denote the set of successor states $\kSuccessorsA{\kStateS} = \{\kStateT \in \kStates \mid \kStateS \kAccessibilityA \kStateT\}$ and
we write $\kPredecessorsA{\kStateT}$ to denote the set of predecessor states $\kPredecessorsA{\kStateT} = \{\kStateS \in \kStates \mid \kStateS \kAccessibilityA \kStateT\}$.

We will be working with a variety of modal logics that are defined by relational properties on Kripke frames.
We define those relational properties here.

\pagebreak

\begin{definition}[Relational properties]
Let $\kStates$ be a set and let $\kAccessibilityRel \subseteq \kStates \times \kStates$ be a binary relation on $\kStates$. 
Then we say that $\kAccessibilityRel$ is \ldots
\begin{itemize}
    \item \ldots {\em serial} if and only if for every $\kStateS \in \kStates$ there exists $\kStateT \in \kStates$ such that $\kStateS \kAccessibility{} \kStateT$.
    \item \ldots {\em reflexive} if and only if for every $\kStateS \in \kStates$: $\kStateS \kAccessibility{} \kStateS$.
    \item \ldots {\em transitive} if and only if for every $\kStateS, \kStateT, \kStateU \in \kStates$: if $\kStateS \kAccessibility{} \kStateT$ and $\kStateT \kAccessibility{} \kStateU$ then $\kStateS \kAccessibility{} \kStateU$.
    \item \ldots {\em symmetric} if and only if for every $\kStateS, \kStateT \in \kStates$: if $\kStateS \kAccessibility{} \kStateT$ then $\kStateT \kAccessibility{} \kStateS$.
    \item \ldots {\em Euclidean} if and only if for every $\kStateS, \kStateT, \kStateU \in \kStates$: if $\kStateS \kAccessibility{} \kStateT$ and $\kStateS \kAccessibility{} \kStateU$ then $\kStateT \kAccessibility{} \kStateU$.
\end{itemize}
\end{definition}

When we ascribe relational properties to Kripke frames we actually ascribe those properties to each of the accessibility relations of the Kripke frame.
So we say that a Kripke frame $\kFrameAndTuple$ is \{serial, reflexive, etc.\} if and only if for every $\agentA \in \agents$ the binary relation $\kAccessibilityA$ is \{serial, reflexive, etc.\}.

We use these relational properties to define the classes of Kripke frames that we will be working with.

\begin{definition}[Classes of Kripke frames]
We define the following classes of Kripke frames:
\begin{itemize}
    \item The class \classK{} of all Kripke frames.
    \item The class \classKF{} of all transitive Kripke frames.
    \item The class \classKFF{} of all transitive and Euclidean Kripke frames.
    \item The class \classKD{} of all serial, transitive and Euclidean Kripke frames.
    \item The class \classS{} of all reflexive, transitive and Euclidean Kripke frames.
\end{itemize}
\end{definition}

In order to interpret the validity of modal formulas containing propositional atoms we augment Kripke frames with valuations of propositional atoms.

\begin{definition}[Kripke models]
A {\em Kripke model}, $\kModelAndTuple$ consists of an {\em underlying Kripke frame} $\kFrameAndTuple$ along with a {\em valuation function} $\kValuation : \atoms \to \powerset(\kStates)$, which is a function from propositional atoms to sets of states.

A {\em pointed Kripke model} $\kPModelAndTuple{\kStateS}$ consists of a Kripke model $\kModelAndTuple$ along with a designated state (the {\em real world}), $\kStateS \in \kStates$.
A {\em multi-pointed Kripke model} $\kPModelAndTuple{\kStatesT}$ consists of a Kripke model $\kModelAndTuple$ along with a non-empty set of designated states, $\kStatesT \subseteq \kStates$.
\end{definition}

As we will usually work with Kripke models rather than frames we overload the notation for classes of Kripke frames to also refer to classes of Kripke models.
When we ascribe relational properties or frame properties to a Kripke model we actually ascribe those properties to its underlying Kripke frame.
We will sometimes treat a (single-)pointed Kripke model as though it were a multi-pointed Kripke model and we assume in this case that $\kPModel{\kStateS}$ is an abbreviation for $\kPModel{\{\kStateS\}}$.

We now define the semantics of modal logic.
The semantics are defined in terms of a parameterised class of Kripke frames, \classC{}, which could stand for \classK{}, \classKF{}, \classKFF{}, etc. or for any other class of Kripke frames so defined.

\begin{definition}[Semantics of modal logic]\label{ml-semantics}
Let $\classC$ be a class of Kripke frames, let $\phi \in \langMl$, and let $\kPModelAndTuple{\kStateS} \in \classC$ be a pointed Kripke model.
The interpretation of the formula $\phi$ in the logic \logicC{} on the pointed Kripke model $\kPModel{\kStateS}$ is defined inductively as:
$$
\begin{array}{lcl}
\kPModel{\kStateS} \entails \atomP & \text{ iff } & \kStateS \in \kValuation(\atomP)\\
\kPModel{\kStateS} \entails \lnot \phi & \text{ iff } & \kPModel{\kStateS} \nentails \phi\\
\kPModel{\kStateS} \entails \phi \land \psi & \text{ iff } & \kPModel{\kStateS} \entails \phi \text{ and } \kPModel{\kStateS} \entails \psi\\
\kPModel{\kStateS} \entails \necessaryA \phi & \text{ iff } & \text{for every } \kStateT \in \kSuccessorsA{\kStateS} : \kPModel{\kStateT} \entails \phi
\end{array}
$$
\end{definition}

In particular we are interested in the modal logics \logicK{}, \logicKF{}, \logicKFF{}, \logicKD{}, and \logicS{} for the respective classes of Kripke frames \classK{}, \classKF{}, \classKFF{}, \classKD{}, and \classS{}.

Let $\kPModel{\kStateS} \in \classC$ be a pointed Kripke model,
let $\kFrame \in \classC$ be a Kripke frame.
If $\kPModel{\kStateS} \entails \phi$ then we say that $\phi$ is {\em valid} on $\kPModel{\kStateS}$.
Let $\Phi \subseteq \langMl$ be a (possibly infinite) set of formulas.
If $\kPModel{\kStateS} \entails \phi$ for every $\phi \in \Phi$ then we say that $\Phi$ is {\em valid} on $\kPModel{\kStateS}$ and we write $\kPModel{\kStateS} \entails \Phi$.
If $\kPModel{\kStateS} \entails \Phi$ for every $\kStateS \in \kStates$ then we say that $\Phi$ is valid on $\kModel$ and we denote this by $\kModel \entails \Phi$.
If $\kModel \entails \Phi$ for every $\kModel$ with the underlying Kripke frame $\kFrame$ then we say that $\Phi$ is valid on $\kFrame$ and we denote this by $\kFrame \entails \Phi$.
If $\kFrame \entails \Phi$ for every $\kFrame \in \classC$ then we say that $\Phi$ is valid on \classC{} and we denote this by $\classC \entails \Phi$.
When \classC{} is clear from context we may simply write $\entails \Phi$ instead of $\classC \entails \Phi$.
If $\kStatesT \subseteq \kStates$ and there exists $\kStateS \in \kStatesT$ such that $\kPModel{\kStateS} \entails \Phi$ then we say that $\Phi$ is {\em satisfiable} in $\kPModel{\kStatesT}$.
If $\Phi$ is satisfiable in $\kPModel{\kStates}$ then we say that $\Phi$ is satisfiable in $\kModel$.
If $\Phi$ is satisfiable in $\kModel$ with the underlying Kripke frame $\kFrame$ then we say that $\Phi$ is satisfiable in $\kFrame$.
If there exists $\kFrame \in \classC$ such that $\Phi$ is satisfiable in $\kFrame$ then we say that $\Phi$ is satisfiable in \classC{}.
If every finite subset of $\Phi$ is satisfiable in \{$\kPModel{\kStatesT}$, $\kModel$, $\kFrame$, \classC{}\} then we say that $\Phi$ is {\em finitely satisfiable} in \{$\kPModel{\kStatesT}$, $\kModel$, $\kFrame$, \classC{}\}.

We write $\interpretation[\kModel]{\phi}$ to denote the set of states where $\phi$ is valid, where $\interpretation[\kModel]{\phi} = \{\kStateS \in \kStates \mid \kPModel{\kStateS} \entails \phi\}$.

\begin{example}\label{example-ml}
Let $\kPModelAndTuple{\kStateS}$ and $\kPModelAndTupleP{\kStateS}$ be pointed Kripke models where:
\begin{eqnarray*}
    \kStates &=& \{\kStateS, \kStateT\}\\
    \kAccessibilityA &=& \{(\kStateS, \kStateS), (\kStateT, \kStateT)\}\\
    \kAccessibilityB &=& \{(\kStateS, \kStateS), (\kStateS, \kStateT), (\kStateT, \kStateS), (\kStateT, \kStateT)\}\\
    \kValuation(\atomP) &=& \{\kStateS\}
\end{eqnarray*}
and:
\begin{eqnarray*}
    \kStatesP &=& \{\kStateS, \kStateT, \kStateU\}\\
    \kAccessibilityPA &=& \{(\kStateS, \kStateS), (\kStateT, \kStateT), (\kStateS, \kStateU), (\kStateU, \kStateS), (\kStateU, \kStateU)\}\\
    \kAccessibilityPB &=& \{(\kStateS, \kStateS), (\kStateS, \kStateT), (\kStateT, \kStateS), (\kStateT, \kStateT), (\kStateU, \kStateU)\}\\
    \kValuationP(\atomP) &=& \{\kStateS, \kStateU\}
\end{eqnarray*}

\begin{figure}[b!]
    \caption{Two example Kripke models, modelling the situations described in Example~\ref{example-ml-formula}.}\label{example-ml-figure}
    \centering
    \begin{tikzpicture}[>=stealth',shorten >=1pt,auto,node distance=7em,thick]

      \node[label=above left:{$\kStateS$}] (s') {\underline{$\{\atomP\}$}};
      \node[label=above left:{$\kStateT$}] (t') [above of=s'] {$\{\}$};
      \node[label=above left:{$\kStateU$}] (u') [below of=s'] {$\{\atomP\}$};
      \node[label=above:{$\kPModelP{\kStateS}$},draw=black, fit=(s') (t') (u'), inner sep=3.5em, dashed, rounded corners] {};

      \path[every node/.style={font=\sffamily\small},->]
        (s') edge [loop right] node {$\agentA, \agentB$} (s')
        (t') edge [loop right] node {$\agentA, \agentB$} (t')
        (u') edge [loop right] node {$\agentA, \agentB$} (u')
        (s') [<->] edge node {$\agentB$} (t')
        (s') edge node {$\agentA$} (u');

      \node[label=above right:{$\kStateS$}] (s) [left=11em of s']{\underline{$\{\atomP\}$}};
      \node[label=above right:{$\kStateT$}] (t) [above of=s] {$\{\}$};
      \node[label=above:{$\kPModel{\kStateS}$},draw=black, fit=(s) (t), inner sep=3.5em, dashed, rounded corners] {};

      \path[every node/.style={font=\sffamily\small},->]
        (s) edge [loop left] node {$\agentA, \agentB$} (s)
        (t) edge [loop left] node {$\agentA, \agentB$} (t)
        (s) [<->] edge node {$\agentB$} (t);
    \end{tikzpicture}
\end{figure}
The Kripke models $\kPModel{\kStateS}$ and $\kPModelP{\kStateS}$ are shown in Figure~\ref{example-ml-figure}.

Recall that in Example~\ref{example-ml-formula} we described a situation where Alice has flipped a coin and, being careful that Bob can't see it, she looks at the coin and sees that it has landed heads up.
The Kripke model $\kPModel{\kStateS}$ models this situation, as we note that:
$\kPModel{\kStateS} \entails \atomP$;
$\kPModel{\kStateS} \entails \atomP$;
$\kPModel{\kStateS} \entails \knowsA \atomP$;
$\kPModel{\kStateS} \entails \lnot \knowsB \atomP$; and
$\kPModel{\kStateS} \entails \knowsA \lnot \knowsB \atomP$.

In Example~\ref{example-ml-formula} a second situation was described where Alice was not as careful and Alice considers it possible that Bob saw the coin.
The Kripke model $\kPModelP{\kStateS}$ models this situation as we note that:
$\kPModelP{\kStateS} \entails \atomP$;
$\kPModelP{\kStateS} \entails \atomP$;
$\kPModelP{\kStateS} \entails \knowsA \atomP$; and
$\kPModelP{\kStateS} \entails \lnot \knowsA \lnot \knowsB \atomP$.
In this particular model we have that $\kPModelP{\kStateS} \entails \lnot \knowsB \atomP$, so although Alice considers it possible that Bob saw the coin, in reality he didn't.
\end{example}

\subsection{Bisimulations and bisimilarity}

We next recall definitions and results related to bisimilarity of Kripke models.
Bisimilarity is an important concept in modal logics.
Notably the modal logic \logicK{} corresponds to the bisimulation-invariant fragment of first-order logic~\cite{vanbenthem:1984}, and there is a partial correspondence between bisimilarity and equivalence under modal validity~\cite{goranko:2006}.

\begin{definition}[Bisimulation]
Let $\kModelAndTuple$ and $\kModelAndTupleP$ be Kripke models.
A non-empty relation $\bisimulation \subseteq \kStates \times \kStatesP$ is a {\em bisimulation} if and only if for every $\atomP \in \atoms$, $\agentA \in \agents$ and $(\kStateS, \kStateSP) \in \bisimulation$ the following conditions, {\bf atoms-$\atomP$}, {\bf forth-$\agentA$} and {\bf back-$\agentA$} holds:

\paragraph{atoms-$\atomP$}
$\kStateS \in \kValuation(\atomP)$ if and only if $\kStateSP \in \kValuationP(\atomP)$.

\paragraph{forth-$\agentA$}
For every $\kStateT \in \kSuccessorsA{\kStateS}$ there exists $\kStateTP \in \kSuccessorsPA{\kStateSP}$ such that $(\kStateT, \kStateTP) \in \bisimulation$.

\paragraph{back-$\agentA$}
For every $\kStateTP \in \kSuccessorsPA{\kStateSP}$ there exists $\kStateT \in \kSuccessorsA{\kStateS}$ such that $(\kStateT, \kStateTP) \in \bisimulation$.

If there exists a bisimulation $\bisimulation$ such that $(\kStateS, \kStateSP) \in \bisimulation$ then we say that $\kPModel{\kStateS}$ and $\kPModelP{\kStateSP}$ are {\em bisimilar} and we denote this by $\kPModel{\kStateS} \bisimilar \kPModelP{\kStateSP}$.
\end{definition}

Bisimulations were first developed by Milner~\cite{milner:1980} and Park~\cite{park:1981} to capture a notion of process equivalence in the process algebra for concurrent systems, and have appeared in the context of modal logics since van Benthem~\cite{vanbenthem:1984}.

We first note that the relational operator $\bisimilar$ for bisimilarity forms an equivalence relation.

\begin{proposition}\label{bisimulation-equivalence-relation}
The relational operator $\bisimilar$ is an equivalence relation (reflexive, transitive and symmetric) on Kripke models.
\end{proposition}

Bisimilar Kripke models are equivalent under modal validity.

\begin{proposition}\label{modal-bisimulation-invariance}
Let $\kPModel{\kStateS}$ and $\kPModelP{\kStateSP}$ be pointed Kripke models such that $\kPModel{\kStateS} \bisimilar \kPModelP{\kStateSP}$.
Then for every $\phi \in \langMl$:
$\kPModel{\kStateS} \entails \phi$ if and only if $\kPModelP{\kStateSP} \entails \phi$.
\end{proposition}

Not all modally equivalent Kripke models are bisimilar.
However we can define a property of Kripke models that guarantees that modal equivalence implies bisimilarity.

\begin{definition}
Let $\kModelAndTuple$ be a Kripke model.
We say that $\kModel$ is {\em modally saturated} if and only if for every $\agentA \in \agents$, $\kStateS \in \kStates$ and $\Phi \subseteq \langMl$: if $\Phi$ is finitely satisfiable in $\kPModel{\kSuccessorsA{\kStateS}}$ then $\Phi$ is satisfiable in $\kPModel{\kSuccessorsA{\kStateS}}$.
\end{definition}

For modally saturated Kripke models, modal equivalence implies bisimilarity.

\begin{proposition}\label{modal-hennessy-milner}
Let $\kModel$ and $\kModelP$ be modally saturated Kripke models such that for every $\phi \in \langMl$: $\kPModel{\kStateS} \entails \phi$ if and only if $\kPModelP{\kStateSP} \entails \phi$.
Then $\kPModel{\kStateS} \bisimilar \kPModelP{\kStateSP}$.
\end{proposition}

A notable class of modally saturated Kripke models are the finite and image-finite Kripke models.

\begin{definition}[Finite Kripke models]
Let $\kModelAndTuple$ be a Kripke model.
We say that $\kModel$ is {\em finite} if and only if $\kStates$ is finite.
\end{definition}

\begin{definition}[Image-finiteness]
Let $\kModelAndTuple$ be a Kripke model.
We say that $\kModel$ is {\em image-finite} if and only if for every $\agentA \in \agents$ and $\kStateS \in \kStates$: $\kSuccessorsA{\kStateS}$ is finite.
\end{definition}

\begin{proposition}
Every finite Kripke model is image-finite.
\end{proposition}

\begin{proposition}
Every image-finite Kripke model is modally saturated.
\end{proposition}

Thus for finite or image-finite Kripke models it is also the case that modal equivalence implies bisimilarity.

Bisimulations can be computed in polynomial time for finite Kripke models.

\begin{proposition}
Let $\kModel$ and $\kModelP$ be Kripke models such that $\kModel \bisimilar \kModelP$.
There is a unique, maximal bisimulation between $\kModel$ and $\kModelP$.
\end{proposition}

\begin{proposition}
Let $\kModel$ and $\kModelP$ be finite Kripke models defined on a finite set of propositional atoms such that $\kModel \bisimilar \kModelP$.
The maximal bisimulation between $\kModel$ and $\kModelP$ can be computed in polynomial time.
\end{proposition}

These results are shown by Goranko and Otto~\cite[pp. 273-274]{goranko:2006}.

We can define a bisimulation-minimal version of a Kripke model by `merging' bisimilar states as in the bisimulation contraction, resulting in a Kripke model where no two states are bisimilar.
Bisimulation contracted Kripke models are useful because bisimilarity of states corresponds precisely to equality of states, so in a bisimulation contracted Kripke model no two distinct states are bisimilar to one another.

\begin{definition}[Bisimulation contraction]
Let $\kModelAndTuple$ be a Kripke model,
let $\bisimulation \subseteq \kStates \times \kStates$ be the maximal bisimulation between $\kModel$ and itself,
and let ${[\kStateS]} = \{\kStateT \in \kStates \mid (\kStateT, \kStateS\}) \in \bisimulation$.
The {\em bisimulation contraction} of $\kModel$ is the quotient model $\kModelAndTupleP$ where:
\begin{eqnarray*}
    \kStatesP &=& \{{[\kStateS]} \mid \kStateS \in \kStates\}\\
    {[\kStateS]} \kAccessibilityPA {[\kStateT]} &\text{ iff }& \text{there exists } \kStateSP \in {[\kStateS]}, \kStateTP \in {[\kStateT]} \text{ such that } \kStateS \kAccessibilityA \kStateT\\
    \kValuationP(\atomP) &=& \{{[\kStateS]} \mid \kStateS \in \kValuation(\atomP)\}
\end{eqnarray*}
\end{definition}

We note that bisimilarity of states corresponds precisely to equality of states in bisimulation contracted Kripke models.

\begin{proposition}
Let $\kModelAndTuple$ be a Kripke model and let $\kModelP$ be the bisimulation contraction of $\kModel$.
Then for every $\kStateS, \kStateT \in \kStates$: $\kPModelP{[\kStateS]} \bisimilar \kPModelP{[\kStateT]}$ if and only if $[\kStateS] = [\kStateT]$.
\end{proposition}

We also note that the bisimulation contraction of a Kripke model is bisimilar to the original Kripke model.

\begin{proposition}
Let $\kModelAndTuple$ be a Kripke model and let $\kModelP$ be the bisimulation contraction of $\kModel$.
Then for every $\kStateS \in \kStates$: $\kPModel{\kStateS} \bisimilar \kPModelP{[\kStateS]}$.
\end{proposition}

We also consider a depth-limited notion of bisimulation.

\begin{definition}[$n$-bisimulation]
Let $\kModelAndTuple$ and $\kModelAndTupleP$ be Kripke models and let $n \in \naturals$.
A list of non-empty relations $\bisimulation_0, \dots, \bisimulation_n$, where for $i = 0, \dots, n$ $\bisimulation_i \subseteq \kStates \times \kStatesP$, is an {\em $n$-bisimulation} if and only if for every $i = 0, \dots, n$, $\atomP \in \atoms$, $\agentA \in \agents$ and $(\kStateS, \kStateSP) \in \bisimulation_i$ the following conditions, {\bf atoms-$i$-$\atomP$}, {\bf forth-$i$-$\agentA$} and {\bf back-$i$-$\agentB$} holds:

\paragraph{atoms-$i$-$\atomP$}
If $i = 0$ then $\kStateS \in \kValuation(\atomP)$ if and only if $\kStateSP \in \kValuationP(\atomP)$.
If $i > 0$ then $\kPModel{\kStateS} \bisimulation_{i-1} \kPModelP{\kStateSP}$.

\paragraph{forth-$i$-$\agentA$}
For every $\kStateT \in \kSuccessorsA{\kStateS}$ there exists $\kStateTP \in \kSuccessorsPA{\kStateSP}$ such that $(\kStateT, \kStateTP) \in \bisimulation_i$.

\paragraph{back-$i$-$\agentA$}
For every $\kStateTP \in \kSuccessorsPA{\kStateSP}$ there exists $\kStateT \in \kSuccessorsA{\kStateS}$ such that $(\kStateT, \kStateTP) \in \bisimulation_i$.

If there exists an $n$-bisimulation $\bisimulation_0, \dots, \bisimulation_n$ such that $(\kStateS, \kStateSP) \in \bisimulation_n$ then we say that $\kPModel{\kStateS}$ and $\kPModelP{\kStateSP}$ are {\em $n$-bisimilar} and we denote this by $\kPModel{\kStateS} \bisimilar[n] \kPModelP{\kStateSP}$.
\end{definition}

\begin{proposition}
The relation $\bisimilar[n]$ is an equivalence relation on Kripke models.
\end{proposition}

Similar to bisimilarity, there is a partial correspondence between $n$-bisimilarity and equivalence under depth-limited modal equivalence.

\begin{definition}[Modal depth]
Let $\phi \in \langMl$.
We denote the {\em modal depth of $\phi$} by $d(\phi)$ and define it inductively as:
\begin{eqnarray*}
    d(\atomP) &=& 0\\
    d(\lnot \phi) &=& d(\phi)\\
    d(\phi \land \psi) &=& max(d(\phi), d(\psi))\\
    d(\necessaryA \phi) &=& d(\phi) + 1
\end{eqnarray*}
\end{definition}

\begin{proposition}
Let $n \in \naturals$ and let $\kPModel{\kStateS}$ and $\kPModelP{\kStateSP}$ be pointed Kripke models such that $\kPModel{\kStateS} \bisimilar[n] \kPModelP{\kStateSP}$.
Then for every $\phi \in \langMl$ such that $d(\phi) \leq n$:
$\kPModel{\kStateS} \entails \phi$ if and only if $\kPModelP{\kStateSP} \entails \phi$.
\end{proposition}

\begin{proposition}
Let $n \in \naturals$ and let $\kModel$ and $\kModelP$ be modally saturated Kripke models such that for every $\phi \in \langMl$ such that $d(\phi) \leq n$: $\kPModel{\kStateS} \entails \phi$ if and only if $\kPModelP{\kStateSP} \entails \phi$.
Then $\kPModel{\kStateS} \bisimilar[n] \kPModelP{\kStateSP}$.
\end{proposition}

These results follow from similar reasoning to the corresponding results for bisimilarity.

We note that if two pointed Kripke models are $n$-bisimilar for all $n \in \naturals$ then they must agree on all modal formulas.
However this does not in general imply that the two pointed Kripke models are bisimilar.

\pagebreak

\subsection{Axiomatisations}

We now introduce the proof theory for modal logics, first providing an axiomatisations for the logic \logicK{}.

\begin{definition}[Axiomatisation \axiomK{}]
The axiomatisation \axiomK{} is a substitution schema consisting of the following axioms and rules:
$$
\begin{array}{rl}
    {\bf P}     & \text{All propositional tautologies}\\
    {\bf K}     & \proves \necessaryA (\phi \implies \psi) \implies (\necessaryA \phi \implies \necessaryA \psi)\\
    {\bf MP}    & \text{From } \proves \phi \implies \psi \text{ and } \proves \phi \text{ infer } \proves \psi\\
    {\bf NecK}  & \text{From } \proves \phi \text{ infer } \proves \necessaryA \phi
\end{array}
$$
where $\phi, \psi \in \langMl$ and $\agentA \in \agents$.
\end{definition}

If $\proves \phi$ is in the least set containing the axioms and closed under the rules of an axiomatisation then we say that $\phi$ is a {\em theorem} of the axiomatisation.
Let $\Phi \subseteq \langMl$ be a (possibly infinite) set of formulas.
If there exists $\phi_1, \dots, \phi_n \in \Phi$ such that $\proves (\phi_1 \land \cdots \land \phi_n) \implies \phi$ then we say that $\phi$ is {\em deducible} from $\Phi$ and we write $\Phi \proves \phi$.
If $\Phi \proves \bot$ then we say that $\Phi$ is {\em inconsistent}, and if $\Phi \nproves \bot$ then we say that $\Phi$ is {\em consistent}.

We also provide axiomatisations for the logics \logicKF{}, \logicKFF{}, \logicKD{}, and \logicS{}, by extending the axiomatisation \axiomK{} with additional axioms.

\begin{definition}[Axiomatisations \axiomKF{}, \axiomKFF{}, \axiomKD{}, and \axiomS{}]
We define the following axioms:
$$
\begin{array}{rl}
    {\bf D}     & \proves \necessaryA \phi \implies \possibleA \phi\\
    {\bf T}     & \proves \necessaryA \phi \implies \phi\\
    {\bf 4}     & \proves \necessaryA \phi \implies \necessaryA \necessaryA \phi\\
    {\bf 5}     & \proves \possibleA \phi \implies \necessaryA \possibleA \phi
\end{array}
$$

\pagebreak

Then the axiomatisations \axiomKF{}, \axiomKFF{}, \axiomKD{}, and \axiomS{} consist of the axioms and rules of \axiomK{} along with the following respective additional axioms:
\begin{itemize}
    \item \axiomKF{}: {\bf 4}
    \item \axiomKFF{}: {\bf 4} and {\bf 5}
    \item \axiomKD{}: {\bf D}, {\bf 4} and {\bf 5}
    \item \axiomS{}: {\bf T} and {\bf 5}
\end{itemize}
\end{definition}

We are interested in properties of the proof theories of the logics we consider.

\begin{definition}[Soundness and completeness]
If every theorem of an axiomatisation is valid in the corresponding semantics then we say that the axiomatisation is {\em sound} with respect to the logic.
If every formula valid in a semantics is a theorem of the corresponding axiomatisation then we say that the axiomatisation is {\em (weakly) complete} with respect to the semantics.
If every set of formulas that is consistent according to an axiomatisation is satisfiable in the corresponding semantics then we say that the axiomatisation is {\em strongly complete} with respect to the semantics.
\end{definition}

\begin{proposition}
The axiomatisations \axiomK{}, \axiomKF{}, \axiomKFF{}, \axiomKD{}, and \axiomS{} are sound and strongly complete with respect to the semantics of the respective logics \logicK{}, \logicKF{}, \logicKFF{}, \logicKD{}, and \logicS{}.
\end{proposition}

\begin{definition}[Substitution of equivalents]
Let $\lang$ be a logical language defined on the propositional atoms $\atoms$.
An axiomatisation is closed under {\em substitution of equivalents} if and only if for every $\phi, \psi, \psi' \in \lang$, and $\atomP \in \atoms$: $\proves \psi \iff \psi'$ implies $\proves \phi[\psi/\atomP] \iff \phi[\psi'/\atomP]$.
\end{definition}

\begin{proposition}
The axiomatisations \axiomK{}, \axiomKF{}, \axiomKFF{}, \axiomKD{}, and \axiomS{} are closed under substitution of equivalents.
\end{proposition}

We are also interested in properties of the logics themselves.

\begin{definition}[Compactness]
If every (possibly infinite) set of formulas that is finitely satisfiable according to the semantics of a logic is satisfiable then we say that the logic is {\em compact}.
\end{definition}

\begin{proposition}
The logics \logicK{}, \logicKF{}, \logicKFF{}, \logicKD{}, and \logicS{} are compact.
\end{proposition}

\begin{definition}[Model-checking problem]
The {\em model-checking problem} for a logic is to determine for a given formula and a given finite pointed Kripke model whether the formula is valid on the pointed Kripke model.
\end{definition}

\begin{proposition}
The model-checking problems for the logics \logicK{}, \logicKF{}, \logicKFF{}, \logicKD{} and \logicS{} are decidable.
\end{proposition}

\begin{definition}[Satisfiability and provability problems]
The {\em satisfiability problem} for a logic is to determine for a given formula whether the formula is satisfiable according to the semantics of the logic.
The {\em provability problem} for a proof system such as an axiomatisation is to determine for a given formula whether the formula is provable according to the proof system.
\end{definition}

We note that for sound and complete proof systems the satisfiability and provability problems are dual problems (a formula is satisfiable if and only if its negation is not provable).

\begin{proposition}
The satisfiability problems for the logics \logicK{}, \logicKF{}, \logicKFF{}, \logicKD{} and \logicS{} are decidable.
\end{proposition}

In the following sections and chapters we will compare the expressive power of the logics that we consider to other logics, often to an underlying modal logic.

\pagebreak

\begin{definition}[Expressivity]
If a logic can express all of the semantic properties that another logic can then we say that the logic is {\em at least as expressive} as the other logic.
If two logics are at least as expressive as each other then we say that the two logics are {\em expressively equivalent}.
If a logic is at least as expressive as but not expressively equivalent to another logic then we say that the logic is {\em strictly less expressive} than the other logic.
If two logics are neither at least as expressive as the other logic then we say that the two logics are {\em incomparable} in expressivity.
\end{definition}
