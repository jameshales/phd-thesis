\chapter{Refinement modal logic: \classK{}}\label{rml-k}

In this chapter we consider results specific to the logic \logicRmlK{} in the setting of \classK{}.
The main result of this chapter is a sound and complete axiomatisation of \logicRmlK{}.
The axiomatisation forms a set of reduction axioms, admitting a provably correct translation from \langRml{} to the underlying modal language \langMl{}.
We use this provably correct translation to show the completeness of the axiomatisation, to show that \logicRmlK{} is expressively equivalent to \logicK{}, and to show that \logicRmlK{} is compact and decidable.
Whereas in the previous chapter we provided definitions and results common to all or most of the settings that we consider, the results in this chapter are specific to \logicRmlK{} and do not trivially generalise to other settings.
However in Chapter~\ref{rml-kd45} and Chapter~\ref{rml-s5} we provide sound and complete axiomatisations, and the same accompanying results for \logicRmlKFF{}, \logicRmlKD{} and \logicRmlS{}, results which build upon the techniques developed in this chapter.

In the following sections we provide a sound and complete axiomatisation for \logicRmlK{}.
In Section~\ref{rml-k-axiomatisation} we provide the axiomatisation for \logicRmlK{}.
In Section~\ref{rml-k-soundness} we show that the axiomatisation is sound.
In Section~\ref{rml-k-completeness} we show that the axiomatisation is complete via a provably correct translation from \langRml{} to \langMl{}.
This provably correct translation uses a disjunctive normal form for modal logic defined using cover operators, followed by applications of the reduction axioms in \axiomRmlK{}.
As a result of this provably correct translation we have as corollaries that \logicRmlK{} is expressively equivalent to \logicK{}, and that \logicRml{} is compact and decidable.

\section{Axiomatisation}\label{rml-k-axiomatisation}

In this section we present the axiomatisation \axiomRmlK{} for the logic \logicRmlK{}.
The axiomatisation relies heavily on the cover operator, which we recall is defined by the syntactic abbreviation $\coversA \Gamma ::= \necessaryA \bigvee_{\gamma \in \Gamma} \gamma \land \bigwedge_{\gamma \in \Gamma} \possibleA \gamma$.
The cover operator also forms the basis of our axiomatisations of \logicRmlKFF{}, \logicRmlKD{}, and \logicRmlS{}, which will be presented in the following chapters.
We now present our axiomatisation for \logicRmlK{} and discuss its features, particularly the use of the cover operator.

\begin{definition}[Axiomatisation \axiomRmlK{}]
    The axiomatisation \axiomRmlK{} is a substitution schema consisting of the axioms and rules of \axiomK{} along with the following additional axioms and rules:
$$
\begin{array}{rl}
    {\bf R} & \proves \allrefsBs (\phi \implies \psi) \implies (\allrefsBs \phi \implies \allrefsBs \psi)\\
    {\bf RP} & \proves \allrefsBs \pi \iff \pi\\
    {\bf RK} & \proves \somerefsBs \coversA \Gamma_\agentA \iff \bigwedge_{\gamma \in \Gamma_\agentA} \possibleA \somerefsBs \gamma \text{ where } \agentA \in \agentsB\\
    {\bf RComm} & \proves \somerefsBs \coversA \Gamma_\agentA \iff \coversA \{\somerefsBs \gamma \mid \gamma \in \Gamma_\agentA\} \text{ where } \agentA \notin \agentsB\\
    {\bf RDist} & \proves \somerefsBs \bigwedge_{\agentC \in \agentsC} \coversC \Gamma_\agentC \iff \bigwedge_{\agentC \in \agentsC} \somerefsBs \coversC \Gamma_\agentC\\
    {\bf NecR} & \text{From } \proves \phi \text{ infer } \proves \allrefsBs \phi
\end{array}
$$
where $\phi, \psi \in \langRml$, $\pi \in \langPl$, $\agentA \in \agents$, $\agentsB, \agentsC \subseteq \agents$, and for every $\agentA \in \agents$: $\Gamma_\agentA \subseteq \langRml$ is a finite set of formulas.
\end{definition}

We note that the axioms {\bf R} and {\bf RP}, and the rule {\bf NecR} are validities established for all variants of \logicRml{} in the previous chapter, in Proposition~\ref{rml-validities}.
We also note that although reflexivity and transitivity are important properties of the relational operator $\simulatesBs$ for refinements, the validities from Proposition~\ref{rml-validities} corresponding to these properties do not feature in this axiomatisation, as we will see that they are not necessary in order to show the completeness of the axiomatisation.

The axioms of \logicRmlK{} have the appearance of reduction axioms, as they allow refinement quantifiers to be ``pushed'' past propositional connectives and modalities, reducing the complexity of the formulas that refinement quantifiers are applied to, or in the case of the axiom {\bf RP}, or some applications of {\bf RK} and {\bf RComm}, allow refinement quantifiers to be removed completely.
In Section~\ref{rml-k-completeness} we will provide a provably correct translation using these reduction axioms, pushing refinement quantifiers past propositional connectives and modalities, until the refinement quantifiers can be removed completely by an application of {\bf RP}, {\bf RK} or {\bf RComm}.
This is similar to the approach used to show the completeness of the axiomatisations for \logicAml{} and \logicPal{}.
However unlike the axiomatisations for \logicAml{} and \logicPal{}, it's not immediately obvious that the reduction axioms of \axiomRmlK{} are applicable to all \langRml{} formulas.
In particular, the reduction axioms can only push refinement quantifiers past negations in propositional formulas, using the axiom {\bf RP}, and can only push refinement quantifiers past conjunctions in specific situations involving cover operators, using the axioms {\bf RK}, {\bf RComm}, and {\bf RDist}.
In Section~\ref{rml-k-completeness} we address these limitation with the introduction of a normal form that restricts negations to propositional formulas and conjunctions to the specific situations handled by {\bf RK}, {\bf RComm}, and {\bf RDist}.

The cover operator features prominently in the axioms, {\bf RK}, {\bf RComm}, and {\bf RDist}.
These axioms describe the interaction between existential refinement quantifiers and conjunctions of modalities, where the cover operator is used as a convenient notation for a conjunction of modalities.
We must have reduction axioms specifically for conjunctions of modalities because of the difficulty in pushing existential refinement quantifiers past conjunctions, or dually, pushing universal refinement quantifiers past disjunctions.
For example, a reduction axiom such as $\proves \somerefsBs (\phi \land \psi) \iff (\somerefsBs \phi \land \somerefsBs \psi)$ would not be sound.
This can be seen if we consider a Kripke model $\kPModel{\kStateS}$ such that $\kPModel{\kStateS} \entails \possibleA \top$.
Then we have $\kPModel{\kStateS} \entails \somerefsA \possibleA \top$, with the witnessing $\agentA$-refinement being $\kPModel{\kStateS}$ itself, and we have $\kPModel{\kStateS} \entails \somerefsA \necessaryA \bot$, with the witnessing $\agentA$-refinement being $\kPModel{\kStateS}$ with its $\agentA$-successors removed, but as $\possibleA \top \land \necessaryA \bot$ is a contradiction then we have that $\kPModel{\kStateS} \nentails \somerefsA (\possibleA \top \land \necessaryA \bot)$.
Clearly there is an interaction between the modalities inside each of the refinement quantifiers, so these two formulas cannot be so easily separated.
Instead we give the axioms {\bf RK}, {\bf RComm}, and {\bf RDist} which consider conjunctions of modalities rather than single modalities, and use the cover operators as an abbreviation for a conjunction of an arbitrary number of modalities.
As the cover operator is defined by a syntactic abbreviation, these axioms could be restated in terms of the more conventional $\necessaryA$ and $\possibleA$ modalities.
However the axioms {\bf RK} and {\bf RComm} would not be sound for arbitrary conjunctions of $\necessaryA$ and $\possibleA$ modalities.
For example, we showed above an example where $\nentails \somerefsA (\possibleA \phi \land \necessaryA \psi) \iff (\somerefsBs \possibleA \phi \land \somerefsBs \necessaryA \psi)$.
In fact if we rewrite $\possibleA \phi \land \necessaryA \psi$ into cover operator form as $\coversA \{\phi \land \psi, \psi\}$ we see from {\bf RK} that $\proves \somerefsA (\possibleA \phi \land \necessaryA \psi) \iff \possibleA \somerefsA (\phi \land \psi) \land \possibleA \somerefsA \psi$.
Hence the cover operator also serves as a convenient notation to restrict conjunctions of modalities to cases where such axioms are sound.
In Section~\ref{rml-k-soundness} we will see that the semantics of the cover operator are convenient for showing the soundness of these axioms.
In Section~\ref{rml-k-completeness} we see that the cover operator allows a convenient disjunctive normal form for modal formulas, which we use in our provably correct translation from \langRml{} to \langMl{}.

Finally we give an example derivation using the axiomatisation \axiomRmlK{}.

\begin{example}\label{rml-k-example-derivation}
We show that $\proves \somerefsA (\necessaryA \atomP \land \lnot \necessaryB \atomP) \iff \possibleB \lnot \atomP$ using the axiomatisation \axiomRmlK{}.
$$
\begin{array}{ll}
    \proves \possibleB \lnot \atomP \iff ((\possibleA \atomP \lor \top) \land \possibleB \lnot \atomP) & {\bf P}\\
    \proves \possibleB \lnot \atomP \iff ((\possibleA \atomP \lor \top) \land \coversB \{\lnot \atomP, \top\}) & \text{Defn. of $\coversB$}\\
    \proves \possibleB \lnot \atomP \iff ((\possibleA \lnot \lnot \atomP \lor \top) \land \coversB \{\lnot \lnot \lnot \atomP, \lnot \lnot \top\}) & {\bf P}\\
    \proves \possibleB \lnot \atomP \iff ((\possibleA \lnot \allrefsA \lnot \atomP \lor \top) \land \coversB \{\lnot \allrefsA \lnot \lnot \atomP, \lnot \allrefsA \lnot \top\}) & {\bf RP}\\
    \proves \possibleB \lnot \atomP \iff ((\possibleA \somerefsA \atomP \lor \top) \land \coversB \{\somerefsA \lnot \atomP, \somerefsA \top\}) & \text{Defn. of $\somerefsA$}\\
    \proves \possibleB \lnot \atomP \iff ((\somerefsA \coversA \{\atomP\} \lor \somerefsA \coversA \emptyset) \land \coversB \{\somerefsA \lnot \atomP, \somerefsA \top\}) & {\bf RK}\\
    \proves \possibleB \lnot \atomP \iff ((\somerefsA \coversA \{\atomP\} \lor \somerefsA \coversA \emptyset) \land \somerefsA \coversB \{\lnot \atomP, \top\}) & {\bf RComm}\\
    \proves \possibleB \lnot \atomP \iff ((\somerefsA \coversA \{\atomP\} \land \somerefsA \coversB \{\lnot \atomP, \top\}) \lor (\somerefsA \coversA \emptyset \land \somerefsA \coversB \{\lnot \atomP, \top\})) & {\bf P}\\
    \proves \possibleB \lnot \atomP \iff (\somerefsA (\coversA \{\atomP\} \land \coversB \{\lnot \atomP, \top\}) \lor \somerefsA (\coversA \emptyset \land \coversB \{\lnot \atomP, \top\})) & {\bf RDist}\\
    \proves \possibleB \lnot \atomP \iff (\somerefsA (\necessaryA \atomP \land \possibleA \atomP \land \possibleB \lnot \atomP) \lor \somerefsA (\necessaryA \bot \land \possibleB \lnot \atomP)) & \text{Defn. of $\coversA$ and $\coversB$}\\
    \proves \possibleB \lnot \atomP \iff (\somerefsA (\necessaryA \atomP \land \possibleA \atomP \land \possibleB \lnot \atomP) \lor \somerefsA (\necessaryA \atomP \land \lnot \possibleA \atomP \land \possibleB \lnot \atomP)) & \text{Modal reasoning}\\
    \proves \possibleB \lnot \atomP \iff \somerefsA (\necessaryA \atomP \land \possibleB \lnot \atomP) & {\bf P}\\
    \proves \possibleB \lnot \atomP \iff \somerefsA (\necessaryA \atomP \land \lnot \necessaryB \atomP) & \text{Defn. of $\possibleB$}
\end{array}
$$
\end{example}

\section{Soundness}\label{rml-k-soundness}

In this section we show that the axiomatisation \axiomRmlK{} is sound with respect to the semantics of the logic \logicRmlK{}.
The axioms {\bf R} and {\bf RP}, and the rule {\bf NecR} are already known to be sound as they were established for all variants of \logicRml{}, in Proposition~\ref{rml-validities}.
What remains to be shown is that the axioms {\bf RK}, {\bf RComm}, and {\bf RDist} are sound.
Each of these axioms share the general form of equivalences, where the left side of the equivalence describes the existence of a single refinement that satisfies a given formula, whilst the right side describes the existence of multiple refinements that satisfy subformulas of the given formula.
Accordingly the proof of soundness for each of these axioms share the same general technique.
For the left-to-right direction of the equivalence, we show that if we have the refinement described on the left of the equivalence, then this same refinement satisfies all that we need for the right of the equivalence.
Conversely, for the right-to-left direction of the equivalence, we show that if we have all of the refinements described on the right of the equivalence, then these refinements can be combined into a single refinement that satisfies the left of the equivalence.
The left-to-right direction is simple to show, whereas the right-to-left direction is more involved.

We begin by showing that the axiom {\bf RK} is sound.
Recall that the axiom {\bf RK} takes the form of $\proves \somerefsBs \coversA \Gamma_\agentA \iff \bigwedge_{\gamma \in \Gamma_\agentA} \possibleA \somerefsBs \gamma$ where $\agentsB \subseteq \agents$, $\agentA \in \agentsB$, and $\Gamma_\agentA \subseteq \langRml$ is a finite set of formulas.

\begin{lemma}\label{rml-k-rk}
The axiom {\bf RK} from the axiomatisation \axiomRmlK{} is sound with respect to the semantics of the logic \logicRmlK{}.
\end{lemma}

\begin{proof}
($\Rightarrow$) 
We show that $\entails \somerefsBs \coversA \Gamma_\agentA \implies \bigwedge_{\gamma \in \Gamma_\agentA} \possibleA \somerefsBs \gamma$ where $\agentA \in \agentsB$.
Let $\kPModelAndTuple{\kStateS} \in \classK$ be a pointed Kripke model such that $\kPModel{\kStateS} \entails \somerefsBs \coversA \Gamma_\agentA$.
There exists $\kPModelAndTupleP{\kStateSP} \in \classK$ such that $\kPModel{\kStateS} \simulatesBs \kPModelP{\kStateSP}$ and $\kPModelP{\kStateSP} \entails \coversA \Gamma_\agentA$.
For every $\gamma \in \Gamma_\agentA$ there exists $\kStateTP_\gamma \in \kSuccessorsPA{\kStateSP}$ such that $\kPModelP{\kStateTP_\gamma} \entails \gamma$.
From {\bf back-$\agentA$} there exists $\kStateT_\gamma \in \kSuccessorsA{\kStateS}$ such that $\kPModel{\kStateT_\gamma} \simulatesBs \kPModelP{\kStateTP_\gamma}$.
Then $\kPModel{\kStateT_\gamma} \entails \somerefsBs \gamma$ and so $\kPModel{\kStateS} \entails \possibleA \somerefsBs \gamma$.
Therefore $\kPModel{\kStateS} \entails \bigwedge_{\gamma \in \Gamma_\agentA} \possibleA \somerefsBs \gamma$.

($\Leftarrow$) 
We show that $\entails \bigwedge_{\gamma \in \Gamma_\agentA} \possibleA \somerefsBs \gamma \implies \somerefsBs \coversA \Gamma_\agentA$ where $\agentA \in \agentsB$.
Let $\kPModelAndTuple{\kStateS} \in \classK$ be a pointed Kripke model such that $\kPModel{\kStateS} \entails \bigwedge_{\gamma \in \Gamma_\agentA} \possibleA \somerefsBs \gamma$.
For every $\gamma \in \Gamma_\agentA$ there exists $\kStateT_\gamma \in \kSuccessorsA{\kStateS}$ and $\kPModelAndTuple[\gamma]{\kStateS[\gamma]} \in \classK$ such that $\kPModel{\kStateT_\gamma} \simulatesBs \kPModel[\gamma]{\kStateS[\gamma]}$ and $\kPModel[\gamma]{\kStateS[\gamma]} \entails \gamma$.
We use these refinements to construct a single larger refinement to satisfy the left-hand-side of the {\bf RK} equivalence.

Let $\kPModelAndTupleP{\kStateSP}$ be a pointed Kripke model where:
\begin{eqnarray*}
    \kStatesP &=& \{\kStateSP\} \cup \kStates \cup \bigcup_{\gamma \in \Gamma_\agentA} \kStates[\gamma]\\
    \kAccessibilityPA &=& \{(\kStateSP, \kStateS[\gamma]) \mid \gamma \in \Gamma_\agentA\} \cup \kAccessibilityA \cup \bigcup_{\gamma \in \Gamma_\agentA} \kAccessibilityA[\gamma]\\
    \kAccessibilityPB &=& \{(\kStateSP, \kStateT) \mid \kStateT \in \kSuccessorsB{\kStateS}\} \cup \kAccessibilityB \cup \bigcup_{\gamma \in \Gamma_\agentA} \kAccessibilityB[\gamma]\\
    \kValuationP(\atomP) &=& \{\kStateSP \mid \kStateS \in \kValuation(\atomP)\} \cup \kValuation(\atomP) \cup \bigcup_{\gamma \in \Gamma_\agentA} \kValuation[\gamma](\atomP)
\end{eqnarray*}
where $\kStateSP$ is a fresh state not appearing in $\kStates$ or $\kStates[\gamma]$ for any $\gamma \in \Gamma_\agentA$, and $\agentB \in \agents \setminus \{\agentA\}$.

\begin{figure}
    \caption{A schematic of the construction used to show soundness of {\bf RK}.}\label{rml-k-rk-construction}
    \centering
    \begin{tikzpicture}[>=stealth',shorten >=1pt,auto,node distance=7em,thick]

        \node (sp) {\underline{$\kPModelP{\kStateSP} \entails \coversA \{\gamma_1, \dots, \gamma_n\}$}};
        \node (mgn) [above of=sp] {$\kPModel[\gamma_n]{\kStateS[\gamma_n]} \entails \gamma_n$};
        \node (mgd) [left of=mgn,node distance=3.5em] {$\cdots$};
        \node (mg1) [left of=mgd,node distance=3.5em] {$\kPModel[\gamma_1]{\kStateS[\gamma_1]} \entails \gamma_1$};
        \node (mtn) [above of=mgn] {$\kPModel{\kStateT_{\gamma_n}} \entails \somerefsBs \gamma_n$};
        \node (mtd) [left of=mtn,node distance=3.5em] {$\cdots$};
        \node (mt1) [left of=mtd,node distance=3.5em] {$\kPModel{\kStateT_{\gamma_1}} \entails \somerefsBs \gamma_1$};
        \node (mtd2) [right of=mtn,node distance=3.5em] {$\cdots$};
        \node (mtr) [right of=mtd2,node distance=3.5em] {$\kPModel{\kStateT} \entails \top$};
        \node (ms) [above of=mtn] {$\kPModel{\kStateS} \entails \possibleA \somerefsBs \gamma_1 \land \cdots \land \possibleA \somerefsBs \gamma_n$};

      \path[every node/.style={font=\sffamily\small},->]
        (mt1) edge [swap,dashed] node {$\simulatesBs$} (mg1)
        (mtn) edge [dashed] node {$\simulatesBs$} (mgn)
        (ms) edge [dashed,bend left=90,min distance=10em] node {$\simulatesBs$} (sp)
        (ms) edge [draw=none,dashed,bend right=90,min distance=10em] node {} (sp)
        (sp) edge node {$\agentA$} (mg1)
             edge [swap] node {$\agentA$} (mgn)
        (ms) edge [swap] node {$\agentA$} (mt1)
             edge node {$\agentA$} (mtr)
             edge node {$\agentA$} (mtn);
    \end{tikzpicture}
\end{figure}

A schematic of the Kripke model $\kPModelP{\kStateSP}$ and an overview of our construction is shown in Figure~\ref{rml-k-rk-construction}.
Here we can see that each of the $\agentsB$-refinements at successors, $\kPModel[\gamma_1]{\kStateT[\gamma_1]}, \dots, \kPModel[\gamma_n]{\kStateT[\gamma_n]}$, are combined into the larger Kripke model $\kPModelP{\kStateSP}$.
From this schematic representation we can clearly see that $\kPModel{\kStateS} \simulatesBs \kPModelP{\kStateSP}$ and $\kPModelP{\kStateSP} \entails \coversA \{\gamma_1, \dots, \gamma_n\}$.
We note that there are $\agentA$-successors of $\kPModel{\kStateS}$ that do not satisfy any $\somerefsBs \gamma_i$ and do not correspond to any $\agentsB$-refinement $\kPModel[\gamma_i]{\kStateT[\gamma_i]}$.
This is permissible as $\agentA \in \agentsB$ and so {\bf forth-$\agentA$} is not required in order for $\kPModelP{\kStateSP}$ to be a $\agentsB$-refinement of $\kPModel{\kStateS}$.

To show that $\kPModel{\kStateS} \entails \somerefsBs \coversA \Gamma_\agentA$ we will show that $\kPModel{\kStateS} \simulatesBs \kPModelP{\kStateSP}$ and $\kPModelP{\kStateSP} \entails \coversA \Gamma_\agentA$.

We first show that $\kPModel{\kStateS} \simulatesBs \kPModelP{\kStateSP}$.

For every $\gamma \in \Gamma_\agentA$ let $\refinement^\gamma \subseteq \kStates \times \kStates[\gamma]$ be a $\agentsB$-refinement from $\kPModel{\kStateT_\gamma}$ to $\kPModel[\gamma]{\kStateS[\gamma]}$.
We define $\refinement \subseteq \kStates \times \kStatesP$ where:
$$
\refinement = \{(\kStateS, \kStateSP)\} \cup \{(\kStateT, \kStateT) \mid \kStateT \in \kStates\} \cup \bigcup_{\gamma \in \Gamma_\agentA} \refinement^\gamma
$$
We show that $\refinement$ is a $\agentsB$-refinement from $\kPModel{\kStateS}$ to $\kPModelP{\kStateSP}$.

Let $\atomP \in \atoms$, $\agentB \in \agents$, $\agentC \in \agents \setminus \agentsB$.
We show by cases that the relationships in $\refinement$ satisfy the conditions {\bf atoms-$\atomP$}, {\bf forth-$\agentC$}, and {\bf back-$\agentB$}.

\begin{description}
    \item[Case $(\kStateS, \kStateSP) \in \refinement$:]
        \hfill
        \begin{description}
            \item[atoms-$\atomP$] 
                By construction $\kStateS \in \kValuation(\atomP)$ if and only if $\kStateSP \in \kValuationP(\atomP)$.
            \item[forth-$\agentC$]
                Let $\kStateT \in \kSuccessorsC{\kStateS}$.
                As $\agentC \in \agents \setminus \agentsB$ and $\agentA \in \agentsB$ then $\agentC \neq \agentA$.
                By construction $\kSuccessorsPC{\kStateSP} = \kSuccessorsC{\kStateSP}$.
                Then $\kStateT \in \kSuccessorsPC{\kStateSP}$ and by construction $(\kStateT, \kStateT) \in \refinement$.
            \item[back-$\agentB$]
                Suppose that $\agentB = \agentA$.
                By construction $\kSuccessorsPA{\kStateSP} = \{\kStateS[\gamma] \mid \gamma \in \Gamma_\agentA\}$.
                Let $\kStateS[\gamma] \in \kSuccessorsPA{\kStateSP}$ where $\gamma \in \Gamma_\agentA$.
                Then by hypothesis $\kStateT_\gamma \in \kSuccessorsA{\kStateS}$ and $(\kStateT_\gamma, \kStateS[\gamma]) \in \refinement^\gamma \subseteq \refinement$.

                Suppose that $\agentB \neq \agentA$.
                Let $\kStateT \in \kSuccessorsPB{\kStateSP}$.
                By construction $\kSuccessorsPB{\kStateSP} = \kSuccessorsB{\kStateS}$.
                Then $\kStateT \in \kSuccessorsB{\kStateS}$ and by construction $(\kStateT, \kStateT) \in \refinement$.
        \end{description}
    \item[Case $(\kStateT, \kStateT) \in \refinement$ where $\kStateT \in \kStates$:]
        \hfill
        \begin{description}
            \item [atoms-$\atomP$] 
                By construction $\kStateT \in \kValuation(\atomP)$ if and only if $\kStateT \in \kValuationP(\atomP)$.
            \item [forth-$\agentC$]
                Let $\kStateU \in \kSuccessorsC{\kStateT}$.
                By construction $\kSuccessorsPC{\kStateT} = \kSuccessorsC{\kStateT}$.
                Then $\kStateU \in \kSuccessorsPC{\kStateT}$ and by construction $(\kStateU, \kStateU) \in \refinement$.
            \item [back-$\agentB$]
                Let $\kStateU \in \kSuccessorsPB{\kStateT}$.
                By construction $\kSuccessorsPB{\kStateT} = \kSuccessorsB{\kStateT}$.
                Then $\kStateU \in \kSuccessorsB{\kStateT}$ and by construction $(\kStateU, \kStateU) \in \refinement$.
        \end{description}
    \item[{Case $(\kStateT, \kStateT[\gamma]) \in \refinement^\gamma \subseteq \refinement$ where $\gamma \in \Gamma_\agentA$:}]
        \hfill
        \begin{description}
            \item [atoms-$\atomP$]
                By {\bf atoms-$\atomP$} for $\refinement^\gamma$ we have $\kStateT \in \kValuation(\atomP)$ if and only if $\kStateT[\gamma] \in \kValuation[\gamma](\atomP)$.
                By construction $\kStateT[\gamma] \in \kValuation[\gamma](\atomP)$ if and only if $\kStateT[\gamma] \in \kValuationP(\atomP)$.
            \item [forth-$\agentC$]
                Let $\kStateU \in \kSuccessorsC{\kStateT}$.
                By {\bf forth-$\agentC$} for $\refinement^\gamma$ there exists $\kStateU[\gamma] \in \kSuccessorsC[\gamma]{\kStateT[\gamma]}$ such that $(\kStateU, \kStateU[\gamma]) \in \refinement^\gamma$.
                By construction $\kSuccessorsPC{\kStateT[\gamma]} = \kSuccessorsC[\gamma]{\kStateT[\gamma]}$.
                Then $\kStateU[\gamma] \in \kSuccessorsPC{\kStateT[\gamma]}$ and by construction $(\kStateU, \kStateU[\gamma]) \in \refinement$.
            \item [back-$\agentB$]
                Let $\kStateU[\gamma] \in \kSuccessorsPB{\kStateT[\gamma]}$.
                By construction $\kSuccessorsPB{\kStateT[\gamma]} = \kSuccessorsB[\gamma]{\kStateT[\gamma]}$.
                Then $\kStateU[\gamma] \in \kSuccessorsB[\gamma]{\kStateT[\gamma]}$.
                By {\bf back-$\agentB$} for $\refinement^\gamma$ there exists $\kStateU \in \kSuccessorsB{\kStateT}$ such that $(\kStateU, \kStateU[\gamma]) \in \refinement^\gamma \subseteq \refinement$.
        \end{description}
\end{description}

Therefore $\refinement$ is a $\agentsB$-refinement and as $(\kStateS, \kStateSP) \in \refinement$ we have that $\kPModel{\kStateS} \simulatesBs \kPModelP{\kStateSP}$.

We finally show that $\kPModelP{\kStateSP} \entails \coversA \Gamma_\agentA$.

Let $\gamma \in \Gamma_\agentA$.
We note that $\kPModel[\gamma]{\kStateS[\gamma]} \bisimilar \kPModelP{\kStateS[\gamma]}$ as by construction the valuations and successors of states of $\kModel[\gamma]$ are left unchanged in $\kModelP$.
As $\kPModel[\gamma]{\kStateS[\gamma]} \entails \gamma$ then by Proposition~\ref{rml-bisimulation-invariance} we have that $\kPModelP{\kStateS[\gamma]} \entails \gamma$.

For every $\gamma \in \Gamma_\agentA$ we have that $\kStateS[\gamma] \in \kSuccessorsPA{\kStateSP}$ and $\kPModelP{\kStateS[\gamma]} \entails \gamma$.
For every $\kStateS[\gamma] \in \kSuccessorsPA{\kStateSP}$ we have that $\kPModelP{\kStateS[\gamma]} \entails \gamma$.
Therefore $\kPModelP{\kStateSP} \entails \coversA \Gamma_\agentA$.

Therefore $\kPModel{\kStateS} \entails \somerefsBs \coversA \Gamma_\agentA$.
\end{proof}

In the previous section we justified the use of the cover operator in the axiomatisation partially by the opinion that it is convenient for the soundness proofs.
In particular we note that in the right-to-left direction of the axioms {\bf RK} and {\bf RComm} there is a one-to-one correspondence between formulas in the cover operator on the left of the equivalence and the refinements described on the right of the equivalence.
As we have just seen in the proof of soundness of {\bf RK}, the refinements described on the right of the equivalence are then directly used in the construction of a single refinement that satisfies the left of the equivalence, and the one-to-one correspondence between refinements and formulas is used to show that the cover operator on the left of the equivalence is satisfied by the constructed refinement.

We next show that the axiom {\bf RComm} is sound.
Recall that the axiom {\bf RComm} takes the form of $\proves \somerefsBs \coversA \Gamma_\agentA \iff \coversA \{\somerefsBs \gamma \mid \gamma \in \Gamma_\agentA\}$ where $\agentsB \subseteq \agents$, $\agentA \notin \agentsB$, and $\Gamma_\agentA \subseteq \langRml$ is a finite set of formulas.
The axiom {\bf RComm} is similar to the axiom {\bf RK}, and the proof strategy is also similar.
Whereas for {\bf RK} we have that $\agentA \in \agentsB$, and therefore a $\agentsB$-refinement need not satisfy {\bf forth-$\agentA$}, for {\bf RComm} we have that $\agentA \notin \agentsB$ and so {\bf forth-$\agentA$} is required.
Accordingly for {\bf RK} our constructed model need not satisfy {\bf forth-$\agentA$} in order to be a $\agentsB$-refinement, so we do not require that every $\agentA$-successor of the original model have a refinement satisfying some $\gamma$.
However for {\bf RComm} our constructed model does require {\bf forth-$\agentA$} in order to be a $\agentsB$-refinement, so we require every $\agentA$-successor of the original model to have a refinement satisfying some $\gamma$.
The difference between {\bf RComm} and {\bf RK} accounts for this additional requirement, ensuring that we can construct an appropriate $\agentsB$-refinement.

\begin{lemma}\label{rml-k-rcomm}
The axiom {\bf RComm} from the axiomatisation \axiomRmlK{} is sound with respect to the semantics of the logic \logicRmlK{}.
\end{lemma}

\begin{proof}
($\Rightarrow$) 
We show that $\entails \somerefsBs \coversA \Gamma_\agentA \implies \coversA \{\somerefsBs \gamma \mid \gamma \in \Gamma_\agentA\}$.
Let $\kPModelAndTuple{\kStateS} \in \classK$ be a pointed Kripke model such that $\kPModel{\kStateS} \entails \somerefsBs \coversA \Gamma_\agentA$.
There exists $\kPModelAndTupleP{\kStateSP} \in \classK$ such that $\kPModel{\kStateS} \simulatesBs \kPModelP{\kStateSP}$ and $\kPModelP{\kStateSP} \entails \coversA \Gamma_\agentA$.
For every $\gamma \in \Gamma_\agentA$ there exists $\kStateTP_\gamma \in \kSuccessorsPA{\kStateSP}$ such that $\kPModelP{\kStateTP_\gamma} \entails \gamma$.
From {\bf back-$\agentA$} there exists $\kStateT_\gamma \in \kSuccessorsA{\kStateS}$ such that $\kPModel{\kStateT_\gamma} \simulatesBs \kPModelP{\kStateTP_\gamma}$.
Then $\kPModel{\kStateT_\gamma} \entails \somerefsBs \gamma$ and so $\kPModel{\kStateS} \entails \possibleA \somerefsBs \gamma$.
For every $\kStateT \in \kSuccessorsA{\kStateS}$ from {\bf forth-$\agentA$} there exists $\kStateTP \in \kSuccessorsPA{\kStateSP}$ such that $\kPModel{\kStateT} \simulatesBs \kPModelP{\kStateTP}$.
As $\kPModelP{\kStateSP} \entails \coversA \Gamma_\agentA$ then $\kPModelP{\kStateTP} \entails \gamma$ for some $\gamma \in \Gamma_\agentA$.
Then $\kPModel{\kStateT} \entails \bigvee_{\gamma \in \Gamma_\agentA} \somerefsBs \gamma$ and so $\kPModel{\kStateS} \entails \necessaryA \bigvee_{\gamma \in \Gamma_\agentA} \somerefsBs \gamma$.
Therefore $\kPModel{\kStateS} \entails \coversA \{\somerefsBs \gamma \mid \gamma \in \Gamma_\agentA\}$.

($\Leftarrow$)
We show that $\entails \coversA \{\somerefsBs \gamma \mid \gamma \in \Gamma_\agentA\} \implies \somerefsBs \coversA \Gamma_\agentA$.
Let $\kPModelAndTuple{\kStateS} \in \classK$ be a pointed Kripke model such that $\kPModel{\kStateS} \entails \coversA \{\somerefsBs \gamma \mid \gamma \in \Gamma_\agentA\}$.
For every $\gamma \in \Gamma_\agentA$ there exists $\kStateT_\gamma \in \kSuccessorsA{\kStateS}$ and $\kPModelAndTuple[\gamma]{\kStateS[\gamma]} \in \classK$ such that $\kPModel{\kStateT_\gamma} \simulatesBs \kPModel[\gamma]{\kStateS[\gamma]}$ and $\kPModel[\gamma]{\kStateS[\gamma]} \entails \gamma$.
For every $\kStateT \in \kSuccessorsA{\kStateS}$ there exists $\gamma \in \Gamma_\agentA$ and $\kPModelAndTuple[\kStateT]{\kStateS[\kStateT]} \in \classK$ such that $\kPModel{\kStateT} \simulatesBs \kPModel[\kStateT]{\kStateS[\kStateT]}$ and $\kPModel[\kStateT]{\kStateS[\kStateT]} \entails \gamma$.
We use these refinements to construct a single larger refinement to satisfy the left-hand-side of the {\bf RComm} equivalence.

Let $\kPModelAndTupleP{\kStateSP}$ be a pointed Kripke model where:
\begin{eqnarray*}
    \kStatesP &=& \{\kStateSP\} \cup \kStates \cup \bigcup_{\gamma \in \Gamma_\agentA} \kStates[\gamma] \cup \bigcup_{\kStateT \in \kSuccessorsA{\kStateS}} \kStates[\kStateT]\\
    \kAccessibilityPA &=& \{(\kStateSP, \kStateS[\gamma]) \mid \gamma \in \Gamma_\agentA\} \cup \{(\kStateSP, \kStateS[\kStateT]) \mid \kStateT \in \kSuccessorsA{\kStateS}\} \cup \kAccessibilityA \cup \bigcup_{\gamma \in \Gamma_\agentA} \kAccessibilityA[\gamma] \cup \bigcup_{\kStateT \in \kSuccessorsA{\kStateS}} \kAccessibilityA[\kStateT]\\
    \kAccessibilityPB &=& \{(\kStateSP, \kStateT) \mid \kStateT \in \kSuccessorsB{\kStateS}\} \cup \kAccessibilityB \cup \bigcup_{\gamma \in \Gamma_\agentA} \kAccessibilityB[\gamma] \cup \bigcup_{\kStateT \in \kSuccessorsA{\kStateS}} \kAccessibilityB[\kStateT]\\
    \kValuationP(\atomP) &=& \{\kStateSP \mid \kStateS \in \kValuation(\atomP)\} \cup \kValuation(\atomP) \cup \bigcup_{\gamma \in \Gamma_\agentA} \kValuation[\gamma](\atomP) \cup \bigcup_{\kStateT \in \kSuccessorsA{\kStateS}} \kValuation[\kStateT](\atomP)
\end{eqnarray*}
where $\kStateSP$ is a fresh state not appearing in $\kStates$, $\kStates[\gamma]$ for any $\gamma \in \Gamma_\agentA$ or $\kStates[\kStateT]$ for any $\kStateT \in \kSuccessorsA{\kStateS}$, and $\agentB \in \agents \setminus \{\agentA\}$.

\begin{figure}
    \caption{A schematic of the construction used to show soundness of {\bf RComm}.}\label{rml-k-rcomm-construction}
    \centering
    \begin{tikzpicture}[>=stealth',shorten >=1pt,auto,node distance=7em,thick]

        \node (sp) {\underline{$\kPModelP{\kStateSP} \entails \coversA \{\gamma_1, \dots, \gamma_n\}$}};
        \node (mgn) [above of=sp] {$\kPModel[\gamma_n]{\kStateS[\gamma_n]} \entails \gamma_n$};
        \node (mgd) [left of=mgn,node distance=3.5em] {$\cdots$};
        \node (mg1) [left of=mgd,node distance=3.5em] {$\kPModel[\gamma_1]{\kStateS[\gamma_1]} \entails \gamma_1$};
        \node (mgd) [right of=mgn,node distance=3.5em] {$\cdots$};
        \node (mgr) [right of=mgd,node distance=3.5em] {$\kPModel[\kStateT]{\kStateS[\kStateT]} \entails \bigvee \Gamma_\agentA$};
        \node (mtn) [above of=mgn] {$\kPModel{\kStateT_{\gamma_n}} \entails \somerefsBs \gamma_n$};
        \node (mtd) [left of=mtn,node distance=3.5em] {$\cdots$};
        \node (mt1) [left of=mtd,node distance=3.5em] {$\kPModel{\kStateT_{\gamma_1}} \entails \somerefsBs \gamma_1$};
        \node (mtd2) [right of=mtn,node distance=3.5em] {$\cdots$};
        \node (mtr) [right of=mtd2,node distance=3.5em] {$\kPModel{\kStateT} \entails \bigvee \somerefsBs \Gamma_\agentA$};
        \node (ms) [above of=mtn] {$\kPModel{\kStateS} \entails \coversA \{\somerefsBs \gamma_1, \dots, \somerefsBs \gamma_n\}$};

      \path[every node/.style={font=\sffamily\small},->]
        (mt1) edge [swap,dashed] node {$\simulatesBs$} (mg1)
        (mtn) edge [dashed] node {$\simulatesBs$} (mgn)
        (mtr) edge [dashed] node {$\simulatesBs$} (mgr)
        (ms) edge [dashed,bend left=90,min distance=10em] node {$\simulatesBs$} (sp)
        (ms) edge [draw=none,dashed,bend right=90,min distance=10em] node {} (sp)
        (sp) edge node {$\agentA$} (mg1)
             edge [swap] node {$\agentA$} (mgn)
             edge [swap] node {$\agentA$} (mgr)
        (ms) edge [swap] node {$\agentA$} (mt1)
             edge node {$\agentA$} (mtr)
             edge node {$\agentA$} (mtn);
    \end{tikzpicture}
\end{figure}

A schematic of the Kripke model $\kPModelP{\kStateSP}$ and an overview of our construction is shown in Figure~\ref{rml-k-rcomm-construction}.
As in the construction used for {\bf RK} we can see that each of the $\agentsB$-refinements at successors, $\kPModel[\gamma_1]{\kStateT[\gamma_1]}, \dots, \kPModel[\gamma_n]{\kStateT[\gamma_n]}$, are combined into the larger Kripke model $\kPModelP{\kStateSP}$.
However in contrast to the construction used for {\bf RK} we note that here every $\agentA$-successor of $\kPModel{\kStateS}$ satisfies $\somerefsBs \gamma$ for some $\gamma \in \Gamma_\agentA$, and corresponds to some $\agentsB$-refinement $\kPModel[\kStateT]{\kStateS[\kStateT]}$.
This is required as $\agentA \notin \agentsB$ and so {\bf forth-$\agentA$} is required in order for $\kPModelP{\kStateSP}$ to be a $\agentsB$-refinement of $\kPModel{\kStateS}$.
From this schematic representation we can clearly see that $\kPModel{\kStateS} \simulatesBs \kPModelP{\kStateSP}$ and $\kPModelP{\kStateSP} \entails \coversA \{\gamma_1, \dots, \gamma_n\}$.

To show that $\kPModel{\kStateS} \entails \somerefsBs \coversA \Gamma_\agentA$ we will show that $\kPModel{\kStateS} \simulatesBs \kPModelP{\kStateSP}$ and $\kPModelP{\kStateSP} \entails \coversA \Gamma_\agentA$.

We first show that $\kPModel{\kStateS} \simulatesBs \kPModelP{\kStateSP}$.

For every $\gamma \in \Gamma_\agentA$ let $\refinement^\gamma \subseteq \kStates \times \kStates[\gamma]$ be a $\agentsB$-refinement from $\kPModel{\kStateT_\gamma}$ to $\kPModel[\gamma]{\kStateS[\gamma]}$ and
for every $\kStateT \in \kSuccessorsA{\kStateS}$ let $\refinement^{\kStateT} \subseteq \kStates \times \kStates[\kStateT]$ be a $\agentsB$-refinement from $\kPModel{\kStateT}$ to $\kPModel[\kStateT]{\kStateS[\kStateT]}$.
We define $\refinement \subseteq \kStates \times \kStatesP$ where:
$$
\refinement = \{(\kStateS, \kStateSP)\} \cup \{(\kStateT, \kStateT) \mid \kStateT \in \kStates\} \cup \bigcup_{\gamma \in \Gamma_\agentA} \refinement^\gamma \cup \bigcup_{\kStateT \in \kSuccessorsA{\kStateS}} \refinement^{\kStateT}
$$
We show that $\refinement$ is a $\agentsB$-refinement from $\kPModel{\kStateS}$ to $\kPModelP{\kStateSP}$.

Let $\atomP \in \atoms$, $\agentB \in \agents$, $\agentC \in \agents \setminus \agentsB$.
We show by cases that the relationships in $\refinement$ satisfy the conditions {\bf atoms-$\atomP$}, {\bf forth-$\agentC$}, and {\bf back-$\agentB$}.

\begin{description}
    \item[Case $(\kStateS, \kStateSP) \in \refinement$:]
        \hfill
        \begin{description}
            \item[atoms-$\atomP$] 
                By construction $\kStateS \in \kValuation(\atomP)$ if and only if $\kStateSP \in \kValuationP(\atomP)$.
            \item[forth-$\agentC$]
                Suppose that $\agentC = \agentA$.
                Let $\kStateT \in \kSuccessorsA{\kStateS}$.
                By construction $\kStateS[\kStateT] \in \kSuccessorsPA{\kStateSP}$ and $(\kStateT, \kStateS[\kStateT]) \in \refinement^{\kStateT} \subseteq \refinement$.

                Suppose that $\agentC \neq \agentA$.
                Let $\kStateT \in \kSuccessorsC{\kStateS}$.
                By construction $\kSuccessorsPC{\kStateSP} = \kSuccessorsC{\kStateSP}$.
                Then $\kStateT \in \kSuccessorsPC{\kStateSP}$ and by construction $(\kStateT, \kStateT) \in \refinement$.
            \item[back-$\agentB$]
                Suppose that $\agentB = \agentA$.
                Let $\kStateS[\gamma] \in \kSuccessorsPA{\kStateSP}$ where $\gamma \in \Gamma_\agentA$.
                Then by hypothesis $\kStateT_\gamma \in \kSuccessorsA{\kStateS}$ and $(\kStateT_\gamma, \kStateS[\gamma]) \in \refinement^\gamma \subseteq \refinement$.
                Let $\kStateS[\kStateT] \in \kSuccessorsPA{\kStateSP}$ where $\kStateT \in \kSuccessorsA{\kStateS}$.
                Then by hypothesis $\kStateT \in \kSuccessorsA{\kStateS}$ and $(\kStateT, \kStateS[\kStateT]) \in \refinement^{\kStateT} \subseteq \refinement$.

                Suppose that $\agentB \neq \agentA$.
                Let $\kStateT \in \kSuccessorsPB{\kStateSP}$.
                By construction $\kStateT \in \kSuccessorsB{\kStateS}$ and $(\kStateT, \kStateT) \in \refinement$.
        \end{description}
    \item[Case $(\kStateT, \kStateT) \in \refinement$ where $\kStateT \in \kStates$:]
        \hfill
        \begin{description}
            \item[atoms-$\atomP$] 
                By construction $\kStateT \in \kValuation(\atomP)$ if and only if $\kStateT \in \kValuationP(\atomP)$.
            \item[forth-$\agentC$]
                Let $\kStateU \in \kSuccessorsC{\kStateT}$.
                By construction $\kSuccessorsPC{\kStateT} = \kSuccessorsC{\kStateT}$.
                Then $\kStateU \in \kSuccessorsPC{\kStateT}$ and by construction $(\kStateU, \kStateU) \in \refinement$.
            \item[back-$\agentB$]
                Let $\kStateU \in \kSuccessorsPB{\kStateT}$.
                By construction $\kSuccessorsPB{\kStateT} = \kSuccessorsB{\kStateT}$.
                Then $\kStateU \in \kSuccessorsB{\kStateT} \subseteq \kStates$ and by construction $(\kStateU, \kStateU) \in \refinement$.
        \end{description}
    \item[{Case $(\kStateU, \kStateU[\gamma]) \in \refinement^\gamma \subseteq \refinement$ where $\gamma \in \Gamma_\agentA$:}]
        \hfill
        \begin{description}
            \item[atoms-$\atomP$] 
                By {\bf atoms-$\atomP$} for $\refinement^\gamma$ we have $\kStateU \in \kValuation(\atomP)$ if and only if $\kStateU[\gamma] \in \kValuation[\gamma](\atomP)$.
                By construction $\kStateU[\gamma] \in \kValuation[\gamma](\atomP)$ if and only if $\kStateU[\gamma] \in \kValuationP(\atomP)$.
            \item[forth-$\agentC$]
                Let $\kStateV \in \kSuccessorsC{\kStateU}$.
                By {\bf forth-$\agentC$} for $\refinement^{\gamma}$ there exists $\kStateV[\gamma] \in \kSuccessorsC[\gamma]{\kStateU[\gamma]}$ such that $(\kStateV, \kStateV[\gamma]) \in \refinement^{\gamma}$.
                By construction $\kSuccessorsPC{\kStateU[\gamma]} = \kSuccessorsC{\kStateU[\gamma]}$.

                Then $\kStateV[\gamma] \in \kSuccessorsPC{\kStateU[\gamma]}$ and $(\kStateV, \kStateV[\gamma]) \in \refinement$.
            \item[back-$\agentB$]
                Let $\kStateV[\gamma] \in \kSuccessorsPB{\kStateU[\gamma]}$.
                By construction $\kSuccessorsPB{\kStateU[\gamma]} = \kSuccessorsB[\gamma]{\kStateU[\gamma]}$.
                Then $\kStateV[\gamma] \in \kSuccessorsB[\gamma]{\kStateU[\gamma]}$.
                By {\bf back-$\agentB$} for $\refinement^{\gamma}$ there exists $\kStateV \in \kSuccessorsB{\kStateU}$ such that $(\kStateV, \kStateV[\gamma]) \in \refinement^{\gamma} \subseteq \refinement$.
        \end{description}
    \item[{Case $(\kStateU, \kStateU[\kStateT]) \in \refinement^{\kStateT} \subseteq \refinement$ where $\kStateT \in \kSuccessorsA{\kStateS}$:}]
        \hfill
        \begin{description}
            \item[atoms-$\atomP$] 
                By {\bf atoms-$\atomP$} for $\refinement^{\kStateT}$ we have $\kStateU \in \kValuation(\atomP)$ if and only if $\kStateU[\kStateT] \in \kValuation[\kStateT](\atomP)$.
                By construction $\kStateU[\kStateT] \in \kValuation[\kStateT](\atomP)$ if and only if $\kStateU[\kStateT] \in \kValuationP(\atomP)$.
            \item[forth-$\agentC$]
                Let $\kStateV \in \kSuccessorsC{\kStateU}$.
                By {\bf forth-$\agentC$} for $\refinement^{\kStateT}$ there exists $\kStateV[\kStateT] \in \kSuccessorsC[\kStateT]{\kStateU[\kStateT]}$ such that $(\kStateV, \kStateV[\kStateT]) \in \refinement^{\kStateT}$.
                By construction $\kSuccessorsPC{\kStateU[\kStateT]} = \kSuccessorsC{\kStateU[\kStateT]}$.
                Then $\kStateV[\kStateT] \in \kSuccessorsPC{\kStateU[\kStateT]}$ and $(\kStateV, \kStateV[\gamma]) \in \refinement$.
            \item[back-$\agentB$]
                Let $\kStateV[\kStateT] \in \kSuccessorsPB{\kStateU[\kStateT]}$.
                By construction $\kSuccessorsPB{\kStateU[\kStateT]} = \kSuccessorsB[\kStateT]{\kStateU[\kStateT]}$.
                Then $\kStateV[\kStateT] \in \kSuccessorsB[\kStateT]{\kStateU[\kStateT]}$.
                By {\bf back-$\agentB$} for $\refinement^{\kStateT}$ there exists $\kStateV \in \kSuccessorsB{\kStateU}$ such that $(\kStateV, \kStateV[\kStateT]) \in \refinement^{\kStateT} \subseteq \refinement$.
        \end{description}
\end{description}

Therefore $\refinement$ is a $\agentsB$-refinement and as $(\kStateS, \kStateSP) \in \refinement$ we have that $\kPModel{\kStateS} \simulatesBs \kPModelP{\kStateSP}$.

Finally $\kPModelP{\kStateSP} \entails \coversA \Gamma_\agentA$ follows from the same reasoning as in the proof of soundness of {\bf RK} in Lemma~\ref{rml-k-rk}.
Therefore $\kPModel{\kStateS} \entails \somerefsBs \coversA \Gamma_\agentA$.
\end{proof}

We next show that the axiom {\bf RDist} is sound.
Recall that the axiom {\bf RDist} takes the form of $\proves \somerefsBs \bigwedge_{\agentC \in \agentsC} \coversC \Gamma_\agentC \iff \bigwedge_{\agentC \in \agentsC} \somerefsBs \coversC \Gamma_\agentC$ where $\agentsB, \agentsC \subseteq \agents$ and for every $\agentC \in \agentsC$: $\Gamma_\agentC \subseteq \langRml$ is a finite set of formulas.

\begin{lemma}\label{rml-k-rdist}
The axiom {\bf RDist} from the axiomatisation \axiomRmlK{} is sound with respect to the semantics of the logic \logicRmlK{}.
\end{lemma}

\begin{proof}
($\Rightarrow$) Let $\kPModel{\kStateS} \in \classK$ be a pointed Kripke model such that $\kPModel{\kStateS} \entails \somerefsBs \bigwedge_{\agentC \in \agentsC} \coversC \Gamma_\agentC$.
There exists $\kPModelP{\kStateSP} \in \classK$ such that $\kPModel{\kStateS} \simulatesBs \kPModelP{\kStateSP}$ and $\kPModelP{\kStateSP} \entails \bigwedge_{\agentC \in \agentsC} \coversC \Gamma_\agentC$.
For every $\agentC \in \agentsC$ we have that $\kPModelP{\kStateSP} \entails \coversC \Gamma_\agentC$ and so $\kPModel{\kStateS} \entails \somerefsBs \coversC \Gamma_\agentC$.
Therefore $\kPModel{\kStateS} \entails \bigwedge_{\agentC \in \agentsC} \somerefsBs \coversC \Gamma_\agentC$.

($\Leftarrow$) Let $\kPModelAndTuple{\kStateS} \in \classK$ be a pointed Kripke model such that $\kPModel{\kStateS} \entails \bigwedge_{\agentC \in \agentsC} \somerefsBs \coversC \Gamma_\agentC$.
For every $\agentC \in \agentsC$ there exists $\kPModel[\agentC]{\kStateS[\agentC]} \in \classK$ such that $\kPModel{\kStateS} \simulatesBs \kPModel[\agentC]{\kStateS[\agentC]}$ and $\kPModel[\agentC]{\kStateS[\agentC]} \entails \coversC \Gamma_\agentC$.
We use these refinements to construct a single larger refinement to satisfy the left-hand-side of the {\bf RDist} equivalence.

Let $\kPModelAndTupleP{\kStateSP}$ be a pointed Kripke model where:
\begin{eqnarray*}
    \kStatesP &=& \{\kStateSP\} \cup \kStates \cup \bigcup_{\agentC \in \agentsC} \kStates[\agentC]\\
    \kAccessibilityC &=& \{(\kStateSP, \kStateT[\agentC]) \mid \kStateT[\agentC] \in \kSuccessorsC[\agentC]{\kStateS[\agentC]}\} \cup \kAccessibilityC \cup \bigcup_{\agentD \in \agentsC} \kAccessibilityC[\agentD] \text{ for } \agentC \in \agentsC\\
    \kAccessibilityB &=& \{(\kStateSP, \kStateT) \mid \kStateT \in \kSuccessorsB{\kStateS}\} \cup \kAccessibilityB \cup \bigcup_{\agentC \in \agentsC} \kAccessibilityB[\agentC] \text{ for } \agentB \notin \agentsC\\
    \kValuation(\atomP) &=& \{\kStateSP \mid \kStateS \in \kValuation(\atomP)\} \cup \kValuation(\atomP) \cup \bigcup_{\agentC \in \agentsC} \kValuation[\agentC](\atomP)
\end{eqnarray*}
where $\kStateSP$ is a fresh state not appearing in $\kStates$ or $\kStates[\agentC]$ for any $\agentC \in \Gamma_\agentA$; $\agentC \in \agentsC$; and, $\agentB \in \agents \setminus \agentsC$.

\begin{figure}
    \caption{A schematic of the construction used to show soundness of {\bf RDist}.}\label{rml-k-rdist-construction}
    \centering
    \begin{tikzpicture}[>=stealth',shorten >=1pt,auto,node distance=7em,thick]

        \node (sp) {\underline{$\kStateSP \entails \somerefsBs (\covers[\agentC_1] \Gamma_{\agentC_1} \land \cdots \land \covers[\agentC_n] \Gamma_{\agentC_n})$}};
        \node (mtid) [above of=sp] {};
        \node (mt1m) [left of=mtid,node distance=3.5em] {$\kStateT[\agentC_1]_m$};
        \node (mt1d) [left of=mt1m,node distance=3.5em] {$\cdots$};
        \node (mt11) [left of=mt1d,node distance=3.5em] {$\kStateT[\agentC_1]_1$};
        \node (mtn1) [right of=mtid,node distance=3.5em] {$\kStateT[\agentC_n]_1$};
        \node (mtnd) [right of=mtn1,node distance=3.5em] {$\cdots$};
        \node (mtnm) [right of=mtnd,node distance=3.5em] {$\kStateT[\agentC_n]_m$};
        \node (mtd) [above of=mtid] {$\cdots$};
        \node (ms) [above of=mtd] {$\kStateS \entails \somerefsBs \covers[\agentC_1] \Gamma_{\agentC_1} \land \cdots \land \somerefsBs \covers[\agentC_n] \Gamma_{\agentC_n}$};
        \node (mt1) [left of=mtd] {$\kPModel[\agentC_1]{\kStateS[\agentC_1]} \entails \covers[\agentC_1] \Gamma_{\agentC_1}$};
        \node (mtn) [right of=mtd] {$\kPModel[\agentC_n]{\kStateS[\agentC_n]} \entails \covers[\agentC_n] \Gamma_{\agentC_n}$};

      \path[every node/.style={font=\sffamily\small},->]
        (mt1) edge [swap] node {$\agentC_1$} (mt11)
              edge node {$\agentC_1$} (mt1m)
        (mtn) edge [swap] node {$\agentC_n$} (mtn1)
              edge node {$\agentC_n$} (mtnm)
        (ms) edge [dashed,bend left=90,min distance=10em] node {$\simulatesBs$} (sp)
        (ms) edge [draw=none,dashed,bend right=90,min distance=10em] node {} (sp)
        (sp) edge node {$\agentC_1$} (mt11)
             edge node {$\agentC_1$} (mt1m)
             edge [swap] node {$\agentC_n$} (mtn1)
             edge [swap] node {$\agentC_n$} (mtnm)
        (ms) edge [swap,dashed] node {$\simulatesBs$} (mt1)
             edge [dashed]node {$\simulatesBs$} (mtn);
    \end{tikzpicture}
\end{figure}

A schematic of the Kripke model $\kPModelP{\kStateSP}$ and an overview of our construction is shown in Figure~\ref{rml-k-rdist-construction}.
Here we can see that $\kPModelP{\kStateSP}$ is formed by taking the $\agentC$-successors for each respective $\agentsB$-refinement $\kPModel[\agentC]{\kStateS[\agentC]}$ and combining them into a single model.
From this schematic representation we can clearly see that  and $\kPModelP{\kStateSP} \entails \coversA \{\gamma_1, \dots, \gamma_n\}$.
It is less clear that $\kPModel{\kStateS} \simulatesBs \kPModelP{\kStateSP}$, but it is straight-forward to show this by combining the refinements $\refinement^\agentC$.

To show that $\kPModel{\kStateS} \entails \somerefsBs \bigwedge_{\agentC \in \agentsC} \coversC \Gamma_\agentC$ we will show that $\kPModel{\kStateS} \simulatesBs \kPModelP{\kStateSP}$ and $\kPModelP{\kStateSP} \entails \bigwedge_{\agentC \in \agentsC} \coversC \Gamma_\agentC$.

We first show that $\kPModel{\kStateS} \simulatesBs \kPModelP{\kStateSP}$.

For every $\agentC \in \agentsC$ let $\refinement^\agentC \subseteq \kStates \times \kStates[\agentC]$ be a $\agentsB$-refinement from $\kPModel{\kStateS}$ to $\kPModel[\agentC]{\kStateS[\agentC]}$.
We define $\refinement \subseteq \kStates \times \kStatesP$ where:
$$
\refinement = \{(\kStateS, \kStateSP)\} \cup \{(\kStateT, \kStateT) \mid \kStateT \in \kStates\} \cup \bigcup_{\agentC \in \agentsC} \refinement^\agentC
$$
We show that $\refinement$ is a $\agentsB$-refinement from $\kPModel{\kStateS}$ to $\kPModelP{\kStateSP}$.

Let $\atomP \in \atoms$, $\agentB \in \agents$, $\agentD \in \agents \setminus \agentsB$.
We show by cases that the relationships in $\refinement$ satisfy the conditions {\bf atoms-$\atomP$}, {\bf forth-$\agentD$}, and {\bf back-$\agentB$}.

\begin{description}
    \item[Case $(\kStateS, \kStateSP) \in \refinement$:]
        \hfill
        \begin{description}
            \item[atoms-$\atomP$] 
                By construction $\kStateS \in \kValuation(\atomP)$ if and only if $\kStateSP \in \kValuationP(\atomP)$.
            \item[forth-$\agentD$]
                Suppose that $\agentD \in \agentsC$.
                Let $\kStateT \in \kSuccessorsD{\kStateS}$.
                By hypothesis $(\kStateS, \kStateS[\agentD]) \in \refinement^\agentD$.
                By {\bf forth-$\agentD$} for $\refinement^\agentD$ there exists $\kStateT[\agentD] \in \kSuccessorsD[\agentD]{\kStateS[\agentD]}$ such that $(\kStateT, \kStateT[\agentD]) \in \refinement^\agentD$.
                By construction $\kSuccessorsPD{\kStateSP} = \kSuccessorsD[\agentD]{\kStates[\agentD]}$.
                Then $\kStateT[\agentD] \in \kSuccessorsPD{\kStateSP}$ and by construction $(\kStateT, \kStateT[\agentD]) \in \refinement$.

                Suppose that $\agentD \notin \agentsC$.
                Let $\kStateT \in \kSuccessorsD{\kStateS}$.
                By construction $\kSuccessorsPD{\kStateSP} = \kSuccessorsD{\kStateS}$.
                Then $\kStateT \in \kSuccessorsPD{\kStateSP}$ and by construction $(\kStateT, \kStateT) \in \refinement$.
            \item[back-$\agentB$]
                Suppose that $\agentB \in \agentsC$.
                By construction $\kSuccessorsPB{\kStateSP} \kSuccessorsB[\agentB]{\kStateS[\agentB]}$.
                Let $\kStateT[\agentB] \in \kSuccessorsB[\agentB]{\kStateS[\agentB]}$.
                By hypothesis $(\kStateS, \kStateS[\agentB]) \in \refinement^\agentB$.
                By {\bf back-$\agentB$} for $\refinement^\agentB$ there exists $\kStateT \in \kSuccessorsB{\kStateS}$ such that $(\kStateT, \kStateT[\agentB]) \in \refinement^\agentB \subseteq \refinement$.

                Suppose that $\agentB \notin \agentsC$.
                Let $\kStateT \in \kSuccessorsPB{\kStateSP}$.
                By construction $\kStateT \in \kSuccessorsB{\kStateS}$ and $(\kStateT, \kStateT) \in \refinement$.
        \end{description}
    \item[{Case $(\kStateT, \kStateT) \in \refinement$ where $\kStateT \in \kStates$:}]
        \hfill
        \begin{description}
            \item[atoms-$\atomP$] 
                By construction $\kStateT \in \kValuation(\atomP)$ if and only if $\kStateT \in \kValuationP(\atomP)$.
            \item[forth-$\agentD$]
                Let $\kStateU \in \kSuccessorsD{\kStateT}$.
                By construction $\kSuccessorsPD{\kStateT} = \kSuccessorsD{\kStateT}$.
                Then $\kStateU \in \kSuccessorsPD{\kStateT}$ and by construction $(\kStateU, \kStateU) \in \refinement$.
            \item[back-$\agentB$]
                Let $\kStateU \in \kSuccessorsPB{\kStateT}$.
                By construction $\kSuccessorsPB{\kStateT} = \kSuccessorsB{\kStateT}$.
                Then $\kStateU \in \kSuccessorsB{\kStateT}$ and by construction $(\kStateU, \kStateU) \in \refinement$.
        \end{description}
    \item[{Case $(\kStateT, \kStateT[\agentC]) \in \refinement^\agentC \subseteq \refinement$ where $\agentC \in \agentsC$:}]
        \hfill
        \begin{description}
            \item[atoms-$\atomP$] 
                By {\bf atoms-$\atomP$} for $\refinement^\agentC$ we have $\kStateT \in \kValuation(\atomP)$ if and only if $\kStateT[\agentC] \in \kValuation[\agentC](\atomP)$.
                By construction $\kStateT[\agentC] \in \kValuation[\agentC](\atomP)$ if and only if $\kStateT[\agentC] \in \kValuationP(\atomP)$.
            \item[forth-$\agentD$]
                Let $\kStateU \in \kSuccessorsD{\kStateT}$.
                By {\bf forth-$\agentD$} for $\refinement^\agentC$ there exists $\kStateU[\agentC] \in \kSuccessorsD[\agentC]{\kStateT[\agentC]}$ such that $(\kStateU, \kStateU[\agentC]) \in \refinement^\agentC$.
                By construction $\kSuccessorsPD{\kStateT[\agentC]} \kSuccessorsD[\agentC]{\kStateT[\agentC]}$.
                Then $\kStateU[\agentC] \in \kSuccessorsPD{\kStateT[\agentC]}$ and $(\kStateU, \kStateU[\agentC]) \in \refinement$.
            \item[back-$\agentB$]
                Let $\kStateU[\agentC] \in \kSuccessorsPB{\kStateT[\agentC]}$.
                By construction $\kSuccessorsPB{\kStateT[\agentC]} = \kSuccessorsB[\agentC]{\kStateT[\agentC]}$.
                Then $\kStateU[\agentC] \in \kSuccessorsB[\agentC]{\kStateT[\agentC]}$.
                By {\bf back-$\agentB$} for $\refinement^\agentC$ there exists $\kStateU \in \kSuccessorsB{\kStateT}$ such that $(\kStateU, \kStateU[\agentC]) \in \refinement^\agentC \subseteq \refinement$.
        \end{description}
\end{description}

Therefore $\refinement$ is a $\agentsB$-refinement and as $(\kStateS, \kStateSP) \in \refinement$ we have that $\kPModel{\kStateS} \simulatesBs \kPModelP{\kStateSP}$.

Finally $\kPModelP{\kStateSP} \entails \bigwedge_{\agentC \in \agentsC} \coversC \Gamma_\agentC$ follows from similar reasoning to the proof of soundness of {\bf RK} in Lemma~\ref{rml-k-rk}.
Therefore $\kPModel{\kStateS} \entails \somerefsBs \bigwedge_{\agentC \in \agentsC} \coversC \Gamma_\agentC$.
\end{proof}

Finally we show that the axiomatisation \axiomRmlK{} is sound.

\begin{lemma}\label{rml-k-sound}
The axiomatisation \axiomRmlK{} is sound with respect to the semantics of the logic \logicRmlK{}.
\end{lemma}

\begin{proof}
The soundness of the axioms and rules of \axiomK{} with respect to the semantics of the logic \logicRmlK{} follow from the same reasoning that they are sound in the logic \logicK{}.
The soundness of {\bf R}, {\bf RP} and {\bf NecR} follow from Proposition~\ref{rml-validities}.
The soundness of {\bf RK}, {\bf RComm} and {\bf RDist} were shown in the previous lemmas.
\end{proof}

\section{Completeness}\label{rml-k-completeness}

In this section we show that the axiomatisation \axiomRmlK{} is complete with respect to the semantics of the logic \logicRmlK{}.
We show that \axiomRmlK{} is complete by demonstrating a provably correct translation from formulas of \langRml{} to the underlying modal language \langMl{}.
As the interpretation of \langMl{} formulas is the same between \logicRmlK{} and \logicK{}, and \logicK{} has a sound and complete axiomatisation \axiomK{} that forms part of the axiomatisation \logicRmlK{}, this allows us to construct proofs of valid \langRml{} formulas by first translating to \langMl{} and then relying on the completeness of \axiomK{}.
As a consequence of this provably correct translation we also have that \logicRmlK{} is expressively equivalent to \logicK{}, and that \logicRmlK{} is compact and decidable (via the compactness and decidability of \logicK{}).

In order to show that the reduction axioms of \axiomRmlK{} are applicable to all \langRml{} formulas we will use a disjunctive normal form for modal logic taken from work by Janin and Walukiewicz~\cite{janin:1995} in the modal $\mu$-calculus.
The disjunctive normal form uses the cover operator, and restricts negations and conjunctions to situations where the reduction axioms are applicable.
Much of our work in this section is simply restating results by Janin and Walukiewicz~\cite{janin:1995} about the disjunctive normal form.
We provide these details because the presentation of the disjunctive normal form of Janin and Walukiewicz~\cite{janin:1995} differs from our presentation, and also includes aspects specific to the modal $\mu$-calculus that are not relevant for our purposes.
In Chapter~\ref{rml-kd45} we introduce a similar normal form in order to show the completeness of the axiomatisations for \logicRmlKFF{} and \logicRmlKD{}.

We first recall the negation normal form, as an intermediate step before the disjunctive normal form.

\begin{definition}[Negation normal form]
A formula in {\em negation normal form} is inductively defined as:
$$
\phi ::= \atomP \mid
         \lnot \atomP \mid
         \phi \land \phi \mid
         \phi \lor \phi \mid
         \necessaryA \phi \mid
         \possibleA \phi
$$
where $\atomP \in \atoms$ and $\agentA \in \agents$.
\end{definition}

\begin{lemma}\label{nnf-equivalent}
Every modal formula is equivalent to a formula in negation normal form in the logic \logicK{}.
\end{lemma}

\begin{proof}
Similar to negation normal forms in propositional logic, we can recursively push the negations inwards until negations are only applied to propositional atoms, using the following equivalences:
\begin{eqnarray*}
    \entails \lnot \lnot \phi &\iff& \phi\\
    \entails \lnot (\phi \land \psi) &\iff& \lnot \phi \lor \lnot \psi\\
    \entails \lnot \necessaryA \phi &\iff& \possibleA \lnot \phi
\end{eqnarray*}
\end{proof}

We next recall the disjunctive normal of Janin and Walukiewicz~\cite{janin:1995}.
We note that the original work by Janin and Walukiewicz~\cite{janin:1995} used a different syntax for the cover operator, and additionally featured the modal $\mu$-calculus fixed-point operators.
Our syntax follows that of Bilkova, Palmigiano and Venema~\cite{bilkova:2008} primarily to be consistent with the established literature in \logicRml{}~\cite{vanditmarsch:2009,vanditmarsch:2010}.

\begin{definition}[Disjunctive normal form]
A formula in {\em disjunctive normal form} is inductively defined as:
$$
\phi ::= \pi \land \bigwedge_{\agentB \in \agentsB} \coverB \Gamma_\agentB \mid \phi \lor \phi
$$
where $\pi \in \langPl$, $\agentsB \subseteq \agents$, and for every $\agentB \in \agentsB$, $\Gamma_\agentB \subseteq \langMl$ is a finite set of formulas in disjunctive normal form.
\end{definition}

\begin{lemma}\label{dnf-equivalent}
Every modal formula is equivalent to a formula in disjunctive normal form in the logic \logicK{}.
\end{lemma}

\begin{proof}
Let $\phi \in \langMl$ be a modal formula.
Without loss of generality, by Lemma~\ref{nnf-equivalent} we may assume that $\phi$ is in negation normal form.
We prove by induction on the modal depth of $\phi$ and the structure of $\phi$ that it is equivalent to a formula in disjunctive normal form.

Suppose that $\phi = \atomP$ or $\phi = \lnot \atomP$ where $\atomP \in \atoms$.
Then $\phi$ is already in disjunctive normal form.

Suppose that $\phi = \psi \lor \chi$ where $\psi, \chi \in \langMl$ in negation normal form.
By the induction hypothesis there exists $\psi', \chi' \in \langMl$ in disjunctive normal form such that $\entails \psi \iff \psi'$ and $\entails \chi \iff \chi'$.
Then $\psi' \lor \chi'$ is in disjunctive normal form and $\entails (\psi \lor \chi) \iff (\psi' \lor \chi')$.

Suppose that $\phi = \necessaryA \psi$ where $\psi \in \langMl$ in negation normal form.
By the induction hypothesis there exists $\psi' \in \langMl$ in disjunctive normal form such that $\entails \psi \iff \psi'$.
Then $\coversA \{\phi'\} \lor \coversA \emptyset$ is in disjunctive normal form and $\entails \necessaryA \phi \iff (\coversA \{\phi'\} \lor \coversA \emptyset)$.

Suppose that $\phi = \possibleA \psi$ where $\psi \in \langMl$ in negation normal form.
By the induction hypothesis there exists $\psi' \in \langMl$ in disjunctive normal form such that $\entails \psi \iff \psi'$.
Then $\coversA \{\phi', \top\}$ is in disjunctive normal form and $\entails \possibleA \phi \iff \coversA \{\phi', \top\}$.

Suppose that $\phi = \psi \land \chi$ where $\psi, \chi \in \langMl$ in negation normal form.
By the induction hypothesis there exists $\psi', \chi' \in \langMl$ in disjunctive normal form such that $\entails \psi \iff \psi'$ and $\entails \chi \iff \chi'$.
As $\psi'$ and $\chi'$ are in disjunctive normal form then $\psi' = \delta_0 \lor \cdots \lor \delta_m$ and $\chi' = \gamma_0 \lor \cdots \lor \gamma_n$ where $m, n \in \naturals$.
We distribute the disjunctions over the conjunction using the following equivalence:
$$
\entails (\psi \land \chi) \iff (\psi' \land \chi') \iff \vee_{\substack{0 \leq i \leq m\\0 \leq j \leq n}} (\delta_i \land \gamma_i)
$$
Here we note that the subformulas $\delta_i \land \gamma_i$ may not be in the appropriate form, and so we use some equivalences to find appropriate substitutes. 
Let $i, j \in \naturals$ such that $0 \leq i \leq m$ and $0 \leq j \leq n$ and suppose that $\delta_i = \pi \land \bigwedge_{\agentB \in \agentsB} \coverB \Gamma_\agentB$ and $\gamma_j = \tau \land \bigwedge_{\agentC \in \agentsC} \coverC \Gamma_\agentC$.
We rewrite the conjunction $\delta_i \land \gamma_j$ using the following equivalence:
$$
\entails (\delta_i \land \gamma_j) \iff 
(\pi \land \tau) \land \left( \bigwedge_{\agentB \in \agentsB} \coverB \Gamma_\agentB \land \bigwedge_{\agentC \in \agentsC} \coverC \Gamma_\agentC \right)
$$
This may leave us with more than one cover operator for some agents, but we can combine these cover operators using the following equivalence:
$$
\entails (\coversA \Gamma \land \coversA \Gamma') \iff \coversA \{\gamma \land \gamma' \mid \gamma \in \Gamma, \gamma' \in \Gamma'\}
$$
This equivalence should be obvious when the cover operator is expanded:
\begin{eqnarray*}
&&\entails \left( \necessaryA \bigvee_{\gamma \in \Gamma} \gamma \land \bigwedge_{\gamma \in \Gamma} \possibleA \gamma \land \necessaryA \bigvee_{\gamma' \in \Gamma'} \gamma' \land \bigwedge_{\gamma' \in \Gamma'} \possibleA \gamma' \right)
\\&&\quad\iff
\left( \necessaryA \bigvee_{\gamma \in \Gamma, \gamma' \in \Gamma'} (\gamma \land \gamma') \land \bigwedge_{\gamma \in \Gamma, \gamma' \in \Gamma'} \possibleA (\gamma \land \gamma') \right)
\end{eqnarray*}
Here we note that $\gamma \land \gamma'$ may not be in disjunctive normal form, and so we will use to the induction hypothesis to find a substitute.
For every $\gamma \in \Gamma$, $\gamma' \in \Gamma'$ as $\gamma \land \gamma'$ has a modal depth less than $\phi$ then by the induction hypothesis there exists an $\epsilon_{\gamma,\gamma'} \in \langMl$ in disjunctive normal form such that $\entails \epsilon_{\gamma,\gamma'} \iff (\gamma \land \gamma')$.
Substituting $\epsilon_{\gamma,\gamma'}$ for $\gamma \land \gamma'$ after applying all of the above equivalences leaves us with a formula in disjunctive normal form.
\end{proof}

We note that we have shown a semantic equivalence between \langMl{} formulas and formulas in disjunctive normal form.
As \axiomK{} is a sound and complete axiomatisation for \logicK{} then this is also a provable equivalence in \axiomK{}, and as the axioms and rules of \axiomK{} are included in the axiomatisation \axiomRmlK{} this is also a provable equivalence in \axiomRmlK{}.

Given the disjunctive normal form, we will show that the reduction axioms of \axiomRmlK{} may be applied to formulas in disjunctive normal form in order to give a provably correct translation.
We first show some useful theorems in \axiomRmlK{}.

\begin{lemma}\label{rml-k-theorems}
The following are theorems of \axiomRmlK{}:
\begin{align}
    \proves & \allrefsBs (\phi \land \psi) \iff (\allrefsBs \phi \land \allrefsBs \psi) \label{rml-k-and}\\
    \proves & \somerefsBs (\phi \lor \psi) \iff (\somerefsBs \phi \lor \somerefsBs \psi) \label{rml-k-or}\\
    \proves & \somerefsBs (\phi \land \psi) \implies (\somerefsBs \phi \land \somerefsBs \psi) \label{rml-k-d-and}\\
    \proves & (\allrefsBs \phi \land \somerefsBs \psi) \implies \somerefsBs (\phi \land \psi) \label{rml-k-db-and}\\
    \proves & (\pi \land \somerefsBs \psi) \iff \somerefsBs (\pi \land \psi) \label{rml-k-pd-and}\\
    \proves & \displaystyle \somerefsBs (\pi \land \bigwedge_{\agentC \in \agentsC} \coverC \Gamma_\agentC) \iff \nonumber\\
            & \displaystyle \quad
            (
            \pi \land
            \bigwedge_{\agentC \in \agentsC \cap \agentsB} \bigwedge_{\gamma \in \Gamma_\agentC} \possibleC \somerefsBs \gamma \land
            \bigwedge_{\agentC \in \agentsC \setminus \agentsB} \coversC \{\somerefsBs \gamma \mid \gamma \in \Gamma_\agentC\} 
            ) \label{rml-k-cover}
\end{align}
where $\phi, \psi \in \langRml$, $\pi \in \langPl$, $\agentA \in \agents$, $\agentsB, \agentsC \subseteq \agents$, and for every $\agentA \in \agents$: $\Gamma_\agentA \subseteq \langRml$ is a finite set of formulas.
\end{lemma}

\begin{proof}
\begin{description}
    \item[(\ref{rml-k-and})]
    We show that $\proves \allrefsBs (\phi \land \psi) \iff (\allrefsBs \phi \land \allrefsBs \psi)$.

    For the left-to-right direction:
    $$
    \begin{array}{ll}
        \proves (\phi \land \psi) \implies \phi & \text{{\bf P}}\\
        \proves \allrefsBs ((\phi \land \psi) \implies \phi) & \text{{\bf NecR}}\\
        \proves \allrefsBs ((\phi \land \psi) \implies \phi) \implies (\allrefsBs (\phi \land \psi) \implies \allrefsBs \phi) & \text{{\bf R}}\\
        \proves \allrefsBs (\phi \land \psi) \implies \allrefsBs \phi & \text{{\bf MP}}\\
        \proves (\phi \land \psi) \implies \psi & \text{{\bf P}}\\
        \proves \allrefsBs ((\phi \land \psi) \implies \psi) & \text{{\bf NecR}}\\
        \proves \allrefsBs ((\phi \land \psi) \implies \psi) \implies (\allrefsBs (\phi \land \psi) \implies \allrefsBs \psi) & \text{{\bf R}}\\
        \proves \allrefsBs (\phi \land \psi) \implies \allrefsBs \psi & \text{{\bf MP}}\\
        \proves \allrefsBs (\phi \land \psi) \implies (\allrefsBs \phi \land \allrefsBs \psi) & \text{{\bf P}}
    \end{array}
    $$

    For the right-to-left direction:
    $$
    \begin{array}{ll}
        \proves \phi \implies (\psi \implies (\phi \land \psi)) & \text{{\bf P}}\\
        \proves \allrefsBs (\phi \implies (\psi \implies (\phi \land \psi))) & \text{{\bf NecR}}\\
        \proves \allrefsBs (\phi \implies (\psi \implies (\phi \land \psi))) \implies \\\quad(\allrefsBs \phi \implies \allrefsBs (\psi \implies (\phi \land \psi))) & \text{{\bf R}}\\
        \proves \allrefsBs \phi \implies \allrefsBs (\psi \implies (\phi \land \psi)) & \text{{\bf MP}}\\
        \proves \allrefsBs (\psi \implies (\phi \land \psi)) \implies (\allrefsBs \psi \implies \allrefsBs (\phi \land \psi)) & \text{{\bf R}}\\
        \proves \allrefsBs \phi \implies (\allrefsBs \psi \implies \allrefsBs (\phi \land \psi)) & \text{{\bf P}}\\
        \proves (\allrefsBs \phi \land \allrefsBs \psi) \implies \allrefsBs (\phi \land \psi)) & \text{{\bf P}}
    \end{array}
    $$

    \item[(\ref{rml-k-or})]
    We show that $\proves \somerefsBs (\phi \lor \psi) \iff (\somerefsBs \phi \lor \somerefsBs \psi)$.

    Given (\ref{rml-k-and}) above we have:
    $$
    \begin{array}{ll}
        \proves \allrefsBs (\lnot \phi \land \lnot \psi) \iff (\allrefsBs \lnot \phi \land \allrefsBs \lnot \psi) & \text{(\ref{rml-k-and})}\\
        \proves \lnot \allrefsBs (\lnot \phi \land \lnot \psi) \iff \lnot (\allrefsBs \lnot \phi \land \allrefsBs \lnot \psi) & \text{{\bf P}}\\
        \proves \lnot \allrefsBs \lnot (\phi \lor \psi) \iff (\lnot \allrefsBs \lnot \phi \lor \lnot \allrefsBs \lnot \psi) & \text{{\bf P}}\\
        \proves \somerefsBs (\phi \lor \psi) \iff (\somerefsBs \phi \lor \somerefsBs \psi) & \text{Defn. of $\somerefsBs$}
    \end{array}
    $$

    \item[(\ref{rml-k-d-and})]
    We show that $\proves \somerefsBs (\phi \land \psi) \implies (\somerefsBs \phi \land \somerefsBs \psi)$.
    $$
    \begin{array}{ll}
        \proves \lnot \phi \implies (\lnot \phi \lor \lnot \psi) & {\bf P}\\
        \proves \allrefsBs (\lnot \phi \implies (\lnot \phi \lor \lnot \psi)) & {\bf NecR}\\
        \proves \allrefsBs (\lnot \phi \implies (\lnot \phi \lor \lnot \psi)) \implies (\allrefsBs \lnot \phi \implies \allrefsBs (\lnot \phi \lor \lnot \psi)) & {\bf R}\\
        \proves \allrefsBs \lnot \phi \implies \allrefsBs (\lnot \phi \lor \lnot \psi) & {\bf MP}\\
        \proves \lnot \somerefsBs \phi \implies \lnot \somerefsBs \lnot (\lnot \phi \lor \lnot \psi) & \text{Defn. of $\somerefsBs$}\\
        \proves \somerefsBs \lnot (\lnot \phi \lor \lnot \psi) \implies \somerefsBs \phi  & {\bf P}\\
        \proves \somerefsBs (\phi \land \psi) \implies \somerefsBs \phi  & {\bf P}\\
        \proves \lnot \psi \implies (\lnot \phi \lor \lnot \psi) & {\bf P}\\
        \proves \allrefsBs (\lnot \psi \implies (\lnot \phi \lor \lnot \psi)) & {\bf NecR}\\
        \proves \allrefsBs (\lnot \psi \implies (\lnot \phi \lor \lnot \psi)) \implies (\allrefsBs \lnot \psi \implies \allrefsBs (\lnot \phi \lor \lnot \psi)) & {\bf R}\\
        \proves \allrefsBs \lnot \psi \implies \allrefsBs (\lnot \phi \lor \lnot \psi) & {\bf MP}\\
        \proves \lnot \somerefsBs \psi \implies \lnot \somerefsBs \lnot (\lnot \phi \lor \lnot \psi) & \text{Defn. of $\somerefsBs$}\\
        \proves \somerefsBs \lnot (\lnot \phi \lor \lnot \psi) \implies \somerefsBs \psi  & {\bf P}\\
        \proves \somerefsBs (\phi \land \psi) \implies \somerefsBs \psi  & {\bf P}\\
        \proves \somerefsBs (\phi \land \psi) \implies (\phi \land \somerefsBs \psi) & {\bf P}
    \end{array}
    $$

    \item[(\ref{rml-k-db-and})]
    We show that $\proves (\pi \land \somerefsBs \psi) \iff \somerefsBs (\pi \land \psi)$.

    For the left-to-right direction:
    $$
    \begin{array}{ll}
        \proves \somerefsBs (\pi \land \psi) \implies (\somerefsBs \pi \land \somerefsBs \psi) & (\ref{rml-k-d-and})\\
        \proves \allrefsBs \lnot \pi \iff \lnot \pi & {\bf RP}\\
        \proves \lnot \allrefsBs \lnot \pi \iff \pi & {\bf P}\\
        \proves \somerefsBs \pi \iff \pi & \text{Defn. of $\somerefsBs$}\\
        \proves \somerefsBs (\pi \land \psi) \implies (\pi \land \somerefsBs \psi) & {\bf P}
    \end{array}
    $$

    For the right-to-left direction:
    $$
    \begin{array}{ll}
        \proves (\allrefsBs \pi \land \somerefsBs \psi) \iff (\allrefsBs \pi \land \lnot \allrefsBs \lnot \psi)& \text{Defn. of $\somerefsBs$}\\
        \proves (\allrefsBs \pi \land \somerefsBs \psi) \iff \lnot (\allrefsBs \pi \implies \allrefsBs \lnot \psi)& {\bf P}\\
        \proves \allrefsBs (\pi \implies \lnot \psi) \implies (\allrefsBs \pi \implies \allrefsBs \lnot \psi)& {\bf R}\\
        \proves \lnot (\allrefsBs \pi \implies \allrefsBs \lnot \psi) \implies \lnot \allrefsBs (\pi \implies \lnot \psi)& {\bf P}\\
        \proves (\allrefsBs \pi \land \lnot \allrefsBs \lnot \psi) \implies \lnot \allrefsBs (\pi \implies \lnot \psi)& {\bf P}\\
        \proves (\allrefsBs \pi \land \somerefsBs \psi) \implies \somerefsBs \lnot (\pi \implies \lnot \psi)& \text{Defn. of $\somerefsBs$}\\
        \proves (\allrefsBs \pi \land \somerefsBs \psi) \implies \somerefsBs (\pi \land \psi)& \text{Defn. of $\somerefsBs$}\\
        \proves \allrefsBs \pi \iff \pi & {\bf RP}\\
        \proves (\pi \land \somerefsBs \psi) \implies \somerefsBs (\pi \land \psi)& {\bf P}\\
    \end{array}
    $$

    \item[(\ref{rml-k-cover})]
    We show that:
    \begin{align*}
        \proves & \displaystyle \somerefsBs (\pi \land \bigwedge_{\agentC \in \agentsC} \coverC \Gamma_\agentC) \iff \\&\quad
            (
            \pi \land
            \bigwedge_{\agentC \in \agentsC \cap \agentsB} \bigwedge_{\gamma \in \Gamma_\agentA} \possibleA \somerefsBs \gamma \land
            \bigwedge_{\agentC \in \agentsC \setminus \agentsB} \coversC \{\somerefsBs \gamma \mid \gamma \in \Gamma_\agentC\} 
            )
    \end{align*}

    By (\ref{rml-k-db-and}) we have:
    \begin{align*}
    \proves & \displaystyle \somerefsBs (\pi \land \bigwedge_{\agentC \in \agentsC} \coverC \Gamma_\agentC) \iff 
            (
            \pi \land
            \somerefsBs (\bigwedge_{\agentC \in \agentsC} \coverC \Gamma_\agentC)
            )
        \end{align*}

    By {\bf RDist} we have:
    \begin{align*}
    \proves & \displaystyle \somerefsBs (\pi \land \bigwedge_{\agentC \in \agentsC} \coverC \Gamma_\agentC) \iff 
            (
            \pi \land
            \bigwedge_{\agentC \in \agentsC} \somerefsBs \coverC \Gamma_\agentC
            )
    \end{align*}

    By {\bf RK} we have:
    \begin{align*}
        \proves & \displaystyle \somerefsBs (\pi \land \bigwedge_{\agentC \in \agentsC} \coverC \Gamma_\agentC) \iff \\&\quad
            (
            \pi \land
            \bigwedge_{\agentC \in \agentsC \cap \agentsB} \bigwedge_{\gamma \in \Gamma_\agentA} \possibleA \somerefsBs \gamma \land
            \bigwedge_{\agentC \in \agentsC \setminus \agentsB} \somerefsBs \coverC \Gamma_\agentC
            )
    \end{align*}

    Finally by {\bf RComm} we have:
    \begin{align*}
        \proves & \displaystyle \somerefsBs (\pi \land \bigwedge_{\agentC \in \agentsC} \coverC \Gamma_\agentC) \iff \\&\quad
            (
            \pi \land
            \bigwedge_{\agentC \in \agentsC \cap \agentsB} \bigwedge_{\gamma \in \Gamma_\agentA} \possibleA \somerefsBs \gamma \land
            \bigwedge_{\agentC \in \agentsC \setminus \agentsB} \coversC \{\somerefsBs \gamma \mid \gamma \in \Gamma_\agentC\} 
            )
    \end{align*}
    \end{description}
\end{proof}

We can now clearly recognise that the equivalences (\ref{rml-k-or}) and (\ref{rml-k-cover}) are reduction axioms that can be used to push refinement quantifiers past propositional connectives and modalities in formulas in disjunctive normal form.
These equivalences form the basis of our provably correct translation from \langRml{} to \langMl{}.

Before we give our provably correct translation we give two lemmas.
First we note that every \axiomK{} theorem is an \axiomRmlK{} theorem.

\begin{lemma}\label{rml-k-ml-provability}
Let $\phi \in \langMl$ be a modal formula.
If $\proves_\axiomK \phi$ then $\proves_\axiomRmlK \phi$.
\end{lemma}

\begin{proof}
Suppose that $\proves_\axiomK \phi$.
Then there exists a proof of $\proves_\axiomK \phi$ using the axioms and rules of \axiomK{}.
As \axiomRmlK{} includes all of the axioms and rules of \axiomK{} then the proof of $\proves_\axiomK \phi$ using the axioms and rules of \axiomK{} is also a proof of $\proves_\axiomRmlK{} \phi$ using the axioms and rules of \axiomRmlK{}.
Therefore $\proves_\axiomRmlK{} \phi$.
\end{proof}

Secondly we show that \axiomRmlK{} is closed under substitution of equivalents.

\begin{lemma}\label{rml-k-substitution-equivalents}
Let $\phi, \psi, \chi \in \langRml$ be formulas and let $\atomP \in \atoms$ be a propositional atom.
If $\proves \psi \iff \chi$ then $\proves \phi[\psi\backslash\atomP] \iff \phi[\chi\backslash\atomP]$.
\end{lemma}

\begin{proof}
We proceed by induction on the structure of $\phi$.

Suppose that $\phi = \atomP$.
Then $\atomP[\psi\backslash\atomP] = \psi$ and $\atomP[\chi\backslash\atomP] = \chi$ and by hypothesis $\proves \psi \iff \chi$ so by {\bf P} we have $\proves \atomP[\psi\backslash\atomP] \iff \atomP[\chi\backslash\atomP]$.

Suppose that $\phi = \atomQ$ where $\atomQ \in \atoms$ and $\atomQ \neq \atomP$.
Then $\atomQ[\psi\backslash\atomP] = \atomQ$ and $\atomQ[\chi\backslash\atomP] = \atomQ$ so by {\bf P} we have $\proves \atomQ[\psi\backslash\atomP] \iff \atomQ[\chi\backslash\atomP]$.

Suppose that $\phi = \lnot \alpha$.
By the induction hypothesis $\proves \alpha[\psi\backslash\atomP] \iff \alpha[\chi\backslash\atomP]$.
Then by {\bf P} we have $\proves \lnot \alpha[\psi\backslash\atomP] \iff \lnot \alpha[\chi\backslash\atomP]$.

Suppose that $\phi = \alpha \land \beta$.
By the induction hypothesis $\proves \alpha[\psi\backslash\atomP] \iff \alpha[\chi\backslash\atomP]$ and $\proves \beta[\psi\backslash\atomP] \iff \beta[\chi\backslash\atomP]$.
Then by {\bf P} we have $\proves (\alpha[\psi\backslash\atomP] \land \beta[\psi\backslash\atomP]) \iff (\alpha[\chi\backslash\atomP] \land \beta[\chi\backslash\atomP])$.

Suppose that $\phi = \necessaryA \alpha$.
By the induction hypothesis $\proves \alpha[\psi\backslash\atomP] \iff \alpha[\chi\backslash\atomP]$.
By {\bf NecK} we have $\proves \necessaryA (\alpha[\psi\backslash\atomP] \implies \alpha[\chi\backslash\atomP])$ and by {\bf K} we have $\proves \necessaryA \alpha[\psi\backslash\atomP] \implies \necessaryA \alpha[\chi\backslash\atomP]$.
Likewise by {\bf NecK} we have $\proves \necessaryA (\alpha[\chi\backslash\atomP] \implies \alpha[\psi\backslash\atomP])$ and by {\bf K} we have $\proves \necessaryA \alpha[\chi\backslash\atomP] \implies \necessaryA \alpha[\psi\backslash\atomP]$.
Then by {\bf P} we have $\proves \necessaryA \alpha[\psi\backslash\atomP] \iff \necessaryA \alpha[\chi\backslash\atomP]$.

Suppose that $\phi = \allrefsBs \alpha$.
By the induction hypothesis $\proves \alpha[\psi\backslash\atomP] \iff \alpha[\chi\backslash\atomP]$.
By {\bf NecR} we have $\proves \allrefsBs (\alpha[\psi\backslash\atomP] \implies \alpha[\chi\backslash\atomP])$ and by {\bf R} we have $\proves \allrefsBs \alpha[\psi\backslash\atomP] \implies \allrefsBs \alpha[\chi\backslash\atomP]$.
Likewise by {\bf NecR} we have $\proves \allrefsBs (\alpha[\chi\backslash\atomP] \implies \alpha[\psi\backslash\atomP])$ and by {\bf R} we have $\proves \allrefsBs \alpha[\chi\backslash\atomP] \implies \allrefsBs \alpha[\psi\backslash\atomP]$.
Then by {\bf P} we have $\proves \allrefsBs \alpha[\psi\backslash\atomP] \iff \allrefsBs \alpha[\chi\backslash\atomP]$.
\end{proof}

Now that we have shown the equivalences of Lemma~\ref{rml-k-theorems}, and that \axiomRmlK{} is closed under substitution of equivalents we can give an alternative version of the proof in Example~\ref{rml-k-example-derivation}.

\begin{example}
We show that $\proves \somerefsA (\necessaryA \atomP \land \lnot \necessaryB \atomP) \iff \possibleB \lnot \atomP$ using the axiomatisation \axiomRmlK{}.
$$
\begin{array}{ll}
    \proves (\necessaryA \atomP \land \lnot \necessaryB \atomP) \iff ((\coversA \{\atomP\} \lor \coversA \emptyset) \land \coversB \{\lnot \atomP, \top\}) & \text{Defn. of $\covers$}\\
    \proves (\necessaryA \atomP \land \lnot \necessaryB \atomP) \iff ((\coversA \{\atomP\} \land \coversB \{\lnot \atomP, \top\}) \lor (\coversA \emptyset \land \coversB \{\lnot \atomP, \top\})) & \text{Defn. of $\covers$}\\
    \proves \somerefsA (\necessaryA \atomP \land \lnot \necessaryB \atomP) \iff \somerefsA ((\coversA \{\atomP\} \land \coversB \{\lnot \atomP, \top\}) \lor (\coversA \emptyset \land \coversB \{\lnot \atomP, \top\})) & \text{Substitution of equivalents}\\
    \proves \somerefsA (\necessaryA \atomP \land \lnot \necessaryB \atomP) \iff (\somerefsA (\coversA \{\atomP\} \land \coversB \{\lnot \atomP, \top\}) \lor \somerefsA (\coversA \emptyset \land \coversB \{\lnot \atomP, \top\})) & \text{(\ref{rml-k-or}) from Lemma~\ref{rml-k-theorems}}\\
    \proves \somerefsA (\necessaryA \atomP \land \lnot \necessaryB \atomP) \iff ((\possibleA \somerefsA \atomP \land \possibleB \somerefsA \lnot \atomP \land \possibleB \somerefsA \top) \lor (\top \land \possibleB \somerefsA \lnot \atomP \land \possibleB \somerefsA \top)) & \text{(\ref{rml-k-cover}) from Lemma~\ref{rml-k-theorems}}\\
    \proves \somerefsA (\necessaryA \atomP \land \lnot \necessaryB \atomP) \iff ((\possibleA \atomP \land \possibleB \lnot \atomP \land \possibleB \top) \lor (\top \land \possibleB \lnot \atomP \land \possibleB \top)) & {\bf RP}\\
    \proves \somerefsA (\necessaryA \atomP \land \lnot \necessaryB \atomP) \iff (\possibleB \lnot \atomP \land \possibleB \top) & {\bf P}\\
    \proves \somerefsA (\necessaryA \atomP \land \lnot \necessaryB \atomP) \iff \possibleB \lnot \atomP & \text{Modal reasoning}\\
\end{array}
$$
\end{example}

We now show that the reduction axioms of \logicRmlK{} admit a provably correct translation from \langRml{} to \langMl{}.
The example above demonstrates the general strategy behind our provably correct translation: convert to disjunctive normal form, then use the provable equivalences from Lemma~\ref{rml-k-theorems} to push refinement quantifiers past modalities and connectives until {\bf RP} may be applied to remove the refinement quantifiers altogether.

\begin{lemma}\label{rml-k-ml-equivalent}
Every refinement modal formula is provably equivalent to a modal formula using the axiomatisation \axiomRmlK{}.
\end{lemma}

\begin{proof}
Let $\phi \in \langRml$ be a refinement modal formula.
Assume without loss of generality that all $\allrefsBs$~operators are expressed instead as $\somerefsBs$~operators.
We show by induction on the number of $\somerefsBs$~operators in $\phi$ ($\somerefsBs$~operators for any $\agentsB \subseteq \agents$) that $\phi$ is provably equivalent to a modal formula.

Suppose that $\phi$ has no $\somerefsBs$~operators.
Then $\phi$ is already a modal formula.

Suppose that $\phi$ has $n + 1$ $\somerefsBs$~operators.
Let $\somerefsBs \psi$ be a subformula of $\phi$ such that $\psi \in \langMl$ is a modal formula.
By Lemma~\ref{dnf-equivalent} there exists $\psi' \in \langMl$ in disjunctive normal form such that $\entails_\logicK{} \psi \iff \psi'$ under the semantics of the logic \logicK{}.
By the completeness of the logic \logicK{}, we have that $\proves_\axiomK{} \psi \iff \psi'$ using the axiomatisation \axiomK{}.
By Lemma~\ref{rml-k-ml-provability} we have that $\proves_\axiomRmlK \psi \iff \psi'$ using the axiomatisation \axiomRmlK{}.
By substitution of equivalents we have $\proves_\axiomRmlK \somerefsBs \psi \iff \somerefsBs \psi'$.
We show by induction on the structure of $\psi'$ that $\somerefsBs \psi'$ is provably equivalent to a modal formula.

Suppose that $\psi' = \alpha \lor \beta$.
By Lemma~\ref{rml-k-theorems} we have that $\proves_\axiomRmlK \somerefsBs (\alpha \lor \beta) \iff (\somerefsBs \alpha \lor \somerefsBs \beta)$.
By the induction hypothesis there exists $\alpha', \beta' \in \langMl$ such that $\proves_\axiomRmlK \somerefsBs \alpha \iff \alpha'$ and $\proves_\axiomRmlK \somerefsBs \beta \iff \beta'$.
Therefore by substitution of equivalents we have that $\proves_\axiomRmlK \somerefsBs (\alpha \lor \beta) \iff (\alpha' \lor \beta')$, where $\alpha' \lor \beta' \in \langMl$.

Suppose that $\psi' = \pi \land \bigwedge_{\agentC \in \agentsC} \coverC \Gamma_\agentC$.
By Lemma~\ref{rml-k-theorems} we have the equivalence (\ref{rml-k-cover}).
By the induction hypothesis for every $\agentC \in \agentsC$, $\gamma \in \Gamma_\agentC$ there exists $\gamma' \in \langMl$ such that $\proves_\axiomRmlK \somerefsBs \gamma \iff \gamma'$.
Therefore by substitution of equivalents we may replace each occurrence of $\somerefsBs \gamma$ with the corresponding $\gamma'$ to yield an equivalent modal formula. 

Therefore $\somerefsBs \psi'$ is provably equivalent to a modal formula $\psi''$.
By substitution of equivalents, substiting $\psi''$ for $\somerefsBs \psi$ in $\phi$ yields a provably equivalent formula $\phi'$ with $n$ $\somerefsBs$~operators.
By the induction hypothesis $\phi'$ is provably equivalent to a modal formula and therefore $\phi$ is provably equivalent to a modal formula.
\end{proof}

Given the provably correct translation we have that \axiomRmlK{} is sound and complete.

\begin{theorem}\label{rml-k-sound-complete}
The axiomatisation \axiomRmlK{} is sound and strongly complete with respect to the semantics of the logic \logicRmlK{}.
\end{theorem}

\begin{proof}
Soundness is shown in Lemma~\ref{rml-k-sound}.

Let $\Phi \subseteq \langRml$ be a set of formulas consistent according to the axiomatisation \axiomRmlK{}.
By Lemma~\ref{rml-k-ml-equivalent}, for every $\phi \in \Phi$ there exists $\phi' \in \langMl$ such that $\proves_\axiomRmlK \phi \iff \phi'$.
Let $\Phi' = \{\phi' \mid \phi \in \Phi\}$.
Then $\Phi'$ is consistent according to the axiomatisation \axiomRmlK{}.
As \axiomRmlK{} contains all of the axioms and rules of \axiomK{} then $\Phi'$ is also consistent according to the axiomatisation \axiomK{}.
By the strong completeness of \axiomK{} it follows that $\Phi'$ is satisfiable with respect to the semantics of the logic \logicK{}.
Suppose that $\kPModel{\kStateS} \in \classK$ is a Kripke model such that $\kPModel{\kStateS} \entails_\logicK{} \Phi'$.
Then $\kPModel{\kStateS} \entails_\logicRmlK{} \Phi'$.
Let $\phi \in \Phi$.
Then $\kPModel{\kStateS} \entails_\logicRmlK{} \phi'$, and as $\proves_\axiomRmlK \phi \iff \phi'$ by the soundness of the axiomatisation \axiomRmlK{} it follows that $\entails_\logicRmlK{} \phi \iff \phi'$ and $\kPModel{\kStateS} \entails_\logicRmlK{} \phi$.
Therefore $\kPModel{\kStateS} \entails_\logicRmlK{} \Phi$.
Therefore $\Phi$ is satisfiable.
\end{proof}

The provably correct translation also obviously implies that \logicRmlK{} is expressively equivalent to \logicK{}.

\begin{corollary}
The logic \logicRmlK{} is expressively equivalent to the logic \logicK{}.
\end{corollary}

Finally, as \logicRmlK{} is expressively equivalent to \logicK{}, and \logicK{} is compact and decidable, we also have that \logicRmlK{} is compact and decidable.

\begin{corollary}
The logic \logicRmlK{} is compact.
\end{corollary}

\begin{corollary}
The satisfiability problem for the logic \logicRmlK{} is decidable.
\end{corollary}
