\chapter{Modal $\mu$-calculus}\label{mu}

In this appendix we define the syntax and semantics of the modal $\mu$-calculus.
The modal $\mu$-calculus extends modal logic with a least fixed point operator $\mu$ and a greatest fixed point operator $\nu$.
We use the modal $\mu$-calculus in Chapter~\ref{rml-k4} to show that the refinement modal logic \logicRmlKF{} is strictly less expressive than the modal $\mu$-calculus \logicMuKF{}.
The proof of this result relies only on the semantics of the modal $\mu$-calculus.
For an introductory text on the modal $\mu$-calculus we direct the reader to the book by Arnold and Niwinski~\cite{arnold:2001}.

Let $\variables$ be a non-empty, countable set of variables.

\begin{definition}[Language of modal $\mu$-calculus]
The {\em language of modal $\mu$-calculus}, \langMu{}, is inductively defined as:
$$
\phi ::= 
    \atomP \mid
    \varX \mid
    \lnot \phi \mid
    (\phi \land \phi) \mid
    \necessaryA \phi \mid
    \lfpX \phi
$$
where $\atomP \in \atoms$, $\agentA \in \agents$ and $\varX \in \variables$, and in $\lfpX \phi$ every free occurrence of $\varX$ in $\phi$ occurs positively (i.e. within the scope of an even number of negations).
\end{definition}

We use all of the standard abbreviations from modal logic, in addition to the abbreviation $\gfpX \phi ::= \lnot \lfpX \lnot \phi$.

We now define the semantics of the modal $\mu$-calculus.
The semantics are defined in terms of a parameterised class of Kripke frames, \classC{}, which could stand for \classK{}, \classKF{}, \classKFF{}, etc. or for any other class of Kripke frames so defined.

\begin{definition}[Semantics of modal $\mu$-calculus]\label{mu-semantics}
Let $\classC$ be a class of Kripke frames, let $\phi \in \langMu$, let $\kPModelAndTuple{\kStateS} \in \classC$ be a pointed Kripke model and let $\kAssignment : \varX \to \powerset(\kStates)$ be a function from variables to sets of states.
The set of states $\interpretation[\kAssignment]{\phi} \subseteq \kStates$ where $\phi$ is satisfied with respect to an assignment $\kAssignment$ is defined inductively as follows:
\begin{eqnarray*}
    \interpretation[\kAssignment]{\atomP} &=& \kValuation(\atomP)\\
    \interpretation[\kAssignment]{\varX} &=& \kAssignment(\varX)\\
    \interpretation[\kAssignment]{\phi \land \psi} &=& \interpretation[\kAssignment]{\phi} \cap \interpretation[\kAssignment]{\psi}\\
    \interpretation[\kAssignment]{\necessaryA \phi} &=& \{ \kStateS \in \kStates \mid \kStateS \kAccessibility{\agentA} \subseteq \interpretation[\kAssignment]{\phi} \}\\
    \interpretation[\kAssignment]{\lfp{\varX} \phi} &=& \bigcap \{ \kStatesT \subseteq \kStates \mid \interpretation[\kAssignmentSubstitute{\kStatesT}{\varX}]{\phi} \subseteq \kStatesT \}
\end{eqnarray*}
where $\kAssignmentSubstitute{\kStatesT}{\varX}$ is the assignment such that $\kAssignmentSubstitute{\kStatesT}{\varX}(\varX) = \kStatesT$ and $\kAssignmentSubstitute{\kStatesT}{\varX}(\varY) = \kAssignment(\varY)$ for $\varY \neq \varX$. 
\end{definition}
