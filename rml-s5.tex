\chapter{Refinement modal logic: \classS{}}\label{rml-s5}

In this chapter we consider results specific to the logic \logicRmlS{} in the setting of \classS{}.
As in previous chapters we present a sound and complete axiomatisation, a provably correct translation from \langRml{} to \langMl{}, and expressive equivalence, compactness and decidability results.
As noted previously, the logics \logicRmlK{}, \logicRmlKFF{}, and \logicKD{} are not sublogics of \logicRmlS{}, so our previous results in \logicRmlK{}, \logicRmlKFF{}, and \logicKD{} do not all apply in this setting.
In particular the axioms {\bf RK}, {\bf RComm}, and {\bf RDist}, and the corresponding axioms from \logicRmlKFF{} and \logicRmlKD{} are not sound in \logicRmlS{}, so once again we must find replacement axioms.

In the following sections we provide a sound and complete axiomatisation for \logicRmlS{}.
In Section~\ref{rml-s5-axiomatisation} we provide the axiomatisation for \logicRmlS{}, which feature modified versions of the axioms {\bf RK}, {\bf RComm}, and {\bf RDist}.
In Section~\ref{rml-s5-soundness} we show that the axiomatisation is sound.
In contrast to \logicRmlK{}, \logicRmlKFF{}, and \logicRmlKD{}, we must show that the Kripke models that are constructed are \classS{} Kripke models.
This additional requirement accounts for the differences in the axioms compared to \logicRmlK{}, \logicRmlKFF{}, and \logicRmlKD{}.
In Section~\ref{rml-s5-completeness} we show that the axiomatisation is complete via a provably correct translation from \langRml{} to \langMl{}.
In contrast to \logicRmlK{}, \logicRmlKFF{}, and \logicRmlKD{}, where disjunctive normal forms were sufficient for the reduction axioms to be applicable, in \logicRmlS{} we must use an even more restricted form to account for the relatively complex syntactic restrictions in the axiomatisation.

\section{Axiomatisation}\label{rml-s5-axiomatisation}

In this section we present the axiomatisation \axiomRmlS{} for the logic \logicRmlS{}.
Similar to the axiomatisations \axiomRmlKFF{} and \axiomRmlKD{}, presented in the previous chapter, the axiomatisation \axiomRmlS{} is a modification of the axiomatisation \axiomRmlK{}.
As in \axiomRmlK{} the cover operator features prominently in this axiomatisation.
We discuss and justify the use of the cover operator in Chapter~\ref{rml-k}, where we introduced the axiomatisation \axiomRmlK{}.
The cover operator serves as a convenient notation for a conjunction of modalities that also restricts conjunctions of modalities to cases where the axioms are sound.
However as in \axiomRmlKFF{} and \axiomRmlKD{} we find that this restriction on notation is not sufficient to ensure that the axioms {\bf RK}, {\bf RComm}, and {\bf RDist} are sound in \logicRmlS{}.

Similar to \logicRmlKFF{} and \logicRmlKD{} we know a priori that some of the rules and axioms of \axiomRmlK{} must not be sound in \logicRmlS{}, as we noted in Proposition~\ref{rml-not-sublogics} that \logicRmlK{} is not a sublogic of \logicRmlS{}.
It is a simple matter to show that the axioms and rules of \axiomS{} are sound for \logicRmlS{}, and the axioms and rules {\bf R}, {\bf RP}, and {\bf NecR} are sound for \logicRmlS{} as they were shown to be sound for all variants of \logicRml{} in Proposition~\ref{rml-validities}.
Hence some or all of {\bf RK}, {\bf RComm}, and {\bf RDist} must not be sound for \logicRmlS{}. 

In Chapter~\ref{rml-kd45} we demonstrated that the \axiomRmlK{} axiom {\bf RK} is not sound in the logics \logicRmlKFF{} and \logicRmlKD{} essentially because of the requirement that refinements be transitive and Euclidean.
In particular we showed that using the axiom {\bf RK} we can derive that $\proves \possibleA (\lnot \atomP \land \possibleA \atomP) \implies \somerefsA (\possibleA \possibleA \atomP \land \lnot \possibleA \atomP)$, but given the modal axiom {\bf 4}, corresponding to transitivity, we can derive the negation.
Likewise using the \axiomRmlK{} axiom {\bf RK} we can derive that $\proves \possibleA \atomP \implies \somerefsA (\possibleA \atomP \land \lnot \necessaryA \possibleA \atomP)$, but given the modal axiom {\bf 5}, corresponding to Euclideaness, we can derive the negation.
These examples also show that the axiom {\bf RK} is not sound in the logic \logicRmlS{}.

In Chapter~\ref{rml-kd45} we modified the axioms {\bf RK}, {\bf RComm}, and {\bf RDist} to be sound in the logics \logicRmlKFF{} and \logicRmlKD{} by placing a syntactic restriction on the axioms, prohibiting the direct nesting of modalities belonging to the same agent.
This syntactic restriction is not sufficient to ensure that the axioms are sound in \logicRmlS{}.
In particular, in \logicRmlKFF{} and \logicRmlKD{} refinements need not be reflexive, so we have $\entails (\lnot \atomP \land \possibleA \atomP) \implies \somerefsA (\lnot \atomP \land \necessaryA \atomP)$, but in \logicRmlS{} all refinements must be reflexive, so we have $\entails \allrefsA \lnot (\lnot \atomP \land \necessaryA \atomP)$ and hence $\entails \lnot ((\lnot \atomP \land \possibleA \atomP) \implies \somerefsA (\lnot \atomP \land \necessaryA \atomP))$.
We can show how this first validity could be derived using the axiomatisations \axiomRmlKFF{} or \axiomRmlKD{}.
$$
\begin{array}{ll}
    \proves \somerefsA (\lnot \atomP \land \necessaryA \atomP) \iff \somerefsA (\lnot \atomP \land \coversA \{\atomP\}) & \text{Defn. of $\coversA$}\\
    \proves \somerefsA (\lnot \atomP \land \necessaryA \atomP) \iff (\lnot \atomP \land \somerefsA \coversA \{\atomP\}) & \text{Lemma~\ref{rml-kd45-theorems}}\\
    \proves \somerefsA (\lnot \atomP \land \necessaryA \atomP) \iff (\lnot \atomP \land \possibleA \somerefsA \atomP) & \text{{\bf RK45} / {\bf RKD45}}\\
    \proves \somerefsA (\lnot \atomP \land \necessaryA \atomP) \iff (\lnot \atomP \land \possibleA \atomP) & {\bf RP}\\
    \proves (\lnot \atomP \land \possibleA \atomP) \implies \somerefsA (\lnot \atomP \land \necessaryA \atomP) & {\bf P}
\end{array}
$$

However in \logicRmlS{} we have reflexivity, represented by the modal axiom {\bf T}, and given this axiom we can very easily show that $\proves \lnot \somerefsA (\lnot \atomP \land \necessaryA \atomP)$.
We provide an informal proof.
$$
\begin{array}{ll}
    \proves \necessaryA \atomP \implies \atomP & {\bf T}\\
    \proves (\lnot \atomP \land \necessaryA \atomP) \iff (\atomP \land \lnot \atomP \land \necessaryA \atomP) & {\bf P}\\
    \proves \lnot (\lnot \atomP \land \necessaryA \atomP) & {\bf P}\\
    \proves \allrefsA \lnot (\lnot \atomP \land \necessaryA \atomP) & {\bf NecR}\\
    \proves \lnot \somerefsA (\lnot \atomP \land \necessaryA \atomP) & \text{Defn. of $\somerefsA$}
\end{array}
$$
The formula $\lnot \atomP \land \possibleA \atomP$ is satisfiable in \logicRmlS{} as it is satisfiable in \logicS{}, and the semantics of these logics agree on all modal formulas.
So for a sound axiomatisation of \logicRmlS{} we must have that $\nproves \lnot (\lnot \atomP \land \possibleA \atomP)$ and therefore $\proves \lnot ((\lnot \atomP \land \possibleA \atomP) \implies \somerefsA (\lnot \atomP \land \necessaryA \atomP))$.
Therefore the derivation of $\proves (\lnot \atomP \land \possibleA \atomP) \implies \somerefsA (\lnot \atomP \land \necessaryA \atomP)$ in \axiomRmlKFF{} and \axiomRmlKD{} above is not sound reasoning for \logicRmlS{}.
We previously noted that the axioms and rules of axioms and rules of \axiomS{} and the axioms and rules {\bf R}, {\bf RP}, and {\bf NecR} are sound for \logicRmlS{}, so the flaw in the derivation must be the use of the axioms {\bf RK45} and {\bf RKD45}, so these axioms are not sound in \logicRmlS{}.

Above we saw that $\proves \lnot (\lnot \atomP \land \necessaryA \atomP)$, and hence $\proves \lnot \somerefsA (\lnot \atomP \land \necessaryA \atomP)$.
This becomes obvious once we convert $\lnot \atomP \land \necessaryA \atomP$ to the equivalent $\atomP \land \lnot \atomP \land \necessaryA \atomP$.
A similar observation was made in Chapter~\ref{rml-kd45} where we converted a formula to a different syntactic form where certain consequences of transitivity and Euclideaness are explicitly represented in the formula, making contradictions due to transitivity and Euclideaness more obvious, and resulting in the {\bf RK} axiom behaving as desired in \logicRmlKFF{} and \logicRmlKD{}.
This idea formed the basis for the modified versions of {\bf RK} used in the axiomatisations \axiomRmlKFF{} and \axiomRmlKD{}.
Perhaps converting formulas to a different syntactic form, where certain consequences of reflexivity are explicitly represented in the formula, would result in the {\bf RK} axiom behaving as desired in \logicRmlS{}.  

Recall that in Chapter~\ref{rml-kd45} we gave an example of a derivation showing that the axiom {\bf RK} is unsound in \logicRmlKFF{} and \logicRmlKD{}.
The problem occurred because we had a set of formulas, $\{\lnot \atomP \land \possibleA \atomP, \lnot \atomP\}$ that was contradictory when taken together in a cover operator, as in $\coverA \{\lnot \atomP \land \possibleA \atomP, \lnot \atomP\}$, but when considered individually each formula is satisfiable in a refinement of a successor, as in $\possibleA \somerefsA (\lnot \atomP \land \possibleA \atomP) \land \possibleA \somerefsA \lnot \atomP$.
The solution was to rewrite the formula $\coversA \{\lnot \atomP \land \possibleA \atomP, \lnot \atomP\}$ into the equivalent $\coversA \{\atomP \land \lnot \atomP, \lnot \atomP\}$ where we now have a contradiction if we consider each formula individually, as in $\possibleA \somerefsA (\atomP \land \lnot \atomP) \land \possibleA \somerefsA \lnot \atomP$.
Rewriting the formula in this way explicitly represents the interaction due to transitivity and Euclideaness between the formulas in the cover operator.
In the rewritten formula there is no interaction due to transitivity and Euclideaness between the formulas in the cover operator, because the formulas in the cover operator do not feature $\agentA$-modalities at the top level.
Therefore any interaction that was implied in the original formula becomes explicit in the rewritten formula.
This makes the contradiction between the formulas due to transitivity and Euclideaness more obvious, and means that considering the formulas individually as is done the {\bf RK} axiom does not result in the contradiction disappearing.

This approach of removing any interaction between the formulas in a cover operator does not work in the presence of reflexivity.
For example in $\coversA \{\necessaryB \necessaryA \atomP, \necessaryB \necessaryA \lnot \atomP\}$ although there are no directly nested modalities belonging to the same agent, such modalities are implied through reflexivity, as $\proves \necessaryB \necessaryA \atomP \implies \necessaryA \atomP$ and $\proves \necessaryB \necessaryA \lnot \atomP \implies \necessaryA \lnot \atomP$.
There is no way of rewriting this formula into a single cover operator where the formulas in the cover operator do not interact through transitivity and Euclideaness. 
Any equivalent formula must imply that $\possibleA \necessaryB \necessaryA \atomP$, meaning that the cover operator must contain a formula that implies that $\necessaryB \necessaryA \atomP$, and through reflexivity this formula would still imply the directly nested modality $\necessaryA \atomP$.
So we use a different strategy.
Rather than removing any interaction between the formulas in a cover operator, we aim to make the interaction between the formulas explicit.
For example, $\necessaryB \necessaryA \atomP$ is equivalent to $\atomP \land \necessaryA \atomP \land \necessaryB \necessaryA \atomP$, and similarly $\necessaryB \necessaryA \lnot \atomP$ is equivalent to $\lnot \atomP \land \necessaryA \lnot \atomP \land \necessaryB \necessaryA \lnot \atomP$, so we can rewrite $\coversA \{\necessaryB \necessaryA \atomP, \necessaryB \necessaryA \lnot \atomP\}$ as $\coversA \{\atomP \land \necessaryA \atomP \land \necessaryB \necessaryA \atomP, \lnot \atomP \land \necessaryA \lnot \atomP \land \necessaryB \necessaryA \lnot \atomP\}$.
We can clearly see how the $\agentA$-modalities in the $\agentA$-cover operator interact due to transitivity and Euclideaness, so we can rewrite the formula to 
$\coversA \{\atomP \land \lnot \atomP \land \necessaryA \atomP \land \necessaryB \necessaryA \atomP, \atomP \land \lnot \atomP \land \necessaryA \lnot \atomP \land \necessaryB \necessaryA \lnot \atomP\}$, where the interaction between the formulas is now explicit and the contradiction is more obvious.
In this form if we consider the formulas individually as in the {\bf RK} axiom the contradiction will not disappear.

Here we define explicit formulas, a syntactic form defined in terms of the cover operator that has been specifically designed so that a set of formulas that are contradictory when taken together in a cover operator features at least one formula that is contradictory when considered individually.
Whereas the syntactic form used in Chapter~\ref{rml-kd45} ensures that there is no interaction between formulas in a cover operator, this syntactic form instead ensures that any interaction between the formulas is explicitly represented in the formula.

\begin{definition}[Explicit formulas]\label{explicit-formulas}
Let:
\begin{itemize}
    \item $\pi \in \langPl$ be a propositional formula
    \item $\lambda_0 \in \powerset(\langMl)$ be a finite set of modal formulas
    \item $\agentsC \subseteq \agents$ be a non-empty, finite set of agents
    \item For every $\agentC \in \agentsC$ let $\Lambda_\agentC \subseteq \powerset(\langMl)$ be a non-empty, finite set of finite sets of modal formulas
    \item $\Delta = \{\delta' \leq \delta \mid \agentC \in \agentsC, \lambda \in \Lambda_\agentC, \delta \in \lambda\}$
    \item $\gamma_0 = \bigwedge_{\delta \in \lambda_0} \delta \land \bigwedge_{\delta \in \Delta \setminus \lambda_0} \lnot \delta$
    \item For every $\agentC \in \agentsC$, $\lambda \in \Lambda_\agentC$ let $\gamma_\lambda = \bigwedge_{\delta \in \lambda} \delta \land \bigwedge_{\delta \in \Delta \setminus \lambda} \lnot \delta$.
    \item For every $\agentC \in \agentsC$ let $\Gamma_\agentC = \{\gamma_\lambda \mid \lambda \in \Lambda_\agentC\}$
\end{itemize}
such that:
\begin{enumerate}
    \item For every $\agentC \in \agentsC$ let $\lambda_0 \in \Lambda_\agentC$
    \item For every $\agentC \in \agentsC$, $\lambda \in \Lambda_\agentC$, $\necessaryC \delta \in \Delta$ : $\necessaryC \delta \in \lambda$ if and only if for every $\lambda' \in \Lambda_\agentC$ we have $\delta \in \lambda'$.
\end{enumerate}

Then an {\em explicit formula} $\phi \in \langMl$ is of the form:
$$
\phi ::= \pi \land \gamma_0 \land \bigwedge_{\agentC \in \agentsC} \coversC \Gamma_\agentC
$$
\end{definition}

It is difficult to fully justify the definition of explicit formulas at this stage, as some aspects of explicit formulas are motivated by details of the soundness proofs in the following section.
However we can make some remarks.
The set of formulas $\lambda_0$ and the corresponding formula $\gamma_0$ represents what is true in the real world.
Condition (1) ensures that each agent considers the real world possible, explicitly representing certain consequences of reflexivity.
Condition (2) ensures that each agent has the same knowledge in each world they consider possible, explicitly representing certain consequences of transitivity and Euclideaness.
Each formula $\gamma_\lambda$ explicitly denotes whether each subformula from $\Delta$ is true or false partly so that condition (2) can handle modalities in disjunctions correctly.
In Section~\ref{rml-s5-soundness} we will show properties of explicit formulas that are used to show the soundness of variants of {\bf RK}, {\bf RComm}, and {\bf RDist} in \logicRmlS{}.
We will also see that each formula $\gamma_\lambda$ explicitly denotes whether each subformula from $\Delta$ is true or false so that when we construct a single refinement from a set of refinements that each satisfy some formula $\gamma_\lambda$ we can show by induction over subformulas that the satisfaction of $\gamma_\lambda$ is preserved by the construction.
In Section~\ref{rml-s5-completeness} we will show that every modal formula is equivalent to an explicit formula under the semantics of \logicS{}, and use this property to demonstrate a provably correct translation from \langRml{} to \langMl{}, similar to \logicRmlK{}, \logicRmlKFF{}, and \logicRmlKD{} in previous chapters.

We now present our axiomatisation for \logicRmlS{}.

\pagebreak

\begin{definition}[Axiomatisation \axiomRmlS{}]
    The axiomatisation \axiomRmlS{} is a substitution schema consisting of the axioms and rules of \axiomS{} along with the following additional axioms and rules:
$$
\begin{array}{rl}
    {\bf R} & \proves \allrefsBs (\phi \implies \psi) \implies (\allrefsBs \phi \implies \allrefsBs \psi)\\
    {\bf RP} & \proves \allrefsBs \atomP \iff \atomP\\
    {\bf RS5} & \proves \somerefsBs (\gamma_0 \land \coversA \Gamma_\agentA) \iff (\somerefsBs \gamma_0 \land \bigwedge_{\gamma \in \Gamma_\agentA} \possibleA \somerefsBs \gamma) \text{ where } \agentA \in \agentsB\\
    {\bf RComm} & \proves \somerefsBs (\gamma_0 \land \coversA \Gamma_\agentA) \iff (\somerefsBs \gamma_0 \land \coversA \{\somerefsBs \gamma \mid \gamma \in \Gamma_\agentA\}) \text{ where } \agentA \notin \agentsB\\
    {\bf RDist} & \proves \somerefsBs (\gamma_0 \land \bigwedge_{\agentA \in \agents} \coversA \Gamma_\agentA) \iff \bigwedge_{\agentA \in \agents} \somerefsBs (\gamma_0 \land \coversA \Gamma_\agentA)\\
    {\bf NecR} & \text{From } \proves \phi \text{ infer } \proves \allrefsBs \phi
\end{array}
$$
where $\phi, \psi \in \langRml$, $\agentA \in \agents$, $\agentsB \subseteq \agents$, $\gamma_0 \land \bigwedge_{\agentA \in \agents} \coversA \Gamma_\agentA$ is an explicit formula and for every $\agentA \in \agents$, $\gamma_0 \land \coversA \Gamma_\agentA$ is an explicit formula.
\end{definition}

Finally we give an example derivation using the axiomatisation \axiomRmlS{}.

\begin{example}\label{rml-s5-example-derivation}
We show that $\proves \somerefsA (\necessaryA \atomP \land \lnot \necessaryB \atomP) \iff (\atomP \land \possibleB \lnot \atomP)$ using the axiomatisation \axiomRmlS{}.
$$
\begin{array}{ll}
    \proves (\atomP \land \possibleB \lnot \atomP) \iff (\atomP \land \possibleA \atomP \land \possibleB \lnot \atomP) & {\bf T}\\
    \proves (\atomP \land \possibleB \lnot \atomP) \iff (\atomP \land \possibleA \atomP \land \coversB \{\lnot \atomP, \atomP\}) & \text{Defn. of $\coversB$}\\
    \proves (\atomP \land \possibleB \lnot \atomP) \iff (\lnot \lnot \atomP \land \possibleA \lnot \lnot \atomP \land \coversB \{\lnot \lnot \lnot \atomP, \lnot \lnot \atomP\}) & {\bf P}\\
    \proves (\atomP \land \possibleB \lnot \atomP) \iff (\lnot \allrefsA \lnot \atomP \land \possibleA \lnot \allrefsA \lnot \atomP \land \coversB \{\lnot \allrefsA \lnot \lnot \atomP, \lnot \allrefsA \lnot \atomP\}) & {\bf RP}\\
    \proves (\atomP \land \possibleB \lnot \atomP) \iff (\somerefsA \atomP \land \possibleA \somerefsA \atomP \land \coversB \{\somerefsA \lnot \atomP, \somerefsA \atomP\}) & \text{Defn. of $\somerefsA$}\\
    \proves (\atomP \land \possibleB \lnot \atomP) \iff (\atomP \land \somerefsA (\atomP \land \coversA \{\atomP\}) \land \coversB \{\somerefsA \lnot \atomP, \somerefsA \atomP\}) & {\bf RS5}\\
    \proves (\atomP \land \possibleB \lnot \atomP) \iff (\somerefsA (\atomP \land \coversA \{\atomP\}) \land \somerefsA (\atomP \land \coversB \{\lnot \atomP, \atomP\})) & {\bf RComm}\\
    \proves (\atomP \land \possibleB \lnot \atomP) \iff \somerefsA (\atomP \land \coversA \{\atomP\} \land \coversB \{\lnot \atomP, \atomP\}) & {\bf RDist}\\
    \proves (\atomP \land \possibleB \lnot \atomP) \iff \somerefsA (\atomP \land \necessaryA \atomP \land \possibleA \atomP \land \possibleB \lnot \atomP) & \text{Defn. of $\coversA$ and $\coversB$}\\
    \proves (\atomP \land \possibleB \lnot \atomP) \iff \somerefsA (\necessaryA \atomP \land \possibleB \lnot \atomP) & \text{Modal reasoning and {\bf T}}\\
    \proves (\atomP \land \possibleB \lnot \atomP) \iff \somerefsA (\necessaryA \atomP \land \possibleB \lnot \atomP) & {\bf P}\\
    \proves (\atomP \land \possibleB \lnot \atomP) \iff \somerefsA (\necessaryA \atomP \land \lnot \necessaryB \atomP) & \text{Defn. of $\possibleB$}
\end{array}
$$
\end{example}

\section{Soundness}\label{rml-s5-soundness}

In this section we show that the axiomatisation \axiomRmlS{} is sound with respect to the semantics of the logic \logicRmlS{}.
As in \logicRmlK{}, the axioms {\bf R} and {\bf RP}, and the rule {\bf NecR} are already known to be sound as they were established for all variants of \logicRml{} in Proposition~\ref{rml-validities}.
What remains to be shown is that the axioms {\bf RS5}, {\bf RComm}, and {\bf RDist} are sound.
These axioms are similar to the corresponding axioms from \axiomRmlK{}, and accordingly our proofs of soundness build upon the techniques used to show the soundness of \axiomRmlK{}.
As in \axiomRmlK{}, the left-to-right direction of these equivalences is simple to show, whereas the right-to-left direction is more involved, relying on a construction that combines the refinements described on the right of the equivalence into a single refinement that satisfies the left of the equivalence.
In the constructions used for the soundness proofs of \axiomRmlK{} the refinements described on the right of the equivalence are combined in such a way that preserves bisimilarity of the original refinements, and hence preserves the satisfaction of all modal formulas.
In the setting of \logicRmlKFF{} and \logicRmlKD{} we noted that a construction that preserves bisimilarity was not generally possible due to the requirement that all refinements satisfy the \classKFF{} or \classKD{} frame conditions.
Hence the constructions used for the soundness proofs of \axiomRmlKFF{} and \axiomRmlKD{} relied on a restricted form of bisimilarity, called $\agentsB$-bisimilarity, which preserves the satisfaction of all $\agentsB$-restricted modal formulas, the syntactic form that was required for the axioms {\bf RK45}, {\bf RKD45}, {\bf RComm}, and {\bf RDist}.
The constructions relied on having bisimilar copies of the refinements described on the right of the equivalence.
Proxy states were introduced which were $\agentsB$-bisimilar to the refinements described on the right of the equivalence by ensuring that the $\agentsB$-successors of the proxy states were the same as the $\agentsB$-successors of the bisimilar copies of the corresponding refinements.
The bisimilar copies were bisimilar because no new outward edges were added to states from the bisimilar copies.
In the setting \logicRmlS{} a similar construction is not possible because of the requirement that all refinements satisfy the \classS{} frame conditions, specifically the requirement that refinements be symmetric.
Due to the requirement of symmetry it is not possible to add an inward edge to a state without also having an outward edge, so the approach used in \logicRmlKFF{} and \logicRmlKD{} to create bisimilar copies will not work in \logicRmlS{}, and therefore the approach used to create $\agentsB$-bisimilar proxy states also will not work in \logicRmlS{}.
This difficulty in achieving $\agentsB$-bisimilarity also explains our abandonment of $\agentsB$-restricted formulas in the axiomatisation \axiomRmlS{} and our choice to use explicit formulas instead.
Since we have no simple replacement for bisimilarity or $\agentsB$-bisimilarity a more complex syntactic form is required in order to guarantee that formulas of this syntactic form are preserved by the construction used in our soundness proofs.

We begin by showing some properties of explicit formulas that we use in the soundness proofs.

\begin{lemma}
    Let $\phi = \gamma_0 \land \bigwedge_{\agentC \in \agentsC} \coversC \Gamma_\agentC$ be an explicit formula such that for every $\agentC \in \agentsC$, $\gamma \in \Gamma_\agentC$, $\gamma$ is consistent with respect to the semantics of \logicRmlS{}, and let $\Delta$ be as defined in Definition~\ref{explicit-formulas}.

    Then for every $\agentC \in \agentsC$, $\gamma \in \Gamma_\agentC$:
    \begin{enumerate}
        \item For every $\delta \in \Delta$: $\proves \gamma \implies \delta$ or $\proves \gamma \implies \lnot \delta$
        \item For every $\necessaryC \delta \in \Delta$: $\proves \gamma \necessaryC \delta$ if and only if for every $\gamma' \in \Gamma_\agentC$ we have $\proves \gamma' \implies \delta$
    \end{enumerate}
\end{lemma}

\begin{proof}
    Let $\agentC \in \agentsC$, $\gamma \in \Gamma_\agentC$.
    By the definition of explicit formulas $\gamma = \gamma_\lambda$ for some $\lambda \in \Lambda_\agentC$, where $\gamma_\lambda = \bigwedge_{\delta' \in \lambda} \delta' \land \bigwedge_{\delta' \in \Delta \setminus \lambda} \lnot \delta'$.

    Let $\delta \in \Delta$.

    Suppose that $\delta \in \lambda$.
    Then $\proves \bigwedge_{\delta' \in \lambda} \delta' \implies \delta$ so $\proves \gamma_\lambda \implies \delta$.

    Suppose that $\delta \notin \lambda$.
    Then $\proves \bigwedge_{\delta' \in \Delta \setminus \lambda} \lnot \delta' \implies \lnot \delta$ so $\proves \gamma_\lambda \implies \lnot \delta$.

    Let $\necessaryC \delta \in \Delta$.

    Suppose that $\proves \gamma_\lambda \implies \necessaryC \delta$.
    As $\gamma_\lambda$ is consistent then $\nproves \gamma_\lambda \implies \lnot \necessaryC \delta$.
    From above $\necessaryC \delta \notin \lambda$ implies that $\proves \gamma_\lambda \implies \lnot \delta$, so by contrapositive we have that $\necessaryC \delta \in \lambda$.
    By the definition of explicit formulas $\necessaryC \delta \in \lambda$ if and only if for every $\lambda' \in \Lambda_\agentC$ we have $\delta \in \lambda'$.
    From above $\delta \in \lambda'$ implies that $\proves \gamma_{\lambda'} \implies \delta$.

    Suppose that for every $\gamma' \in \Gamma_\agentC$ we have $\proves \gamma' \implies \delta$.
    As $\gamma_{\lambda'}$ is consistent then $\nproves \gamma_{\lambda'} \implies \lnot \delta$.
    From above $\delta \notin \lambda$ implies that $\proves \gamma_{\lambda'} \implies \lnot \delta$, so by contrapositive we have that $\delta \in \lambda'$ for every $\lambda' \in \Lambda_\agentC$.
    By the definition of explicit formulas $\necessaryC \delta \in \lambda$ if and only if for every $\lambda' \in \Lambda_\agentC$ we have $\delta \in \lambda'$.
    From above $\necessaryC \delta \in \lambda$ implies that $\proves \gamma_\lambda \implies \necessaryC \delta$.
\end{proof}

We use this lemma to show the soundness of {\bf RS5}, {\bf RComm}, and {\bf RDist}.

We next show that the axiom {\bf RS5} is sound.
Recall that the axiom {\bf RS5} takes the form of $\proves \somerefsBs (\gamma_0 \land \coversA \Gamma_\agentA) \iff (\somerefsBs \gamma_0 \land \bigwedge_{\gamma \in \Gamma_\agentA} \possibleA \somerefsBs \gamma)$ where $\agentsB \subseteq \agents$, $\agentA \in \agentsB$ and $\gamma_0 \land \coversA \Gamma_\agentA$ is an explicit formula.

\begin{lemma}\label{rml-s5-rs5}
The axiom {\bf RS5} is sound with respect to the semantics of the logic \logicRmlS{}.
\end{lemma}

\begin{proof}
($\Rightarrow$) 
Let $\kPModel{\kStateS} \in \classS$ be a pointed Kripke model such that $\kPModel{\kStateS} \entails \somerefsBs (\gamma_0 \land \coversA \Gamma_\agentA)$.
We show that $\kPModel{\kStateS} \entails \somerefsBs \gamma_0 \land \bigwedge_{\gamma \in \Gamma_\agentA} \possibleA \somerefsBs \gamma$ using essentially the same reasoning used in the proof of soundness of {\bf RK} in Lemma~\ref{rml-k-rk}.
The only additional consideration required for \logicRmlS{} is that the refinement must be a \classS{} Kripke model, but this is given by the semantics of $\somerefsBs$ in \logicRmlS{}.

($\Leftarrow$)
Let $\kPModelAndTuple{\kStateS} \in \classS$ be a pointed Kripke model such that $\kPModel{\kStateS} \entails \somerefsBs \gamma_0 \land \bigwedge_{\gamma \in \Gamma_\agentA} \possibleA \somerefsBs \gamma$.
For every $\gamma \in \Gamma_\agentA$ there exists $\kStateT_\gamma \in \kSuccessorsA{\kStateS}$ and $\kPModelAndTuple[\gamma]{\kStateS[\gamma]} \in \classS$ such that $\kPModel{\kStateT_\gamma} \simulatesBs \kPModel[\gamma]{\kStateS[\gamma]}$ and $\kPModel[\gamma]{\kStateS[\gamma]} \entails \gamma$.
By Lemma~\ref{refinement-expansion}, without loss of generality we assume for every $\gamma \in \Gamma_\agentA$ that $\kPModel[\gamma]{\kStateS[\gamma]}$ is such that $\kPModel{\kStateT_\gamma} \simulatesBs \kPModel[\gamma]{\kStateS[\gamma]}$ via an expanded $\agentsB$-refinement $\refinement^{\gamma} \subseteq \kStates \times \kStates[\gamma]$.
As $\kPModel{\kStateS} \entails \somerefsBs \gamma_0$ and $\gamma_0 \in \Gamma_\agentA$, without loss of generality we assume that $\kStateT_{\gamma_0} = \kStateS$.
Without loss of generality we assume that each of the $\kStates[\gamma]$ are pair-wise disjoint.
We use these refinements to construct a single larger refinement to satisfy the left-hand-side of the {\bf RS5} equivalence.

Let $\kPModelAndTupleP{\kStateSP_{\gamma_0}}$ be a pointed Kripke model where:
\begin{eqnarray*}
    \kStatesP &=& \bigcup_{\gamma \in \Gamma_\agentA} (\{\kStateSP_\gamma\} \cup \kStates[\gamma])\\
    \kAccessibilityPA &=& \{(\kStateSP_\gamma, \kStateSP_{\gamma'}) \mid \gamma, \gamma' \in \Gamma_\agentA\} \cup \bigcup_{\gamma \in \Gamma_\agentA} \kStates[\gamma]\\
    \kAccessibilityPB &=& \bigcup_{\gamma \in \Gamma_\agentA} (\{(\kStateSP_\gamma, \kStateSP_\gamma), (\kStateSP_\gamma, \kStateT[\gamma]), (\kStateT[\gamma], \kStateSP_\gamma) \mid \kStateT[\gamma] \in \kSuccessorsB[\gamma]{\kStateS[\gamma]}\} \cup \kAccessibilityB[\gamma])\\
    \kValuationP(\atomP) &=& \bigcup_{\gamma \in \Gamma_\agentA} (\{\kStateSP_\gamma \mid \kStateS[\gamma] \in \kValuation[\gamma](\atomP)\} \cup \kValuation[\gamma](\atomP))
\end{eqnarray*}
where $\kStateSP_\gamma$ for every $\gamma \in \Gamma_\agentA$ are fresh states not appearing in $\kStates[\gamma]$ for any $\gamma \in \Gamma_\agentA$, and $\agentB \in \agents \setminus \{\agentA\}$.

We note that by construction $\kModelP \in \classS$.

\begin{figure}
    \caption{A schematic of the construction used to show soundness of {\bf RS5}.}\label{rml-s5-rs5-construction}
    \centering
    \begin{tikzpicture}[>=stealth',shorten >=1pt,auto,node distance=7em,thick]

        \node (sp) {\underline{$\kPModelP{\kStateSP} \entails \gamma_0 \land \coversA \{\gamma_1, \dots, \gamma_n\}$}};
        \node (mpn) [above of=sp] {$\kPModelP{\kStateSP_{\gamma_n}} \entails \gamma_n$};
        \node (mpd) [left of=mpn,node distance=5.25em] {$\cdots$};
        \node (mp1) [left of=mpd,node distance=5.25em] {$\kPModelP{\kStateSP_{\gamma_1}} \entails \gamma_1$};
        \node (msn) [above of=mpn] {$\vdots$};
        \node (ms1) [left of=msn,node distance=10.5em] {$\vdots$};
        \node (mgn) [above of=msn] {$\kPModel[\gamma_n]{\kStateS[\gamma_n]} \entails \gamma_n$};
        \node (mgd) [left of=mgn,node distance=5.25em] {$\cdots$};
        \node (mg1) [left of=mgd,node distance=5.25em] {$\kPModel[\gamma_1]{\kStateS[\gamma_1]} \entails \gamma_1$};
        \node (mtn) [above of=mgn] {$\kPModel{\kStateT_{\gamma_n}} \entails \somerefsBs \gamma_n$};
        \node (mtd) [left of=mtn,node distance=5.25em] {$\cdots$};
        \node (mt1) [left of=mtd,node distance=5.25em] {$\kPModel{\kStateT_{\gamma_1}} \entails \somerefsBs \gamma_1$};
        \node (ms) [above of=mtn] {$\kPModel{\kStateS} \entails \somerefsBs \gamma_0 \land \possibleA \somerefsBs \gamma_1 \land \cdots \land \possibleA \somerefsBs \gamma_n$};
        \node (mtr) [right of=mtn,node distance=10.5em] {$\kPModel{\kStateT} \entails \top$};

      \path[every node/.style={font=\sffamily\small},<->]
        (mt1) edge [->,swap,dashed] node {$\simulatesBs$} (mg1)
        (mtn) edge [->,dashed] node {$\simulatesBs$} (mgn)
        (mg1) edge [loop left,distance=1em] node {$\agents$} (mg1)
        (mgn) edge [loop right,distance=1em] node {$\agents$} (mgn)
        (ms) edge [->,dashed,bend left=90,min distance=10em] node {$\simulatesBs$} (sp)
        (sp) edge node {$\agentA$} (mp1)
             edge [swap] node {$\agentA$} (mpn)
             edge [loop below,distance=1em] node {$\agents$} (sp)
        (ms) edge [swap] node {$\agentA$} (mt1)
             edge node {$\agentA$} (mtr)
             edge node {$\agentA$} (mtn)
             edge [loop above,distance=1em] node {$\agents$} (ms)
        (mt1) edge [swap,loop left,distance=1em] node {$\agents$} (mt1)
        (mtn) edge [swap,loop right,distance=1em] node {$\agents$} (mtn)
        (mtr) edge [swap,loop right,distance=1em] node {$\agents$} (mtr)
        (mt1) edge [<->,swap,bend left] node {$\agentA$} (mtn)
        (mtn) edge [<->,swap,bend left] node {$\agentA$} (mtr)
        (mt1) edge [<->,swap,bend left] node {$\agentA$} (mtr)
        (mp1) edge [swap,loop left,distance=1em] node {$\agents$} (mp1)
        (mpn) edge [swap,loop right,distance=1em] node {$\agents$} (mpn)
        (mp1) edge [<->,swap,bend right] node {$\agentA$} (mpn)
        (mp1) edge [swap] node {$\agents \setminus \{\agentA\}$} (ms1)
        (mpn) edge node {$\agents \setminus \{\agentA\}$} (msn)
        (mg1) edge node {$\agents$} (ms1)
        (mgn) edge [swap] node {$\agents$} (msn);
    \end{tikzpicture}
\end{figure}

A schematic of the Kripke model $\kPModelP{\kStateSP}$ and an overview of our construction is shown in Figure~\ref{rml-s5-rs5-construction}.
The construction is similar in essence to the construction used for the soundness proof of soundness of {\bf RK45} in Lemma~\ref{rml-k45-rk45}.
Here we can see that each of the $\agentsB$-refinements at successors, $\kPModel[\gamma_1]{\kStateT[\gamma_1]}, \dots, \kPModel[\gamma_n]{\kStateT[\gamma_n]}$, are combined into the larger Kripke model $\kPModelP{\kStateSP}$.
We can see the use of the proxy states $\kPModelP{\kStateS_{\gamma_1}}, \dots, \kPModelP{\kStateS_{\gamma_n}}$, which have all of the $(\agents \setminus \{\agentA\})$-successors of the respective refinements $\kPModel[\gamma_1]{\kStateT[\gamma_1]}, \dots, \kPModel[\gamma_n]{\kStateT[\gamma_n]}$.
Unlike the construction used for {\bf RK45} the proxy states are {\em not} $(\agents \setminus \{\agentA\})$-bisimilar to the respective refinements states.
This is because in order to ensure that $\kPModelP \in \classS$ the $(\agents \setminus \{\agentA\})$-edges from proxy states to refinement states must be symmetrical.
From this schematic representation we can clearly see that $\kPModelP{\kStateSP} \entails \coversA \{\gamma_1, \dots, \gamma_n\}$.
It is less clear that $\kPModel{\kStateS} \simulatesBs \kPModelP{\kStateSP}$, or that $\kPModelP{\kStateSP_{\gamma_i}} \entails \gamma_i$, but we show these next.
We note that there are $\agentA$-successors of $\kPModel{\kStateS}$ that do not satisfy any $\somerefsBs \gamma_i$ and do not correspond to any $\agentsB$-refinement $\kPModel[\gamma_i]{\kStateT[\gamma_i]}$.
This is permissible as $\agentA \in \agentsB$, so {\bf forth-$\agentA$} is not required in order for $\kPModel{\kStateS} \simulatesBs \kPModelP{\kStateSP}$ to hold.

To show that $\kPModelP{\kStateSP_{\gamma_0}} \entails \coversA (\gamma_0 \land \Gamma_\agentA)$ we will show that $\kPModel{\kStateS} \simulatesBs \kPModelP{\kStateSP_{\gamma_0}}$ and $\kPModelP{\kStateSP_{\gamma_0}} \entails \coversA (\gamma_0 \land \Gamma_\agentA)$.

We first show that $\kPModel{\kStateS} \simulatesBs \kPModelP{\kStateSP_{\gamma_0}}$.

We define $\refinement \subseteq \kStates \times \kStatesP$ where:
$$
\refinement = \{(\kStateT_\gamma, \kStateSP_\gamma) \mid \gamma \in \Gamma_\agentA\} \cup \bigcup_{\gamma \in \Gamma_\agentA} \refinement^\gamma
$$
We show that $\refinement$ is a $\agentsB$-refinement from $\kPModel{\kStateS}$ to $\kPModelP{\kStateSP_{\gamma_0}}$.

Let $\atomP \in \atoms$, $\agentB \in \agents$, $\agentC \in \agents \setminus \agentsB$.
We show by cases that the relationships in $\refinement$ satisfy the conditions {\bf atoms-$\atomP$}, {\bf forth-$\agentC$}, and {\bf back-$\agentB$}.

\begin{description}
    \item[{Case $(\kStateT_\gamma, \kStateSP_\gamma) \in \refinement$ where $\gamma \in \Gamma_\agentA$:}]
        \hfill
        \begin{description}
            \item[atoms-$\atomP$] 
                By {\bf atoms-$\atomP$} for $\refinement^\gamma$, $\kStateT_\gamma \in \kValuation(\atomP)$ if and only if $\kStateS[\gamma] \in \kValuation[\gamma](\atomP)$.
                By construction $\kStateS[\gamma] \in \kValuation[\gamma](\atomP)$ if and only if $\kStateSP_\gamma \in \kValuationP(\atomP)$.
            \item[forth-$\agentC$]
                As $\agentA \in \agentsB$ and $\agentC \in \agents \setminus \agentsB$ then $\agentC \neq \agentA$.
                Let $\kStateU \in \kSuccessorsC{\kStateT_\gamma}$.
                By hypothesis $(\kStateT_\gamma, \kStateS[\gamma]) \in \refinement^\gamma$.
                By {\bf forth-$\agentC$} for $\refinement^\gamma$ there exists $\kStateU[\gamma] \in \kSuccessorsC[\gamma]{\kStateS[\gamma]} \subseteq \kSuccessorsPC{\kStateSP_\gamma}$ such that $(\kStateU, \kStateU[\gamma]) \in \refinement^\gamma \subseteq \refinement$.
            \item[back-$\agentB$]
                Suppose that $\agentB = \agentA$.
                Let $\kStateSP_{\gamma'} \in \kSuccessorsPA{\kStateSP_\gamma}$ where $\gamma' \in \Gamma_\agentA$.
                By hypothesis $\kStateT_{\gamma'} \in \kSuccessorsA{\kStateT_\gamma}$.
                By construction $(\kStateT_{\gamma'}, \kStateSP_{\gamma'}) \in \refinement$.

                Suppose that $\agentB \neq \agentA$.
                Consider $\kStateSP_\gamma \in \kSuccessorsPB{\kStateSP_\gamma}$.
                By the reflexivity of $\kModel$ we have that $\kStateT_\gamma \in \kSuccessorsB{\kStateT_\gamma}$.
                By construction $(\kStateT_\gamma, \kStateSP_\gamma) \in \refinement$.
                Consider $\kStateT[\gamma] \in \kSuccessorsB[\gamma]{\kStateS[\gamma]} \subseteq \kSuccessorsPB{\kStateSP_\gamma}$.
                By hypothesis $(\kStateT_\gamma, \kStateS[\gamma]) \in \refinement^\gamma$.
                By {\bf back-$\agentB$} for $\refinement^\gamma$ there exists $\kStateU \in \kSuccessorsB{\kStateT_\gamma}$ such that $(\kStateU, \kStateT[\gamma]) \in \refinement^\gamma \subseteq \refinement$.
        \end{description}
    \item[{Case $(\kStateT, \kStateT[\gamma]) \in \refinement^\gamma \subseteq \refinement$ where $\gamma \in \Gamma_\agentA$:}]
        \hfill
        \begin{description}
            \item[atoms-$\atomP$] 
                By {\bf atoms-$\atomP$} for $\refinement^\gamma$, $\kStateT \in \kValuation(\atomP)$ if and only if $\kStateT[\gamma] \in \kValuation[\gamma](\atomP)$.
                By construction $\kStateT[\gamma] \in \kValuation[\gamma](\atomP)$ if and only if $\kStateT[\gamma] \in \kValuationP(\atomP)$.
            \item[forth-$\agentC$]
                Let $\kStateU \in \kSuccessorsC{\kStateT}$.
                By {\bf forth-$\agentC$} for $\refinement^\gamma$ there exists $\kStateU[\gamma] \in \kSuccessorsC[\gamma]{\kStateT[\gamma]} \subseteq \kSuccessorsPC{\kStateT[\gamma]}$ such that $(\kStateU, \kStateU[\gamma]) \in \refinement^\gamma \subseteq \refinement$.
            \item[back-$\agentB$]
                Let $\kStateSP_\gamma \in \kSuccessorsPB{\kStateT[\gamma]}$.
                By construction this is only the case if $\agentB \neq \agentA$.
                By construction $\kStateS[\gamma] \in \kSuccessorsB[\gamma]{\kStateT[\gamma]}$.
                By hypothesis $(\kStateT_\gamma, \kStateS[\gamma]) \in \refinement^\gamma$.
                By {\bf back-$\agentB$} for $\refinement^\gamma$ there exists $\kStateU \in \kSuccessorsB{\kStateT_\gamma}$ such that $(\kStateU, \kStateS[\gamma]) \in \refinement^\gamma$.
                By hypothesis $(\kStateT_\gamma, \kStateS[\gamma]) \in \refinement^\gamma$ and as $\refinement^\gamma$ is an expanded $\agentsB$-refinement there exists a unique $\kStateU \in \kStates$ such that $(\kStateU, \kStateT[\gamma]) \in \refinement^\gamma$ and so $\kStateU = \kStateT$.
                Therefore $\kStateT_\gamma \in \kSuccessorsB{\kStateT}$ and $(\kStateT_\gamma, \kStateS[\gamma]) \in \refinement^\gamma$.
                By construction $(\kStateT_\gamma, \kStateSP_\gamma) \in \refinement$.

                Let $\kStateU[\gamma] \in \kSuccessorsB[\gamma]{\kStateT[\gamma]} \subseteq \kSuccessorsPB{\kStateT[\gamma]}$.
                By {\bf back-$\agentB$} for $\refinement^\gamma$ there exists $\kStateU \in \kSuccessorsB{\kStateT}$ such that $(\kStateU, \kStateU[\gamma]) \in \refinement^\gamma \subseteq \refinement$.
        \end{description}
\end{description}

Therefore $\refinement$ is a $\agentsB$-refinement from $\kPModel{\kStateS}$ to $\kPModelP{\kStateSP_{\gamma_0}}$.

We next show that $\kPModelP{\kStateSP_{\gamma_0}} \entails \coversA (\gamma_0 \land \Gamma_\agentA)$.

We will show for every $\gamma \in \Gamma_\agentA$ that $\kPModelP{\kStateSP_\gamma} \entails \gamma$.
Unlike in the constructions used for {\bf RK} and {\bf RKD45}, the construction used here does not preserve the bisimilarity of states from each of the refinements $\kModel[\gamma]$, so we need a different approach to show that $\kPModelP{\kStateSP_\gamma} \entails \gamma$.

Let $\Delta = \{\delta' \leq \delta \mid \agentC \in \agentsC, \lambda \in \Lambda_\agentC, \delta \in \lambda\}$, as defined in the definition of explicit formulas in Definition~\ref{explicit-formulas}.
We show by induction on the structure of formulas in $\Delta$, for every $\delta \in \Delta$, $\gamma \in \Gamma_\agentA$ that:
\begin{enumerate}
    \item $\kPModelP{\kStateSP_\gamma} \entails \delta$ if and only if $\kPModel[\gamma]{\kStateS[\gamma]} \entails \delta$
    \item For every $\kStateT[\gamma] \in \kStates[\gamma]$: $\kPModelP{\kStateT[\gamma]} \entails \delta$ if and only if $\kPModel[\gamma]{\kStateT[\gamma]} \entails \delta$.
\end{enumerate}

Let $\delta \in \Delta$, $\gamma \in \Gamma_\agentA$, and $\kStateT[\gamma] \in \kStates[\gamma]$.
We show by cases that the above properties hold.

\begin{enumerate}
    \item
        We show that $\kPModelP{\kStateSP_\gamma} \entails \delta$ if and only if $\kPModel[\gamma]{\kStateS[\gamma]} \entails \delta$:
        \begin{description}
            \item[Case $\delta = \atomP$ where $\atomP \in \atoms$:]
                \hfill\\
                By the semantics $\kPModelP{\kStateSP_\gamma} \entails \atomP$ if and only if $\kStateSP_\gamma \in \kValuationP(\atomP)$.
                By construction $\kStateSP_\gamma \in \kValuationP(\atomP)$ if and only if $\kStateS[\gamma] \in \kValuation[\gamma](\atomP)$.
                Then $\kStateS[\gamma] \in \kValuation[\gamma](\atomP)$ if and only if $\kPModel[\gamma]{\kStateS[\gamma]} \entails \atomP$.
            \item[Case $\delta = \lnot \phi$ where $\phi \in \Delta$:]
                \hfill\\
                Follows directly from the induction hypothesis.
            \item[Case $\delta = \phi \land \psi$ where $\phi, \psi \in \Delta$:]
                \hfill\\
                Follows directly from the induction hypothesis.
            \item[Case $\delta = \necessaryA \phi$ where $\phi \in \Delta$:]
                \hfill\\
                Suppose $\kPModelP{\kStateSP_\gamma} \entails \necessaryA \phi$.
                For every $\gamma' \in \Gamma_\agentA$ we have $\kPModelP{\kStateSP_{\gamma'}} \entails \phi$.
                As $\gamma_0 \land \coversA \Gamma_\agentA$ is an explicit formula then either $\proves \gamma' \implies \phi$ or $\proves \gamma' \implies \lnot \phi$.
                By hypothesis $\kPModel[\gamma']{\kStateS[\gamma']} \entails \gamma'$ and by the induction hypothesis $\kPModel[\gamma']{\kStateS[\gamma']} \entails \phi$ and so we must have $\proves \gamma' \implies \phi$.
                As $\gamma_0 \land \coversA \Gamma_\agentA$ is an explicit formula and for every $\gamma' \in \Gamma_\agentA$ we have $\proves \gamma' \implies \phi$ then $\proves \gamma \implies \necessaryA \phi$.
                By hypothesis $\kPModel[\gamma]{\kStateS[\gamma]} \entails \gamma$.
                Therefore $\kPModel[\gamma]{\kStateS[\gamma]} \entails \necessaryA \phi$.

                Suppose $\kPModel[\gamma]{\kStateS[\gamma]} \entails \necessaryA \phi$.
                As $\gamma_0 \land \coversA \Gamma_\agentA$ is an explicit formula then either $\proves \gamma \implies \necessaryA \phi$ or $\proves \gamma \implies \lnot \necessaryA \phi$.
                By hypothesis $\kPModel[\gamma]{\kStateS[\gamma]} \entails \gamma$ and from above $\kPModel[\gamma]{\kStateS[\gamma]} \entails \necessaryA \phi$ and so we must have $\proves \gamma \implies \necessaryA \phi$.
                As $\gamma_0 \land \coversA \Gamma_\agentA$ is an explicit formula and $\proves \gamma \implies \necessaryA \phi$ then for every $\gamma' \in \Gamma_\agentA$ we have $\proves \gamma' \implies \phi$.
                By hypothesis for every $\gamma' \in \Gamma_\agentA$ we have $\kPModel[\gamma']{\kStateS[\gamma']} \entails \gamma'$ and so $\kPModel[\gamma']{\kStateS[\gamma']} \entails \phi$.
                By the induction hypothesis $\kPModelP{\kStateSP_{\gamma'}} \entails \phi$.
                Therefore $\kPModelP{\kStateSP_\gamma} \entails \necessaryA \phi$.
            \item[Case $\delta = \necessaryB \phi$ where $\phi \in \Delta$ and $\agentB \neq \agentA$:]
                \hfill\\
                Suppose $\kPModelP{\kStateSP_\gamma} \entails \necessaryB \phi$.
                For every $\kStateT[\gamma] \in \kSuccessorsB[\gamma]{\kStateS[\gamma]} \subseteq \kSuccessorsPB{\kStateSP_\gamma}$ we have $\kPModelP{\kStateT[\gamma]} \entails \phi$.
                By the induction hypothesis for every $\kStateT[\gamma] \in \kSuccessorsB[\gamma]{\kStateS[\gamma]}$ we have $\kPModel[\gamma]{\kStateT[\gamma]} \entails \phi$.
                Therefore $\kPModel[\gamma]{\kStateS[\gamma]} \entails \necessaryB \phi$.

                Suppose $\kPModel[\gamma]{\kStateS[\gamma]} \entails \necessaryB \phi$.
                For every $\kStateT[\gamma] \in \kSuccessorsB[\gamma]{\kStateS[\gamma]}$ we have $\kPModel[\gamma]{\kStateT[\gamma]} \entails \phi$.
                By the induction hypothesis for every $\kStateT[\gamma] \in \kSuccessorsB[\gamma]{\kStateS[\gamma]} \subseteq \kSuccessorsPB{\kStateSP_\gamma}$ we have $\kPModelP{\kStateT[\gamma]} \entails \phi$.
                Also by the induction hypothesis as $\kPModel[\gamma]{\kStateS[\gamma]} \entails \phi$ we have $\kPModelP{\kStateSP_\gamma} \entails \phi$.
                Therefore $\kPModelP{\kStateSP_\gamma} \entails \necessaryB \gamma$.
        \end{description}
    \item
        We show that $\kPModelP{\kStateT[\gamma]} \entails \delta$ if and only if $\kPModel[\gamma]{\kStateT[\gamma]} \entails \delta$:
        \begin{description}
            \item[Case $\delta = \atomP$ where $\atomP \in \atoms$:]
                \hfill\\
                By the semantics $\kPModelP{\kStateT[\gamma]} \entails \atomP$ if and only if $\kStateT[\gamma] \in \kValuationP(\atomP)$.
                By construction $\kStateT[\gamma] \in \kValuationP(\atomP)$ if and only if $\kStateT[\gamma] \in \kValuation[\gamma](\atomP)$.
                Then $\kStateT[\gamma] \in \kValuation[\gamma](\atomP)$ if and only if $\kPModel[\gamma]{\kStateT[\gamma]} \entails \atomP$.
            \item[Case $\delta = \lnot \phi$ where $\phi \in \Delta$:]
                \hfill\\
                Follows directly from the induction hypothesis.
            \item[Case $\delta = \phi \land \psi$ where $\phi, \psi \in \Delta$:]
                \hfill\\
                Follows directly from the induction hypothesis.
            \item[Case $\delta = \necessaryA \phi$ where $\phi \in \Delta$:]
                \hfill\\
                Suppose $\kPModelP{\kStateT[\gamma]} \entails \necessaryA \phi$.
                For every $\kStateU[\gamma] \in \kSuccessorsA[\gamma]{\kStateT[\gamma]} \subseteq \kSuccessorsPA{\kStateT[\gamma]}$ we have $\kPModelP{\kStateU[\gamma]} \entails \phi$.
                By the induction hypothesis for every $\kStateU[\gamma] \in \kSuccessorsA[\gamma]{\kStateT[\gamma]}$ we have $\kPModel[\gamma]{\kStateU[\gamma]} \entails \phi$.
                Therefore $\kPModel[\gamma]{\kStateT[\gamma]} \entails \necessaryA \phi$.

                Suppose $\kPModel[\gamma]{\kStateT[\gamma]} \entails \necessaryA \phi$.
                For every $\kStateU[\gamma] \in \kSuccessorsA[\gamma]{\kStateT[\gamma]}$ we have $\kPModel[\gamma]{\kStateU[\gamma]} \entails \phi$.
                By the induction hypothesis for every $\kStateU[\gamma] \in \kSuccessorsA[\gamma]{\kStateT[\gamma]}$ we have $\kPModelP{\kStateU[\gamma]} \entails \phi$.
                By construction $\kSuccessorsPA{\kStateT[\gamma]} = \{\kStateSP_\gamma\} \cup \kSuccessorsA[\gamma]{\kStateT[\gamma]}$ or $\kSuccessorsPA{\kStateT[\gamma]} = \kSuccessorsA[\gamma]{\kStateT[\gamma]}$.
                Suppose that $\kStateSP_\gamma \notin \kSuccessorsPA{\kStateT[\gamma]}$.
                Then $\kPModelP{\kStateT[\gamma]} \entails \necessaryA \phi$.
                Suppose that $\kStateSP_\gamma \in \kSuccessorsPA{\kStateT[\gamma]}$.
                Then $\kStateS[\gamma] \in \kSuccessorsA[\gamma]{\kStateT[\gamma]}$ so from above $\kPModel[\gamma]{\kStateS[\gamma]} \entails \phi$ and by the induction hypothesis $\kPModelP{\kStateSP_\gamma} \entails \phi$.
                Therefore $\kPModelP{\kStateT[\gamma]} \entails \necessaryA \phi$.
            \item[Case $\delta = \necessaryB \phi$ where $\phi \in \Delta$ and $\agentB \neq \agentA$:]
                \hfill\\
                Suppose $\kPModelP{\kStateT[\gamma]} \entails \necessaryB \phi$.
                For every $\kStateU[\gamma] \in \kSuccessorsB[\gamma]{\kStateT[\gamma]} \subseteq \kSuccessorsPB{\kStateT[\gamma]}$ we have $\kPModelP{\kStateU[\gamma]} \entails \phi$.
                By the induction hypothesis for every $\kStateU[\gamma] \in \kSuccessorsB[\gamma]{\kStateT[\gamma]}$ we have $\kPModel[\gamma]{\kStateU[\gamma]} \entails \phi$.
                Therefore $\kPModel[\gamma]{\kStateT[\gamma]} \entails \necessaryB \phi$.

                Suppose $\kPModel[\gamma]{\kStateT[\gamma]} \entails \necessaryB \phi$.
                For every $\kStateU[\gamma] \in \kSuccessorsB[\gamma]{\kStateT[\gamma]}$ we have $\kPModel[\gamma]{\kStateU[\gamma]} \entails \phi$.
                By the induction hypothesis for every $\kStateU[\gamma] \in \kSuccessorsB[\gamma]{\kStateT[\gamma]} \subseteq \kSuccessorsPB{\kStateT[\gamma]}$ we have $\kPModelP{\kStateU[\gamma]} \entails \phi$.
                Suppose that $\kStateSP_\gamma \in \kSuccessorsPB{\kStateT[\gamma]}$.
                By construction $\kStateS[\gamma] \in \kSuccessorsB[\gamma]{\kStateT[\gamma]}$ and so from above $\kPModel[\gamma]{\kStateS[\gamma]} \entails \phi$.
                By the induction hypothesis $\kPModelP{\kStateSP_\gamma} \entails \phi$.
                Therefore $\kPModelP{\kStateT[\gamma]} \entails \necessaryB \gamma$.
        \end{description}
\end{enumerate}

Therefore for every $\gamma \in \Gamma_\agentA$ we have $\kPModelP{\kStateSP_\gamma} \entails \gamma$.
Then $\kPModelP{\kStateSP_{\gamma_0}} \entails \gamma_0 \land \coversA \Gamma_\agentA$ follows from similar reasoning to that used in the proof of soundness of {\bf RK} in Lemma~\ref{rml-k-rk}.
Therefore $\kPModel{\kStateS} \entails \somerefsBs (\gamma_0 \land \coversA \Gamma_\agentA)$.
\end{proof}

We next show that the axiom {\bf RComm} is sound.
Recall that the axiom {\bf RComm} takes the form of $\proves \somerefsBs (\gamma_0 \land \coversA \Gamma_\agentA) \iff (\somerefsBs \gamma_0 \land \coversA \{\somerefsBs \gamma \mid \gamma \in \Gamma_\agentA\})$ where $\agentsB \subseteq \agents$, $\agentA \notin \agentsB$ and $\gamma_0 \land \coversA \Gamma_\agentA$ is an explicit formula.
Also recall the differences between the soundness proofs for {\bf RK} and {\bf RComm} in \axiomRmlK{}.
Whereas for {\bf RK} we had that $\agentA \in \agentsB$ and therefore a $\agentsB$-refinement need not satisfy {\bf forth-$\agentA$}, for {\bf RComm} we had that $\agentA \notin \agentsB$ and so {\bf forth-$\agentA$} is required.
This accounted for the additional refinements $\kPModel[\kStateT]{\kStateS[\kStateT]}$ used in the construction for {\bf RComm} in \axiomRmlK{}.
Similar accommodations must be made for the soundness proof for {\bf RComm} in \axiomRmlS{} as compared to the soundness proof for {\bf RS}.

\begin{lemma}\label{rml-s5-rcomm}
The axiom {\bf RComm} is sound with respect to the semantics of the logic \logicRmlS{}.
\end{lemma}

\begin{proof}
($\Rightarrow$)
Let $\kPModel{\kStateS} \in \classS$ be a pointed Kripke model such that $\kPModel{\kStateS} \entails \somerefsBs (\gamma_0 \land \coversA \Gamma_\agentA)$.
We show that $\kPModel{\kStateS} \entails \somerefsBs \gamma_0 \land \coversA \{\somerefsBs \gamma \mid \gamma \in \Gamma_\agentA\}$ using essentially the same reasoning to that used in the proof of soundness of {\bf RComm} in Lemma~\ref{rml-k-rcomm}.
The only additional consideration required for \logicRmlS{} is that the refinement must be a \classS{} Kripke model, but this is given by the semantics of $\somerefsBs$ in \logicRmlS{}.

($\Leftarrow$)
Let $\kPModelAndTuple{\kStateS} \in \classS$ be a pointed Kripke model such that $\kPModel{\kStateS} \entails \somerefsBs \gamma_0 \land \bigwedge_{\gamma \in \Gamma_\agentA} \possibleA \somerefsBs \gamma$.
For every $\gamma \in \Gamma_\agentA$ there exists $\kStateT_\gamma \in \kSuccessorsA{\kStateS}$ and $\kPModelAndTuple[\gamma]{\kStateS[\gamma]} \in \classS$ such that $\kPModel{\kStateT_\gamma} \simulatesBs \kPModel[\gamma]{\kStateS[\gamma]}$ and $\kPModel[\gamma]{\kStateS[\gamma]} \entails \gamma$.
For every $\kStateT \in \kSuccessorsA{\kStateS}$ there exists $\gamma \in \Gamma_\agentA$ and $\kPModelAndTuple[\kStateT]{\kStateS[\kStateT]} \in \classS$ such that $\kPModel{\kStateT} \simulatesBs \kPModel[\kStateT]{\kStateS[\kStateT]}$ and $\kPModel[\kStateT]{\kStateS[\kStateT]} \entails \gamma$.
For notational consistency, for every $\kStateU \in \kSuccessorsA{\kStateS}$ we define $\kStateT_{\kStateU} = \kStateU$.
By Lemma~\ref{refinement-expansion}, without loss of generality we assume for every $\gamma \in \Gamma_\agentA$ that $\kPModel[\gamma]{\kStateS[\gamma]}$ is such that $\kPModel{\kStateT_\gamma} \simulatesBs \kPModel[\gamma]{\kStateS[\gamma]}$ via an expanded $\agentsB$-refinement $\refinement^{\gamma} \subseteq \kStates \times \kStates[\gamma]$.
Likewise without loss of generality we assume for every $\kStateT \in \kSuccessorsA{\kStateS}$ that $\kPModel[\kStateT]{\kStateS[\kStateT]}$ is such that $\kPModel{\kStateT} \simulatesBs \kPModel[\kStateT]{\kStateS[\kStateT]}$ via an expanded $\agentsB$-refinement $\refinement^{\kStateT} \subseteq \kStates \times \kStates[\kStateT]$.
As $\kPModel{\kStateS} \entails \somerefsBs \gamma_0$ and $\gamma_0 \in \Gamma_\agentA$, without loss of generality we assume that $\kStateT_{\gamma_0} = \kStateS$.
Without loss of generality we assume that each of the $\kStates[\gamma]$ and $\kStates[\kStateT]$ are pair-wise disjoint.
We use these refinements to construct a single larger refinement to satisfy the left-hand-side of the {\bf RComm} equivalence.

Let $\kPModelAndTupleP{\kStateSP_{\gamma_0}}$ be a pointed Kripke model where:
\begin{eqnarray*}
    \kStatesP &=& \bigcup_{x \in \Gamma_\agentA \cup \kSuccessorsA{\kStateT}} (\{\kStateSP_x\} \cup \kStates[x])\\
    \kAccessibilityPA &=& \{(\kStateSP_x, \kStateSP_y) \mid x, y \in \Gamma_\agentA \cup \kSuccessorsA{\kStateS}\} \cup \bigcup_{x \in \Gamma_\agentA \cup \kSuccessorsA{\kStateT}} \kStates[x]\\
    \kAccessibilityPB &=& \bigcup_{x \in \Gamma_\agentA \cup \kSuccessorsA{\kStateS}} (\{(\kStateSP_x, \kStateSP_x), (\kStateSP_x, \kStateT[x]), (\kStateT[x], \kStateSP_x) \mid \kStateT[x] \in \kSuccessorsB[x]{\kStateS[x]}\}\\
    \kValuationP(\atomP) &=& \bigcup_{x \in \Gamma_\agentA \cup \kSuccessorsA{\kStateS}} (\{\kStateSP_x \mid \kStateS[x] \in \kValuation[x](\atomP)\} \cup \kValuation[x](\atomP))
\end{eqnarray*}
where $\kStateSP_x$ for every $x \in \Gamma_\agentA \cup \kSuccessorsA{\kStateS}$ are fresh states not appearing in $\kStates[y]$ for any $y \in \Gamma_\agentA \cup \kSuccessorsA{\kStateS}$, and $\agentB \in \agents \setminus \{\agentA\}$.

We note that by construction $\kModelP \in \classS$.

\begin{figure}
    \caption{A schematic of the construction used to show soundness of {\bf RComm}.}\label{rml-s5-rcomm-construction}
    \centering
    \begin{tikzpicture}[>=stealth',shorten >=1pt,auto,node distance=7em,thick]

        \node (sp) {\underline{$\kPModelP{\kStateSP} \entails \coversA \{\gamma_1, \dots, \gamma_n\}$}};
        \node (mpn) [above of=sp] {$\kPModelP{\kStateSP_{\gamma_n}} \entails \gamma_n$};
        \node (mpr) [right of=mpn,node distance=10.5em] {$\kPModelP{\kStateSP_{\kStateT}} \entails \bigvee \Gamma_\agentA$};
        \node (mpd) [left of=mpn,node distance=5.25em] {$\cdots$};
        \node (mp1) [left of=mpd,node distance=5.25em] {$\kPModelP{\kStateSP_{\gamma_1}} \entails \gamma_1$};
        \node (msn) [above of=mpn] {$\vdots$};
        \node (msr) [right of=msn,node distance=10.5em] {$\vdots$};
        \node (ms1) [left of=msn,node distance=10.5em] {$\vdots$};
        \node (mgn) [above of=msn] {$\kPModel[\gamma_n]{\kStateS[\gamma_n]} \entails \gamma_n$};
        \node (mgr) [above of=msr] {$\kPModel[\kStateT]{\kStateS[\kStateT]} \entails \bigvee \Gamma_\agentA$};
        \node (mgd) [left of=mgn,node distance=5.25em] {$\cdots$};
        \node (mg1) [left of=mgd,node distance=5.25em] {$\kPModel[\gamma_1]{\kStateS[\gamma_1]} \entails \gamma_1$};
        \node (mtn) [above of=mgn] {$\kPModel{\kStateT_{\gamma_n}} \entails \somerefsBs \gamma_n$};
        \node (mtd) [left of=mtn,node distance=5.25em] {$\cdots$};
        \node (mt1) [left of=mtd,node distance=5.25em] {$\kPModel{\kStateT_{\gamma_1}} \entails \somerefsBs \gamma_1$};
        \node (ms) [above of=mtn] {$\kPModel{\kStateS} \entails \possibleA \somerefsBs \gamma_1 \land \cdots \land \possibleA \somerefsBs \gamma_n$};
        \node (mtr) [right of=mtn,node distance=10.5em] {$\kPModel{\kStateT} \entails \top$};

      \path[every node/.style={font=\sffamily\small},<->]
        (mt1) edge [->,swap,dashed] node {$\simulatesBs$} (mg1)
        (mtn) edge [->,dashed] node {$\simulatesBs$} (mgn)
        (mtr) edge [->,dashed] node {$\simulatesBs$} (mgr)
        (mg1) edge [loop left,distance=1em] node {$\agents$} (mg1)
        (mgn) edge [loop right,distance=1em] node {$\agents$} (mgn)
        (mgr) edge [loop right,distance=1em] node {$\agents$} (mgr)
        (ms) edge [dashed,bend left=90,min distance=10em] node {$\simulatesBs$} (sp)
        %(ms) edge [draw=none,dashed,bend right=90,min distance=10em] node {} (sp)
        (sp) edge node {$\agentA$} (mp1)
             edge [swap] node {$\agentA$} (mpn)
             edge [swap] node {$\agentA$} (mpr)
             edge [loop below,distance=1em] node {$\agents$} (sp)
        (ms) edge [swap] node {$\agentA$} (mt1)
             edge node {$\agentA$} (mtr)
             edge node {$\agentA$} (mtn)
             edge [loop above,distance=1em] node {$\agents$} (ms)
        (mt1) edge [swap,loop left,distance=1em] node {$\agents$} (mt1)
        (mtn) edge [swap,loop right,distance=1em] node {$\agents$} (mtn)
        (mtr) edge [swap,loop right,distance=1em] node {$\agents$} (mtr)
        (mt1) edge [<->,swap,bend left] node {$\agentA$} (mtn)
        (mtn) edge [<->,swap,bend left] node {$\agentA$} (mtr)
        (mt1) edge [<->,swap,bend left] node {$\agentA$} (mtr)
        (mp1) edge [swap,loop left,distance=1em] node {$\agents$} (mp1)
        (mpn) edge [swap,loop right,distance=1em] node {$\agents$} (mpn)
        (mpr) edge [swap,loop right,distance=1em] node {$\agents$} (mpr)
        (mp1) edge [<->,swap,bend right] node {$\agentA$} (mpn)
        (mpn) edge [<->,swap,bend right] node {$\agentA$} (mpr)
        (mp1) edge [<->,swap,bend right] node {$\agentA$} (mpr)
        (mp1) edge [swap] node {$\agents \setminus \{\agentA\}$} (ms1)
        (mpn) edge node {$\agents \setminus \{\agentA\}$} (msn)
        (mpr) edge node {$\agents \setminus \{\agentA\}$} (msr)
        (mg1) edge node {$\agents$} (ms1)
        (mgn) edge [swap] node {$\agents$} (msn)
        (mgr) edge [swap] node {$\agents$} (msr);
    \end{tikzpicture}
\end{figure}

A schematic of the Kripke model $\kPModelP{\kStateSP}$ and an overview of our construction is shown in Figure~\ref{rml-s5-rcomm-construction}.
The construction is similar in essence to the construction used for the soundness proof of soundness of {\bf RComm} in Lemma~\ref{rml-k45-rcomm}.
As in the construction used for {\bf S5} we can see that each of the $\agentsB$-refinements at successors, $\kPModel[\gamma_1]{\kStateT[\gamma_1]}, \dots, \kPModel[\gamma_n]{\kStateT[\gamma_n]}$, are combined into the larger Kripke model $\kPModelP{\kStateSP}$.
However in contrast to the construction used for {\bf S5} we note that here every $\agentA$-successor of $\kPModel{\kStateS}$ satisfies $\somerefsBs \gamma$ for some $\gamma \in \Gamma_\agentA$, and corresponds to some $\agentsB$-refinement $\kPModel[\kStateT]{\kStateS[\kStateT]}$.
This is required as $\agentA \in \agentsB$ and so {\bf forth-$\agentA$} is required in order for $\kPModelP{\kStateSP}$ to be a $\agentsB$-refinement of $\kPModel{\kStateS}$.
From this schematic representation we can clearly see that $\kPModelP{\kStateSP} \entails \coversA \{\gamma_1, \dots, \gamma_n\}$.
It is less clear that $\kPModel{\kStateS} \simulatesBs \kPModelP{\kStateSP}$, but we will show this next.

To show that $\kPModelP{\kStateSP_{\gamma_0}} \entails \coversA (\gamma_0 \land \Gamma_\agentA)$ we will show that $\kPModel{\kStateS} \simulatesBs \kPModelP{\kStateSP_{\gamma_0}}$ and $\kPModelP{\kStateSP_{\gamma_0}} \entails \coversA (\gamma_0 \land \Gamma_\agentA)$.

We first show that $\kPModel{\kStateS} \simulatesBs \kPModelP{\kStateSP_{\gamma_0}}$.

We define $\refinement \subseteq \kStates \times \kStatesP$ where:
$$
\refinement = \bigcup_{x \in \Gamma_\agentA \cup \kSuccessorsA{\kStateS}} (\{(\kStateT_x, \kStateS[x])\} \cup \refinement^x)
$$
We show that $\refinement$ is a $\agentsB$-refinement from $\kPModel{\kStateS}$ to $\kPModelP{\kStateSP_{\gamma_0}}$.

Let $\atomP \in \atoms$, $\agentB \in \agents$, $\agentC \in \agents \setminus \agentsB$.
We show by cases that the relationships in $\refinement$ satisfy the conditions {\bf atoms-$\atomP$}, {\bf forth-$\agentC$}, and {\bf back-$\agentB$}.

\begin{description}
    \item[{Case $(\kStateT_x, \kStateSP_x) \in \refinement$ where $x \in \Gamma_\agentA \cup \kSuccessorsA{\kStateS}$:}]
        \hfill
        \begin{description}
            \item[atoms-$\atomP$] 
                By {\bf atoms-$\atomP$} for $\refinement^x$, $\kStateT_x \in \kValuation(\atomP)$ if and only if $\kStateS[x] \in \kValuation[x](\atomP)$.
                By construction $\kStateS[x] \in \kValuation[x](\atomP)$ if and only if $\kStateSP_x \in \kValuationP(\atomP)$.
            \item[forth-$\agentC$]
                Suppose that $\agentC = \agentA$.
                Let $\kStateU \in \kSuccessorsA{\kStateT_x}$.
                By the transitivity of $\kModel$ we have that $\kStateU \in \kSuccessorsA{\kStateS}$.
                By construction $\kStateSP_{\kStateU} \in \kSuccessorsPA{\kStateSP_x}$ and $(\kStateU, \kStateSP_{\kStateU}) \in \refinement$.

                Suppose that $\agentC \neq \agentA$.
                Let $\kStateU \in \kSuccessorsC{\kStateT_x}$.
                By hypothesis $(\kStateT_x, \kStateS[x]) \in \refinement^x$.
                By {\bf forth-$\agentC$} for $\refinement^x$ there exists $\kStateU[x] \in \kSuccessorsC[x]{\kStateS[x]} \subseteq \kSuccessorsPC{\kStateSP_x}$ such that $(\kStateU, \kStateU[x]) \in \refinement^x \subseteq \refinement$.
            \item[back-$\agentB$]
                Suppose that $\agentB = \agentA$.
                Let $\kStateSP_{x'} \in \kSuccessorsPA{\kStateSP_x}$ where $x' \in \Gamma_\agentA \cup \kSuccessorsA{\kStateS}$.
                By hypothesis $\kStateT_{x'} \in \kSuccessorsA{\kStateT_x}$.
                By construction $(\kStateT_{x'}, \kStateSP_{x'}) \in \refinement$.

                Suppose that $\agentB \neq \agentA$.
                Consider $\kStateSP_x \in \kSuccessorsPB{\kStateSP_x}$.
                By the reflexivity of $\kModel$ we have that $\kStateT_x \in \kSuccessorsB{\kStateT_x}$.
                By construction $(\kStateT_x, \kStateSP_x) \in \refinement$.
                Consider $\kStateT[x] \in \kSuccessorsB[x]{\kStateS[x]} \subseteq \kSuccessorsPB{\kStateSP_x}$.
                By hypothesis $(\kStateT_x, \kStateS[x]) \in \refinement^x$.
                By {\bf back-$\agentB$} for $\refinement^x$ there exists $\kStateU \in \kSuccessorsB{\kStateT_x}$ such that $(\kStateU, \kStateT[x]) \in \refinement^x \subseteq \refinement$.
        \end{description}
    \item[{Case $(\kStateT, \kStateT[x]) \in \refinement^x \subseteq \refinement$ where $x \in \Gamma_\agentA \cup \kSuccessorsA{\kStateS}$:}]
        \hfill
        \begin{description}
            \item[atoms-$\atomP$] 
                By {\bf atoms-$\atomP$} for $\refinement^x$, $\kStateT \in \kValuation(\atomP)$ if and only if $\kStateT[x] \in \kValuation[x](\atomP)$.
                By construction $\kStateT[x] \in \kValuation[x](\atomP)$ if and only if $\kStateT[x] \in \kValuationP(\atomP)$.
            \item[forth-$\agentC$]
                Let $\kStateU \in \kSuccessorsC{\kStateT}$.
                By {\bf forth-$\agentC$} for $\refinement^x$ there exists $\kStateU[x] \in \kSuccessorsC[x]{\kStateT[x]} \subseteq \kSuccessorsPC{\kStateT[x]}$ such that $(\kStateU, \kStateU[x]) \in \refinement^x \subseteq \refinement$.
            \item[back-$\agentB$]
                Let $\kStateSP_x \in \kSuccessorsPB{\kStateT[x]}$.
                By construction this is only the case if $\agentB \neq \agentA$.
                By construction $\kStateS[x] \in \kSuccessorsB[x]{\kStateT[x]}$.
                By {\bf back-$\agentB$} for $\refinement^x$ there exists $\kStateU \in \kSuccessorsB{\kStateT_x}$ such that $(\kStateU, \kStateT[x]) \in \refinement^x$.
                By hypothesis $(\kStateT_x, \kStateS[x]) \in \refinement^x$ and as $\refinement^x$ is an expanded $\agentsB$-refinement there exists a unique $\kStateU \in \kStates$ such that $(\kStateU, \kStateT[x]) \in \refinement^x$ and so $\kStateU = \kStateT$.
                Therefore $\kStateT_x \in \kSuccessorsB{\kStateT}$ and $(\kStateT_x, \kStateS[x]) \in \refinement^x$.
                By construction $(\kStateT_x, \kStateSP_x) \in \refinement$.

                Let $\kStateU[x] \in \kSuccessorsB[x]{\kStateT[x]} \subseteq \kSuccessorsPB{\kStateT[x]}$.
                By {\bf back-$\agentB$} for $\refinement^x$ there exists $\kStateU \in \kSuccessorsB{\kStateT}$ such that $(\kStateU, \kStateU[x]) \in \refinement^x \subseteq \refinement$.
        \end{description}
\end{description}

Therefore $\refinement$ is a $\agentsB$-refinement from $\kPModel{\kStateS}$ to $\kPModelP{\kStateSP_{\gamma_0}}$.

We note for every $\gamma \in \Gamma_\agentA$ that $\kPModelP{\kStateSP_\gamma} \entails \gamma$ and for every $\kStateT \in \kSuccessorsA{\kStateS}$ that $\kPModelP{\kStateSP_{\kStateT}} \entails \bigvee_{\gamma \in \Gamma_\agentA} \gamma$.
This follows from essentially the same reasoning to that used in the proof of soundness of {\bf RS5} in Lemma~\ref{rml-s5-rs5}.
Then $\kPModelP{\kStateSP_{\gamma_0}} \entails \gamma_0 \land \coversA \Gamma_\agentA$ follows from similar reasoning to that used in the proof of soundness of {\bf RK} in Lemma~\ref{rml-k-rk}.

Therefore $\kPModel{\kStateS} \entails \somerefsBs (\gamma_0 \land \coversA \Gamma_\agentA)$.
\end{proof}

We next show that the axiom {\bf RDist} is sound.
Recall that the axiom {\bf RDist} takes the form $\proves \somerefsBs (\gamma_0 \land \bigwedge_{\agentA \in \agents} \coversA \Gamma_\agentA) \iff \bigwedge_{\agentA \in \agents} \somerefsBs (\gamma_0 \land \coversA \Gamma_\agentA)$ where $\agentsB \subseteq \agents$ and $\gamma_0 \land \bigwedge_{\agentA \in \agents} \coversA \Gamma_\agentA$ is an explicit formula.
We note that unlike the proof of soundness of {\bf RDist} in \logicRmlKFF{} and \logicRmlKD{}, which was a direct copy of the soundness proof of {\bf RDist} in \logicRmlK{}, the soundness proof for {\bf RDist} in \logicRmlS{} is more involved.
Due to the requirement that refinements be reflexive, the construction used for \logicRmlK{} does not work in \logicRmlS{}.
The construction used for \logicRmlK{}, \logicRmlKFF{}, and \logicRmlKD{} relied on including in the construction bisimilar copies of the refinements described on the right of the equivalence, which as we remarked earlier is not possible in general in \logicRmlS{}.
Similar to the soundness proofs for {\bf RS5} and {\bf RComm} we must rely on the properties of explicit formulas in order to show that our constructed refinement satisfies the required explicit formula.

\begin{lemma}\label{rml-s5-rdist}
The axiom {\bf RDist} is sound with respect to the semantics of the logic \logicRmlS{}.
\end{lemma}

\begin{proof}
($\Rightarrow$)
Let $\kPModel{\kStateS} \in \classS$ be a pointed Kripke model such that $\kPModel{\kStateS} \entails \somerefsBs (\gamma_0 \land \bigwedge_{\agentA \in \agents} \coversA \Gamma_\agentA)$.
We show that $\kPModel{\kStateS} \entails \bigwedge_{\agentA \in \agents} \somerefsBs (\gamma_0 \land \coversA \Gamma_\agentA)$ using essentially the same reasoning to that used in the proof of soundness of {\bf RDist} in Lemma~\ref{rml-k-rdist}.
The only additional consideration required for \logicRmlS{} is that the refinement must be a \classS{} Kripke model, but this is given by the semantics of $\somerefsBs$ in \logicRmlS{}.

($\Leftarrow$)
Let $\kPModelAndTuple{\kStateS} \in \classS$ be a pointed Kripke model such that $\kPModel{\kStateS} \entails \bigwedge_{\agentA \in \agents} \somerefsBs (\gamma_0 \land \coversA \Gamma_\agentA)$.
For every $\agentA \in \agents$ there exists $\kPModel[\agentA]{\kStateS[\agentA]} \in \classS$ such that $\kPModel{\kStateS} \simulatesBs \kPModel[\agentA]{\kStateS[\agentA]}$ and $\kPModel[\agentA]{\kStateS[\agentA]} \entails \gamma_0 \land \coversA \Gamma_\agentA$.
By Lemma~\ref{refinement-expansion}, without loss of generality we assume for every $\agentA \in \agents$ that $\kPModel[\agentA]{\kStateS[\agentA]}$ is such that $\kPModel{\kStateS} \simulatesBs \kPModel[\agentA]{\kStateS[\agentA]}$ via an expanded $\agentsB$-refinement $\refinement^{\agentA} \subseteq \kStates \times \kStates[\agentA]$ where for every $\kStateT[\agentA] \in \kStates[\agentA]$.
Without loss of generality we assume that each of the $\kStates[\agentA]$ are pair-wise disjoint.
We use these refinements to construct a single larger refinement to satisfy the left-hand-side of the {\bf RDist} equivalence.

Let $\kPModelAndTupleP{\kStateSP}$ be a pointed Kripke model where:
\begin{eqnarray*}
    \kStatesP &=& \{\kStateSP\} \cup \bigcup_{\agentA \in \agents} \kStates[\agentA]\\
    \kAccessibilityA &=& (\{\kStateSP\} \cup \kSuccessorsA[\agentA]{\kStateS[\agentA]})^2 \cup \bigcup_{\agentD \in \agents} \kAccessibilityA[\agentD] \text{ for } \agentA \in \agents\\
    \kValuation(\atomP) &=& \{\kStateSP \mid \kStateS \in \kValuation(\atomP)\} \cup \bigcup_{\agentA \in \agents} \kValuation[\agentA](\atomP)
\end{eqnarray*}
where $\kStateSP$ is a fresh state not appearing in $\kStates$ or $\kStates[\agentA]$ for any $\agentA \in \Gamma_\agentA$, and $\agentA \in \agents$.

We note that by construction $\kModelP \in \classS$.

\begin{figure}
    \caption{A schematic of the construction used to show soundness of {\bf RDist}.}\label{rml-s5-rdist-construction}
    \centering
    \begin{tikzpicture}[>=stealth',shorten >=1pt,auto,node distance=7em,thick]

        \node (sp) {\underline{$\kStateSP \entails \somerefsBs (\covers[\agentC_1] \Gamma_{\agentC_1} \land \cdots \land \covers[\agentC_n] \Gamma_{\agentC_n})$}};
        \node (mtid) [above of=sp] {};
        \node (mt1m) [left of=mtid,node distance=3.5em] {$\kStateT[\agentC_1]_m$};
        \node (mt1d) [left of=mt1m,node distance=3.5em] {$\cdots$};
        \node (mt11) [left of=mt1d,node distance=3.5em] {$\kStateT[\agentC_1]_1$};
        \node (mtn1) [right of=mtid,node distance=3.5em] {$\kStateT[\agentC_n]_1$};
        \node (mtnd) [right of=mtn1,node distance=3.5em] {$\cdots$};
        \node (mtnm) [right of=mtnd,node distance=3.5em] {$\kStateT[\agentC_n]_m$};
        \node (mtd) [above of=mtid] {$\cdots$};
        \node (ms) [above of=mtd] {$\kStateS \entails \somerefsBs \covers[\agentC_1] \Gamma_{\agentC_1} \land \cdots \land \somerefsBs \covers[\agentC_n] \Gamma_{\agentC_n}$};
        \node (mt1) [left of=mtd] {$\kPModel[\agentC_1]{\kStateS[\agentC_1]} \entails \covers[\agentC_1] \Gamma_{\agentC_1}$};
        \node (mtn) [right of=mtd] {$\kPModel[\agentC_n]{\kStateS[\agentC_n]} \entails \covers[\agentC_n] \Gamma_{\agentC_n}$};

      \path[every node/.style={font=\sffamily\small},<->]
        (mt1) edge [swap] node {$\agentC_1$} (mt11)
              edge node {$\agentC_1$} (mt1m)
              edge [loop left,distance=1em] node {$\agents$} (mt1)
        (mt11) edge [bend left] node {$\agentC_1$} (mt1m)
        (mtn) edge [swap] node {$\agentC_n$} (mtn1)
              edge node {$\agentC_n$} (mtnm)
              edge [loop right,distance=1em] node {$\agents$} (mtn)
        (mtn1) edge [bend left] node {$\agentC_1$} (mtnm)
        (mt11) edge [loop left,distance=1em] node {$\agents$} (mt11)
        (mt1m) edge [loop right,distance=1em] node {$\agents$} (mt1m)
        (mtn1) edge [loop left,distance=1em] node {$\agents$} (mtn1)
        (mtnm) edge [loop right,distance=1em] node {$\agents$} (mtnm)
        (ms) edge [->,dashed,bend left=90,min distance=10em] node {$\simulatesBs$} (sp)
        %(ms) edge [draw=none,dashed,bend right=90,min distance=10em] node {} (sp)
        (sp) edge node {$\agentC_1$} (mt11)
             edge node {$\agentC_1$} (mt1m)
             edge [swap] node {$\agentC_n$} (mtn1)
             edge [swap] node {$\agentC_n$} (mtnm)
             edge [loop below,distance=1em] node {$\agents$} (sp)
        (ms) edge [->,swap,dashed] node {$\simulatesBs$} (mt1)
             edge [->,dashed]node {$\simulatesBs$} (mtn)
             edge [loop above,distance=1em] node {$\agents$} (ms);
    \end{tikzpicture}
\end{figure}

To show that $\kPModel{\kStateS} \entails \somerefsBs (\gamma_0 \land \bigwedge_{\agentA \in \agents} \coversA \Gamma_\agentA)$ we will show that $\kPModel{\kStateS} \simulatesBs \kPModelP{\kStateSP}$ and $\kPModelP{\kStateSP} \entails \somerefsBs (\gamma_0 \land \bigwedge_{\agentA \in \agents} \coversA \Gamma_\agentA)$.

We first show that $\kPModel{\kStateS} \simulatesBs \kPModelP{\kStateSP}$.

For every $\agentA \in \agents$ let $\refinement^\agentA \subseteq \kStates \times \kStates[\agentA]$ be a $\agentsB$-refinement from $\kPModel{\kStateS}$ to $\kPModel[\agentA]{\kStateS[\agentA]}$.
We define $\refinement \subseteq \kStates \times \kStatesP$ where:
$$
\refinement = \{(\kStateS, \kStateSP)\} \cup \bigcup_{\agentA \in \agents} \refinement^\agentA
$$
We show that $\refinement$ is a $\agentsB$-refinement from $\kPModel{\kStateS}$ to $\kPModelP{\kStateSP}$.

Let $\atomP \in \atoms$, $\agentB \in \agents$, $\agentD \in \agents \setminus \agentsB$.
We show by cases that the relationships in $\refinement$ satisfy the conditions {\bf atoms-$\atomP$}, {\bf forth-$\agentD$}, and {\bf back-$\agentB$}.

\begin{description}
    \item[Case $(\kStateS, \kStateSP) \in \refinement$:]
        \hfill
        \begin{description}
            \item[atoms-$\atomP$]
                By construction $\kStateS \in \kValuation(\atomP)$ if and only if $\kStateSP \in \kValuationP(\atomP)$.
            \item[forth-$\agentD$]
                Let $\kStateT \in \kSuccessorsD{\kStateS}$.
                By hypothesis $(\kStateS, \kStateS[\agentD]) \in \refinement^\agentD$.
                By {\bf forth-$\agentD$} for $\refinement^\agentD$ there exists $\kStateT[\agentD] \in \kSuccessorsD{\kStateS[\agentD]}$ such that $(\kStateT, \kStateT[\agentD]) \in \refinement^\agentD$.
                By construction $\kSuccessorsD{\kStateS[\agentD]} \subseteq \kSuccessorsPD{\kStateSP}$.
                Then $\kStateT[\agentD] \in \kSuccessorsPD{\kStateSP}$ and by construction $(\kStateT, \kStateT[\agentD]) \in \refinement$.
            \item[back-$\agentB$]
                Let $\kStateSP \in \kSuccessorsPB{\kStateSP}$.
                By the reflexivity of $\kModel$ we have $\kStateS \in \kSuccessorsB{\kStateS}$.
                By construction $(\kStateS, \kStateSP) \in \refinement$.

                Let $\kStateT[\agentB] \in \kSuccessorsB[\agentB]{\kStateS[\agentB]} \subseteq \kSuccessorsPB{\kStateSP}$.
                By hypothesis $(\kStateS, \kStateS[\agentB]) \in \refinement^\agentB$.
                By {\bf back-$\agentB$} for $\refinement^\agentB$ there exists $\kStateT \in \kSuccessorsB{\kStateS}$ such that $(\kStateT, \kStateT[\agentB]) \in \refinement^\agentB$.
                Then $(\kStateT, \kStateT[\agentB]) \in \refinement$.
        \end{description}
    \item[{Case $(\kStateT, \kStateT[\agentA]) \in \refinement^\agentA \subseteq \refinement$ where $\agentA \in \agents$:}]
        \hfill
        \begin{description}
            \item[atoms-$\atomP$]
                By {\bf atoms-$\atomP$} for $\refinement^\agentA$, $\kStateT \in \kValuation(\atomP)$ if and only if $\kStateT[\agentA] \in \kValuation[\agentA](\atomP)$.
                By construction $\kStateT[\agentA] \in \kValuation[\agentA](\atomP)$ if and only if $\kStateT[\agentA] \in \kValuationP(\atomP)$.
            \item[forth-$\agentD$]
                Let $\kStateU \in \kSuccessorsD{\kStateT}$.
                By {\bf forth-$\agentD$} for $\refinement^\agentA$ there exists $\kStateU[\agentA] \in \kSuccessorsD[\agentA]{\kStateT[\agentA]}$ such that $(\kStateU, \kStateU[\agentA]) \in \refinement^\agentA \subseteq \refinement$.
                By construction $\kSuccessorsD{\kStateT[\agentA]} \subseteq \kSuccessorsPD{\kStateT[\agentA]}$.
                Then $\kStateU[\agentA] \in \kSuccessorsPD{\kStateT[\agentA]}$ and $(\kStateU, \kStateU[\agentA]) \in \refinement$.
            \item[back-$\agentB$]
                Let $\kStateSP \in \kSuccessorsPB{\kStateT[\agentA]}$.
                By construction $\kStateS[\agentA] \in \kSuccessorsB[\agentA]{\kStateT[\agentA]}$.
                By {\bf back-$\agentB$} for $\refinement^\agentA$ there exists $\kStateU \in \kSuccessorsB{\kStateT}$ such that $(\kStateU, \kStateS[\agentB]) \in \refinement^\agentA$.
                By hypothesis $(\kStateS, \kStateS[\agentA]) \in \refinement^\agentA$ and as $\refinement^\agentA$ is an expanded $\agentsB$-refinement there exists a unique $\kStateU \in \kStates$ such that $(\kStateU, \kStateT[\agentA]) \in \refinement^\agentA$ and so $\kStateU = \kStateS$.
                Therefore $\kStateS \in \kSuccessorsA{\kStateT}$.
                By construction $(\kStateS, \kStateSP) \in \refinement$.

                Let $\kStateU[\agentA] \in \kSuccessorsPB{\kStateT[\agentA]}$.
                By construction $\kStateU[\agentA] \in \kSuccessorsB[\agentA]{\kStateT[\agentA]}$.
                By {\bf back-$\agentB$} for $\refinement^\agentA$ there exists $\kStateU \in \kSuccessorsB{\kStateT}$ such that $(\kStateU, \kStateU[\agentA]) \in \refinement^\agentA$.
                Then $(\kStateU, \kStateU[\agentA]) \in \refinement$.
        \end{description}
\end{description}

Therefore $\refinement$ is a $\agentsB$-refinement and as $(\kStateS, \kStateSP) \in \refinement$ we have that $\kPModel{\kStateS} \simulatesBs \kPModelP{\kStateSP}$.

Let $\Delta = \{\delta' \leq \delta \mid \agentC \in \agentsC, \lambda \in \Lambda_\agentC, \delta \in \lambda\}$, as defined in the definition of explicit formulas in Definition~\ref{explicit-formulas}.
We show by induction on the structure of formulas in $\Delta$, for every $\delta \in \Delta$, $\gamma \in \Gamma_\agentA$ that:
\begin{enumerate}
    \item For every $\agentA \in \agents$: $\kPModelP{\kStateSP} \entails \delta$ if and only if $\kPModel[\agentA]{\kStateS[\agentA]} \entails \delta$
    \item For every $\agentA \in \agents$, $\kStateT[\agentA] \in \kStates[\agentA]$: $\kPModelP{\kStateT[\agentA]} \entails \delta$ if and only if $\kPModel[\agentA]{\kStateT[\agentA]} \entails \delta$
\end{enumerate}

Let $\delta \in \Delta$, $\agentA \in \agents$, and $\kStateT[\agentA] \in \kStates[\agentA]$.

\begin{enumerate}
    \item We show that $\kPModelP{\kStateSP} \entails \delta$ if and only if $\kPModel[\agentA]{\kStateS[\agentA]} \entails \delta$:
        \begin{description}
            \item[Case $\delta = \atomP$ where $\atomP \in \atoms$]:
                \hfill\\
                By the semantics $\kPModelP{\kStateSP} \entails \atomP$ if and only if $\kStateSP \in \kValuationP(\atomP)$.
                By construction $\kStateSP \in \kValuationP(\atomP)$ if and only if $\kStateS \in \kValuation(\atomP)$ if and only if $\kStateS[\agentA] \in \kValuation[\agentA](\atomP)$.
                Then $\kStateS[\agentA] \in \kValuation[\agentA](\atomP)$ if and only if $\kPModel[\agentA]{\kStateS[\agentA]} \entails \atomP$.
            \item[Case $\delta = \lnot \phi$ where $\phi \in \Delta$]:
                \hfill\\
                Follows directly from the induction hypothesis.
            \item[Case $\delta = \phi \land \psi$ where $\phi, \psi \in \Delta$]:
                \hfill\\
                Follows directly from the induction hypothesis.
            \item[Case $\delta = \necessaryB \phi$ where $\phi \in \Delta$]
                \hfill\\
                Suppose $\kPModelP{\kStateSP} \entails \necessaryB \phi$.
                For every $\kStateT[\agentB] \in \kSuccessorsB[\agentB]{\kStateS[\agentB]} \subseteq \kSuccessorsPB{\kStateSP}$ we have $\kPModelP{\kStateT[\agentB]} \entails \phi$.
                By the induction hypothesis for every $\kStateT[\agentB] \in \kSuccessorsB[\agentB]{\kStateS[\agentB]}$ we have $\kPModel[\agentB]{\kStateT[\agentB]} \entails \phi$.
                Therefore $\kPModel[\agentB]{\kStateS[\agentB]} \entails \necessaryB \phi$.
                By hypothesis $\gamma_0 \land \bigwedge_{\agentA \in \agents} \coversA \Gamma_\agentA$ is an explicit formula so $\proves \gamma_0 \implies \necessaryB \phi$ or $\proves \gamma_0 \implies \lnot \necessaryB \phi$.
                By hypothesis $\kPModel[\agentB]{\kStateS[\agentB]} \entails \gamma_0$ and from above $\kPModel[\agentB]{\kStateS[\agentB]} \entails \necessaryB \phi$ so we must have $\proves \gamma_0 \implies \necessaryB \phi$.
                By hypothesis $\kPModel[\agentA]{\kStateS[\agentA]} \entails \gamma_0$ and from above $\proves \gamma_0 \implies \necessaryB \phi$ therefore $\kPModel[\agentA]{\kStateS[\agentA]} \entails \necessaryB \phi$.

                Suppose $\kPModel[\agentA]{\kStateS[\agentA]} \entails \necessaryB \phi$.
                From the same reasoning as above we must have $\kPModel[\agentB]{\kStateS[\agentB]} \entails \necessaryB \phi$.
                For every $\kStateT[\agentB] \in \kSuccessorsB[\agentB]{\kStateS[\agentB]}$ we have $\kPModel[\agentB]{\kStateT[\agentB]} \entails \phi$.
                By the induction hypothesis we have $\kPModelP{\kStateSP} \entails \phi$ and for every $\kStateT[\agentB] \in \kSuccessorsB[\agentB]{\kStateS[\agentB]}$ we have $\kPModelP{\kStateT[\agentB]} \entails \phi$.
                Therefore $\kPModelP{\kStateSP} \entails \necessaryB \phi$.
        \end{description}
    \item We show that $\kPModelP{\kStateT[\agentA]} \entails \delta$ if and only if $\kPModel[\agentA]{\kStateT[\agentA]} \entails \delta$:
        \begin{description}
            \item[Case $\delta = \atomP$ where $\atomP \in \atoms$]:
                \hfill\\
                By the semantics $\kPModelP{\kStateT[\agentA]} \entails \atomP$ if and only if $\kStateT[\agentA] \in \kValuationP(\atomP)$.
                By construction $\kStateT[\agentA] \in \kValuationP(\atomP)$ if and only if $\kStateT[\agentA] \in \kValuation[\agentA](\atomP)$.
                Then $\kStateT[\agentA] \in \kValuation[\agentA](\atomP)$ if and only if $\kPModel[\agentA]{\kStateT[\agentA]} \entails \atomP$.
            \item[Case $\delta = \lnot \phi$ where $\phi \in \Delta$]:
                \hfill\\
                Follows directly from the induction hypothesis.
            \item[Case $\delta = \phi \land \psi$ where $\phi, \psi \in \Delta$]:
                \hfill\\
                Follows directly from the induction hypothesis.
            \item[Case $\delta = \necessaryB \phi$ where $\phi \in \Delta$]
                \hfill\\
                Suppose $\kPModelP{\kStateT[\agentA]} \entails \necessaryB \phi$.
                For every $\kStateU[\agentA] \in \kSuccessorsB[\agentA]{\kStateT[\agentA]} \subseteq \kSuccessorsPB{\kStateT[\agentA]}$ we have $\kPModelP{\kStateU[\agentA]} \entails \phi$.
                By the induction hypothesis for every $\kStateU[\agentA] \in \kSuccessorsB[\agentA]{\kStateT[\agentA]}$ we have $\kPModel[\agentA]{\kStateU[\agentA]} \entails \phi$.
                Therefore $\kPModel[\agentA]{\kStateT[\agentA]} \entails \necessaryB \phi$.

                Suppose $\kPModel[\agentA]{\kStateT[\agentA]} \entails \necessaryB \phi$.
                For every $\kStateU[\agentA] \in \kSuccessorsB[\agentA]{\kStateT[\agentA]}$ we have $\kPModel[\agentA]{\kStateU[\agentA]} \entails \phi$.
                By the induction hypothesis for every $\kStateU[\agentA] \in \kSuccessorsB[\agentA]{\kStateT[\agentA]}$ we have $\kPModelP{\kStateU[\agentA]} \entails \phi$.
                By construction $\kSuccessorsPB{\kStateT[\agentA]} = \{\kStateSP\} \cup \kSuccessorsB[\agentA]{\kStateT[\agentA]}$ or $\kSuccessorsPB{\kStateT[\agentA]} = \kSuccessorsB[\agentA]{\kStateT[\agentA]}$.
                Suppose that $\kStateSP \notin \kSuccessorsPB{\kStateT[\agentA]}$.
                Then $\kPModelP{\kStateT[\agentA]} \entails \necessaryB \phi$.
                Suppose that $\kStateSP \in \kSuccessorsPB{\kStateT[\agentA]}$.
                Then $\kStateS[\agentA] \in \kSuccessorsB[\agentA]{\kStateT[\agentA]}$ so from above $\kPModel[\agentA]{\kStateS[\agentA]} \entails \phi$ and by the induction hypothesis $\kPModelP{\kStateSP} \entails \phi$.
                Therefore $\kPModelP{\kStateT[\agentA]} \entails \necessaryB \phi$.
        \end{description}
\end{enumerate}

Then $\kPModelP{\kStateSP} \entails \gamma_0 \land \bigwedge_{\agentA \in \agents} \coversA \Gamma_\agentA$ follows from similar reasoning to that used in the proof of soundness of {\bf RK} in Lemma~\ref{rml-k-rk}.
Therefore $\kPModel{\kStateS} \entails \somerefsBs (\gamma_0 \land \bigwedge_{\agentA \in \agents} \coversA \Gamma_\agentA)$.
\end{proof}

Finally we show that the axiomatisation \axiomRmlS{} is sound.

\begin{lemma}\label{rml-s5-sound}
The axiomatisation \axiomRmlS{} is sound with respect to the semantics of the logic \logicRmlS{}.
\end{lemma}

\begin{proof}
The soundness of the axioms and rules of \axiomS{} with respect to the semantics of the logic \logicRmlS{} follow from the same reasoning that they are sound in the logic \logicS{}.
The soundness of {\bf R}, {\bf RP} and {\bf NecR} follow from Proposition~\ref{rml-validities}.
The soundness of {\bf RS5}, {\bf RComm} and {\bf RDist} were shown in the previous lemmas.
\end{proof}

\section{Completeness}\label{rml-s5-completeness}

In this section we show that the axiomatisation \axiomRmlS{} is complete with respect to the semantics of the logic \logicRmlS{}.
As for \axiomRmlK{}, we show that \axiomRmlS{} is complete by demonstrating a provably correct translation from formulas of \langRml{} to the underlying modal language \langMl{}.
As a consequence of this provably correct translation we also have that \logicRmlS{} is expressively equivalent to \logicS{}, and that \logicRmlS{} is compact and decidable (via the compactness and decidability of \logicS{}).

Similar to \axiomRmlK{} we rely on a special syntactic form for modal logics for our provably correct translation, which we call explicit formulas.
We show that every modal formula is equivalent to a disjunction of explicit formulas, under the semantics of \logicS{}.

\begin{lemma}\label{explicit-equivalent}
Every modal formula is equivalent to a disjunction of explicit formulas of at most the same modal depth, under the semantics of the logic \logicS{}.
\end{lemma}

\begin{proof}
Let $\phi \in \langMl$ be a modal formula.
Without loss of generality, by Lemma~\ref{dnf-equivalent} we may assume that $\phi$ is in disjunctive normal form.
Then $\phi$ is a disjunction of formulas of the form $\psi = \pi \land \bigwedge_{\agentC \in \agentsC} \coversC \Gamma_\agentC$ where $\pi \in \langPl$, $\agentsC \subseteq \agents$, and for every $\agentC \in \agentsC$, $\Gamma_\agentC \subseteq \langMl$ is a finite set of modal formulas.
Let $\Delta = \{\delta \leq \gamma \mid \agentC \in \agentsC, \gamma \in \Gamma_\agentC\}$.

Our strategy is essentially to transform $\psi$ into a disjunction of cover operators where each formula in each cover operator explicitly denotes whether each subformula from $\Delta$ is true or false.
We achieve this by adding for each formula in the cover operator and each $\delta \in \Delta$ a vacuously true disjunction $\delta \lor \lnot \delta$, and then ``pulling'' the disjunctions out of the cover operator to the top level.
Once in the desired form we show that each disjunct is either equivalent to an explicit formula, or is inconsistent, in which case we can safely remove the disjunct from the overall disjunction.
The \axiomS{} axioms {\bf T}, {\bf 4}, and {\bf 5} ensure that if a disjunct is consistent then it satisfies conditions (1) and (2) of explicit formulas.

For every $\lambda \in \powerset(\Delta)$ we define $\tau(\lambda) = \bigwedge_{\delta \in \lambda} \delta \land \bigwedge_{\delta \in \Delta \setminus \lambda} \lnot \delta$.

Let $\lambda, \lambda' \in \powerset(\Delta)$ such that $\lambda \neq \lambda'$.
Without loss of generality we assume that there exists $\delta \in \lambda$ such that $\delta \notin \lambda'$.
As $\delta \in \Delta$ then $\delta \in \Delta \setminus \lambda'$.
Then $\proves \tau(\lambda) \implies \delta$ and $\proves \tau(\lambda') \implies \lnot \delta$ so $\proves \lnot (\tau(\lambda) \land \tau(\lambda'))$.
Therefore for every $\lambda, \lambda' \in \powerset(\Delta)$ if $\lambda \neq \lambda'$ then $\tau(\lambda)$ and $\tau(\lambda')$ are inconsistent,
so by contrapositive, if $\tau(\lambda)$ and $\tau(\lambda')$ are consistent then $\lambda = \lambda'$.

By propositional reasoning we have:
$$
\proves \bigvee_{\lambda \in \powerset(\Delta)} \tau(\lambda)
$$

Let $\agentC \in \agentsC$, $\gamma \in \Gamma_\agentC$.
By propositional reasoning we have:
$$
\proves \gamma \iff \left(\gamma \land \bigvee_{\lambda \in \powerset(\Delta)} \tau(\lambda)\right)
$$
and:
$$
\proves \gamma \iff \left(\bigvee_{\lambda \in \powerset(\Delta)} (\gamma \land \tau(\lambda))\right)
$$

Let $\lambda \in \powerset(\Delta)$ such that $\gamma \notin \lambda$.
As $\gamma \in \Delta$ then $\gamma \in \Delta \setminus \lambda$, so $\proves \tau(\lambda) \implies \lnot \gamma$ and $\proves \lnot (\gamma \land \tau(\lambda))$.
So we can safely remove such disjuncts $\gamma \land \tau(\lambda)$ from the disjunction.

Let $\powerset^{+\gamma}(\Delta) = \{\lambda \in \powerset(\Delta) \mid \gamma \in \lambda\}$.
Let $\lambda \in \powerset^{+\gamma}(\Delta)$.
As $\gamma \in \lambda$ then $\proves \tau(\lambda) \implies \gamma$ so $\proves (\gamma \land \tau(\lambda)) \iff \tau(\lambda)$.
Then:
$$
\proves \gamma \iff \left(\bigvee_{\lambda \in \powerset^{+\gamma}(\Delta)} \tau(\lambda)\right)
$$

Let $\agentC \in \agentsC$.
Then:
$$
\proves \coversC \Gamma_\agentC \iff \coversC \left\{\bigvee_{\lambda \in \powerset^{+\gamma}(\Delta)} \tau(\lambda) \mid \gamma \in \Gamma_\agentC\right\}
$$

We can pull the disjunctions inside the cover operator out to the top level.
We note that:
$$
\proves \coversC (\{\alpha \lor \beta\} \cup \Gamma) \iff (\coversC (\{\alpha\} \cup \Gamma) \lor \coversC (\{\beta\} \cup \Gamma) \lor \coversC (\{\alpha, \beta\} \cup \Gamma))
$$
We can generalise this to disjunctions of an arbitrary number of formulas:
$$
\proves \coversC (\{\bigvee_{\sigma \in \Sigma} \sigma\} \cup \Gamma) \iff \bigvee_{\emptyset \subset \Sigma' \subseteq \Sigma} \coversC (\Sigma' \cup \Gamma)
$$
And we can generalise this further to sets of disjunctions of formulas:
$$
\proves \coversC \{\bigvee_{\sigma \in \Sigma_i} \sigma \mid i = 1, \dots, n\} \iff \bigvee_{\emptyset \subset \Sigma'_1 \subseteq \Sigma_1} \cdots  \bigvee_{\emptyset \subset \Sigma'_n \subseteq \Sigma_n} \coversC \bigcup_{i = 1}^n \Sigma'_i
$$
If we enumerate $\Gamma_\agentC$ as $\Gamma_\agentC = \{\gamma^\agentC_1, \dots, \gamma^\agentC_{n^\agentC}\}$ then we get:
$$
\proves \coversC \Gamma'_\agentC \iff \bigvee_{\emptyset \subset \Lambda^\agentC_1 \subseteq \powerset^{+\gamma^\agentC_1}(\Delta)} \cdots \bigvee_{\emptyset \subset \Lambda^\agentC_{n^\agentC} \subseteq \powerset^{+\gamma^\agentC_{n^\agentC}}(\Delta)} \coversC \bigcup_{i = 1}^{n^\agentC} \{\tau(\lambda) \mid \lambda \in \Lambda^\agentC_i\}
$$
If we enumerate $\agentsC$ as $\agentsC = \{\agentC_1, \dots, \agentC_m\}$ then we get:
$$
\begin{array}{rcl}
\proves \psi &\iff& 
\bigvee_{\emptyset \subset \Lambda^{\agentC_1}_1 \subseteq \powerset^{+\gamma^{\agentC_1}_1}(\Delta)} \cdots 
\bigvee_{\emptyset \subset \Lambda^{\agentC_1}_{n^{\agentC_1}} \subseteq \powerset^{+\gamma^{\agentC_1}_{n^{\agentC_1}}}(\Delta)} \cdots\\&&\quad
\bigvee_{\emptyset \subset \Lambda^{\agentC_m}_1 \subseteq \powerset^{+\gamma^{\agentC_m}_1}(\Delta)} \cdots 
\bigvee_{\emptyset \subset \Lambda^{\agentC_m}_{n^{\agentC_m}} \subseteq \powerset^{+\gamma^{\agentC_m}_{n^{\agentC_m}}}(\Delta)} \\&&\quad
(\pi \land \bigwedge_{\agentC \in \agentsC} \coversC \bigcup_{i = 1}^{n^\agentC} \{\tau(\lambda) \mid \lambda \in \Lambda^\agentC_i\}
\end{array}
$$
So $\psi$ is equivalent to a disjunction of formulas of the form $\chi = \pi \land \bigwedge_{\agentC \in \agentsC} \coversC \bigcup_{i = 1}^{n^\agentC} \{\tau(\lambda) \mid \lambda \in \Lambda^\agentC_i\}$, where for every $\agentC \in \agentsC$, $i = 1, \dots, n^\agentC$: $\emptyset \subset \Lambda^\agentC_i \subseteq \powerset^{+\gamma^\agentC_i}(\Delta)$.

For every $\agentC \in \agentsC$ let $\Lambda_\agentC = \bigcup_{i = 1}^{n^c} \lambda^\agentC_i$.
Then:
$$
\proves \chi \iff \pi \land \bigwedge_{\agentC \in \agentsC} \coversC \{\tau(\lambda) \mid \lambda \in \Lambda_\agentC\}
$$

We note that as $\proves \coversC \Gamma \implies \necessaryC \bigvee \Gamma$ then by the \axiomS{} axiom {\bf T} we have $\proves \coversC \Gamma \implies \bigvee \Gamma$.
From this follows:
$$
\proves \chi \iff \left( \pi \land \bigwedge_{\agentC \in \agentsC} \left(\bigvee_{\lambda_\agentC \in \Lambda_\agentC} \tau(\lambda_\agentC) \land \coversC \{\tau(\lambda) \mid \lambda \in \Lambda_\agentC\}\right) \right)
$$
and by propositional reasoning we have:
$$
\proves \chi \iff \bigvee_{\lambda_{\agentC_1} \in \Lambda_{\agentC_1}} \cdots \bigvee_{\lambda_{\agentC_m} \in \Lambda_{\agentC_m}} \left(\pi \land \bigwedge_{\agentC \in \agentsC} \left(\lambda_\agentC \land \coversC \{\tau(\lambda) \mid \lambda \in \Lambda_\agentC\} \right) \right)
$$

So $\chi$ is equivalent to a disjunction of formulas of the form:
$$\omega = \pi \land \bigwedge_{\agentC \in \agentsC} \left(\lambda_\agentC \land \coversC \{\tau(\lambda) \mid \lambda \in \Lambda_\agentC\} \right)$$
where for every $\agentC \in \agentsC$, $\lambda_\agentC \in \Lambda_\agentC$.
We will show that if $\omega$ is consistent then it is equivalent to an explicit formula.
If $\omega$ is inconsistent then we can safely remove it from the overall disjunction.

Suppose that there exists $\agentC, \agentD \in \agentsC$ such that $\lambda_\agentC \neq \lambda_\agentD$.
From above it follows that $\tau(\lambda_\agentC)$ and $\tau(\lambda_\agentD)$ are inconsistent.
Therefore $\omega$ is inconsistent.
So we can safely remove such disjuncts $\omega$ from the disjunction.
By contrapositive it follows that the remaining, consistent disjuncts have for every $\agentC, \agentD \in \agentsC$ that $\lambda_\agentC = \lambda_\agentD$.
Let $\lambda_0 = \lambda_{\agentC_1} = \cdots = \lambda_{\agentC_m}$.
Then for every $\agentC \in \agentsC$ we have that $\lambda_0 \in \Lambda_\agentC$, satisfying condition (1) of explicit formulas.
Then:
$$
\proves \omega \iff \left( \pi \land \tau(\lambda_0) \land \bigwedge_{\agentC \in \agentsC} \coversC \{\tau(\lambda) \mid \lambda \in \Lambda_\agentC\} \right)
$$

Let $\gamma'_0 = \tau(\lambda_0)$, for every $\agentC \in \agentsC$, $\lambda \in \Lambda_\agentC$, let $\gamma'_\lambda = \tau(\lambda)$, and for every $\agentC \in \agentsC$ let $\Gamma'_\agentC = \{\gamma'_\lambda \mid \lambda \in \Lambda_\agentC\}$.
Then:
$$
\proves \omega \iff \left( \pi \land \gamma'_0 \land \bigwedge_{\agentC \in \agentsC} \coversC \Gamma'_\agentC \right)
$$
This is in the correct syntactic form for an explicit formula.

Suppose that there exists $\agentC \in \agentsC$, $\lambda \in \Lambda_\agentC$, $\necessaryC \delta \in \Delta$ such that $\necessaryC \delta \in \lambda$ and there exists $\lambda' \in \Lambda_\agentC$ such that $\delta \notin \lambda'$.
Then $\proves \tau(\lambda) \implies \necessaryC \delta$ so $\proves \coversC \Gamma'_\agentC \implies \possibleC \necessaryC \delta$, and by the contrapositive of the \axiomS{} axiom {\bf 5} it follows that $\proves \coversC \Gamma'_\agentC \implies \necessaryC \delta$.
Also $\proves \tau(\lambda') \implies \lnot \delta$ so $\proves \coversC \Gamma'_\agentC \implies \possibleC \lnot \delta$ or equivalently, $\proves \coversC \Gamma'_\agentC \implies \lnot \necessaryC \delta$.
Therefore $\omega$ is inconsistent.
So we can safely remove such disjuncts $\omega$ from the disjunction.
By contrapositive it follows that the remaining, consistent disjuncts have for every $\agentC \in \agentsC$, $\lambda \in \Lambda_\agentC$, $\necessaryC \delta \in \Delta$ that if $\necessaryC \delta \in \lambda$ then for every $\lambda' \in \Lambda_\agentC$ we have $\delta \in \lambda'$.

Suppose that there exists $\agentC \in \agentsC$, $\lambda \in \Lambda_\agentC$, $\necessaryC \delta \in \Delta$ such that $\necessaryC \delta \notin \lambda$ and for every $\lambda' \in \Lambda_\agentC$ we have $\delta \in \lambda'$.
Then $\proves \tau(\lambda) \implies \lnot \necessaryC \delta$ so $\proves \coversC \Gamma'_\agentC \implies \possibleC \lnot \necessaryC \delta$, or equivalently $\proves \coversC \Gamma'_\agentC \implies \lnot \necessaryC \necessaryC \delta$ and by the contrapositive of the \axiomS{} axiom {\bf 4} it follows that $\proves \coversC \Gamma'_\agentC \implies \lnot \necessaryC \delta$.
Also for every $\lambda' \in \Lambda_\agentC$ we have $\tau(\lambda') \implies \delta$ and $\proves \coversC \Gamma \implies \necessaryC \bigvee{\lambda' \in \Lambda_\agentC} \tau(\lambda')$ so $\proves \coversC \Gamma \implies \necessaryC \delta$.
Therefore $\omega$ is inconsistent.
So we can safely remove such disjuncts $\omega$ from the disjunction.
By contrapositive it follows that the remaining, consistent disjuncts have for every $\agentC \in \agentsC$, $\lambda \in \Lambda_\agentC$, $\necessaryC \delta \in \Delta$ that if for every $\lambda' \in \Lambda_\agentC$ we have $\delta \in \lambda'$ then $\necessaryC \delta \in \lambda$.

From the above it follows that the remaining, consistent disjuncts have for every $\agentC \in \agentsC$, $\lambda \in \Lambda_\agentC$, $\necessaryC \delta \in \Delta$ that $\necessaryC \delta \in \lambda$ if and only if for every $\lambda' \in \Lambda_\agentC$ we have $\delta \in \lambda'$, satisfying condition (2) of explicit formulas.
Therefore if $\omega$ is consistent then it is equivalent to an explicit formula.

Therefore $\phi$ is equivalent to a disjunction of explicit formulas.
\end{proof}

We note that we have shown a semantic equivalence between \langMl{} formulas and disjunctions of explicit formulas.
As \axiomS{} is a sound and complete axiomatisation for \logicS{} then this is also a provable equivalence in \axiomS{}, and as the axioms and rules of \axiomS{} are included in the axiomatisation \axiomRmlS{} this is also a provable in \axiomRmlS{}.

We also note that converting a modal formula to a disjunction of explicit formulas can result in an exponential increase in the size compared to the original formula.
This is essentially because the conversion requires introducing disjuncts considering power sets of subformulas.

Given this equivalence with disjunctions of explicit formulas, we will show that the reduction axioms of \axiomRmlS{} may be applied to disjunctions of explicit formulas in order to give a provably correct translation.

We first show some useful theorems in \axiomRmlS{}.

\begin{lemma}\label{rml-s5-theorems}
The following are theorems of \axiomRmlKFF{}:
\begin{align}
    \proves & \allrefsBs (\phi \land \psi) \iff (\allrefsBs \phi \land \allrefsBs \psi) \label{rml-s5-and}\\
    \proves & \somerefsBs (\phi \lor \psi) \iff (\somerefsBs \phi \lor \somerefsBs \psi) \label{rml-s5-or}\\
    \proves & \somerefsBs (\phi \land \psi) \implies (\somerefsBs \phi \land \somerefsBs \psi) \label{rml-s5-d-and}\\
    \proves & (\allrefsBs \phi \land \somerefsBs \psi) \implies \somerefsBs (\phi \land \psi) \label{rml-s5-db-and}\\
    \proves & (\pi \land \somerefsBs \psi) \iff \somerefsBs (\pi \land \psi) \label{rml-s5-pd-and}\\
    \proves & \displaystyle \somerefsBs (\pi \land \gamma_0 \land \bigwedge_{\agentA \in \agents} \coverA \Gamma_\agentA) \iff \nonumber\\
            & \displaystyle \quad
            (
            \pi \land
            \somerefsBs \gamma_0 \land
            \bigwedge_{\agentA \in \agents \cap \agentsB} \bigwedge_{\gamma \in \Gamma_\agentA} \possibleC \somerefsBs \gamma \land
            \bigwedge_{\agentA \in \agents \setminus \agentsB} \coversC \{\somerefsBs \gamma \mid \gamma \in \Gamma_\agentA\} 
            ) \label{rml-s5-cover}
\end{align}
where $\phi, \psi \in \langRml$, $\pi \in \langPl$, $\agentA \in \agents$, $\agentsB \subseteq \agents$, $\gamma_0 \land \bigwedge_{\agentA \in \agents} \coversA \Gamma_\agentA$ is an explicit formula and for every $\agentA \in \agents$, $\gamma_0 \land \coversA \Gamma_\agentA$ is an explicit formula.
\end{lemma}

\begin{proof}
These theorems can be shown using essentially the same proofs given for Lemma~\ref{rml-k-theorems} for similar theorems in \axiomRmlK{}.
The only consideration that must be made for \axiomRmlS{} is for theorem (\ref{rml-s5-cover}) where we must use {\bf RS5} instead of {\bf RK}, 
and we require that $\gamma_0 \land \bigwedge_{\agentA \in \agents} \coversA \Gamma_\agentA$ is an explicit formula and for every $\agentA \in \agents$, $\gamma_0 \land \coversA \Gamma_\agentA$ is an explicit formula, in order for {\bf RS5}, {\bf RComm}, and {\bf RDist} to be applicable, but that requirement is satisfied by hypothesis.
\end{proof}

We can now clearly recognise that equivalences (\ref{rml-s5-or}) and (\ref{rml-s5-cover}) are reduction axioms that can be used to push refinement quantifiers past propositional connectives and modalities in disjunctions of explicit formulas.
However unlike the reduction axioms of \logicRmlK{}, which operated on formulas in disjunctive normal form, or the reduction axioms of \logicRmlKFF{} and \logicRmlKD{}, which operated on formulas in alternating disjunctive normal form, we note that in explicit formulas the sets of formulas $\Gamma_\agentA$ are not themselves explicit formulas.
We must modify our provably correct translation appropriately to account for this.

Before we give our provably correct translations we give two lemmas.
We note that every \axiomS{} theorem is an \axiomRmlS{} theorem.

\begin{lemma}\label{rml-s5-ml-provability}
Let $\phi \in \langMl$ be a modal formula.
If $\proves_\axiomS \phi$ then $\proves_\axiomRmlS \phi$.
\end{lemma}

We also note that \axiomRmlS{} are closed under substitution of equivalents.

\begin{lemma}\label{rml-s5-substitution-equivalents}
Let $\phi, \psi, \chi \in \langRml$ be formulas and let $\atomP \in \atoms$ be a propositional atom.
If $\proves_\axiomRmlS \psi \iff \chi$ then $\proves_\axiomRmlS \phi[\psi\backslash\atomP] \iff \phi[\chi\backslash\atomP]$.
\end{lemma}

These lemmas follow from essentially the same reasoning to that used to show Lemma~\ref{rml-k-ml-provability} and Lemma~\ref{rml-k-substitution-equivalents} for \axiomRmlK{}.

We now show that the reduction axioms of \logicRmlS{} admit a provably correct translation from \langRml{} to \langMl{}.

\begin{lemma}\label{rml-s5-ml-equivalent}
Every refinement modal formula is equivalent to a modal formula under the semantics of the logic \logicRmlS{}.
\end{lemma}

\begin{proof}
We use essentially the same reasoning to that used to show the same result in \axiomRmlK{} in Lemma~\ref{rml-k-ml-equivalent}.
We convert subformulas to disjunctions of explicit formulas instead of formulas in disjunctive normal form, allowing the equivalences from Lemma~\ref{rml-s5-theorems} to be applied.
In order for the equivalences from Lemma~\ref{rml-s5-theorems} to be applied inductively to the subformula we must at each stage convert to a disjunction of explicit formulas again.
We note that by Lemma~\ref{explicit-equivalent} converting a formula to a disjunction of explicit formulas does not increase the modal depth of the formula, so the induction remains well-founded despite these additional conversion steps, as at each step the modal depth of the formula decreases.
\end{proof}

\pagebreak

Given the provably correct translation we have that \axiomRmlS{} is sound and complete.

\begin{theorem}
The axiomatisation \axiomRmlS{} is sound and strongly complete with respect to the semantics of the logic \logicRmlS{}.
\end{theorem}

\begin{proof}
Soundness is shown in Lemma~\ref{rml-s5-sound}.
Strong completeness follows from similar reasoning used in the proof of strong completeness of \logicRmlK{} in Lemma~\ref{rml-k-sound-complete}.
\end{proof}

We note that, much like the provably correct translation for \logicRmlK{}, the provably correct translations we have presented here can result in a non-elementary increase in the size compared to the original formula.

The provably correct translation also implies that \logicRmlS{} is expressively equivalent to \logicS{}.

\begin{corollary}
The logic \logicRmlS{} is expressively equivalent to the logic \logicS{}.
\end{corollary}

Finally from expressive equivalence we have that \logicRmlS{} is compact and decidable.

\begin{corollary}
The logic \logicRmlS{} is compact.
\end{corollary}

\begin{corollary}
The model-checking problem for the logic \logicRmlS{} is decidable.
\end{corollary}

\begin{corollary}
The satisfiability problem for the logic \logicRmlS{} is decidable.
\end{corollary}

As we noted above, the provably correct translation from \langRml{} to \langMl{} may result in a non-elementary increase in size compared to the original formula.
Therefore any algorithm that relies on the provably correct translation will have a non-elementary complexity.
Unlike \logicRmlK{} complexity bounds for the model-checking and satisfiability problems have not been considered for \logicRmlS{}, neither has the succinctness of \logicRmlS{} been considered.
We leave the consideration of better complexity bounds and succinctness results for \logicRmlS{} to future work.
