\chapter{Introduction}\label{introduction}

Epistemic logic is the logic of knowledge, used to reason about the knowledge of a collection of agents hold regarding the truth of propositional atoms and each other's knowledge.
As a modal logic, situations involving knowledge are represented by relational structures known as Kripke models, and are reasoned about using modal operators that denote that an agent knows that a statement is true.
Epistemic logic only considers static situations involving knowledge, where knowledge and the truth of the propositional atoms that the knowledge is about do not change.
However many practical situations involving knowledge are not static, and many natural questions about knowledge directly concern changes in knowledge.
For example, your knowledge may change as a result of reading this dissertation, and you might ask ``What will I know after reading this dissertation?'', ``Will I know about epistemic updates after reading this dissertation?'' or ``How can I learn about epistemic updates?''.

Dynamic epistemic logics are logics of change of knowledge, used to reason about how knowledge changes in response to epistemic updates, events that provide agents with additional information.
Examples of epistemic updates include the direct observation of information by an agent, communication of information between agents, and epistemic protocols formed by composing simpler epistemic updates sequentially or concurrently.
For our purposes, when we discuss epistemic updates we assume that they are purely informative in nature, so they may cause knowledge to change, but not the truth of the propositional atoms that the knowledge is about.
We also assume that epistemic updates provide additional information monotonically, so they may not cause agents to forget or revise information.

Previous work in dynamic epistemic logic has considered how knowledge changes in response to specific epistemic updates.
Notable works include the public announcement logic of Plaza~\cite{plaza:1989} and Gerbrandy and Groenveld~\cite{gerbrandy:1997}, and the action model logic of Baltag, Moss and Solecki~\cite{baltag:1998,baltag:2004}, which each introduce models for epistemic updates and logics for reasoning about the effects of specific epistemic updates using these models.
These logics extend epistemic logic with operators that denote that a specific epistemic update results in a statement becoming true, allowing us to answer questions such as ``Will I know about quantifying over epistemic updates after reading this dissertation?''.
Both logics represent changes in knowledge as operations that take a Kripke model representing a situation involving knowledge, and a model representing an epistemic update, and produces a new Kripke model, representing the result of the epistemic update.
These operations for performing epistemic updates on Kripke models give a powerful method for modelling and reasoning about the full effects of a specific epistemic update in a specific situation, allowing us to answer questions such as ``What will I know after reading this dissertation?''.
However many natural questions about changes in knowledge cannot be expressed as questions about specific epistemic updates, such as ``How can I learn about quantifying over epistemic updates?''.

More recent work in dynamic epistemic logic has considered the knowledge changes in response to arbitrary epistemic updates, by quantifying over epistemic updates.
Notable works include the arbitrary public announcement logic (\logicApal{}) of Balbiani, et al.~\cite{balbiani:2007} and the group announcement logic (GAL) of {\AA}gotnes, et al.~\cite{agotnes:2008,agotnes:2010}, which each introduce logics for quantifying over epistemic updates.
These logics extend public announcement logic with quantifiers that denote either that every epistemic update or some epistemic update results in a statement becoming true, allowing us to answer questions such as ``Can I learn about quantifying over epistemic updates (through some epistemic update)?''.
Supposing that the answer is in the affirmative we might subsequently ask ``How can I learn about quantifying over epistemic updates?'', expecting an example of a specific epistemic update that will result in the desired change in knowledge.
In principle, such questions may be answered by model-checking, decision and synthesis procedures for these logics.
However although both of these logics have model-checking procedures~\cite{agotnes:2010}, they are undecidable~\cite{agotnes:2014}.
In the present work we introduce several decidable logics for quantifying over epistemic updates, in the same style as \logicApal{} and GAL, but quantifying over different models for epistemic updates: refinement modal logic, arbitrary action model logic, and arbitrary positive announcement logic.

Refinement modal logic (\logicRml{}) is an extension of epistemic logic that introduces quantifiers over {\em refinements} of Kripke models.
Refinements correspond to the results of a very general notion of epistemic updates, in accordance with our informal understanding of epistemic updates as purely informative and monotonically increasing information.
The refinements of a Kripke model partially correspond to the results of action models, but are more general.
However unlike public announcements or action models, refinements in general are not backed by a model or operation for epistemic updates that produces the results.
We consider \logicRml{} in a variety of modal settings, including \classK{}, \classKF{}, \classKFF{}, \classKD{} and \classS{}.
In the settings of \classK{}, \classKFF{}, \classKD{} and \classS{} we provide sound and complete axiomatisations, we show that \logicRml{} is decidable and that RML is expressively equivalent to modal logic.
In the setting of \classKF{} we show that \logicRml{} is decidable, and that its expressivity lies strictly between that of modal logic and the modal $\mu$-calculus.

Arbitrary action model logic (\logicAaml{}) is an extension of action model logic that introduces quantifiers over action models.
Like \logicRml{}, we consider \logicAaml{} in a variety of settings, including \classK{}, \classKFF{} and \classS{}.
In these settings we show provide sound and complete axiomatisations, we show that \logicAaml{} is decidable and that \logicAaml{} is expressively equivalent to modal logic.
Moreover we show that action model quantification is equivalent to refinement quantification and we provide a synthesis procedure that constructs a specific action model that results in a desired change in knowledge whenever such an action model exists.

Arbitrary positive announcement logic (\logicPapal{}) is an extension of public announcement logic and a variant of \logicApal{} that introduces quantifiers over public announcements of positive formulas.
Positive formulas consist only of positive knowledge statements, allowing statements that an agent knows a particular statement is true, but not that an agent doesn't know a statement is true.
In contrast to public announcements of modal formulas in general, all true positive formulas can be publicly announced successfully, resulting in common knowledge of the formula, the truth of a publicly announced positive formula cannot be effected by subsequent public announcements, and repeating a public announcement of a positive formula has no effect.
Like \logicApal{}, we show that the model-checking problem is PSPACE-complete, we provide a sound and complete axiomatisation, and we show that \logicPapal{} is strictly more expressive than epistemic logic and not at least as expressive as \logicApal{}.
Unlike \logicApal{}, \logicPapal{} is decidable and we provide a decision procedure.

In Chapter~\ref{literature} we provide an overview of literature in epistemic logic and dynamic epistemic logic, giving context and motivation to the present work.
In Chapter~\ref{technical} we recall technical definitions and results used in the following chapters.
In Chapter~\ref{rml} we introduce the refinement modal logic in various settings, providing semantic results about refinements, sound and complete axiomatisations, and expressivity and decidability results.
In Chapter~\ref{aaml} we introduce the arbitrary action model logic in various settings, providing sound and complete axiomatisations, expressivity and decidability results, methods for action model synthesis, and we show that action model quantifiers are equivalent to refinement quantifiers.
In Chapter~\ref{papal} we introduce the arbitrary positive announcement logic, showing that the model-checking problem is PSPACE-complete, and providing a sound and complete axiomatisation, and expressivity and decidability results.
Finally in Chapter~\ref{conclusion} we summarise our results, and outline on-going work and open questions.
