\section{Public announcement logic}\label{pal}

We recall definitions and results from the public announcement logic of Plaza~\cite{plaza:1989}.
For a comprehensive introduction to public announcement logic we direct the reader to the book by van Ditmarsch, van der Hoek and Kooi~\cite{vanditmarsch:2007}.

\begin{definition}[Language of public announcement logic]
The {\em language of public announcement logic}, \langPal{}, is inductively defined as:
$$
\phi ::= 
    \atomP \mid
    \lnot \phi \mid
    \phi \land \phi \mid
    \necessaryA \phi \mid
    \announceA{\phi} \phi
$$
where $\atomP \in \atoms$ and $\agentA \in \agents$.
\end{definition}

We use all of the standard abbreviations from modal logic, in addition to the abbreviation $\announceE{\phi} \psi ::= \lnot \announceA{\phi} \lnot \psi$.

The formula $\announceA{\phi} \psi$ may be read as ``if $\phi$ is true then after publicly announcing that it is true, $\psi$ is true''.
The formula $\announceE{\phi} \psi$ may be read as ``$\phi$ is true and after publicly announcing that it is true, $\psi$ is true''.

As in modal logics, formulas of public announcement logic are interpreted over Kripke models.
The act of publicly announcing that a statement is true is modelled as an operation on Kripke models, a model restriction, that gives a new Kripke model representing the knowledge of the agents after witnessing the public announcement.

\begin{definition}[Model restriction]
Let $\kModelAndTuple$ be a Kripke model and let $\kStatesT \subseteq \kStates$.
We denote the {\em restriction of $\kModel$ to $\kStatesT$} by $\restrict{\kModel}{\kStatesT}$ and
define it as $\restrict{\kModel}{\kStatesT} = (\kStatesT, \restrict{\kAccessibilityRel}{\kStatesT}, \restrict{\kValuation}{\kStatesT})$ where:
\begin{eqnarray*}
    \restrict{\kAccessibilityA}{\kStatesT} &=& \kAccessibilityA \cap (\kStatesT \times \kStatesT)\\
    \restrict{\kValuation}{\kStatesT}(\atomP) &=& \kValuation(\atomP) \cap \kStatesT
\end{eqnarray*}

Let \lang{} be a logical language with an interpretation on $\kModel$, and let $\phi \in \lang$.
We denote the {\em restriction of $\kModel$ to $\phi$} by $\restrict{\kModel}{\phi}$ and define it as $\restrict{\kModel}{\phi} = \restrict{\kModel}{\interpretation[\kModel]{\phi}}$.
\end{definition}

We model the public announcement that a formula $\phi$ is true in an initial Kripke model $\kModel$ as a restriction $\restriction{\kModel}{\phi}$ of $\kModel$ to the states where $\phi$ was true in $\kModel$.
We thus define the semantics of public announcement logic.

\begin{definition}[Semantics of public announcement logic]\label{pal-semantics}
Let $\phi \in \langPal$ and let $\kPModelAndTuple{\kStateS} \in \classS$ be a pointed Kripke model.
The interpretation of the formula $\phi$ in the logic \logicPalS{} on the pointed Kripke model $\kPModel{\kStateS}$ is the same as its interpretation in modal logic, defined in Definition~\ref{ml-semantics}, with the additional inductive case:
$$
\begin{array}{lcl}
\kPModel{\kStateS} \entails \announceA{\phi} \psi & \text{ iff } & \kPModel{\kStateS} \entails \phi \text{ implies } \restrict{\kPModel{\kStateS}}{\phi} \entails \psi
\end{array}
$$
\end{definition}

We note that we only consider public announcement logic in the setting of \classS{}, as in the source material~\cite{plaza:1989}.

\begin{example}\label{example-pal}
\begin{figure}
    \caption{The result of publicly announcing $\knowsA \atomP$ in the epistemic model of Figure~\ref{example-kripke-model}.}\label{example-public-announcement}
    \centering
    \begin{tikzpicture}[>=stealth',shorten >=1pt,auto,node distance=7em,thick]

        \node[label=left:{$\kStateS$}] (s) {\underline{$\{\atomP, \atomQ\}$}};
        \node[label=right:{$\kStateT$}] (t) [right of=s] {$\{\atomP\}$};

      \path[every node/.style={font=\sffamily\small},->]
        (s) edge [loop above] node {$\agentA, \agentB$} (s)
        (t) edge [loop above] node {$\agentA, \agentB$} (t)
        (s) [<->] edge node {$\agentA$} (t);
    \end{tikzpicture}
\end{figure}
Consider the epistemic model $\kPModel{\kStateS}$ from Example~\ref{example-ml}.
Then $\interpretation[\kModel]{\knowsA \atomP} = \{\kStateS, \kStateT\}$, and so $\restrict{\kPModel{\kStateS}}{\knowsA \atomP}$ is the model restriction of $\kPModel{\kStateS}$ to the states $\{\kStateS, \kStateT\}$, depicted in Figure~\ref{example-public-announcement}.  
In the resulting model restriction we have that $\restrict{\kPModel{\kStateS}}{\knowsA \atomP} \entails \knowsB \atomP$, and so in the original model we therefore have $\kPModel{\kStateS} \entails \announceE{\knowsA \atomP} \knowsB \atomP$.
That is, agent $\agentB$ knows that $\atomP$ is true after it is publicly announced that agent $\agentA$ knows that $\atomP$ is true.
\end{example}

We provide an axiomatisation for the logic \logicPalS{}.

\begin{definition}[Axiomatisation \axiomPalS{}]
The axiomatisation \axiomPalS{} is a substitution schema consisting of the axioms and rules of \axiomS{} along with the following additional axioms and rules:
$$
\begin{array}{rl}
    {\bf AP} & \proves \announceA{\phi} \atomP \iff (\phi \implies \atomP)\\
    {\bf AN} & \proves \announceA{\phi} \lnot \psi \iff (\phi \implies \lnot \announceA{\phi} \psi)\\
    {\bf AC} & \proves \announceA{\phi} (\psi \land \chi) \iff (\announceA{\phi} \psi \land \announceA{\phi} \chi)\\
    {\bf AK} & \proves \announceA{\phi} \necessaryA \psi \iff (\phi \implies \necessaryA \announceA{\phi} \psi)\\
    {\bf AA} & \proves \announceA{\phi} \announceA{\psi} \chi \iff \announceA{\phi \land \announceA{\phi} \psi} \chi\\
    {\bf NecA} & \text{From } \proves \phi \text{ infer } \proves \announceA{\psi} \phi
\end{array}
$$
\end{definition}

We note that the axiomatisation \axiomPalS{} forms a set of reduction axioms, admitting a provably correct translation from the language \langPal{} of public announcement logic to the language \langMl{} of modal logic.
This works as the axioms {\bf AN}, {\bf AC}, {\bf AK}, and {\bf AA} may be applied to ``push'' the public announcement operators past negations, conjunctions, modalities and other public announcement operators, until they are applied only to propositional atoms, where the axiom {\bf AP} may be applied to replace the formula with a modal formula.
Therefore we have that \logicPalS{} is expressively equivalent to \logicS{} and so we get other results for the logic from the corresponding results for \logicS{}.

\begin{proposition}
The axiomatisation \axiomPalS{} is sound and strongly complete with respect to the semantics of the logic \logicPalS{}.
\end{proposition}

\begin{proposition}
The logic \logicPalS{} is expressively equivalent to the logic \logicS{}.
\end{proposition}

\begin{proposition}
The logic \logicPalS{} is bisimulation invariant.
\end{proposition}

\begin{proposition}
The logic \logicPalS{} is compact.
\end{proposition}

\begin{proposition}
The satisfiability problem for the logic \logicPalS{} is decidable.
\end{proposition}
