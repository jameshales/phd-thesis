\chapter{Refinement modal logic: \classKF{}}\label{rml-k4}

In this chapter we consider results specific to the logic \logicRmlKF{} in the setting of \classKF{}.
The main results of this chapter are expressivity results.
We show that the logic \logicRmlKF{} is strictly more expressive than the underlying modal logic \logicKF{}, and strictly less expressive than the bisimulation quantified modal logic \logicBqmlKF{}, and the modal $\mu$-calculus \logicMuKF{}.
A corollary of the latter results are that \logicRmlKF{} is decidable, via a semantically correct translation from \langRml{} to \langBqml{}.
Unlike previous chapters we do not present a sound and complete axiomatisation or a provably correct translation from \langRml{} to \langMl{}.
As \logicRmlKF{} is strictly more expressive than \logicKF{} a provably correct translation from \langRml{} to \langMl{} is not possible, so a different strategy for proving the completeness of a candidate axiomatisation is required.
The axiomatisation of \logicRmlKF{} is left as an open problem.

In the following sections we provide expressivity results for \logicRmlKF{} with respect to various logics.
In Section~\ref{rml-k4-ml} we show that \logicRmlKF{} is strictly more expressive than the underlying modal logic \logicKF{} by demonstrating a semantic property that can be expressed as a \langRml{} formula but not as a \langMl{} formula.
In Section~\ref{rml-k4-bqml} we show that \logicRmlKF{} is (non-strictly) less expressive than \logicBqmlKF{} and \logicMuKF{} by demonstrating a semantic translation from \langRml{} to \langBqml{}.
In Section~\ref{rml-k4-mu} we show that \logicRmlKF{} is strictly less expressive than \logicMuKF{} and \logicBqmlKF{}, by demonstrating a semantic property that can be expressed as a \langMu{} formula but not as a \langRml{} formula.

\section{Expressivity: modal logic}\label{rml-k4-ml}

In this section we show that \logicRmlKF{} is strictly more expressive than the underlying modal logic \logicKF{}.
That \logicRmlKF{} is at least as expressive as \logicKF{} is obvious, as \logicRmlKF{} generalises the syntax and semantics of \logicKF{}.
To show that \logicRmlKF{} is strictly more expressive than \logicKF{} we demonstrate a \classKF{} Kripke model with two designated states that can be distinguished by the validity of a \langRml{} formula under the semantics of \logicRmlKF{}, but that cannot be distinguished by any \langMl{} formula under the semantics of \logicKF{}. 

Although we do not show it formally here, the distinguishing \langRml{} formula that we will use corresponds to the semantic property that there exists an infinite path starting from the designated state in a pointed Kripke model.
To show that no \langMl{} formula corresponds to this semantic property we demonstrate a \classKF{} Kripke model with two designated states, one with an infinite path, and one without.
Both states also have terminating paths of length $n$ for every $n \in \naturals$.
The Kripke model is constructed in such a way that the two designated states are $n$-bisimilar for all $n \in \naturals$, and so they agree on the interpretation of all \langMl{} formulas.
However as one designated state has an infinite path and the other doesn't, they disagree on the interpretation of the given \langRml{} formula.

\begin{theorem}
The logic \logicRmlKF{} is strictly more expressive than \logicKF{}.
\end{theorem}

\begin{proof}
Let $\kModelAndTuple$ be a Kripke model where:
\begin{eqnarray*}
\kStateS &=& \naturals \cup \{\omega, \omega'\}\\
\kAccessibility{} &=& \{(m, n), (\omega, n), (\omega', n) \mid m, n \in \naturals, m > n\} \cup \{(\omega', \omega')\}\\
\kValuation(\atomP) &=& \emptyset
\end{eqnarray*}

\begin{figure}
    \centering
    \begin{tikzpicture}[>=stealth',shorten >=1pt,auto,node distance=7em,thick]

      \node (0) {0};
      \node (1) [right of=0] {1};
      \node (2) [right of=1] {2};
      \node (3) [right of=2] {\ldots};
      \node (omega) [above right of=3] {$\omega$};
      \node (omega') [below right of=3] {$\omega'$};

      \path[every node/.style={font=\sffamily\small},->]
        (1) edge node {} (0)
        (2) edge node {} (1)
        (3) edge node {} (2)
        (omega) edge node {} (3)
        (omega') edge node {} (3)
            edge [loop right] node {} (omega');
    \end{tikzpicture}
    \caption{The model $\kModel$, omitting implied transitive edges. The state $\omega$ has only finite paths whilst the state $\omega'$ has an infinite path due to its reflexive edge.}\label{model-naturals}
\end{figure}

The model $\kModel$ is represented in Figure~\ref{model-naturals}.
We note that $\kModel \in \classKF$.

We will show that no modal formula can distinguish between $\kPModel{\omega}$ and $\kPModel{\omega'}$.

To show this we show the following intermediate results for every $n \in \naturals$:
\begin{enumerate}
    \item $\kPModel{i} \bisimilar[n] \kPModel{j}$ for $i, j \in \naturals$ where $i, j \geq n$.
    \item $\kPModel{n} \bisimilar[n] \kPModel{\omega'}$.
    \item $\kPModel{\omega} \bisimilar[n] \kPModel{\omega'}$.
\end{enumerate}

We proceed by induction on $n \in \naturals$:

\begin{enumerate}
    \item We show that $\kPModel{i} \bisimilar[n] \kPModel{j}$ for $i, j \in \naturals$ where $i, j \geq n$.

        Suppose that $n = 0$. Then we trivially have that $\kPModel{i} \bisimilar[0] \kPModel{j}$.

        Suppose that $n > 0$.
        \begin{description}
            \item[atoms] Trivial.

            \item[forth] Let $k \in \kSuccessors{}{i}$.

            Suppose that $k \geq n - 1$. By the induction hypothesis $\kPModel{k} \bisimilar [n-1] \kPModel{j - 1}$.

            Suppose that $k < n - 1$. Then $k < n - 1 < j$ and $k \in \kSuccessors{}{j}$, so we trivially have that $\kPModel{k} \bisimilar[n - 1] \kPModel{k}$.

            \item[back] Symmetrical reasoning to {\bf forth}.
        \end{description}

    \item We show that $\kPModel{n} \bisimilar[n] \kPModel{\omega'}$.

        Suppose that $n = 0$. Then we trivially have that $\kPModel{0} \bisimilar[0] \kPModel{\omega'}$.

        Suppose that $n > 0$.

        \begin{description}
            \item[atoms] Trivial.

            \item[forth] Let $k \in \kSuccessors{}{n}$. Then $k \in \kSuccessors{}{\omega'}$ and we trivially have that $\kPModel{k} \bisimilar[n - 1] \kPModel{k}$.

            \item[back] Let $k \in \kSuccessors{}{\omega'}$. 

                Suppose that $k = \omega'$. Then $n - 1 \in \kSuccessors{}{n}$ and by the induction hypothesis $\kPModel{n - 1} \bisimilar[n - 1] \kPModel{\omega'}$. 

                Suppose that $k \neq \omega'$ and $k < n$. Then $k \in \kSuccessors{}{n}$ and we trivially have that $\kPModel{k} \bisimilar[n - 1] \kPModel{k}$.

                Suppose that $k \neq \omega'$ and $k \geq n$. Then $n - 1 \in \kSuccessors{}{n}$ and from above $\kPModel{n - 1} \bisimilar[n - 1] \kPModel{k}$.
        \end{description}

    \item We show that $\kPModel{\omega} \bisimilar[n] \kPModel{\omega'}$.

        Suppose that $n = 0$. Then we trivially have that $\kPModel{\omega} \bisimilar[0] \kPModel{\omega'}$.

        Suppose that $n > 0$.

        \begin{description}
            \item[atoms] Trivial.

            \item[forth] Let $k \in \kSuccessors{}{\omega}$. Then $k \in \kSuccessors{}{\omega'}$ and from above we have that $\kPModel{k} \bisimilar[n - 1] \kPModel{k}$.

            \item[back] Let $k \in \kSuccessors{}{\omega'}$.

                Suppose that $k = \omega'$. Then $n - 1 \in \kSuccessors{}{\omega}$ and from above $\kPModel{n - 1} \bisimilar[n - 1] \kPModel{\omega'}$.

                Suppose that $k \neq \omega'$. Then $k \in \kSuccessors{}{\omega}$ and we trivially have that $\kPModel{k} \bisimilar[n - 1] \kPModel{k}$.
        \end{description}
\end{enumerate}

Therefore $\kPModel{\omega} \bisimilar[n] \kPModel{\omega'}$ for every $n \in \naturals$.

Let $\phi \in \langMl$ and let $n = d(\phi)$ be the modal depth of $\phi$.
From above $\kPModel{\omega} \bisimilar[n] \kPModel{\omega'}$ so $\kPModel{\omega} \entails \phi$ if and only if $\kPModel{\omega'} \entails \phi$.
Therefore $\kPModel{\omega}$ is modally indistinguishable from $\kPModel{\omega'}$.

Next we will show that the refinement modal formula $\somerefs (\possible \top \land \necessary \possible \top)$ can distinguish between $\kPModel{\omega}$ and $\kPModel{\omega'}$.
Although we do not show it formally here, this distinguishing formula corresponds to the semantic property that there exists an infinite path starting from the designated state in a pointed Kripke model.
It should be clear from the construction of $\kModel$ that $\omega'$ has an infinite path, consisting of repeatedly following the reflexive edge, whereas $\omega$ does not have an infinite path, as the longest path from any successor $n \in \kSuccessors{}{\omega}$ has at most $n - 1$ steps until it reaches the state $0$ where the path must end.

We proceed with model checking using to show that $\kPModel{\omega}$ and $\kPModel{\omega'}$ disagree on the interpretation of this distinguishing formula.

We first show that $\kPModel{\omega'} \entails \somerefs (\possible \top \land \necessary \possible \top)$.
Let $\kPModelAndTupleP{\omega'}$ where:
\begin{eqnarray*}
    \kStatesP &=& \{\omega'\}\\
    \kAccessibilityP{} &=& \{(\omega', \omega')\}\\
    \kValuationP(\atomP) &=& \emptyset
\end{eqnarray*}

We note that by construction $\kPModelP{\omega'} \in \classKF$.
We also note that $\kPModel{\omega'} \simulatesBs \kPModelP{\omega'}$, and $\kPModelP{\omega'} \entails \possible \top \land \necessary \possible \top$.
Therefore $\kPModelP{\omega'} \entails \somerefs (\possible \top \land \necessary \possible \top)$.

We next show that $\kPModel{\omega} \nentails \somerefs (\possible \top \land \necessary \possible \top)$.

Let $\kPModelAndTupleP{\kStateSP} \in \classKF$ such that $\kPModel{\omega} \simulatesBs \kPModelP{\kStateSP}$, via some refinement $\refinement \subseteq \kStates \times \kStatesP$.

Suppose that $\kSuccessorsP{}{\kStateSP} = \emptyset$.
Then $\kPModelP{\kStateSP} \nentails \possible \top$ so $\kPModelP{\kStateSP} \nentails \possible \top \land \necessary \possible \top$.

Suppose that $\kSuccessorsP{}{\kStateSP} \neq \emptyset$.
Let $n \in \naturals$, $\kStateTP \in \kSuccessorsP{}{\kStateSP}$ such that $(n, \kStateTP) \in \refinement$.
Suppose there exists $\kStateUP \in \kSuccessorsP{}{\kStateTP}$.
By transitivity we have $\kStateUP \in \kSuccessorsP{}{\kStateSP}$.
By {\bf back} for $\refinement$ there exists $m \in \kSuccessors{}{n}$ such that $(m, \kStateUP) \in \refinement$.
By construction we must have $m < n$.
Then there exists $m \in \naturals$, $\kStateUP \in \kSuccessorsP{}{\kStateSP}$ such that $m < n$ and $(m, \kStateUP) \in \refinement$.
Therefore if there exists $\kStateUP \in \kSuccessorsP{}{\kStateTP}$ then there exists $m \in \naturals$, $\kStateUP \in \kSuccessorsP{}{\kStateSP}$ such that $m < n$ and $(m, \kStateUP) \in \refinement$.
By contrapositive if there is no $m \in \naturals$, $\kStateUP \in \kSuccessorsP{}{\kStateSP}$ such that $m < n$ and $(m, \kStateUP) \in \refinement$, then there is no $\kStateUP \in \kSuccessorsP{}{\kStateTP}$.

Let $n \in \naturals$ be the smallest natural number such that there exists $\kStateTP \in \kSuccessorsP{}{\kStateSP}$ such that $(n, \kStateTP) \in \refinement$.
From above we have $\kSuccessorsP{}{\kStateTP} = \emptyset$. 
Then $\kPModelP{\kStateTP} \nentails \possible \top$ so $\kPModelP{\kStateSP} \nentails \necessary \possible \top$ and $\kPModelP{\kStateSP} \nentails \possible \top \land \necessary \possible \top$.

Therefore $\kPModel{\omega} \nentails \somerefs (\possible \top \land \necessary \possible \top)$.

Therefore $\kPModel{\omega}$ is refinement modally distinguishable from $\kPModel{\omega'}$.

Therefore \logicRmlKF{} is strictly more expressive than \logicKF{}.
\end{proof}

\section{Expressivity: bisimulation quantified modal logic}\label{rml-k4-bqml}

In this section we show that \logicRmlKF{} is (non-strictly) less expressive than the bisimulation quantified modal logic \logicBqmlKF{}.
As a corollary we also have that \logicRmlKF{} is non-strictly less expressive than the modal $\mu$-calculus \logicMuKF{}, as \logicBqmlKF{} and \logicMuKF{} are expressively equivalent.
We direct the reader to Appendix~\ref{bqml} for the required technical background for bisimulation quantified modal logics.

We recall that \logicBqml{} extends modal logic with quantifiers over the pointed Kripke models that are bisimilar to the currently considered Kripke model, except for the valuation of a given propositional atom.
This notion of bisimilarity except for a given propositional atom is called $\atomP$-bisimilarity, and is the same as the usual notion of bisimulation except that the condition {\bf atoms-$\atomP$} is relaxed just for the given atom $\atomP$.
In \logicBqml{} the formula $\allbisimsP \phi$ may be read as ``in every $\atomP$-bisimilar Kripke model $\phi$ is true'' and the formula $\somerefsBs \phi$ may be read as ``in some $\atomP$-bisimilar Kripke model $\phi$ is true''.
Similar to refinement modal logic, different variants of \logicBqml{} restrict the $\atomP$-bisimilar Kripke models that the quantifiers consider to Kripke models from a given class of Kripke frames.
So in \logicBqmlKF{} the quantifiers only consider $\atomP$-bisimilar Kripke models from \classKF{}.

In previous chapters we considered \logicRml{} in the settings of \classK{}, \classKFF{}, \classKD{}, and \classS{}, and in each setting showed that \logicRml{} is expressively equivalent to the respective underlying modal logic.
As a consequence we trivially get that \logicRml{} is non-strictly less expressive than \logicBqml{} in these settings.
In the previous section we showed that \logicRmlKF{} is strictly more expressive than \logicKF{}, so we cannot simply show that \logicRmlKF{} is non-strictly less expressive than \logicBqmlKF{} as a consequence of \logicRmlKF{} being expressively equivalent to \logicKF{}.
Bozzelli, et al.~\cite{bozzelli:2014b} previously showed that \logicRmlK{} is non-strictly less expressive than \logicBqmlK{} by demonstrating a translation from \langRml{} to \langBqml{}.
These results are specific to the setting of \classK{}, however are easily adapted to \classKF{}.
Specifically, as \logicRmlK{} and \logicRmlKF{} quantify over different classes of refinements, and \logicBqmlK{} and \logicBqmlKF{} quantify over different classes of bisimilar Kripke models, to adapt the results of Bozzelli, et al.~\cite{bozzelli:2014b} to \logicRmlKF{} we must demonstrate that certain refinements and bisimilar Kripke models in the results belong to \classKF{}.

Bozzelli, et al.~\cite{bozzelli:2014b} partially characterised refinements as bisimulations followed by restrictions of the accessibility relation.
This partial characterisation was more closely related to bisimulation quantification by partially characterising refinements as $\atomP$-bisimulations followed by a restriction of the accessibility relation to $\atomP$, removing edges to states not in the valuation of $\atomP$.
As the bisimulation quantifiers of \logicBqml{} quantify over $\atomP$-bisimilar Kripke models, this allows a characterisation of refinement quantifiers in terms of bisimulation quantifiers.
This characterisation was demonstrated by Bozzelli, et al.~\cite{bozzelli:2014b} using a translation from \langRml{} formulas to \langBqml{} formulas.
This translation replaces each refinement quantifier $\allrefs \phi$ with a bisimulation quantifier $\allbisimsP \phi^\atomP$ where $\atomP$ is a fresh propositional atom, and $\phi^\atomP$ is the formula $\phi$ ``relativised'' with respect to the propositional atom $\atomP$.
The relativisation of a formula with respect to $\atomP$ has the effect of restricting modalities to only consider states in which $\atomP$ is valid, essentially by replacing each modality $\necessary \phi$ with a restricted modality $\necessary (\atomP \implies \phi)$.
Interpreting the relativised formula $\phi^\atomP$ on a Kripke model is equivalent to interpreting the original formula $\phi$ on the Kripke model with its accessibility relation restricted so that states are only related to other states that have the atom $\atomP$ in their valuation.
Thus the formula $\allbisimsP \phi^\atomP$ quantifies over all $\atomP$-bisimilar Kripke models and interprets the formula $\phi$ on each Kripke model with its accessibility relation so-restricted by the atom $\atomP$.
As refinements were partially characterised as $\atomP$-bisimulations followed by a restriction of the accessibility relation to $\atomP$, Bozzelli, et al.~\cite{bozzelli:2014b} showed that $\allbisimsP \phi^\atomP$ is equivalent to $\allrefs \phi$ in the setting of \classK{}.

In this section we mostly restate the results and reasoning by Bozzelli, et al.~\cite{bozzelli:2014b}, with minor modifications to show that the results hold in the setting of \classKF{}.
In line with our previous results we also adapt these results to use our notion of multi-agent refinement rather than the notion of single-agent refinement used by Bozzelli, et al.~\cite{bozzelli:2014b}.

We first define the notion of model restriction that we will use.

\begin{definition}[Model with accessibility restricted by an atomic proposition]
Let $\agentsB \subseteq \agents$ be a set of agents, let $\atomP \in \atoms$ be a propositional atom and let $\kModelAndTuple$ be a Kripke model.
Then the {\em $(\agentsB, \atomP)$-restriction of $\kModel$} is the Kripke model $\kModel[(\agentsB, \atomP)]$ where $\kModel[(\agentsB, \atomP)] = (\kStates, \kAccessibilityP{}, \kValuation)$ where for every $\agentB \in \agentsB$: $\kAccessibilityPB \subseteq \kAccessibilityB$ where $(\kStateS, \kStateT) \in \kAccessibilityPB$ if and only if $(\kStateS, \kStateT) \in \kAccessibilityB$ and $\kPModel{\kStateT} \entails \atomP$; and for every $\agentC \in \agents \setminus \agentsB$: $\kAccessibilityPC = \kAccessibilityC$.
\end{definition}

We note that all Kripke models restricted in such a way are refinements of the original Kripke model.

\begin{lemma}\label{rml-k4-restriction-refinement}
Let $\agentsB \subseteq \agents$ be a set of agents, let $\atomP \in \atoms$ be a propositional atom, let $\kPModel{\kStateS} \in \classKF$ be a Kripke model, and let $\kPModel[(\agentsB, \atomP)]{\kStateS}$ be the $(\agentsB, \atomP)$-restriction of $\kModel$.
Then $\kPModel{\kStateS} \simulatesBs \kPModel[(\agentsB, \atomP)]{\kStateS}$.
\end{lemma}

\begin{proof}
As $\kModel[(\agentsB, \atomP)]$ is defined by removing $\agentsB$-edges from $\kModel$ then this follows directly from Proposition~\ref{subrelations-refinement}.
\end{proof}

We also note that in the setting of \classKF{} all Kripke models restricted in such a way are \classKF{} models.

\begin{lemma}\label{rml-k4-restriction-k4}
Let $\agentsB \subseteq \agents$ be a set of agents, let $\atomP \in \atoms$ be a propositional atom, let $\kModel \in \classKF$ be a Kripke model, and let $\kModel[(\agentsB, \atomP)]$ be the $(\agentsB, \atomP)$-restriction of $\kModel$.
Then $\kModelP \in \classKF$.
\end{lemma}

\begin{proof}
Let $\agentB \in \agentsB$, and let $(\kStateS, \kStateT), (\kStateT, \kStateU) \in \kAccessibilityPB$.
By construction $\kAccessibilityPB \subseteq \kAccessibilityB$ so $(\kStateS, \kStateT), (\kStateT, \kStateU) \in \kAccessibilityB$.
Then $(\kStateS, \kStateU) \in \kAccessibilityB$ follows from the transitivity of $\kAccessibilityB$.
By construction as $(\kStateT, \kStateU) \in \kAccessibilityPB$ then $\kPModel{\kStateU} \entails \atomP$.
Then by construction $(\kStateS, \kStateU) \in \kAccessibilityPB$.
Therefore $\kAccessibilityPB$ is transitive.
Let $\agentC \in \agents \setminus \agentsB$.
By construction $\kAccessibilityPC = \kAccessibilityC$ so as $\kAccessibilityC$ is transitive so is $\kAccessibilityPC$.
Therefore $\kModelP \in \classKF$.
\end{proof}

We now adapt a lemma from Bozzelli, et al.~\cite{bozzelli:2014b} to the setting of \classKF{}, partially characterising $\agentsB$-refinements as $(\agentsB, \atomP)$-restrictions of $\atomP$-bisimilar Kripke models.

\begin{lemma}\label{rml-k4-refinement-p-bisimulation}
Let $\agentsB \subseteq \agents$ be a set of agents, let $\atomP \in \atoms$ be a propositional atom, and let $\kPModelAndTuple{\kStateS}, \kPModelAndTuplePP{\kStateSPP} \in \classKF$ be pointed Kripke models such that $\kPModel{\kStateS} \simulatesB \kPModelPP{\kStateSPP}$.
There exists a pointed Kripke model $\kPModelP{\kStateSP} \in \classKF$ where $\kPModel[\prime(\agentsB, \atomP)]{\kStateSP}$ is $(\agentsB, \atomP)$-restriction of $\kModelP$, such that $\kPModel{\kStateS} \bisimilar[\atomP] \kPModelP{\kStateS}$ and $\kPModelPP{\kStateSP} \bisimilar[\atomP] \kPModel[\prime(\agentsB, \atomP)]{\kStateSP}$.
\end{lemma}

\begin{proof}
By Lemma~\ref{refinement-expansion} there exists a pointed Kripke model $\kPModelPPP{\kStateSPPP}$ such that $\kPModelPP{\kStateSPP} \bisimilar \kPModelPPP{\kStateSPPP}$ and $\kPModel{\kStateS} \simulates[\agentsB] \kPModelPPP{\kStateSPPP}$ via an expanded $\agentsB$-refinement.
Suppose that there exists a pointed Kripke model $\kPModelP{\kStateSP} \in \classKF$ where $\kPModel[\prime(\agentsB, \atomP)]{\kStateSP}$ is the $(\agentsB, \atomP)$-restriction of $\kModelP$, such that $\kPModel{\kStateS} \bisimilar[\atomP] \kPModelP{\kStateS}$ and $\kPModelPPP{\kStateSPPP} \bisimilar[\atomP] \kPModel[\prime(\agentsB, \atomP)]{\kStateSP}$.
As $\kPModelPP{\kStateSPP} \bisimilar \kPModelPPP{\kStateSPPP}$ then we have that $\kPModelPP{\kStateSPP} \bisimilar[\atomP] \kPModelPPP{\kStateSPPP}$ and so $\kPModelPP{\kStateSPP} \bisimilar[\atomP] \kPModel[\prime(\agentsB, \atomP)]{\kStateSP}$.

Then without loss of generality we assume that $\kModel$ and $\kModelPP$ are disjoint Kripke models such that $\kPModel{\kStateS} \simulates[\agentsB] \kPModelPP{\kStateSPP}$ via an expanded $\agentsB$-refinement $\refinement \subseteq \kStates \times \kStatesPP$.
For every $\kStateTPP \in \kStatesPP$ we denote by $\refinement^{-1}(\kStateTPP)$ the unique $\kStateT \in \kStates$ such that $(\kStateT, \kStateTPP) \in \refinement^{-1}$.

Let $\kPModelAndTupleP{\kStateSPP}$ where:
\begin{eqnarray*}
    \kStatesP &=& \kStates \cup \kStatesPP\\
    \kAccessibilityPB &=& \kAccessibilityB \cup \kAccessibilityPPB \cup \{(\kStateTPP, \kStateU)  \mid (\kStateT, \kStateTPP) \in \refinement, \kStateU \in \kSuccessorsB{\kStateT}\}\\
    \kAccessibilityPC &=& \kAccessibilityC \cup \kAccessibilityPPC\\
    \kValuationP(\atomP) &=& \kStatesPP\\
    \kValuationP(\atomQ) &=& \kValuation(\atomQ) \cup \kValuationPP(\atomQ)
\end{eqnarray*}
where $\agentB \in \agentsB$, $\agentC \in \agents \setminus \agentsB$, and $\atomQ \in \atoms \setminus \{\atomP\}$.

We show that $\kModelP \in \classKF$.
Let $\agentA \in \agents$.
We note that $\kAccessibilityPA$ is composed from the union of $\kAccessibilityA$ and $\kAccessibilityPPA$ (which are relations defined over disjoint domains) and if $\agentA \in \agentsB$ some additional relationships from states in $\kStatesPP$ to states in $\kStates$.
So we consider the following cases:

\begin{description}
    \item[Case $(\kStateT, \kStateU), (\kStateU, \kStateV) \in \kAccessibilityA \subseteq \kAccessibilityPA$:] \hfill\\
        $(\kStateT, \kStateV) \in \kAccessibilityA \subseteq \kAccessibilityPA$ follows from the transitivity of $\kAccessibilityA$.
    \item[Case $(\kStateTPP, \kStateUPP), (\kStateUPP, \kStateVPP) \in \kAccessibilityPPA \subseteq \kAccessibilityPA$:] \hfill\\
        $(\kStateTPP, \kStateVPP) \in \kAccessibilityPPA \subseteq \kAccessibilityPA$ follows from the transitivity of $\kAccessibilityPPA$.
    \item[Case $(\kStateTPP, \kStateUPP) \in \kAccessibilityPPA \subseteq \kAccessibilityPA$ and $(\kStateUPP, \kStateV) \in \kAccessibilityPA$ for some $(\kStateU, \kStateUPP) \in \refinement$, $\kStateV \in \kSuccessorsA{\kStateU}$:] \hfill\\
        As $\refinement$ is an expanded refinement then $\kStateT = \refinement^{-1}(\kStateTPP) \in \kStates$ is the unique state such that $(\kStateT, \kStateTPP) \in \refinement$.
        As $\kStateUPP \in \kSuccessorsPPA{\kStateTPP}$ then by {\bf back-$\agentA$} for $\refinement$ there exists $\kStateX \in \kSuccessors{}{\kStateT}$ such that $(\kStateX, \kStateUPP) \in \refinement$.
        As $\refinement$ is an expanded refinement then there is a unique state $\kStateX \in \kStates$ such that $(\kStateX, \kStateUPP) \in \refinement$, and as $(\kStateU, \kStateUPP) \in \refinement$ then $\kStateX = \kStateU$.
        Then $(\kStateT, \kStateU), (\kStateU, \kStateV) \in \kAccessibilityA$ so by the transitivity of $\kAccessibilityA$ we have $(\kStateT, \kStateV) \in \kAccessibilityA$.
        As $(\kStateT, \kStateTPP) \in \refinement$ and $\kStateV \in \kSuccessors{}{\kStateT}$ then by construction $(\kStateTPP, \kStateV) \in \kAccessibilityPA$.
\end{description}

Therefore $\kAccessibilityPA$ is transitive and $\kModelP \in \classKF$.

We show that $\kPModel{\kStateS} \bisimilar[\atomP] \kPModelP{\kStateSPP}$.
Let $\bisimulation' \subseteq \kStates \times \kStatesP$ where $\bisimulation' = \refinement \cup \{(\kStateT, \kStateT) \mid \kStateT \in \kStates\}$.
We show that $\bisimulation'$ is a $\atomP$-bisimulation between $\kPModel{\kStateS}$ and $\kPModelP{\kStateSPP}$.
Let $\atomQ \in \atoms \setminus \{\atomP\}$, $\agentA \in \agents$.
We show by cases that the relationships in $\bisimulation'$ satisfy the conditions {\bf atoms-$\atomQ$}, {\bf forth-$\agentA$}, and {\bf back-$\agentA$}.

\begin{description}
    \item[Case $(\kStateT, \kStateT) \in \bisimulation'$ where $\kStateT \in \kStates$:]
        \hfill
        \begin{description}
            \item[atoms-$\atomQ$] 
                By construction $\kStateT \in \kValuation(\atomQ)$ if and only if $\kStateT \in \kValuationP(\atomQ)$.
            \item[forth-$\agentA$]
                Let $\kStateU \in \kSuccessorsA{\kStateT}$.
                By construction $\kSuccessorsA{\kStateT} \subseteq \kSuccessorsPA{\kStateT}$.
                Then $\kStateU \in \kSuccessorsPA{\kStateT}$ and by construction $(\kStateU, \kStateU) \in \bisimulation'$.
            \item[back-$\agentA$]
                Let $\kStateUP \in \kSuccessorsPA{\kStateT}$.
                By construction $\kSuccessorsPA{\kStateT} = \kSuccessorsA{\kStateT}$.
                Then $\kStateUP \in \kSuccessorsA{\kStateT}$ and $(\kStateUP, \kStateUP) \in \bisimulation'$.
        \end{description}
    \item[Case $(\kStateT, \kStateTPP) \in \refinement \subseteq \bisimulation'$:]
        \hfill
        \begin{description}
            \item[atoms-$\atomQ$] 
                By {\bf atoms-$\atomQ$} for $\refinement$, $\kStateT \in \kValuation(\atomQ)$ if and only if $\kStateTPP \in \kValuationPP(\atomQ)$.
                As $\atomQ \neq \atomP$ then by construction $\kStateTPP \in \kValuationPP(\atomQ)$ if and only if $\kStateTPP \in \kValuationP(\atomQ)$.
            \item[forth-$\agentA$]
                Let $\kStateU \in \kSuccessorsA{\kStateT}$.

                Suppose that $\agentA \in \agentsB$.
                As $\kStateT \in \refinement^{-1}(\kStateTPP)$ and $\kStateU \in \kSuccessorsA{\kStateT}$ then by construction $\kStateU \in \kSuccessorsPA{\kStateTPP}$.
                By construction $(\kStateU, \kStateU) \in \bisimulation'$.

                Suppose that $\agentA \notin \agentsB$.
                By {\bf forth-$\agentA$} for $\refinement$ there exists $\kStateUPP \in \kSuccessorsPPA{\kStateTPP}$ such that $(\kStateU, \kStateUPP) \in \refinement \subseteq \bisimulation'$.
                By construction $\kSuccessorsPPA{\kStateTPP} \subseteq \kSuccessorsPA{\kStateTPP}$.
                Then $\kStateUPP \in \kSuccessorsPA{\kStateTPP}$.
            \item[back-$\agentA$]
                Let $\kStateUPP \in \kSuccessorsPA{\kStateTPP}$.

                Suppose that $\kStateUPP \in \kStates$.
                By construction $\kStateUPP \in \kSuccessorsA{\refinement^{-1}(\kStateTPP)}$.
                By construction $(\kStateUPP, \kStateUPP) \in \bisimulation'$.

                Suppose that $\kStateUPP \in \kStatesPP$.
                Then by construction $\kStateUPP \in \kSuccessorsPPA{\kStateTPP}$.
                By {\bf back-$\agentA$} for $\refinement$ there exists $\kStateU \in \kSuccessorsA{\kStateT}$ such that $(\kStateU, \kStateUPP) \in \refinement \subseteq \bisimulation'$.
        \end{description}
\end{description}

Therefore $\bisimulation'$ is a $\atomP$-bisimulation between $\kPModel{\kStateS}$ and $\kPModelP{\kStateSPP}$ and $\kPModel{\kStateS} \bisimilar[\atomP] \kPModelP{\kStateSPP}$.

Let $\kPModel[\prime(\agentsB, \atomP)]{\kStateSPP} = ((\kStateSP, \kAccessibility[\prime(\agentsB, \atomP)]{}, \kValuationP), \kStateSPP)$ be the $(\agentsB, \atomP)$-restriction of $\kPModelP{\kStateSPP}$.

We show that $\kPModelPP{\kStateSP} \bisimilar[\atomP] \kPModel[\prime(\agentsB, \atomP)]{\kStateSPP}$.
Let $\bisimulation'' \subseteq \kStatesPP \times \kStatesP$ where $\bisimulation'' = \{\kStateTPP, \kStateTPP) \mid \kStateTPP \in \kStatesPP\}$.
We show that $\bisimulation''$ is a $\atomP$-bisimulation between $\kPModelPP{\kStateSPP}$ and $\kPModel[\prime(\agentsB, \atomP)]{\kStateSPP}$.
Let $\atomQ \in \atoms \setminus \{\atomP\}$, $\agentA \in \agents$, and $(\kStateTPP, \kStateTPP) \in \bisimulation''$ where $\kStateTPP \in \kStatesPP$.
We show by cases that the relationships in $\bisimulation''$ satisfy the conditions {\bf atoms-$\atomQ$}, {\bf forth-$\agentA$}, and {\bf back-$\agentA$}.

\begin{description}
    \item[atoms-$\atomQ$] 
        By construction $\kStateTPP \in \kValuationPP(\atomQ)$ if and only if $\kStateTPP \in \kValuationP(\atomQ)$.
    \item[forth-$\agentA$]
        Let $\kStateUPP \in \kSuccessorsPPA{\kStateTPP}$.
        By construction $\kSuccessorsPPA{\kStateT} \subseteq \kSuccessorsPA{\kStateTPP}$ so $\kStateUPP \in \kSuccessorsPA{\kStateTPP}$.
        By construction as $\kStateUPP \in \kStatesPP$ then $\kStateUPP \in \kValuationP(\atomP)$ so $\kPModelP{\kStateUPP} \entails \atomP$.
        Then $\kStateUPP \in \kSuccessorsA[\prime(\agentsB, \atomP)]{\kStateTPP}$ and by construction $(\kStateUPP, \kStateUPP) \in \bisimulation''$.
    \item[back-$\agentA$]
        Let $\kStateUPP \in \kSuccessorsA[\prime(\agentsB, \atomP)]{\kStateT}$.
        By construction $\kSuccessorsA[\prime(\agentsB, \atomP)]{\kStateT} \subseteq \kSuccessorsPPA{\kStateT}$.
        Then $\kStateUPP \in \kSuccessorsPPA{\kStateTPP}$ and by construction $(\kStateUPP, \kStateUPP) \in \bisimulation''$.
\end{description}

Therefore $\bisimulation''$ is a $\atomP$-bisimulation between $\kPModelPP{\kStateSPP}$ and $\kPModel[\prime(\agentsB, \atomP)]{\kStateSPP}$, and $\kPModelPP{\kStateSP} \bisimilar[\atomP] \kPModel[\prime(\agentsB, \atomP)]{\kStateSPP}$.
\end{proof}

As previously mentioned, Bozzelli, et al.~\cite{bozzelli:2014b} demonstrated a translation from \langRml{} to \langBqml{} that relies on a notion of relativisation of \langBqml{} formulas with respect to a specific agent and propositional atom.
We use essentially the same notion of relativisation, but generalised to match our multi-agent notion of refinement.

\begin{definition}[Relativisation]
Let $\agentsB \subseteq \agents$ be a set of agents, let $\atomP \in \atoms$ be a propositional atom, and let $\phi \in \langBqml$ be a bisimulation quantified modal formula not containing $\atomP$.
The {\em $(\agentsB, \atomP)$-relativisation} $\phi^{(\agentsB, \atomP)}$ of a formula $\phi$ of the agents $\agentsB$ to the propositional atom $\atomP$ is defined inductively as follows:
\begin{eqnarray*}
    \atomQ^{(\agentsB, \atomP)} &=& \atomQ\\
    (\lnot \phi)^{(\agentsB, \atomP)} &=& \lnot (\phi^{(\agentsB, \atomP)})\\
    (\phi \land \psi)^{(\agentsB, \atomP)} &=& \phi^{(\agentsB, \atomP)} \land \psi^{(\agentsB, \atomP)}\\
    (\necessaryB \phi)^{(\agentsB, \atomP)} &=& \necessaryB (\atomP \implies \phi^{(\agentsB, \atomP)})\\
    (\necessaryC \phi)^{(\agentsB, \atomP)} &=& \necessaryC \phi^{(\agentsB, \atomP)}\\
    (\allbisimsQ \phi)^{(\agentsB, \atomP)} &=& \allbisimsQ \phi^{(\agentsB, \atomP)}\\
\end{eqnarray*}
where $\atomQ \neq \atomP$, $\agentB \in \agentsB$, and $\agentC \in \agents \setminus \agentsB$.
\end{definition}

\begin{figure}
    \centering
    \begin{tikzpicture}[>=stealth',shorten >=1pt,auto,node distance=7em,thick]

        \node (s) {$\entails \necessaryA (\atomP \implies \phi)$};
        \node (++) [above of=s] {$\entails \atomP \land \phi$};
        \node (-+) [left of=++] {$\entails \lnot \atomP \land \phi$};
        \node (--) [right of=++] {$\entails \lnot \atomP \land \lnot \phi$};

        \node (s') [right=14em of s] {$\entails \necessaryA \phi$};
        \node (++') [above of=s'] {$\entails \phi$};

      \path[every node/.style={font=\sffamily\small},->]
          (s)  edge node {$\agentA$} (++)
               edge node {$\agentA$} (-+)
               edge [swap] node {$\agentA$} (--)
          (s') edge node {$\agentA$} (++');
    \end{tikzpicture}
    \caption{A schematic showing the intended relationship between the $(\agentA, \atomP)$-restriction of a Kripke model and the $(\agentA, \atomP)$-relativisation of a formula. The Kripke model on the right is the $(\agentA, \atomP)$-restriction of the Kripke model on the left, and the formula on the left is the $(\agentA, \atomP)$-relativisation of the formula on the right.}\label{rml-k4-relativisation-restriction-diagram}
\end{figure}

The intent of $(\agentsB, \atomP)$-relativisation is to capture the notion of $(\agentsB, \atomP)$-restriction syntactically.
Bozzelli, et al.~\cite{bozzelli:2014b} gave a result demonstrating that given a \langBqml{} formula and a Kripke model, the interpretation of the $(\agentsB, \atomP)$-relativisation of the formula on the Kripke model is equivalent to the interpretation of the formula on the $(\agentsB, \atomP)$-restriction of the Kripke model.
This is represented with a schematic diagram in Figure~\ref{rml-k4-relativisation-restriction-diagram}.
We adapt this result to the setting of \classKF{} and for our modified definitions.

\begin{lemma}\label{rml-k4-relativisation-restriction}
Let $\agentsB \subseteq \agents$ be a set of agents, let $\atomP \in \atoms$ be a propositional atom, let $\phi \in \langBqml$ be a bisimulation quantified modal formula, let $\kPModel{\kStateS} \in \classKF$ be a pointed Kripke model, and let $\kPModel[(\agentsB, \atomP)]{\kStateS}$ be the $(\agentsB, \atomP)$-restriction of $\kPModel{\kStateS}$.
Then $\kPModel{\kStateS} \entails_\logicBqmlKF \phi^{(\agentsB, \atomP)}$ if and only if $\kPModel[(\agentsB, \atomP)]{\kStateS} \entails_\logicBqmlKF \phi$.
\end{lemma}

\begin{proof}
We show by induction on the structure of $\phi$ that for every $\kPModel{\kStateS} \in \classKF$: $\kPModel{\kStateS} \entails \phi^{(\agentsB, \atomP)}$ if and only if $\kPModel[(\agentsB, \atomP)]{\kStateS} \entails \phi$, where $\kPModel[(\agentsB, \atomP)]{\kStateS}$ is the $(\agentsB, \atomP)$-restriction of $\kPModel{\kStateS}$.
Let $\kPModel{\kStateS} \in \classKF$.

We consider each case:
\begin{description}
    \item[Case $\phi = \atomQ$ where $\atomQ \neq \atomP$:] \hfill\\
        By definition $\kPModel{\kStateS} \entails \atomQ^{(\agentsB, \atomP)}$ if and only if $\kPModel{\kStateS} \entails \atomQ$.
        By construction $\kPModel{\kStateS} \entails \atomQ$ if and only if $\kPModel[(\agentsB, \atomP)]{\kStateS} \entails \atomQ$.
    \item[Case $\phi = \lnot \psi$ and $\phi = \psi \land \chi$:] \hfill\\
        Follows trivially from the induction hypothesis.
    \item[Case $\phi = \necessaryB \psi$ where $\agentB \in \agentsB$:] \hfill\\
        By definition $\kPModel{\kStateS} \entails (\necessaryB \psi)^{(\agentsB, \atomP)}$ if and only if $\kPModel{\kStateS} \entails \necessaryB (\atomP \implies \psi^{(\agentsB, \atomP)})$.
        By definition $\kPModel{\kStateS} \entails \necessaryB (\atomP \implies \psi^{(\agentsB, \atomP)})$ if and only if for every $\kStateT \in \kSuccessorsB{\kStateS}$: $\kPModel{\kStateT} \entails \atomP$ implies that  $\kPModel{\kStateS} \entails \psi^{(\agentsB, \atomP)}$.
        By the induction hypothesis $\kPModel{\kStateT} \entails \psi^{(\agentsB, \atomP)}$ if and only if $\kPModel[(\agentsB, \atomP)]{\kStateT} \entails \psi$.
        Then for every $\kStateT \in \kSuccessorsB{\kStateS}$: $\kPModel{\kStateT} \entails \atomP$ implies that  $\kPModel{\kStateS} \entails \psi^{(\agentsB, \atomP)}$; if and only if for every $\kStateT \in \kSuccessorsB{\kStateS}$: $\kPModel{\kStateT} \entails \atomP$ implies that $\kPModel[(\agentsB, \atomP)]{\kStateS} \entails \psi$.
        By construction $\kStateT \in \kSuccessorsB[(\agentsB, \atomP)]{\kStateS}$ if and only if $\kStateT \in \kSuccessorsB{\kStateS}$ and $\kPModel{\kStateT} \entails \atomP$.
        Then for every $\kStateT \in \kSuccessorsB{\kStateS}$: $\kPModel{\kStateT} \entails \atomP$ implies that $\kPModel[(\agentsB, \atomP)]{\kStateS} \entails \psi$; if and only if for every $\kStateT \in \kSuccessorsB[(\agentsB, \atomP)]{\kStateS}$: $\kPModel[(\agentsB, \atomP)]{\kStateS} \entails \psi$.
        By definition for every $\kStateT \in \kSuccessorsB[(\agentsB, \atomP)]{\kStateS}$: $\kPModel[(\agentsB, \atomP)]{\kStateS} \entails \psi$; if and only if $\kPModel[(\agentsB, \atomP)]{\kStateS} \entails \necessaryB \psi$.
\pagebreak
    \item[Case $\phi = \necessaryC \psi$ where $\agentC \in \agents \setminus \agentsB$]:
        By definition $\kPModel{\kStateS} \entails (\necessaryC \psi)^{(\agentsB, \atomP)}$ if and only if $\kPModel{\kStateS} \entails \necessaryC \psi^{(\agentsB, \atomP)}$.
        By definition $\kPModel{\kStateS} \entails \necessaryC \psi^{(\agentsB, \atomP)}$ if and only if for every $\kStateT \in \kSuccessorsC{\kStateS}$: $\kPModel{\kStateT} \entails \psi^{(\agentsB, \atomP)}$.
        By the induction hypothesis $\kPModel{\kStateT} \entails \psi^{(\agentsB, \atomP)}$ if and only if $\kPModel[(\agentsB, \atomP)]{\kStateT} \entails \psi$ and by construction $\kSuccessorsC[(\agentsB, \atomP)]{\kStateT} = \kSuccessorsC{\kStateT}$.
        Then for every $\kStateT \in \kSuccessorsC{\kStateS}$: $\kPModel{\kStateT} \entails \psi^{(\agentsB, \atomP)}$; if and only if for every $\kStateT \in \kSuccessorsC[(\agentsB, \atomP)]{\kStateS}$: $\kPModel[(\agentsB, \atomP)]{\kStateT} \entails \psi$.
        By definition for every $\kStateT \in \kSuccessorsC[(\agentsB, \atomP)]{\kStateS}$: $\kPModel[(\agentsB, \atomP)]{\kStateT} \entails \psi^{(\agentsB, \atomP)}$; if and only if $\kPModel[(\agentsB, \atomP)]{\kStateS} \entails \necessaryC \psi$. 
    \item[Case $\phi = \somebisimsQ \psi$ where $\atomQ \neq \atomP$:] \hfill\\
        By definition $\kPModel{\kStateS} \entails (\somebisimsQ \psi)^{(\agentsB, \atomP)}$ if and only if $\kPModel{\kStateS} \entails \somebisimsQ \psi^{(\agentsB, \atomP)}$.
        By definition $\kPModel{\kStateS} \entails \somebisimsQ \psi^{(\agentsB, \atomP)}$ if and only if there exists $\kPModelP{\kStateSP} \in \classKF$ such that $\kPModel{\kStateS} \bisimilar[\atomQ] \kPModelP{\kStateSP}$ and $\kPModelP{\kStateSP} \entails \psi^{(\agentsB, \atomP)}$.
        By the induction hypothesis $\kPModelP{\kStateSP} \entails \psi^{(\agentsB, \atomP)}$ if and only if $\kPModel[\prime(\agentsB, \atomP)]{\kStateSP} \entails \psi$.
        So $\kPModel{\kStateS} \entails (\somebisimsQ \psi)^{(\agentsB, \atomP)}$ if and only if there exists $\kPModelP{\kStateSP} \in \classKF$ such that $\kPModel{\kStateS} \bisimilar[\atomQ] \kPModelP{\kStateSP}$ and $\kPModel[\prime(\agentsB, \atomP)]{\kStateSP} \entails \psi$.

        By definition we also have that $\kPModel[(\agentsB, \atomP)]{\kStateS} \entails \somebisimsQ \psi$ if and only if there exists $\kPModelPP{\kStateSPP} \in \classKF$ such that $\kPModel[(\agentsB, \atomP)]{\kStateS} \bisimilar[\atomQ] \kPModelPP{\kStateSPP}$ and $\kPModelPP{\kStateSPP} \entails \psi$.

        We will show that there exists $\kPModelP{\kStateSP} \in \classKF$ such that $\kPModel{\kStateS} \bisimilar[\atomQ] \kPModelP{\kStateSP}$ and $\kPModel[\prime(\agentsB, \atomP)]{\kStateSP} \entails \psi$ if and only if there exists $\kPModelPP{\kStateSPP} \in \classKF$ such that $\kPModel[(\agentsB, \atomP)]{\kStateS} \bisimilar[\atomQ] \kPModelPP{\kStateSPP}$ and $\kPModelPP{\kStateSPP} \entails \psi$.

        ($\Rightarrow$)
        Suppose there exists $\kPModelP{\kStateSP} \in \classKF$ such that $\kPModel{\kStateS} \bisimilar[\atomQ] \kPModelP{\kStateSP}$, via some $\atomQ$-bisimulation $\bisimulation \subseteq \kStates \times \kStatesP$, and $\kPModel[\prime(\agentsB, \atomP)]{\kStateSP} \entails \psi$.
        We show that $\bisimulation$ is also a $\atomQ$-bisimulation between $\kPModel[(\agentsB, \atomP)]{\kStateS}$ and $\kPModel[\prime(\agentsB, \atomP)]{\kStateSP}$.
        Let $\atomR \in \atoms \setminus \{\atomQ\}$, $\agentB \in \agentsB$, $\agentC \in \agents \setminus \agentsB$, and $(\kStateT, \kStateTP) \in \refinement$:

        \paragraph{atoms-$\atomR$}
        By definition $\kStateT \in \kValuation[(\agentsB, \atomP)](\atomR)$ if and only if $\kStateT \in \kValuation(\atomR)$.
        By {\bf atoms-$\atomR$} for $\bisimulation$ between $\kPModel{\kStateS}$ and $\kPModelP{\kStateSP}$ we have $\kStateT \in \kValuation(\atomR)$ if and only if $\kStateTP \in \kValuationP(\atomR)$.
        By definition $\kStateTP \in \kValuationP(\atomR)$ if and only if $\kStateT \in \kValuation[\prime(\agentsB, \atomP)](\atomR)$.

        \paragraph{forth-$\agentB$}
        Let $\kStateU \in \kSuccessorsB[(\agentsB, \atomP)]{\kStateT}$.
        By definition $\kSuccessorsB[(\agentsB, \atomP)]{\kStateT} \subseteq \kSuccessorsB{\kStateT}$ so $\kStateU \in \kSuccessorsB{\kStateT}$.
        By {\bf forth-$\agentB$} for $\bisimulation$ between $\kPModel{\kStateS}$ and $\kPModelP{\kStateSP}$ there exists $\kStateUP \in \kSuccessorsPB{\kStateTP}$ such that $(\kStateU, \kStateUP) \in \bisimulation$.
        As $\kStateU \in \kSuccessorsB[(\agentsB, \atomP)]{\kStateT}$ then $\kPModel{\kStateU} \entails \atomP$ so $\kStateU \in \kValuation(\atomP)$.
        By {\bf atoms-$\atomP$} for $\bisimulation$ between $\kPModel{\kStateS}$ and $\kPModelP{\kStateSP}$ we have $\kStateU \in \kValuation(\atomP)$ if and only if $\kStateUP \in \kValuationP(\atomP)$.
        Then $\kPModelP{\kStateUP} \entails \atomP$ so by construction $\kStateUP \in \kSuccessorsPB{\kStateTP}$.

        \paragraph{forth-$\agentC$}
        Let $\kStateU \in \kSuccessorsC[(\agentsC, \atomP)]{\kStateT}$.
        By definition $\kSuccessorsC[(\agentsC, \atomP)]{\kStateT} \subseteq \kSuccessorsC{\kStateT}$ so $\kStateU \in \kSuccessorsC{\kStateT}$.
        By {\bf forth-$\agentC$} for $\bisimulation$ between $\kPModel{\kStateS}$ and $\kPModelP{\kStateSP}$ there exists $\kStateUP \in \kSuccessorsPC{\kStateTP}$ such that $(\kStateU, \kStateUP) \in \bisimulation$.
        By construction $\kSuccessorsC[\prime(\agentsC, \atomP)]{\kStateTP} = \kSuccessorsPC{\kStateTP}$ so $\kStateUP \in \kSuccessorsPC{\kStateTP}$.

        \paragraph{back-$\agentB$}
        Follows from symmetrical reasoning to {\bf forth-$\agentB$}.

        \paragraph{back-$\agentC$}
        Follows from symmetrical reasoning to {\bf forth-$\agentC$}.

        Therefore $\kPModel[(\agentsB, \atomP)]{\kStateS} \bisimilar[\atomQ] \kPModel[\prime(\agentsB, \atomP)]{\kStateSP}$ and $\kPModel[\prime(\agentsB, \atomP)]{\kStateSP} \entails \psi$.

        ($\Leftarrow$)
        Suppose there exists $\kPModelPP{\kStateSPP} \in \classKF$ such that $\kPModel[(\agentsB, \atomP)]{\kStateS} \bisimilar[\atomQ] \kPModelPP{\kStateSPP}$, via some $\atomQ$-bisimulation $\bisimulation \subseteq \kStates \times \kStatesPP$, and $\kPModelPP{\kStateSPP} \entails \psi$.
        Let $\kPModelAndTupleP{\kStateSPP}$ where:
        \begin{eqnarray*}
            \kStatesP &=& \kStates \cup \kStatesPP\\
            \kAccessibilityPB &=& \kAccessibilityB \cup \kAccessibilityPPB \cup \{(\kStateTP, \kStateU) \mid (\kStateT, \kStateTP) \in \bisimulation, \kStateU \in \kSuccessorsB{\kStateT} \cap \kValuation(\atomP)\}\\
            \kAccessibilityPC &=& \kAccessibilityC \cup \kAccessibilityPPC\\
            \kValuationP(\atomR) &=& \kValuation(\atomR) \cup \kValuationPP(\atomR)
        \end{eqnarray*}
        where $\agentB \in \agentsB$, $\agentC \in \agents \setminus \agentsB$, $\atomR \in \atoms$.

        We note that $\kModelP \in \classKF$ by the same reasoning as used for the similar construction in Lemma~\ref{rml-k4-refinement-p-bisimulation}.

        We show that $\kPModel{\kStateS} \bisimilar[\atomQ] \kPModelP{\kStateSPP}$.
        Let $\bisimulation' \subseteq \kStates \times \kStatesP$ where $\bisimulation' = \bisimulation \cup \{(\kStateT, \kStateT) \mid \kStateT \in \kStates\}$.
        We show that $\bisimulation'$ is a $\atomQ$-bisimulation between $\kPModel{\kStateS}$ and $\kPModelP{\kStateSPP}$.
        Let $\atomR \in \atoms \setminus \{\atomQ\}$, $\agentB \in \agentsB$, $\agentC \in \agents \setminus \agentsB$.
        We show by cases that the relationships in $\bisimulation'$ satisfy the conditions {\bf atoms-$\atomP$}, {\bf forth-$\agentB$}, {\bf back-$\agentB$}, {\bf forth-$\agentC$}, and {\bf back-$\agentC$}.
        \begin{description}
            \item[{Case $(\kStateT, \kStateTPP) \in \bisimulation \subseteq \bisimulation'$:}]
                \hfill
                \begin{description}
                    \item[atoms-$\atomR$] 
                        By construction $\kStateT \in \kValuation(\atomR)$ if and only if $\kStateT \in \kValuation[(\agentsB, \atomP)](\atomR)$.
                        By {\bf atoms-$\atomR$} for $\bisimulation$ we have $\kStateT \in \kValuation[(\agentsB, \atomP)](\atomR)$ if and only if $\kStateTPP \in \kValuationPP(\atomR)$.
                        By construction $\kStateTPP \in \kValuationPP(\atomR)$ if and only if $\kStateTPP \in \kValuationP(\atomR)$.
                    \item[forth-$\agentB$]
                        Let $\kStateU \in \kSuccessorsB{\kStateT}$.

                        Suppose that $\kStateU \in \kValuation(\atomP)$.
                        Then by construction $\kStateU \in \kSuccessorsB[(\agentsB, \atomP)]{\kStateT}$.
                        By {\bf forth-$\agentB$} for $\bisimulation$ there exists $\kStateUPP \in \kSuccessorsPPB{\kStateTPP} \subseteq \kSuccessorsPB{\kStateTPP}$ such that $(\kStateU, \kStateUPP) \in \bisimulation \subseteq \bisimulation'$.
                        Suppose that $\kStateU \notin \kValuation(\atomP)$.
                        Then by construction $\kStateU \in \kSuccessorsPB{\kStateT}$ and $(\kStateU, \kStateU) \in \bisimulation'$.
                    \item[back-$\agentB$]
                        Let $\kStateUPP \in \kSuccessorsPB{\kStateTPP}$.
                        By construction as $\kStateUPP \in \kStatesPP$ then $\kStateUPP \in \kSuccessorsPPB{\kStateTPP}$.
                        By {\bf forth-$\agentB$} for $\bisimulation$ there exists $\kStateU \in \kSuccessorsB[(\agentsB, \atomP)] \subseteq \kSuccessorsB{\kStateT}$ such that $(\kStateU, \kStateUPP) \in \bisimulation \subseteq \bisimulation'$.
                    \item[forth-$\agentC$]
                        Let $\kStateU \in \kSuccessorsC{\kStateT}$.
                        By construction $\kStateU \in \kSuccessorsC[(\agentsB, \atomP)]{\kStateT}$.
                        By {\bf forth-$\agentC$} for $\bisimulation$ there exists $\kStateUPP \in \kSuccessorsPPC{\kStateTPP} \subseteq \kSuccessorsPC{\kStateTPP}$ such that $(\kStateU, \kStateUPP) \in \bisimulation \subseteq \bisimulation'$.
                    \item[back-$\agentC$]
                        Follows from the same reasoning as {\bf back-$\agentB$}.
                \end{description}
            \item[{Case $(\kStateT, \kStateT) \in \bisimulation'$ where $\kStateT \in \kStates$:}]
                \hfill
                \begin{description}
                    \item[atoms-$\atomR$] 
                        By construction $\kStateT \in \kValuation(\atomR)$ if and only if $\kStateT \in \kValuationP(\atomR)$.
                    \item[forth-$\agentB$]
                        Let $\kStateU \in \kSuccessorsB{\kStateT}$.
                        By construction $\kStateU \in \kSuccessorsPB{\kStateT}$ and $(\kStateU, \kStateU) \in \bisimulation'$.
                    \item[back-$\agentB$]
                        Let $\kStateU \in \kSuccessorsPB{\kStateT}$.
                        By construction $\kStateU \in \kSuccessorsB{\kStateT}$ and $(\kStateU, \kStateU) \in \bisimulation'$.
                    \item[forth-$\agentC$]
                        Follows from the same reasoning as {\bf forth-$\agentB$}.
                    \item[back-$\agentC$]
                        Follows from the same reasoning as {\bf back-$\agentB$}.
                \end{description}
        \end{description}

        Therefore $\bisimulation'$ is a $\atomQ$-bisimulation between $\kPModel{\kStateS}$ and $\kPModelP{\kStateSPP}$ and $\kPModel{\kStateS} \bisimilar[\atomQ] \kPModelP{\kStateSPP}$.

        We show that $\kPModel{\kStateS} \bisimilar \kPModel[\prime(\agentsB, \atomP)]{\kStateSPP}$.
        Let $\bisimulation'' \subseteq \kStatesPP \times \kStatesP$ where $\bisimulation'' = \{(\kStateTPP, \kStateTPP) \mid (\kStateT, \kStateTPP) \in \bisimulation\}$.
        We show that $\bisimulation'$ is a bisimulation between $\kPModelPP{\kStateSPP}$ and $\kPModel[\prime(\agentsB, \atomP)]{\kStateSPP}$.
        Let $\atomR \in \atoms$, $\agentA \in \agents$, $(\kStateTPP, \kStateTPP) \in \bisimulation''$ for some $(\kStateT, \kStateTPP) \in \bisimulation$.
        We show that the relationships in $\bisimulation'$ satisfy the conditions {\bf atoms-$\atomP$}, {\bf forth-$\agentA$}, and {\bf back-$\agentA$}.

        \paragraph{atoms-$\atomR$}
        By construction $\kStateTPP \in \kValuationPP(\atomR)$ if and only if $\kStateTPP \in \kValuationP(\atomR)$.

        \paragraph{forth-$\agentA$}
        Let $\kStateUPP \in \kSuccessorsPPA{\kStateTPP}$.

        Suppose that $\agentA \in \agentsB$.
        By construction there exists $\kStateT \in \kStates$ such that $(\kStateT, \kStateTPP) \in \bisimulation$.
        By {\bf back-$\agentA$} for $\bisimulation$ there exists $\kStateU \in \kSuccessorsA[(\agentsB, \atomP)]{\kStateT}$ such that $(\kStateU, \kStateUPP) \in \bisimulation$.
        By construction $\kStateU \in \kValuation(\atomP)$.
        By {\bf atoms-$\atomP$} for $\bisimulation''$ we have $\kStateUPP \in \kValuation(\atomP)$.
        By construction $\kStateUPP \in \kSuccessorsA[\prime(\agentsB, \atomP)]{\kStateTPP}$ and $(\kStateUPP, \kStateUPP) \in \bisimulation''$.

        Suppose that $\agentA \notin \agentsB$.
        By {\bf back-$\agentA$} for $\bisimulation$ there exists $\kStateU \in \kSuccessorsA[(\agentsB, \atomP)]{\kStateT}$ such that $(\kStateU, \kStateUPP) \in \bisimulation$.
        By construction $\kSuccessorsA[\prime(\agentsB, \atomP)]{\kStateTPP} = \kSuccessorsPPA{\kStateTPP}$.
        Then $\kStateUPP \in \kSuccessorsA[\prime(\agentsB, \atomP)]{\kStateTPP}$ and by construction $(\kStateUPP, \kStateUPP) \in \bisimulation''$.

        \paragraph{back-$\agentA$}
        Let $\kStateUPP \in \kSuccessorsA[\prime(\agentsB, \atomP)]{\kStateTPP}$.
        By construction there exists $\kStateT \in \kStates$ such that $(\kStateT, \kStateTPP) \in \bisimulation$.
        By {\bf back-$\agentA$} for $\bisimulation$ there exists $\kStateU \in \kSuccessorsA[(\agentsB, \atomP)]{\kStateT}$ such that $(\kStateU, \kStateUPP) \in \bisimulation$.
        By construction as $\kStateUPP \in \kSuccessorsA[\prime(\agentsB, \atomP)]{\kStateTPP}$ then $\kStateUPP \notin \kValuationP(\atomP)$.
        Then $\kStateUPP \in \kSuccessorsPPA{\kStateTPP}$ and by construction $(\kStateUPP, \kStateUPP) \in \bisimulation''$.

        Therefore $\bisimulation''$ is a bisimulation between $\kPModelPP{\kStateSPP}$ and $\kPModel[\prime(\agentsB, \atomP)]{\kStateSPP}$ and $\kPModel{\kStateS} \bisimilar \kPModel[\prime(\agentsB, \atomP)]{\kStateSPP}$.
        As $\kPModelPP{\kStateSPP} \entails \psi$ then by bisimulation invariance we have $\kPModel[\prime(\agentsB, \atomP)]{\kStateSPP} \entails \psi$.

        Therefore $\kPModel{\kStateS} \bisimilar[\atomQ] \kPModelP{\kStateSPP}$ and $\kPModel[\prime(\agentsB, \atomP)]{\kStateSPP} \entails \psi$.

        Therefore $\kPModelP{\kStateSP} \in \classKF$ such that $\kPModel{\kStateS} \bisimilar[\atomQ] \kPModelP{\kStateSP}$ and $\kPModel[\prime(\agentsB, \atomP)]{\kStateSP} \entails \psi$ if and only if there exists $\kPModelPP{\kStateSPP} \in \classKF$ such that $\kPModel[(\agentsB, \atomP)]{\kStateS} \bisimilar[\atomQ] \kPModelPP{\kStateSPP}$ and $\kPModelPP{\kStateSPP} \entails \psi$.

        Therefore $\kPModel{\kStateS} \entails (\somebisimsQ \psi)^{(\agentsB, \atomP)}$ if and only if $\kPModel[(\agentsB, \atomP)]{\kStateS} \entails \somebisimsQ \psi$.
\end{description}

Therefore by induction on the structure of $\psi$ we have for every $\kPModel{\kStateS} \in \classKF$: $\kPModel{\kStateS} \entails \psi^{(\agentsB, \atomP)}$ if and only if $\kPModel[(\agentsB, \atomP)]{\kStateS} \entails \psi$.
\end{proof}

Using relativisation we can define a translation from \langRml{} formulas to \langBqml{} formulas.

\begin{definition}
We define the translation $\tau : \langRml \to \langBqml$ by the following inductive definition:
\begin{eqnarray*}
    \tau(\atomQ) &=& \atomQ\\
    \tau(\lnot \phi) &=& \lnot \tau(\phi)\\
    \tau(\phi \land \psi) &=& \tau(\phi) \land \tau(\psi)\\
    \tau(\necessaryA \phi) &=& \necessaryA \tau(\phi)\\
    \tau(\allrefsBs \phi) &=& \allbisimsP (\tau(\phi))^{(\agentsB, \atomP)}
\end{eqnarray*}
where $\atomQ \in \atoms$, $\agentA \in \agents$, $\agentsB \subseteq \agents$, and $\atomP \in \atoms$ where $\atomP$ is a fresh atom that does not appear in $\tau(\phi)$.
\end{definition}

Finally we can show that this translation is a semantically correct translation from \langRml{} to \langBqml{} under the semantics of \logicRmlKF{} and \logicBqmlKF{}.
The following result is an adaptation of the analogous result by Bozzelli, et al.~\cite{bozzelli:2014b} to the setting of \classKF{}.
We rely on the partial characterisation of $\agentsB$-refinements as $(\agentsB, \atomP)$-restrictions of $\atomP$-bisimilar Kripke models, and on the correspondence between $(\agentsB, \atomP)$-restricted Kripke models and the interpretation of $(\agentsB, \atomP)$-relativised formulas that we demonstrated in the previous lemmas.

\begin{theorem}\label{rml-k4-bqml-translation}
Let $\phi \in \langRml$ be a refinement modal formula.
Then for every $\kPModel{\kStateS} \in \classKF$: $\kPModel{\kStateS} \entails_\logicRmlKF \phi$ if and only if $\kPModel{\kStateS} \entails_\logicBqmlKF \tau(\phi)$.
\end{theorem}

\begin{proof}
We show by induction on the structure of $\phi \in \langRml$ that for every $\kPModel{\kStateS} \in \classKF$: $\kPModel{\kStateS} \entails_\logicRmlKF \phi$ if and only if$\kPModel{\kStateS} \entails_\logicBqmlKF \tau(\phi)$.
The propositional and modal cases follow directly from the semantics of \logicRmlKF{} and \logicBqmlKF{} and the induction hypothesis, so we show only the case involving refinement quantifiers.

Let $\kPModel{\kStateS} \in \classKF$.

($\Rightarrow$)
Suppose that $\kPModel{\kStateS} \entails_\logicRmlKF \somerefsBs \phi$.
Then there exists $\kPModelP{\kStateSP} \in \classKF$ such that $\kPModel{\kStateS} \simulatesBs \kPModelP{\kStateSP}$ and $\kPModelP{\kStateSP} \entails_\logicRmlKF \phi$.
By the induction hypothesis we have $\kPModelP{\kStateSP} \entails_\logicBqmlKF \tau(\phi)$, so $\kPModelP{\kStateSP} \entails_\logicBqmlKF \tau(\phi)$.
By Lemma~\ref{rml-k4-refinement-p-bisimulation} there exists $\kPModelPP{\kStateSPP} \in \classKF$ such that $\kPModel{\kStateS} \bisimilar[\atomP] \kPModelPP{\kStateSPP}$ and $\kPModelP{\kStateSP} \bisimilar[\atomP] \kPModel[\prime\prime(\agentsB, \atomP)]{\kStateSPP}$.
As $\atomP$ does not appear in $\tau(\phi)$, $\kPModelP{\kStateSP} \entails_\logicBqmlKF \tau(\phi)$ and $\kPModelP{\kStateSP} \bisimilar[\atomP] \kPModel[\prime\prime(\agentsB, \atomP)]{\kStateSPP}$ then $\kPModel[\prime\prime(\agentsB, \atomP)]{\kStateSPP} \entails_\logicBqmlKF \tau(\phi)$.
As $\kPModel[\prime\prime(\agentsB, \atomP)]{\kStateSPP} \entails_\logicBqmlKF \tau(\phi)$ then by Lemma~\ref{rml-k4-relativisation-restriction} we have $\kPModelPP{\kStateSPP} \entails_\logicBqmlKF (\tau(\phi))^{(\agentsB, \atomP)}$.
Then there exists $\kPModelPP{\kStateSPP} \in \classKF$ such that $\kPModel{\kStateS} \bisimilar[\atomP] \kPModelPP{\kStateSPP}$ and $\kPModelPP{\kStateSPP} \entails_\logicBqmlKF (\tau(\phi))^{(\agentsB, \atomP)}$.
Therefore $\kPModel{\kStateS} \entails_\logicBqmlKF \somebisimsP (\tau(\phi))^{(\agentsB, \atomP)}$.

($\Leftarrow$)
Suppose that $\kPModel{\kStateS} \entails_\logicBqmlKF \somebisimsP (\tau(\phi))^{(\agentsB, \atomP)}$.
Then there exists $\kPModelP{\kStateSP} \in \classKF$ such that $\kPModel{\kStateS} \bisimilar[\atomP] \kPModelP{\kStateSP}$, via some $\atomP$-bisimulation $\bisimulation \subseteq \kStates \times \kStatesP$, and $\kPModelP{\kStateSP} \entails_\logicBqmlKF (\tau(\phi))^{(\agentsB, \atomP)}$.
By Lemma~\ref{rml-k4-relativisation-restriction} we have $\kPModel[\prime(\agentsB, \atomP)]{\kStateSP} \entails_\logicBqmlKF \tau(\phi)$.
By Lemma~\ref{rml-k4-restriction-refinement} we note that $\kPModelP{\kStateSP} \simulatesBs \kPModel[\prime(\agentsB, \atomP)]{\kStateSP}$, say via some $\agentsB$-refinement $\refinement' \subseteq \kStatesP \times \kStatesP$, and by Lemma~\ref{rml-k4-restriction-k4} we note that $\kPModel[\prime(\agentsB, \atomP)]{\kStateSP} \in \classKF$.
By the induction hypothesis we have $\kPModel[\prime(\agentsB, \atomP)]{\kStateSPP} \entails_\logicRml \phi$.

Let $\kPModel[\prime(\agentsB, \atomP)]{\kStateSP} = \kPModelTuplePP{\kStateSP}$ where:
\begin{eqnarray*}
    \kStatesPP &=& \kStatesP\\
    \kAccessibilityPPA &=& \kAccessibilityA[\prime(\agentsB, \atomP)]\\
    \kValuationPP(\atomP) &=& \{\kStateTP \mid (\kStateT, \kStateUP) \in \bisimulation, (\kStateUP, \kStateTP) \in \refinement', \kStateT \in \kValuation(\atomP)\}\\
    \kValuationPP(\atomQ) &=& \kValuationP(\atomQ)
\end{eqnarray*}
where $\agentA \in \agents$ and $\atomQ \in \atoms \setminus \{\atomP\}$.

We show that $\kPModel{\kStateS} \simulatesBs \kPModel[\prime(\agentsB, \atomP)]{\kStateSP}$.
Let $\refinement'' \subseteq \kStates \times \kStatesPP$ where $\refinement'' = \bisimulation \circ \refinement'$.
We show that $\refinement''$ is a $\agentsB$-refinement from $\kPModel{\kStateS}$ to $\kPModel[\prime(\agentsB, \atomP)]{\kStateSP}$.
Let $\atomQ \in \atoms$, $\agentC \in \agents \setminus \agentsB$, $\agentA \in \agents$, and $(\kStateT, \kStateTP) \in \refinement''$ for some $(\kStateT, \kStateTP) \in \bisimulation$ and $(\kStateTP, \kStateTPP) \in \refinement'$.

\paragraph{atoms-$\atomQ$}
Suppose that $\atomQ = \atomP$.
By hypothesis $(\kStateT, \kStateTP) \in \bisimulation$ and $(\kStateTP, \kStateTPP) \in \refinement'$.
Then by construction $\kStateT \in \kValuation(\atomP)$ if and only if $\kStateTPP \in \kValuationPP(\atomP)$.

Suppose that $\atomQ \neq \atomP$.
By {\bf atoms-$\atomQ$} for $\bisimulation$ we have $\kStateT \in \kValuation(\atomQ)$ if and only if $\kStateTP \in \kValuationP(\atomQ)$.
By {\bf atoms-$\atomQ$} for $\refinement'$ we have $\kStateTP \in \kValuationP(\atomQ)$ if and only if $\kStateTPP \in \kValuationP(\atomQ)$.
By construction $\kStateTPP \in \kValuationPP(\atomQ)$ if and only if $\kStateTPP \in \kValuationP(\atomQ)$.

\paragraph{forth-$\agentC$}
Let $\kStateU \in \kSuccessorsC{\kStateT}$.
By {\bf forth-$\agentC$} for $\bisimulation$ there exists $\kStateUP \in \kSuccessorsPC{\kStateTP}$ such that $(\kStateU, \kStateUP) \in \bisimulation$.
By {\bf forth-$\agentC$} for $\refinement'$ there exists $\kStateUPP \in \kSuccessorsC[\prime(\agentsB, \atomP)]{\kStateTPP} = \kSuccessorsPPC{\kStateTPP}$ such that $(\kStateUP, \kStateUPP) \in \refinement'$.
Then $(\kStateU, \kStateUPP) \in \refinement''$.

\paragraph{back-$\agentA$}
Follows from symmetrical reasoning to {\bf forth-$\agentC$}.

Therefore $\refinement''$ is a $\agentsB$-refinement from $\kPModel{\kStateS}$ to $\kPModel[\prime(\agentsB, \atomP)]{\kStateSP}$ so $\kPModel{\kStateS} \simulatesBs \kPModel[\prime(\agentsB, \atomP)]{\kStateSP}$.
Therefore $\kPModel{\kStateS} \entails \somerefsBs \phi$.
\end{proof}

\begin{corollary}
The logic \logicBqmlKF{} is at least as expressive as \logicRmlKF{}.
\end{corollary}

As \logicBqmlKF{} is expressively equivalent to \logicMuKF{} we also trivially get the following corollary. 

\begin{corollary}
The logic \logicMuKF{} is at least as expressive as \logicRmlKF{}.
\end{corollary}

\section{Expressivity: modal $\mu$-calculus}\label{rml-k4-mu}

In this section we show that \logicRmlKF{} is strictly less expressive than the modal $\mu$-calculus \logicMuKF{}.
As a corollary we also have that \logicRmlKF{} is strictly less expressive than the bisimulation quantified modal logic \logicBqmlKF{}, as \logicMuKF{} and \logicBqmlKF{} are expressively equivalent.
That \logicMuKF{} is at least as expressive follows from the results in the previous section.
To show that \logicRmlKF{} is strictly less expressive than \logicMuKF{} we demonstrate a \classKF{} Kripke model with two states that can be distinguished by the validity of a \langMu{} formula under the semantics of \logicMuKF{}, but that cannot be distinguished by any \langRml{} formula under the semantics of \logicRmlKF{}. 
We direct the reader to Appendix~\ref{mu} for the required technical background for modal $\mu$-calculus.

In a previous section we used a similar strategy to show that \logicRmlKF{} is strictly more expressive than \logicKF{}.
The distinguishing \langRml{} formula that we used corresponds to the semantic property that there exists an infinite path starting from the designated state in a pointed Kripke model.
To show that no \langMl{} formula corresponds to this semantic property we demonstrated a \classKF{} Kripke model with two designated states, one with an infinite path, and one without.
Both states also had terminating paths of length $n$ for every $n \in \naturals$.
The Kripke model was constructed in such a way that the two designated states are $n$-bisimilar for all $n \in \naturals$, and so they agree on the interpretation of all \langMl{} formulas.
However as one designated state has an infinite path and the other doesn't, they disagree on the interpretation of the given \langRml{} formula.

We use a very similar semantic property and construction here to show that \logicRmlKF{} is strictly less expressive than \logicMuKF{}.
The distinguishing \langMu{} formula that we will use corresponds roughly to the semantic property that there exists an infinite path starting from the designated state in a pointed Kripke model, along which $\possible \necessary \atomP$ and $\possible \necessary \lnot \atomP$ are satisfied infinitely often.
To show that no \langRml{} formula corresponds to this semantic property we demonstrate a \classKF{} Kripke model with two designated states, one with an appropriate infinite path, and one without.
Both states also had terminating paths of length $n$ for every $n \in \naturals$, along which $\possible \necessary \atomP$ and $\possible \necessary \lnot \atomP$ are satisfied $n$ times.
To show that the two states agree on the interpretation of all \langRml{} formulas we use a notion similar to $n$-bisimilarity, which we call $n$-mutual refinements.
We show that the two designated states of the Kripke model are $n$-mutual refinements and therefore they agree on the interpretation of all \langRml{} formulas. 
However as one designated state has an appropriate infinite path and the other doesn't, they disagree on the interpretation of the given \langMu{} formula.

We first define $n$-mutual refinements.

\begin{definition}[$n$-mutual refinements]
Let $\kPModelAndTuple{\kStateS} \in \classK$ and $\kPModelAndTupleP{\kStateSP} \in \classK$ be pointed Kripke models.

We say that $\kPModel{\kStateS}$ and $\kPModelP{\kStateSP}$ are {\em $0$-mutual refinements} and we write $\kPModel{\kStateS} \bisimref[0] \kPModelP{\kStateSP}$ if and only if for every $\emptyset \subset \agentsB \subseteq \agents$: $\kPModel{\kStateS} \refinesBs \kPModelP{\kStateSP}$ and $\kPModel{\kStateS} \simulatesBs \kPModelP{\kStateSP}$.

We say that $\kPModel{\kStateS}$ and $\kPModelP{\kStateSP}$ are {\em $n$-mutual refinements} for some $n \in \naturals$ and we write $\kPModel{\kStateS} \bisimref[n] \kPModelP{\kStateSP}$ if and only if for every $\agentA \in \agents$ the following, {\bf mutual refinements}, {\bf forth-$\agentA$} and {\bf back-$\agentA$} hold:

\paragraph{mutual refinements} 
For every $\emptyset \subset \agentsB \subseteq \agents$: $\kPModel{\kStateS} \refinesBs \kPModelP{\kStateSP}$ and $\kPModel{\kStateS} \simulatesBs \kPModelP{\kStateSP}$ 

\paragraph{forth-$a$}
For every $\kStateT \in \kStateS \kAccessibility{\agentA}$ there exists $\kStateTP \in \kStateSP \kAccessibilityP{\agentA}$ such that $\kPModel{\kStateT} \bisimref[n-1] \kPModelP{\kStateTP}$.

\paragraph{back-$a$}
For every $\kStateTP \in \kStateSP \kAccessibilityP{\agentA}$ there exists $\kStateT \in \kStateS \kAccessibility{\agentA}$ such that $\kPModel{\kStateT} \bisimref[n-1] \kPModelP{\kStateTP}$.
\end{definition}

We show that if two pointed Kripke models are $n$-mutual refinements then they agree on the interpretation of all \langRml{} formulas of modal depth up to $n$.

\begin{lemma}
Let $\classC$ be a class of Kripke frames,
let $n \in \naturals$,
let $\phi \in \langRml$ such that $d(\phi) \leq n$
and let $\kPModelAndTuple{\kStateS}, \kPModelAndTupleP{\kStateSP} \in \classC$ be pointed Kripke models
such that $\kPModel{\kStateS} \bisimref[n] \kPModelP{\kStateSP}$.
Then $\kPModel{\kStateS} \entails_\logicRmlC \phi$ if and only if $\kPModelP{\kStateSP} \entails_\logicRmlC \phi$.
\end{lemma}

\begin{proof}
We proceed by induction on the modal depth and structure of $\phi$.

Suppose that $\phi = \atomP$ where $\atomP \in \atoms$. 
As $\kPModel{\kStateS}$ and $\kPModelP{\kStateSP}$ are $n$-mutual refinements then $\kPModel{\kStateS} \refines \kPModelP{\kStateSP}$ and from {\bf atoms-$\atomP$} we have that $\kStateS \in \kValuation(\atomP)$ if and only if $\kStateSP \in \kValuationP(\atomP)$ and therefore $\kPModel{\kStateS} \entails \atomP$ if and only if $\kPModelP{\kStateSP} \entails \atomP$.

Suppose that $\phi = \neg \psi$ or $\phi = \psi \land \chi$.
These cases follow directly from the induction hypothesis.

Suppose that $\phi = \necessaryA \psi$ and $\kPModel{\kStateS} \entails \necessaryA \psi$.
Then for every $\kStateT \in \kStateS \kAccessibility{\agentA}$ we have $\kPModel{\kStateT} \entails \psi$.
Let $\kStateTP \in \kStateSP \kAccessibility{\agentA}$.
By {\bf back-$\agentA$} there exists $\kStateT \in \kStateS \kAccessibility{\agentA}$ such that $\kPModel{\kStateT}$ is $(n-1)$-mutual refinements to $\kPModelP{\kStateTP}$.
As $\kPModel{\kStateT} \entails \psi$ and $d(\psi) \leq n - 1$ then by the induction hypothesis we have that $\kPModelP{\kStateTP} \entails \psi$.
So for every $\kStateTP \in \kSuccessorsA{\kStateSP}$ we have $\kPModelP{\kStateTP} \entails \psi$.
Therefore $\kPModelP{\kStateSP} \entails \necessaryA \psi$.
The converse follows from symmetrical reasoning.

Suppose that $\phi = \allrefsBs \psi$ and $\kPModel{\kStateS} \entails \allrefsBs \psi$.
Then there exists $\kPModelPP{\kStateSPP} \in \classC$ such that $\kPModel{\kStateS} \simulatesBs \kPModelPP{\kStateSPP}$ and $\kPModelPP{\kStateS} \entails \psi$.
As $\kPModelP{\kStateSP} \simulatesBs \kPModel{\kStateS}$ then by Lemma~\ref{refinements-preorder} we have that $\kPModelP{\kStateSP} \simulatesBs \kPModelPP{\kStateSPP}$.
Therefore $\kPModelP{\kStateSP} \entails \allrefsBs \psi$.
The converse follows from symmetrical reasoning.
\end{proof}

We now show our expressivity result.

\begin{theorem}
The logic \logicRmlKF{} is strictly less expressive than \logicMuKF{}.
\end{theorem}

\begin{proof}
Let $\kModelAndTuple$ be a Kripke model where:
\begin{eqnarray*}
    \kStates &=& \{n, n^+, n^- \mid n \in \naturals\} \cup \{\omega, \omega'\}\\
    \kAccessibility{} &=& \{(n, m), (n, m^+), (n, m^-), (n, n^+), \\&&\quad(n, n^-), (n^+, n^+), (n^-, n^-) \mid n, m \in \naturals, n > m\} \\&&
        \cup \{0, 0^+, 0^-\} \times \{0^+, 0^-\}
        \cup \{(0, 0)\} \\&&
        \cup \{\omega, \omega'\} \times \{n, n^+, n^- \mid n \in \naturals\}
        \cup \{(\omega', \omega')\}\\
    \kValuation(\atomP) &=& \{n, n^+ \mid n \in \naturals\} \cup \{\omega, \omega'\}
\end{eqnarray*}
The model $\kModel$ is represented in Figure~\ref{model-caterpillar}.

\begin{figure}
    \centering
    \begin{tikzpicture}[>=stealth',shorten >=1pt,auto,node distance=7em,thick]

      \node (0) {$0$};
      \node (0+) [above left of=0] {$0^+$};
      \node (0-) [below left of=0] {$0^-$};
      \node (1) [right of=0] {$1$};
      \node (1+) [above left of=1] {$1^+$};
      \node (1-) [below left of=1] {$1^-$};
      \node (2) [right of=1] {$2$};
      \node (2+) [above left of=2] {$2^+$};
      \node (2-) [below left of=2] {$2^-$};
      \node (3) [right of=2] {\ldots};
      \node (omega) [above right of=3] {$\omega$};
      \node (omega') [below right of=3] {$\omega'$};

      \path[every node/.style={font=\sffamily\small},->]
        (0) edge [loop left] node {} (0)
            edge node {} (0+)
            edge node {} (0-)
        (0+) edge [loop above] node  {} (0+)
            edge [<->,bend right] node {} (0-)
        (0-) edge [loop below] node  {} (0-)
        (1) edge node {} (0)
            edge node {} (1+)
            edge node {} (1-)
        (1+) edge [loop above] node {} (1+)
        (1-) edge [loop below] node {} (1-)
        (2) edge node {} (1)
            edge node {} (2+)
            edge node {} (2-)
        (2+) edge [loop above] node {} (2+)
        (2-) edge [loop below] node {} (2-)
        (3) edge node {} (2)
        (omega) edge node {} (3)
        (omega') edge node {} (3)
            edge [loop right] node {} (omega');
    \end{tikzpicture}
    \caption{The model $\kModel$, omitting implied transitive edges.}\label{model-caterpillar}
\end{figure}

To show that $\kPModel{\omega}$ and $\kPModel{\omega'}$ are indistinguishable by the refinement modal logic we show that $\kPModel{\omega} \bisimref[n] \kPModel{\omega'}$ for all $n \in \naturals$. To do so we first show that $\kPModel{\omega}$ and $\kPModel{\omega'}$ are mutual refinements.

That $\kPModel{\omega} \refines \kPModel{\omega'}$ is trivial, as $\omega$ is the same as $\omega'$ except for the reflexive edge, so we need only show that $\kPModel{\omega'} \refines \kPModel{\omega}$.

Let $\refinement \subseteq \kStates \times \kStates$ be defined as follows:
\begin{eqnarray*}
\refinement &=& \{(\kStateS, \kStateS) \mid \kStateS \in \kStates\} \cup
    \{(0, \omega), (0, \omega'), (\omega, \omega'), (\omega', \omega)\} \cup  \\&&\quad
    \{(0, n), (n, 0), (0^+, n^+), (0^-, n^-), (\omega, n), (\omega', n) \mid n \in \naturals\}
\end{eqnarray*}

We show that $\refinement$ satisfies {\bf atoms-$\atomP$} and {\bf back} for every $(\kStateS, \kStateSP) \in \refinement$.

\begin{description}
    \item[Case: $(\kStateS, \kStateS) \in \refinement$]\hfill
        \begin{description}
            \item[atoms-$\atomP$]
                Trivial.
            \item[back]
                Let $\kStateT \in \kSuccessors{}{\kStateS}$. 
                By construction $(\kStateT, \kStateT) \in \refinement$.
        \end{description}
    \item[Case: $(0, n) \in \refinement$ where $n \in \naturals$:]\hfill
        \begin{description}
            \item[atoms-$\atomP$]
                By construction $0 \in \kValuation(\atomP)$ and $n \in \kValuation(\atomP)$.
            \item[back]
                Let $k^* \in \kSuccessors{}{n}$.

                Suppose that $k^* = k$ where $k \in \naturals$.
                By construction $0 \in \kSuccessors{}{0}$ and $(0, k) \in \refinement$.

                Suppose that $k^* = k^+$ where $k \in \naturals$.
                By construction $0^+ \in \kSuccessors{}{0}$ and $(0^+, k^+) \in \refinement$.

                Suppose that $k^* = k^-$ where $k \in \naturals$.
                By construction $0^- \in \kSuccessors{}{0}$ and $(0^-, k^-) \in \refinement$.
        \end{description}
    \item[Case: $(n, 0) \in \refinement$ where $n \in \naturals$:]\hfill
        \begin{description}
            \item[atoms-$\atomP$]
                By construction $n \in \kValuation(\atomP)$ and $0 \in \kValuation(\atomP)$.
            \item[back]
                Let $k^* \in \kSuccessors{}{0}$.
                By construction $k^* \in \kSuccessors{}{n}$ and $(k^*, k^*) \in \refinement$.
        \end{description}
    \item[Case: $(0^+, n^+) \in \refinement$ where $n \in \naturals$:]\hfill
        \begin{description}
            \item[atoms-$\atomP$]
                By construction $0^+ \in \kValuation(\atomP)$ and $n^+ \in \kValuation(\atomP)$.
            \item[back]
                Let $n^+ \in \kSuccessors{}{n^+}$.
                By construction $0^+ \in \kSuccessors{}{n^+}$ and $(0^+, n^+) \in \refinement$.
        \end{description}
    \item[Case: $(0^-, n^-) \in \refinement$ where $n \in \naturals$:]\hfill
        \begin{description}
            \item[atoms-$\atomP$]
                By construction $0^- \notin \kValuation(\atomP)$ and $n^- \notin \kValuation(\atomP)$.
            \item[back]
                Let $n^- \in \kSuccessors{}{n^0}$.
                By construction $0^- \in \kSuccessors{}{n^-}$ and $(0^-, n^-) \in \refinement$.
        \end{description}
    \item[Case: $(\omega, n) \in \refinement$ where $n \in \naturals$:]\hfill
        \begin{description}
            \item[atoms-$\atomP$]
                By construction $\omega \in \kValuation(\atomP)$ and $n \in \kValuation(\atomP)$.
            \item[back]
                Let $k^* \in \kSuccessors{}{n}$.
                By construction $k^* \in \kSuccessors{}{\omega}$ and $(k^*, k^*) \in \refinement$.
        \end{description}
    \item[Case: $(\omega', n) \in \refinement$ where $n \in \naturals$:]\hfill
        \begin{description}
            \item[atoms-$\atomP$]
                By construction $\omega' \in \kValuation(\atomP)$ and $n \in \kValuation(\atomP)$.
            \item[back]
                Let $k^* \in \kSuccessors{}{n}$.
                By construction $k^* \in \kSuccessors{}{\omega}$ and $(k^*, k^*) \in \refinement$.
        \end{description}
    \item[Case: $(0, \omega) \in \refinement$:]\hfill
        \begin{description}
            \item[atoms-$\atomP$]
                By construction $0 \in \kValuation(\atomP)$ and $\omega \in \kValuation(\atomP)$.
            \item[back]
                Let $k^* \in \kSuccessors{}{\omega}$.
                Suppose that $k^* = k$ where $k \in \naturals$.
                By construction $0 \in \kSuccessors{}{0}$ and $(0, k) \in \refinement$.

                Suppose that $k^* = k^+$ where $k \in \naturals$.
                By construction $0^+ \in \kSuccessors{}{0}$ and $(0^+, k^+) \in \refinement$.

                Suppose that $k^* = k^-$ where $k \in \naturals$.
                By construction $0^- \in \kSuccessors{}{0}$ and $(0^-, k^-) \in \refinement$.
        \end{description}
    \item[Case: $(0, \omega') \in \refinement$:]\hfill
        \begin{description}
            \item[atoms-$\atomP$]
                By construction $0 \in \kValuation(\atomP)$ and $\omega' \in \kValuation(\atomP)$.
            \item[back]
                Let $k^* \in \kSuccessors{}{\omega'}$.

                Suppose that $k^* = \omega'$.
                By construction $0 \in \kSuccessors{}{\omega'}$ and $(0, \omega') \in \refinement$.

                Suppose that $k^* = k$ where $k \in \naturals$.
                By construction $0 \in \kSuccessors{}{0}$ and $(0, k) \in \refinement$.

                Suppose that $k^* = k^+$ where $k \in \naturals$.
                By construction $0^+ \in \kSuccessors{}{0}$ and $(0^+, k^+) \in \refinement$.

                Suppose that $k^* = k^-$ where $k \in \naturals$.
                By construction $0^- \in \kSuccessors{}{0}$ and $(0^-, k^-) \in \refinement$.
        \end{description}
    \item[Case: $(\omega, \omega') \in \refinement$:]\hfill
        \begin{description}
            \item[atoms-$\atomP$]
                By construction $\omega \in \kValuation(\atomP)$ and $\omega' \in \kValuation(\atomP)$.
            \item[back]
                Let $k^* \in \kSuccessors{}{\omega'}$.

                Suppose that $k^* = \omega'$.
                By construction $0 \in \kSuccessors{}{\omega}$ and $(0, \omega') \in \refinement$.

                Suppose that $k^* \neq \omega'$.
                By construction $k^* \in \kSuccessors{}{\omega}$ and $(k^*, k^*) \in \refinement$.
        \end{description}
    \item[Case: $(\omega', \omega) \in \refinement$:]\hfill
        \begin{description}
            \item[atoms-$\atomP$]
                By construction $\omega' \in \kValuation(\atomP)$ and $\omega \in \kValuation(\atomP)$.
            \item[back]
                Let $k^* \in \kSuccessors{}{\omega}$.
                By construction $k^* \in \kSuccessors{}{\omega'}$ and $(k^*, k^*) \in \refinement$.
        \end{description}
\end{description}

We next show that $\kPModel{\omega} \bisimref[n] \kPModel{\omega'}$ for every $n \in \naturals$. 
To show this we show the following intermediate results for every $n \in \naturals$:
\begin{enumerate}
    \item $\kPModel{i} \bisimref[n] \kPModel{j}$ for $i, j$ where $i, j \geq n$.
    \item $\kPModel{n} \bisimref[n] \kPModel{\omega'}$.
    \item $\kPModel{\omega} \bisimref[n] \kPModel{\omega'}$.
\end{enumerate}

We proceed by induction on $n \in \naturals$.

\begin{enumerate}
    \item We show that $\kPModel{i} \bisimref[n] \kPModel{j}$ for $i, j$ where $i, j \geq n$.

    Suppose that $n = 0$. 
    From above, $\kPModel{i} \refines \kPModel{0} \refines \kPModel{j}$ and $\kPModel{i} \simulates \kPModel{0} \simulates \kPModel{j}$ so we have that $\kPModel{i} \bisimref[0] \kPModel{j}$.

    Suppose that $n > 0$.

    \paragraph{mutual refinements}

    From above, $\kPModel{i} \refines \kPModel{0} \refines \kPModel{j}$
    and $\kPModel{i} \simulates \kPModel{0} \simulates \kPModel{j}$.

    \paragraph{forth}

    Let $k^* \in \kSuccessors{}{i}$.

    Suppose that $k^* = k$ where $k \in \naturals$ and $k \geq n - 1$.
    Then from the induction hypothesis $\kPModel{k} \bisimilar [n-1] \kPModel{j - 1}$.

    Suppose that $k^* = k$ where $k \in \naturals$ and $k < n - 1$.
    Then $k < n - 1 < j$ so $k \in \kSuccessors{}{j}$ and we trivially have that $\kPModel{k} \bisimilar \kPModel{k}$.

    Suppose that $k^* = 0^+$.
    Then $0^+ \in \kSuccessors{}{j}$ and we trivially have that $\kPModel{0^+} \bisimilar \kPModel{0^+}$.

    Suppose that $k^* = k^+$ where $k \in \naturals$ and $k > 0$.
    Then $j^+ \in \kSuccessors{}{j}$ and as $j \geq n > 0$ we trivially have that $\kPModel{k^+} \bisimilar \kPModel{k^+}$.

    Suppose that $k^* = k^-$ for $k \in \naturals$. This follows from similar reasoning to the case where $k^* = k^+$.

    \paragraph{back}

    Symmetrical reasoning to {\bf forth}.

    \item We show that $\kPModel{n} \bisimref[n] \kPModel{\omega'}$.

    Suppose that $n = 0$. 
    From above, $\kPModel{n} \refines \kPModel{0} \refines \kPModel{\omega'}$ and $\kPModel{n} \simulates \kPModel{\omega'}$ so we have that $\kPModel{0} \bisimref[0] \kPModel{\omega'}$.

    Suppose that $n > 0$.

    \paragraph{mutual refinements}

    From above, $\kPModel{n} \refines \kPModel{0} \refines \kPModel{\omega'}$ and $\kPModel{n} \simulates \kPModel{\omega'}$.

    \paragraph{forth}

    Let $k \in \kSuccessors{}{n}$. 
    Then $k \in \kSuccessors{}{\omega'}$ and we trivially have that $\kPModel{k} \bisimilar \kPModel{k}$.

    \paragraph{back}

    Let $k \in \kSuccessors{}{\omega'}$. 

    Suppose that $k = \omega'$. 
    Then $n - 1 \in \kSuccessors{}{n}$ and by the induction hypothesis $\kPModel{n - 1} \bisimilar{n - 1} \kPModel{\omega'}$.

    Suppose that $k \neq \omega'$ and $k < n$. Then $k \in \kSuccessors{}{n}$ and we trivially have that $\kPModel{k} \bisimilar \kPModel{k}$.

    Suppose that $k \geq n$. Then $n - 1 \in \kSuccessors{}{n}$ and from above we have that $\kPModel{k} \bisimref[n - 1] \kPModel{n - 1}$.
    
    \item We show that $\kPModel{\omega} \bisimref[n] \kPModel{\omega'}$.

    Suppose that $n = 0$. 
    From above $\kPModel{\omega} \refines \kPModel{\omega'}$ and  $\kPModel{\omega} \simulates \kPModel{\omega'}$ so we have that $\kPModel{\omega} \bisimref[0] \kPModel{\omega'}$.

    Suppose that $n > 0$.

    \paragraph{mutual refinements}

    From above $\kPModel{\omega} \refines \kPModel{\omega'}$ and  $\kPModel{\omega} \simulates \kPModel{\omega'}$.

    \paragraph{forth}

    Let $k \in \kSuccessors{}{\omega}$.
    Then $k \in \kSuccessors{}{\omega'}$ and we trivially have that $\kPModel{k} \bisimilar \kPModel{k}$.

    \paragraph{back}

    Let $k \in \kSuccessors{}{\omega'}$.

    Suppose that $k = \omega'$.
    Then $n - 1 \in \kSuccessors{}{\omega}$ and from above we have that $\kPModel{n - 1} \bisimref[n - 1] \kPModel{\omega'}$.

    Suppose that $k \neq \omega'$.
    Then $k \in \kSuccessors{}{\omega}$ and we trivially have that $\kPModel{k} \bisimilar \kPModel{k}$.
\end{enumerate}

Therefore $\kPModel{\omega} \bisimref[n] \kPModel{\omega'}$ for every $n \in \naturals$.

Let $\phi \in \langRml$ and let $n = d(\phi)$ be the modal depth of $\phi$.
From above $\kPModel{\omega} \bisimref[n] \kPModel{\omega'}$ so $\kPModel{\omega} \entails \phi$ if and only if $\kPModel{\omega'} \entails \phi$.
Therefore $\kPModel{\omega}$ is refinement modally indistinguishable from $\kPModel{\omega'}$.

We next show that the states $\kPModel{\omega}$ and $\kPModel{\omega'}$ are distinguishable by the modal $\mu$-calculus logic formula $\gfp{\varX} (\possible (\varX \land \possible \necessary \atomP) \land \possible (\varX \land \possible \necessary \neg \atomP))$.
Although we do not show it formally here, this distinguishing formula corresponds to the semantic property that there exists an infinite path starting from the designated state in a pointed Kripke model, along which there is always a successor state on the path where $\possible \necessary \atomP$ is satisfied and there is always a successor state on the path where $\possible \necessary \lnot \atomP$ is satisfied.
It should be clear from the construction of $\kModel$ that $\omega'$ has such a infinite path, consisting of repeatedly following the reflexive edge, whereas $\omega$ does not have such an infinite path, as any infinite path from $\omega$ must include one of the reflexive states, either: a $k^+$ state for $k \in \naturals$, where no successors satisfy $\possible \necessary \lnot \atomP$; a $k^-$ state for $k \in \naturals$, where no successors satisfy $\possible \necessary \atomP$; or $0$, where no successors satisfy either $\possible \necessary \atomP$ or $\possible \necessary \lnot \atomP$.

We proceed with model checking using the modal $\mu$-calculus to show that $\kPModel{\omega}$ and $\kPModel{\omega'}$ disagree on the interpretation of this distinguishing formula.

For any assignment $\kAssignment$ we have the following:
\begin{eqnarray*}
    \interpretation[\kAssignment]{\atomP} &=& \{n, n^+ \mid n \in \naturals\} \cup \{\omega, \omega'\}\\
    \interpretation[\kAssignment]{\necessary \atomP} &=& \{n^+ \mid n \in \naturals, n > 0\}\\
    \interpretation[\kAssignment]{\possible \necessary \atomP} &=& \{n, n^+ \mid n \in \naturals, n > 0\} \cup \{\omega, \omega'\}\\
    \interpretation[\kAssignment]{\neg \atomP} &=& \{n^- \mid n \in \naturals\}\\
    \interpretation[\kAssignment]{\necessary \neg \atomP} &=& \{n^- \mid n \in \naturals, n > 0\}\\
    \interpretation[\kAssignment]{\possible \necessary \neg \atomP} &=& \{n, n^- \mid n \in \naturals, n > 0\} \cup \{\omega, \omega'\}
\end{eqnarray*}

For any assignment $\kAssignment$ where $\kAssignment(\varX) = \kStates$ we have the following:
\begin{eqnarray*}
    \interpretation[\kAssignment]{\varX} &=& \kStates\\
    \interpretation[\kAssignment]{\varX \land \possible \necessary \atomP} &=& \{n, n^+ \mid n \in \naturals, n > 0\} \cup \{\omega, \omega'\}\\
    \interpretation[\kAssignment]{\possible (\varX \land \possible \necessary \atomP)} &=& \{n, n^+ \mid n \in \naturals, n > 0\} \cup \{\omega, \omega'\}\\
    \interpretation[\kAssignment]{\varX \land \possible \necessary \neg \atomP} &=& \{n, n^- \mid n \in \naturals, n > 0\} \cup \{\omega, \omega'\}\\
    \interpretation[\kAssignment]{\possible (\varX \land \possible \necessary \neg \atomP)} &=& \{n, n^- \mid n \in \naturals, n > 0\} \cup \{\omega, \omega'\}\\
    \interpretation[\kAssignment]{\possible (\varX \land \possible \necessary \atomP) \land \possible (\varX \land \possible \necessary \neg \atomP)} &=& \{n \mid n \in \naturals, n > 0\} \cup \{\omega, \omega'\}
\end{eqnarray*}

For any assignment $\kAssignment$ where $\kAssignment(\varX) = \{n \mid n \in \naturals, n > m\} \cup \{\omega, \omega'\}$ for some $m \in \naturals$ we have the following:
\begin{eqnarray*}
    \interpretation[\kAssignment]{\varX} &=& \{n \mid n \in \naturals, n > m\} \cup \{\omega, \omega'\}\\
    \interpretation[\kAssignment]{\varX \land \possible \necessary \atomP} &=& \{n \mid n \in \naturals, n > m\} \cup \{\omega, \omega'\}\\
    \interpretation[\kAssignment]{\possible (\varX \land \possible \necessary \atomP)} &=& \{n \mid n \in \naturals, n > m + 1\} \cup \{\omega, \omega'\}\\
    \interpretation[\kAssignment]{\varX \land \possible \necessary \neg \atomP} &=& \{n \mid n \in \naturals, n > m\} \cup \{\omega, \omega'\}\\
    \interpretation[\kAssignment]{\possible (\varX \land \possible \necessary \neg \atomP)} &=& \{n \mid n \in \naturals, n > m + 1\} \cup \{\omega, \omega'\}\\
    \interpretation[\kAssignment]{\possible (\varX \land \possible \necessary \atomP) \land \possible (\varX \land \possible \necessary \neg \atomP)} &=& \{n \mid n \in \naturals, n > m + 1\} \cup \{\omega, \omega'\}
\end{eqnarray*}

For any assignment $\kAssignment$ where $\kAssignment(\varX) = \{\omega, \omega'\}$ for some $m \in \naturals$ we have the following:
\begin{eqnarray*}
    \interpretation[\kAssignment]{\varX} &=& \{\omega, \omega'\}\\
    \interpretation[\kAssignment]{\varX \land \possible \necessary \atomP} &=& \{\omega, \omega'\}\\
    \interpretation[\kAssignment]{\possible (\varX \land \possible \necessary \atomP)} &=& \{\omega'\}\\
    \interpretation[\kAssignment]{\varX \land \possible \necessary \neg \atomP} &=& \{\omega, \omega'\}\\
    \interpretation[\kAssignment]{\possible (\varX \land \possible \necessary \neg \atomP)} &=& \{\omega'\}\\
    \interpretation[\kAssignment]{\possible (\varX \land \possible \necessary \atomP) \land \possible (\varX \land \possible \necessary \neg \atomP)} &=& \{\omega'\}
\end{eqnarray*}

Therefore for any assignment $\kAssignment$ we have that: $$\interpretation[\kAssignment]{\gfp{\varX} (\possible (\varX \land \possible \necessary \atomP) \land \possible (\varX \land \possible \necessary \neg \atomP))} = \{\omega'\}$$

Therefore $\kPModel{\omega'} \entails \gfp{\varX} (\possible (\varX \land \possible \necessary \atomP) \land \possible (\varX \land \possible \necessary \neg \atomP))$, 
but $\kPModel{\omega} \nentails \gfp{\varX} (\possible (\varX \land \possible \necessary \atomP) \land \possible (\varX \land \possible \necessary \neg \atomP))$.

Therefore $\kPModel{\omega}$ is distinguishable from $\kPModel{\omega'}$ using the modal $\mu$-calculus.

Therefore \logicRmlKF{} is strictly less expressive than \logicMuKF{}.
\end{proof}

As \logicMuKF{} is expressively equivalent to \logicBqmlKF{} we also trivially get the following corollary. 

\begin{corollary}
The logic \logicRmlKF{} is strictly less expressive than \logicBqmlKF{}.
\end{corollary}
