\section{Syntax and semantics}\label{aaml-semantics}

In this section we introduce the syntax and semantics of the arbitrary action model logic.
Like our treatment of \logicRml{}, we consider \logicAaml{} in different settings, including \classK{}, \classKFF{}, and \classS{}.
The definitions that we give here generalise to these different settings.
Unlike our treatment of \logicRml{}, we don't give any semantic results that are common to all of these settings.
As our main results in \logicAaml{} are to show that the action model quantifiers of \logicAaml{} are equivalent to the refinement quantifiers of \logicRml{} in the settings we consider, all of the semantic results from \logicRml{} also apply to \logicAaml{} in these settings.
For the same reason we use the same syntax for action model quantifiers and refinement quantifiers.

We begin with a definition of the syntax of \logicAaml{}.
As in action model logic, the syntax of \logicAaml{} is parameterised by a class of Kripke models, \classC{}, and a set of action signatures, \aSignatureFamily{}.

\begin{definition}[Language of arbitrary action model logic]
Let \aSignatureFamily{} be a non-empty, countable set of action signatures.
The {\em language of arbitrary action model logic} with action signatures \aSignatureFamily{}, $\langAaml(\aSignatureFamily)$, is inductively defined as:
$$
\phi ::= 
    \atomP \mid
    \lnot \phi \mid
    \phi \land \phi \mid
    \necessaryA \phi \mid
    \actionA{\aSignature \aStatesT, \phi, \dots, \phi} \phi \mid
    \allactsBs \phi
$$
where $\atomP \in \atoms$, $\agentA \in \agents$, $\agentsB \subseteq \agents$, $\aSignatureAndTuple \in \aSignatureFamily$, $\aStatesT \subseteq \aStates$, and the number of parameters to a given action signature $\aSignature$ is determined by the number of designated actions in the action signature.
\end{definition}

We use all of the standard abbreviations from modal logic, in addition to the abbreviations $\somerefsBs \phi ::= \lnot \allrefsBs \lnot \phi$, $\allrefs \phi ::= \allrefsAs \phi$, and $\allrefsA \phi ::= \allrefs[\{\agentA\}] \phi$.

The formula $\allacts \phi$ may be read as ``every action model results in $\phi$ becoming true'' and the formula $\someacts \phi$ may be read as ``some action model results in $\phi$ becoming true''.

The use of the subscript $\agentsB$ in the quantifiers $\allactsBs$ and $\someactsBs$ restricts the action models under consideration to action models that result in a $\agentsB$-refinement of the original Kripke model.
So the formula $\allactsBs \phi$ may be read as ``every action model results in $\phi$ becoming true if it results in a $\agentsB$-refinement'' and the formula $\someactsBs \phi$ may be read as ``some action model results in $\phi$ becoming true and results in a $\agentsB$-refinement''.
This addition is mostly for the purposes of showing a full correspondence between action model quantifiers and refinement quantifiers.
Although we do not consider it in greater detail here, the notion of restricting the results of executing action models to $\agentsB$-refinements seems like it would be useful.
For example, $\agentsB$-refinements can be partially characterised as those Kripke models that preserve the truth of $\agentsB$-positive formulas, restricting the learning of new information to agents in $\agentsB$.
So an alternative reading of the formula $\allactsBs \phi$ may be ``every action model where only agents in $\agentsB$ learn new information results in $\phi$ becoming true'' and an alternative reading of the formula $\someactsBs \phi$ may be ``some action model where only agents in $\agentsB$ learn new information results in $\phi$ becoming true''.

We define the semantics of \logicAaml{}.

\begin{definition}[Semantics of arbitrary action model logic]
Let \classC{} be a class of Kripke models and let \aSignatureFamily{} be a non-empty, countable set of action signatures, let $\phi \in \langAaml(\aSignatureFamily)$, and let $\kPModelAndTuple{\kStateS} \in \classC$ be a pointed Kripke model.
The interpretation of the formula $\phi$ in the logic \logicAamlC{} on the pointed Kripke model $\kPModel{\kStateS}$ is the same as its interpretation in action model logic, defined in Definition~\ref{aml-semantics}, with the additional inductive case:
$$
\begin{array}{lcl}
    \kPModel{\kStateS} \entails \allactsBs \phi & \text{ iff } & \text{for every } \aPModel{\aStateS} \in \aSignatureFamily \text{ if } \kPModel{\kStateS} \entails \aPrecondition(\aStateS) \text{ and } \kPModel{\kStateS} \simulatesBs \kPModel{\kStateS} \exec \aPModel{\aStateS}\\&&\quad \text{ then } \kPModel{\kStateS} \exec \aPModel{\aStateS} \entails \phi
\end{array}
$$
\end{definition}

We are interested in the following variants of arbitrary action model logic:
\begin{itemize}
    \item \logicAamlK{} interpreted over the class of \classK{} Kripke frames and the language of arbitrary action model logic $\langAaml(\classK)$ with action signatures defined on the class of finite \classK{} Kripke frames.
    \item \logicAamlKFF{} interpreted over the class of \classKFF{} Kripke frames and the language of arbitrary action model logic $\langAaml(\classKFF)$ with action signatures defined on the class of finite \classKFF{} Kripke frames.
    \item \logicAamlS{} interpreted over the class of \classS{} Kripke frames and the language of arbitrary action model logic $\langAaml(\classS)$ with action signatures defined on the class of finite \classS{} Kripke frames.
\end{itemize}

We note that by Proposition~\ref{action-models-refinements} the result of executing any action model is a $\agents$-refinement, so the action model quantifiers $\allactsAs$ and $\someactsAs$, abbreviated as $\allacts$ and $\someacts$ respectively, correspond to unrestricted action model quantification.
Given this we can observe that the semantics of the action model quantifier $\allacts$ is defined similarly to the semantics of the public announcement quantifier of \logicApal{}.
One notable difference is that whilst \logicApal{} permits public announcements of formulas containing public announcement quantifiers, the public announcement quantifiers do not quantify over public announcements that themselves contain quantifiers, whereas the action model quantifier of \logicAaml{} makes no such restriction.
We will show in the following sections that such a restriction is unnecessary in the settings that we consider; the logics \logicAamlK{}, \logicAamlKFF{}, and \logicAamlS{} are expressively equivalent to their underlying modal logics, \logicK{}, \logicKFF{}, and \logicS{} respectively, so any action model containing quantifiers is equivalent to an action model without quantifiers.
Despite the action model quantifiers quantifying over more epistemic updates that public announcement quantifiers, the public announcement quantifiers are in fact more powerful than action model quantifiers, at least in the setting of \classS{}, as \logicApalS{} is strictly more expressive than \logicS{} and is undecidable, whereas \logicAamlS{} is expressively equivalent to \logicS{} and is decidable.

\todo[inline]{Example of \logicAaml{}}

In the following sections we show that the action model quantifiers of \logicAaml{} are equivalent to the refinement quantifiers of \logicRml{} in the settings of \classK{}, \classKFF{}, and \classS{}.
We show the equivalence by showing that if there exists a refinement where a given formula is satisfied then we can construct a finite action model that results in that formula being satisfied.
The converse we have already shown; if there exists a (possibly infinite) action model that results in a given formula being satisfied then by Proposition~\ref{action-models-refinements} the result of executing the action model is itself a refinement, so there exists a refinement where the formula is satisfied.
We rely heavily on results from action model logic and \logicRml{}, particularly the axioms from both, so we find it useful to define a logic, which we call refinement action model logic (\logicRaml{}), that extends action model logic with refinement quantifiers.

\begin{definition}[Semantics of refinement action model logic]
Let \classC{} be a class of Kripke models and let \aSignatureFamily{} be a non-empty, countable set of action signatures, let $\phi \in \langAaml(\aSignatureFamily)$, and let $\kPModelAndTuple{\kStateS} \in \classC$ be a pointed Kripke model.
The interpretation of the formula $\phi$ in the logic \logicRamlC{} on the pointed Kripke model $\kPModel{\kStateS}$ is the same as its interpretation in action model logic, defined in Definition~\ref{aml-semantics}, with the additional inductive case:
$$
\begin{array}{lcl}
    \kPModel{\kStateS} \entails \allactsBs \phi & \text{ iff } & \text{for every } \kPModelP{\kStateSP} \in \classC \text{ if } \kPModel{\kStateS} \simulatesBs \kPModelP{\kStateSP} \text{ then } \kPModelP{\kStateSP} \entails \phi
\end{array}
$$
\end{definition}

As with \logicAaml{} we are interested in the following variants of refinement action model logic:
\begin{itemize}
    \item \logicRamlK{} interpreted over the class of \classK{} Kripke frames and the language of arbitrary action model logic $\langAaml(\classK)$ with action signatures defined on the class of finite \classK{} Kripke frames.
    \item \logicRamlKFF{} interpreted over the class of \classKFF{} Kripke frames and the language of arbitrary action model logic $\langAaml(\classKFF)$ with action signatures defined on the class of finite \classKFF{} Kripke frames.
    \item \logicRamlS{} interpreted over the class of \classS{} Kripke frames and the language of arbitrary action model logic $\langAaml(\classS)$ with action signatures defined on the class of finite \classS{} Kripke frames.
\end{itemize}

In the following sections we will show that results from action model logic and \logicRml{} apply to the combined logic \logicRaml{}, and use these results to show that formulas of \langAaml{} have the same interpretation in both \logicAaml{} and \logicRaml{}.
