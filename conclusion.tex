\chapter{Conclusion}\label{conclusion}

In this work we considered a number of logics for quantifying over epistemic updates: 
arbitrary action model logic (\logicAaml{}), which quantifies over action models; and
refinement modal logic (\logicRml{}), which quantifies over refinements, which have a partial correspondence with the results of action models, but are more general.
Compared to previous dynamic epistemic logics such as the public announcement logic of Plaza~\cite{plaza:1989} and Gerbrandy and Groenveld~\cite{gerbrandy:1997}, and the action model logic of Baltag, Moss and Solecki~\cite{baltag:1998,baltag:2004}, which reason about the results of specific epistemic updates, the logics we consider reason about the results of arbitrary epistemic updates, allowing us to ask questions such as ``Is there an epistemic update that results in the desired change in knowledge?'', and ``What is a specific epistemic update that results in the desired change in knowledge?''.
Compared to previous dynamic epistemic logics such as the arbitrary public announcement logic (\logicApal{}) of Balbiani, et al.~\cite{balbiani:2007} and the group announcement logic (GAL) of {\AA}gotnes, et al.~\cite{agotnes:2008,agotnes:2010}, which quantify over relatively restricted forms of epistemic updates, the logics we consider quantify over very general forms of epistemic updates.

We justified our interpretation of refinements as a very general form of epistemic updates by partially characterising the refinements of a Kripke model as the Kripke models that preserve the truth of positive formulas.
The preservation of positive formulas ensures that refinements are purely informative and only provide additional information, in line with our general notion of an epistemic update.
The characterisation of refinements as the Kripke models that preserve the truth of positive formulas is analogous to the characterisation of bisimilar Kripke models as the Kripke models that preserve the truth of modal formulas.
This forms a strong case in favour of refinements as a very general notion of epistemic updates.
We further justified our interpretation of refinements as a very general form of epistemic updates by partially characterising the refinements of a Kripke model as the results of executing action models on that Kripke model.
Refinements are actually more general than action models, as there are refinements that do not correspond to the result of any action model, however it turns out that the action model quantifiers of \logicAaml{} are equivalent to the refinement quantifiers of \logicRml{}.

\logicRml{} was first introduced by van Ditmarsch and French~\cite{vanditmarsch:2009}, and initial results were given by van Ditmarsch, French and Pinchinat~\cite{vanditmarsch:2010} in the setting of single-agent \classK{}.
This work has generalised these results to other modal settings, particularly multi-agent settings.
We considered \logicRml{} in the settings of multi-agent \classK{}, \classKF{}, \classKFF{}, \classKD{}, and \classS{}.
In the settings of multi-agent \classK{}, \classKFF{}, \classKD{} and \classS{} we provided sound and complete axiomatisations for \logicRml{}, we showed that \logicRml{} is compact and decidable, and that \logicRml{} is expressively equivalent to modal logic, via a provably correct translation from the language of \logicRml{} to the language of the underlying modal logic.
In the setting of \classKF{} we showed that \logicRml{} is decidable, and that its expressivity lies strictly between that of modal logic and the modal $\mu$-calculus.
We also showed a variety of semantic results that apply to all variants of \logicRml{} and some that apply to several of the variants of \logicRml{} that we considered in this work.

\logicAaml{} was first proposed by Balbiani, et al.~\cite{balbiani:2007} as a possible generalisation for \logicApal{}, and the syntax and semantics of \logicAaml{} and \logicApal{} are accordingly very similar.
We considered \logicAaml{} in the settings of multi-agent \classK{}, \classKFF{}, and \classS{}.
In these settings we provided sound and complete axiomatisations, we showed that \logicAaml{} is compact and decidable, and that \logicAaml{} is expressively equivalent to modal logic, via a provably correct translation from the language of \logicAaml{} to the language of the underlying modal logic.
We achieved these results simply by showing that the action model quantifiers of \logicAaml{} are equivalent to the refinement quantifiers of \logicRml{}, and we 
showed this equivalence by providing a synthesis procedure that, given a desired change in knowledge, constructs a specific action model that will result in the desired change in knowledge whenever a refinement exists where that change in knowledge is satisfied.

There are several immediate avenues for future work based on the results presented in this work.

We have not shown complexity or succinctness results for the multi-agent variants of \logicRml{} and \logicAaml{}.
Bozzelli, van Ditmarsch and Pinchinat~\cite{bozzelli:2014a} gave complexity bounds for the decision problem and succinctness results for the single-agent variant of \logicRmlK{}, and Achilleos and Lampis~\cite{achilleos:2013} provided complexity results for the model-checking problem in addition to tighter complexity bounds for the decision problem, both for the single-agent variant of \logicRmlK{}.
Hales, French and Davies~\cite{hales:2011b} described a decision procedure for the single-agent variants of \logicRmlKD{} and \logicRmlS{}.
Decision procedures for the multi-agent logics \logicRmlK{}, \logicRmlKFF{}, \logicRmlKD{}, and \logicRmlS{}, can be formed by combining the provably correct translation for each with a decision procedure for the respective underlying modal logic.
However the provably correct translations result in a non-elementary increase in formula size, so such decision procedures will have a non-elementary complexity, which is less than ideal.

Although we provided decidability and expressivity results for \logicRmlKF{} we do not yet have a sound and complete axiomatisation.
As \logicRmlKF{} is strictly more expressive than \logicKF{} a provably correct translation from \langRml{} to modal logic is not possible, so a different strategy for proving the completeness of a candidate axiomatisation is required.
A candidate axiomatisation must also greatly differ from the axiomatisations of \logicRmlK{}, \logicRmlKFF{}, \logicRmlKD{}, and \logicRmlS{}, as these axiomatisations admit provably correct translations, which would be unsound in \logicRmlKF{}.

As this work has focussed on generalising \logicRml{} to different modal settings, a natural avenue for future work would be generalisation to even more modal settings.
Bozzelli, et al.~\cite{bozzelli:2014b} suggest that refinement quantifiers may be applicable to any modal logic, not just epistemic modal logics.
A strong candidate for future consideration is \logicRmlSF{}, in the setting of reflexive and transitive Kripke frames.
We have briefly considered expressivity and decidability results for \logicRmlSF{}, and we conjecture that the expressivity results of \logicRmlKF{} can be adapted to \logicRmlSF{}.
We conjecture that the expressivity of \logicRmlSF{} lies strictly between that of the modal logic \logicSF{} and the modal $\mu$-calculus \logicMuSF{}, similar to the expressivity of \logicRmlKF{}.
Work on these results is on-going.

We have not yet considered the addition of common knowledge operators to \logicRml{} and \logicAaml{}.
Many natural questions in epistemic logics and dynamic epistemic logics are about common knowledge, so it seems natural that we would want to consider common knowledge in connection with quantifiers over epistemic updates.
In principle this would allow us to answer questions such as whether or not desired common knowledge is attainable, or how desired common knowledge can be achieved through a specific epistemic update.

Although we provided a synthesis procedure for \logicAaml{} we have not yet considered efficient or optimal synthesis procedures for \logicAaml{}.
The provided synthesis procedure relies on the expressive equivalence of \logicAaml{} with the corresponding underlying modal logic.
Expressive equivalence of \logicAaml{} and modal logic is shown via a provably correct translation that results in a non-elementary increase in formula size, so a synthesis procedure that uses this provably correct translation will have a non-elementary complexity.
In addition to an efficient synthesis procedure it would be desirable to have a synthesis procedure that results in ``optimal'' action models, such as by minimising the overall size of the action model as measured by the number of states, the size of the preconditions, and so on.

Finally, we have not yet considered in detail the relationship between \logicAaml{} and the DEL-sequents of Aucher~\cite{aucher:2011,aucher:2012}.
In contrast to \logicAaml{}, which introduces syntactic quantifiers over action models, the system of DEL-sequents does not extend the syntax or semantics of action model logic with quantifiers.
However the DEL-sequents can answer similar questions to those considered by \logicAaml{} by performing reasoning at the meta-logical level.
We showed most of our results in \logicAaml{} by showing that the action model quantifiers of \logicAaml{} are equivalent to the refinement quantifiers of \logicRml{}, however it may be possible to show similar results by relating the action model quantifiers of \logicAaml{} to the meta-logical reasoning that can be performed with the DEL-sequents.
As we have shown that the action model quantifiers of \logicAaml{} are equivalent to the refinement quantifiers of \logicRml{} this also suggests that results in \logicAaml{} derived from the DEL-sequents would also be valid in \logicRml{}.
