\section{K}

\begin{definition}[Axiomatisation \axiomAamlK{}]
The axiomatisation \axiomAamlK{} is a substitution schema consisting of the axioms and rules of \axiomAmlK{} and the axioms and rules of \axiomRmlK{}.
\end{definition}

\begin{theorem}\label{aaml-k-sound-complete}
The axiomatisation \axiomRmlK{} is sound and strongly complete with respect to the semantics of the logic \logicAamlK{}.
\end{theorem}

\begin{proof}[Proof (Sketch)]
Soundness of the axioms and rules from \axiomAmlK{} and \axiomRmlK{} follow from the same reasoning used to show that they are sound in \logicAmlK{} and \logicRmlK{} respectively.
Strong completeness follows from similar reasoning as in the proof of strong completeness of \axiomRmlK{} in Lemma~\ref{rml-k-sound-complete}.
We note that the axioms of \axiomAmlK{} form a set of reduction axioms that can be used to provably translate any formula containing action model operators but not refinement quantifiers into an equivalent modal formula.
Likewise the axioms of \axiomRmlK{} form a set of reduction axioms that can be used to provably translate any formula containing refinement quantifiers but not action model operators into an equivalent modal formula.
These methods can be combined to provably translate any formula containing action model operators and refinement quantifiers into an equivalent modal formula, by iteratively translating subformulas containing action model operators but not refinement quantifiers using the former method, and subformulas containing refinement quantifiers but not action model operators using the latter method.
\end{proof}

\begin{corollary}\label{aaml-k-expressive-equivalence}
The logic \logicAamlK{} is expressively equivalent to the logic \logicK{}.
\end{corollary}

\begin{corollary}
The logic \logicAamlK{} is compact.
\end{corollary}

\begin{corollary}
The satisfiability problem for the logic \logicAamlK{} is decidable.
\end{corollary}

\begin{lemma}\label{aaml-k-choice}
Let $\agentsB \subseteq \agents$, 
let $\phi = \alpha \lor \beta \in \langAaml$, and 
let $\aPModel[\alpha]{\aStatesT[\alpha]} \in \classAmK$ and $\aPModel[\beta]{\aStatesT[\beta]} \in \classAmK$ be action models such that 
$\entails \actionA{\aPModel[\alpha]{\aStatesT[\alpha]}} \alpha$, 
$\entails \actionE{\aPModel[\alpha]{\aStatesT[\alpha]}} \alpha \iff \someactsBs \alpha$, 
$\entails \actionA{\aPModel[\beta]{\aStatesT[\beta]}} \beta$,
$\entails \actionE{\aPModel[\beta]{\aStatesT[\beta]}} \beta \iff \someactsBs \beta$,
for every $\aStateT[\alpha] \in \aStatesT[\alpha]$, $\kPModel{\kStateS} \in \classK$ if $\kPModel{\kStateS} \entails \aPrecondition[\alpha](\aStateT[\alpha])$ then $\kPModel{\kStateS} \simulatesBs \kPModel{\kStateS} \exec \aPModel[\alpha]{\aStateT[\alpha]}$, and
for every $\aStateT[\beta] \in \aStatesT[\beta]$, $\kPModel{\kStateS} \in \classK$ if $\kPModel{\kStateS} \entails \aPrecondition[\beta](\aStateT[\beta])$ then $\kPModel{\kStateS} \simulatesBs \kPModel{\kStateS} \exec \aPModel[\beta]{\aStateT[\beta]}$.
Then there exists an action model $\aPModel{\aStatesT} \in \classAmK$ such that 
$\entails \actionA{\aPModel{\aStatesT}} \phi$,
$\entails \actionE{\aPModel{\aStatesT}} \phi \iff \someactsBs \phi$, and
for every $\aStateT \in \aStatesT$, $\kPModel{\kStateS} \in \classK$ if $\kPModel{\kStateS} \entails \aPrecondition(\aStateT)$ then $\kPModel{\kStateS} \simulatesBs \kPModel{\kStateS} \exec \aPModel{\aStateT}$.
\end{lemma}

\begin{proof}
Without loss of generality we assume that $\aModel[\alpha]$ and $\aModel[\beta]$ are disjoint.
We construct the action model $\aPModel{\aStatesT} = \aPModel[\alpha]{\aStatesT[\alpha]} \choice \aPModel[\beta]{\aStatesT[\beta]} = \aPModelTuple{\aStatesT}$ where the disjoint union of action models is defined in Definition~\ref{aml-choice}.  
As $\aModel$ is formed by the disjoint union of $\aModel[\alpha]$ and $\aModel[\beta]$ we note that each state of $\aModel[\alpha]$ and $\aModel[\beta]$ is bisimilar to the corresponding state in $\aModel$.

We first show that for every $\aStateT \in \aStatesT$, $\kPModel{\kStateS} \in \classK$ if $\kPModel{\kStateS} \entails \aPrecondition(\aStateT)$ then $\kPModel{\kStateS} \simulatesBs \kPModel{\kStateS} \exec \aPModel{\aStateT}$.
Let $\gamma \in \{\alpha, \beta\}$, $\aStateT[\gamma] \in \aStatesT[\gamma] \subseteq \aStatesT$, and $\kPModel{\kStateS} \in \classK$ such that $\kPModel{\kStateS} \entails \aPrecondition(\aStateT[\gamma])$.
By construction $\aPrecondition(\aStateT[\gamma]) = \aPrecondition[\gamma](\aStateT[\gamma])$ and so $\kPModel{\kStateS} \entails \aPrecondition[\gamma](\aStateT[\gamma])$. 
By hypothesis then $\kPModel{\kStateS} \simulatesBs \kPModel{\kStateS} \exec \aPModel[\gamma]{\aStateT[\gamma]}$.
From above $\aPModel{\aStateT[\gamma]} \bisimilar \aPModel[\gamma]{\aStateT[\gamma]}$ and so from Proposition~\ref{action-bisimulation-results} we have that $\kPModel{\kStateS} \exec \aPModel{\aStateT[\gamma]} \bisimilar \kPModel{\kStateS} \exec \aPModel[\gamma]{\aStateT[\gamma]}$.
From Corollary~\ref{bisimilar-refinement} and Proposition~\ref{refinements-preorder} we have that $\kPModel{\kStateS} \simulatesBs \kPModel{\kStateS} \exec \aPModel{\aStateT[\gamma]}$.

We next show that $\entails \actionA{\aPModel{\aStatesT}} \phi$.
\begin{eqnarray}
    &&\entails \actionA{\aPModel[\alpha]{\aStatesT[\alpha]}} \alpha \land \actionA{\aPModel[\beta]{\aStatesT[\beta]}} \beta \label{aaml-k-choice-1}\\
    &&\entails \actionA{\aPModel{\aStatesT[\alpha]}} \alpha \land \actionA{\aPModel{\aStatesT[\beta]}} \beta \label{aaml-k-choice-2}\\
    &&\entails \actionA{\aPModel{\aStatesT[\alpha]}} (\alpha \lor \beta) \land \actionA{\aPModel{\aStatesT[\beta]}} (\alpha \lor \beta) \label{aaml-k-choice-3}\\
    &&\entails \actionA{\aPModel{\aStatesT}} (\alpha \lor \beta) \label{aaml-k-choice-4}
\end{eqnarray}
(\ref{aaml-k-choice-1}) follows from hypothesis;
(\ref{aaml-k-choice-2}) follows from the above note that $\aPModel[\alpha]{\aStatesT[\alpha]} \bisimilar \aPModel{\aStatesT[\alpha]}$ and $\aPModel[\beta]{\aStatesT[\beta]} \bisimilar \aPModel{\aStatesT[\beta]}$ and Proposition~\ref{aml-bisimilar-actions};
(\ref{aaml-k-choice-3}) follows from propositional disjunction introduction; and
(\ref{aaml-k-choice-4}) follows from \axiomAamlK{} axiom {\bf AU}, as $\aStatesT = \aStatesT[\alpha] \cup \aStatesT[\beta]$.

Finally we show that $\entails \actionE{\aPModel{\aStatesT}} \phi \iff \someactsBs \phi$.
\begin{eqnarray}
    &&\entails \someactsBs (\alpha \lor \beta) \implies (\someactsBs \alpha \lor \someactsBs \beta) \label{aaml-k-choice-5}\\
    &&\entails \someactsBs (\alpha \lor \beta) \implies (\actionE{\aPModel[\alpha]{\aStatesT[\alpha]}} \alpha \land \actionE{\aPModel[\beta]{\aStatesT[\beta]}} \beta) \label{aaml-k-choice-6}\\
    &&\entails \someactsBs (\alpha \lor \beta) \implies (\actionE{\aPModel{\aStatesT[\alpha]}} \alpha \land \actionE{\aPModel{\aStatesT[\beta]}} \beta) \label{aaml-k-choice-7}\\
    &&\entails \someactsBs (\alpha \lor \beta) \implies (\actionE{\aPModel{\aStatesT[\alpha]}} (\alpha \lor \beta) \land \actionE{\aPModel{\aStatesT[\beta]}} (\alpha \lor \beta)) \label{aaml-k-choice-8}\\
    &&\entails \someactsBs (\alpha \lor \beta) \implies \actionE{\aPModel{\aStatesT}} (\alpha \lor \beta) \label{aaml-k-choice-9}
\end{eqnarray}
(\ref{aaml-k-choice-5}) follows from \axiomAamlK{} axiom {\bf R};
(\ref{aaml-k-choice-6}) follows from hypothesis;
(\ref{aaml-k-choice-7}) follows from the above note that $\aPModel[\alpha]{\aStatesT[\alpha]} \bisimilar \aPModel{\aStatesT[\alpha]}$ and $\aPModel[\beta]{\aStatesT[\beta]} \bisimilar \aPModel{\aStatesT[\beta]}$ and Proposition~\ref{aml-bisimilar-actions};
(\ref{aaml-k-choice-8}) follows from propositional disjunction introduction; and
(\ref{aaml-k-choice-9}) follows from \axiomAamlK{} axiom {\bf AU}, as $\aStatesT = \aStatesT[\alpha] \cup \aStatesT[\beta]$.

The converse, that $\entails \actionE{\aPModel{\aStatesT}} \phi \implies \someactsBs \phi$ follows from a simple semantic argument.
Let $\kPModel{\kStateS} \in \classK$ and suppose that $\kPModel{\kStateS} \entails \actionE{\aPModel{\aStatesT}} \phi$.
Then there exists $\aStateS \in \aStatesT$ such that $\kPModel{\kStateS} \entails \aPrecondition(\aStateS)$ and $\kPModel{\kStateS} \exec \aPModel{\aStateS} \entails \phi$.
From above, $\aModel$ is a $\agentsB$-action model, so $\kPModel{\kStateS} \simulatesBs \kPModel{\kStateS} \exec \aPModel{\aStateS}$ and so $\kPModel{\kStateS} \entails \someactsBs \phi$.
Therefore $\entails \actionE{\aPModel{\aStatesT}} \phi \implies \someactsBs \phi$
\end{proof}

\begin{lemma}\label{aaml-k-covers}
Let $\agentsB, \agentsC \subseteq \agents$, 
let $\phi = \pi \land \bigwedge_{\agentC \in \agentsC} \coverC \Gamma_\agentC \in \langAaml$ where $\pi \in \langPl$, and 
for every $\agentC \in \agentsC$, $\gamma \in \agentsC$
let $\aPModelAndTuple[\gamma]{\aStatesT[\gamma]} \in \classAmK$ be a $\agentsB$-action model such that 
$\entails \actionA{\aPModel[\gamma]{\aStatesT[\gamma]}} \gamma$,
$\entails \actionE{\aPModel[\gamma]{\aStatesT[\gamma]}} \gamma \iff \someactsBs \gamma$, and
for every $\aStateT[\gamma] \in \aStatesT[\gamma]$, $\kPModel{\kStateS} \in \classK$ if $\kPModel{\kStateS} \entails \aPrecondition[\gamma](\aStateT[\gamma])$ then $\kPModel{\kStateS} \simulatesBs \kPModel{\kStateS} \exec \aPModel[\gamma]{\aStateT[\gamma]}$.
Then there exists a $\agentsB$-action model $\aPModel{\aStatesT} \in \classAmK$ such that 
$\entails \actionA{\aPModel{\aStatesT}} \phi$, and 
$\entails \actionE{\aPModel{\aStatesT}} \phi \iff \someactsBs \phi$.
for every $\aStateT \in \aStatesT$, $\kPModel{\kStateS} \in \classK$ if $\kPModel{\kStateS} \entails \aPrecondition(\aStateT)$ then $\kPModel{\kStateS} \simulatesBs \kPModel{\kStateS} \exec \aPModel{\aStateT}$.
\end{lemma}

\begin{proof}
Without loss of generality we assume that each $\aModel[\gamma]$ for every $\agentC \in \agentsC$, $\gamma \in \Gamma_\agentC$ is disjoint.

We construct the action model $\aPModelAndTuple{\aStateTest}$ where:
\begin{eqnarray*}
    \aStates &=& \{\aStateTest, \aStateSkip\} \cup \bigcup_{\agentC \in \agentsC, \gamma \in \Gamma_\agentC} \aStates[\gamma]\\
    \aAccessibilityA &=& \{(\aStateTest, \aStateT[\gamma]) \mid \gamma \in \Gamma_\agentA, \aStateT[\gamma] \in \aStatesT[\gamma]\} \cup \{(\aStateSkip, \aStateSkip)\} \cup \bigcup_{\agentC \in \agentsC, \gamma \in \Gamma_\agentC} \aAccessibilityA[\gamma] \text{ for } \agentA \in \agentsC\\
    \aAccessibilityA &=& \{(\aStateTest, \aStateSkip), (\aStateSkip, \aStateSkip)\} \cup \bigcup_{\agentC \in \agentsC, \gamma \in \Gamma_\agentC} \aAccessibilityA[\gamma] \text{ for } \agentA \in \agents \setminus \agentsC\\
    \aPrecondition &=& \{(\aStateTest, \someactsBs \phi), (\aStateSkip, \top)\} \cup \bigcup_{\agentC \in \agentsC, \gamma \in \Gamma_\agentC} \aPrecondition[\gamma]
\end{eqnarray*}

We note for every $\agentC \in \agentsC$, $\gamma \in \Gamma_\agentC$, $\aStateS[\gamma] \in \aStates[\gamma]$ that $\aPModel{\aStateS[\gamma]} \bisimilar \aPModel[\gamma]{\aStateS[\gamma]}$, as by construction $\aModel$ contains the disjoint union of each $\aModel[\gamma]$ and no outward-facing edges are added to any state from $\aStates[\gamma]$ in $\aModel$.

We first show that for every $\kPModel{\kStateS} \in \classK$ if $\kPModel{\kStateS} \entails \aPrecondition(\aStateTest)$ then $\kPModel{\kStateS} \simulatesBs \kPModel{\kStateS} \exec \aPModel{\aStateTest}$.
Let $\kPModel{\kStateS} \in \classK$ such that $\kPModel{\kStateS} \entails \aPrecondition(\aStateTest)$.
For every $\agentC \in \agentsC$, $\gamma \in \Gamma_\agentC$, $\aStateT[\agentC, \gamma] \in \aStatesT[\agentC, \gamma]$, $\kStateT \in \kSuccessorsC{\kStateS}$ such that $\kPModel{\kStateT} \entails \aPrecondition[\agentC,\gamma](\aStateT[\agentC,\gamma])$ we have that $\kPModel{\kStateT} \simulatesBs \kPModel{\kStateT} \exec \aPModel[\agentC, \gamma]{\aStateT[\agentC, \gamma]}$.
From above $\aPModel[\agentC, \gamma]{\aStateT[\agentC, \gamma]} \bisimilar \aPModel{\aStateT[\agentC, \gamma]}$ and so by 
Proposition~\ref{action-bisimulation-results} we have that $\kPModel{\kStateT} \exec \aPModel[\agentC, \gamma]{\aStateT[\agentC, \gamma]} \bisimilar \kPModel{\kStateT} \exec \aPModel{\aStateT[\agentC, \gamma]}$.
From Corollary~\ref{bisimilar-refinement} and Proposition~\ref{refinements-preorder} we have that $\kPModel{\kStateT} \simulatesBs \kPModel{\kStateT} \exec \aPModel{\aStateT[\agentC, \gamma]}$ (say via a $\agentsB$-refinement $\refinement^{\kStateT, \aStateT[\agentC, \gamma]}$.

Let $\kPModelP{(\kStateS, \aStateTest)} = \kPModel{\kStateS} \exec \aPModel{\aStateTest}$.
We define $\refinement \subseteq \kStates \times \kStatesP$ where:
\begin{eqnarray*}
\refinement &=& 
\{(\kStateS, (\kStateS, \aStateTest))\} \cup 
\{(\kStateT, (\kStateT, \aStateSkip)) \mid \kStateT \in \kStates\}  \\&&\quad \cup
\bigcup \{\{(\kStateT, (\kStateT, \aPStateT[\agentC, \gamma]))\} \cup \refinement^{\kStateT, \aStateT[\agentC, \gamma]} \mid \agentC \in \agentsC, \gamma \in \Gamma_\agentC, \aStateT[\agentC, \gamma] \in \aStatesT[\agentC, \gamma], \kStateT \in \kSuccessorsC{\kStateT}, \kPModel{\kStateT} \entails \aPrecondition(\aStateT[\agentC, \gamma])\}
\end{eqnarray*}

We claim that $\refinement$ is a $\agentsB$-refinement from $\kModel{\kStateS}$ to $\kModelP{(\kStateS, \aStateTest)}$.
Let $\atomP \in \atoms$, $\agentA \in \agents$ and $\agentD \in \agents \setminus \agentsB$.

\paragraph{atoms-$\atomP$}
Consider $(\kStateS, (\kStateS, \aStateTest)) \in \refinement$.
By construction $\kStateS \in \kValuation(\atomP)$ if and only if $(\kStateS, \aStateTest) \in \kValuationP(\atomP)$.

Consider $(\kStateT, (\kStateT, \aStateSkip)) \in \refinement$ where $\kStateT \in \kStates$.
By construction $\kStateT \in \kValuation(\atomP)$ if and only if $(\kStateT, \aStateSkip) \in \kValuationP(\atomP)$.

Consider $(\kStateT, (\kStateT, \aPStateT[\agentC, \gamma])) \in \refinement$ where $\agentC \in \agentsC$, $\gamma \in \Gamma_\agentC$, $\aStateT[\agentC, \gamma] \in \aStatesT[\agentC, \gamma]$, $\kStateT \in \kStates$, and $\kPModel{\kStateT} \entails \aPrecondition(\aStateT[\agentC, \gamma])$.
By construction $\kStateT \in \kValuation(\atomP)$ if and only if $(\kStateT, \aStateT[\agentC, \gamma]) \in \kValuationP(\atomP)$.

Consider $(\kStateT, \kStateTP) \in \refinement^{\kStateT, \aStateT[\agentC, \gamma]} \subseteq \refinement$ where $\agentC \in \agentsC$, $\gamma \in \Gamma_\agentC$, $\aStateT[\agentC, \gamma] \in \aStatesT[\agentC, \gamma]$, $\kStateT \in \kStates$, and $\kPModel{\kStateT} \entails \aPrecondition(\aStateT[\agentC, \gamma])$.
From {\bf atoms-$\atomP$} for $\refinement^{\kStateT, \aStateT[\agentC, \gamma]}$ we have that $\kStateT \in \kValuation(\atomP)$ if and only if $\kStateTP \in \kValuationP(\atomP)$.

\paragraph{forth-$\agentD$}
Consider $(\kStateS, (\kStateS, \aStateTest)) \in \refinement$.
Let $\kStateT \in \kSuccessorsD{\kStateS}$.
Suppose that $\agentD \in \agentsC$.
By hypothesis $\kPModel{\kStateS} \entails \someactsBs (\pi \land \bigwedge_{\agentC \in \agentsC} \coversC \Gamma_\agentC)$, and in particular $\kPModel{\kStateS} \entails \someactsBs \coversD \Gamma_\agentD$.
By the \axiomAamlK{} axiom {\bf RComm} we have that $\kPModel{\kStateS} \entails \coversD \{\someactsBs \gamma' \mid \gamma' \in \Gamma_\agentD\}$ and by the definition of the cover operator we have that $\necessaryD \bigvee_{\gamma' \in \Gamma_\agentD} \someactsBs \gamma'$ so there exists $\gamma' \in \Gamma_\agentD$ such that $\kPModel{\kStateT} \entails \someactsBs \gamma'$.
By hypothesis $\entails \someactsBs \gamma' \implies \actionE{\aPModel[\agentC, \gamma']{\aStatesT[\agentC, \gamma']}} \gamma'$ so there exists $\aStateT[\agentC, \gamma'] \in \aStatesT[\agentC, \gamma']$ such that $\kPModel{\kStateT} \entails \aPrecondition[\agentC, \gamma'](\aStateT[\agentC, \gamma'])$.
By construction $\aStateT[\agentC, \gamma'] \in \aSuccessorsD{\aStateTest}$ and $\aPrecondition(\aStateT[\agentC, \gamma']) = \aPrecondition[\agentC, \gamma'](\aStateT[\agentC, \gamma'])$ so $\kPModel{\kStateT} \entails \aPrecondition(\aStateT[\agentC, \gamma'])$, $(\kStateT, \aStateT[\agentC, \gamma']) \in \kSuccessorsPD{(\kStateS, \aStateTest)}$, and $(\kStateT, (\kStateT, \aStateT[\agentC, \gamma'])) \in \refinement^{\kStateT, \aStateT[\agentC, \gamma']} \subseteq \refinement$.
Suppose that $\agentD \notin \agentsC$.
By construction $\aStateSkip \in \aSuccessorsD{\aStateTest}$ so $(\kStateT, \aStateSkip) \in \kSuccessorsPD{(\kStateS, \aStateTest)}$ and $(\kStateT, (\kStateT, \aStateSkip)) \in \refinement$.

Consider $(\kStateT, (\kStateT, \aStateSkip)) \in \refinement$ where $\kStateT \in \kStates$.
Let $\kStateU \in \kSuccessorsD{\kStateT}$.
By construction $\aStateSkip \in \aSuccessorsD{\aStateSkip}$ so $(\kStateU, \aStateSkip) \in \kSuccessorsPD{(\kStateT, \aStateSkip)}$ and $(\kStateU, (\kStateU, \aStateSkip)) \in \refinement$.

Consider $(\kStateT, (\kStateT, \aPStateT[\agentC, \gamma])) \in \refinement$ where $\agentC \in \agentsC$, $\gamma \in \Gamma_\agentC$, $\aStateT[\agentC, \gamma] \in \aStatesT[\agentC, \gamma]$, $\kStateT \in \kStates$, and $\kPModel{\kStateT} \entails \aPrecondition(\aStateT[\agentC, \gamma])$.
Let $\kStateU \in \kSuccessorsD{\kStateT}$.
As $\kPModel{\kStateT} \entails \aPrecondition(\aStateT[\agentC, \gamma])$ then by hypothesis $(\kStateT, (\kStateT, \aStateT[\agentC, \gamma])) \in \refinement^{\kStateT, \aStateT[\agentC, \gamma]}$.
By {\bf forth-$\agentD$} for $\refinement^{\kStateT, \aStateT[\agentC, \gamma]}$ there exists $(\kStateV, \aStateV[\agentC, \gamma]) \in \kSuccessorsPD{(\kStateT, \aStateT[\agentC, \gamma])}$ such that $(\kStateU, (\kStateV, \aStateV[\agentC, \gamma])) \in \refinement^{\kStateT, \aStateT[\agentC, \gamma]} \subseteq \refinement$.

Consider $(\kStateT, \kStateTP) \in \refinement^{\kStateT, \aStateT[\agentC, \gamma]} \subseteq \refinement$ where $\agentC \in \agentsC$, $\gamma \in \Gamma_\agentC$, $\aStateT[\agentC, \gamma] \in \aStatesT[\agentC, \gamma]$, $\kStateT \in \kStates$, and $\kPModel{\kStateT} \entails \aPrecondition(\aStateT[\agentC, \gamma])$.
Let $\kStateU \in \kSuccessorsD{\kStateT}$.
By {\bf forth-$\agentD$} for $\refinement^{\kStateT, \aStateT[\agentC, \gamma]}$ there exists $\kStateUP \in \kSuccessorsPD{\kStateTP}$ such that $(\kStateU, \kStateUP) \in \refinement^{\kStateT, \aStateT[\agentC, \gamma]} \subseteq \refinement$.

\paragraph{back-$\agentA$}
Consider $(\kStateS, (\kStateS, \aStateTest)) \in \refinement$.
Suppose that $\agentA \in \agentsC$.
Let $(\kStateT, \aStateT[\agentA, \gamma]) \in \kSuccessorsPA{(\kStateS, \aStateTest)}$ where $\gamma \in \Gamma_\agentA$ and $\aStateT[\agentA, \gamma] \in \aStatesT[\agentA, \gamma]$.
By construction $\kStateT \in \kSuccessorsA{\kStateS}$ and $\kPModel{\kStateT} \entails \aPrecondition(\aStateT[\agentA, \gamma])$ so by hypothesis $(\kStateT, (\kStateT, \aStateT[\agentA, \gamma])) \in \refinement^{\kStateT, \aStateT[\agentC, \gamma]} \subseteq \refinement$.
Suppose that $\agentA \notin \agentsC$.
Let $(\kStateT, \aStateSkip) \in \kSuccessorsPA{(\kStateS, \aStateTest)}$.
By construction $\kStateT \in \kSuccessorsA{\kStateS}$ and $(\kStateT, (\kStateT, \aStateSkip)) \in \refinement$.

Consider $(\kStateT, (\kStateT, \aStateSkip)) \in \refinement$ where $\kStateT \in \kStates$.
Let $(\kStateU, \aStateSkip) \in \kSuccessorsPA{(\kStateT, \aStateSkip)}$.
By construction $\kStateU \in \kSuccessorsA{\kStateT}$ and $(\kStateU, (\kStateU, \aStateSkip)) \in \refinement$.

Consider $(\kStateT, (\kStateT, \aPStateT[\agentC, \gamma])) \in \refinement$ where $\agentC \in \agentsC$, $\gamma \in \Gamma_\agentC$, $\aStateT[\agentC, \gamma] \in \aStatesT[\agentC, \gamma]$, $\kStateT \in \kStates$, and $\kPModel{\kStateT} \entails \aPrecondition(\aStateT[\agentC, \gamma])$.
Let $(\kStateU, \aStateU[\agentC, \gamma]) \in \kSuccessorsPA{(\kStateT, \aPStateT[\agentC, \gamma])}$ where $\aStateU[\agentC, \gamma] \in \aSuccessorsA{\aStateT[\agentC, \gamma]} = \aSuccessorsA{\aPStateT[\agentC, \gamma]}$.

As $\kPModel{\kStateT} \entails \aPrecondition(\aStateT[\agentC, \gamma])$ then by hypothesis $(\kStateT, (\kStateT, \aStateT[\agentC, \gamma])) \in \refinement^{\kStateT, \aStateT[\agentC, \gamma]}$.
By {\bf back-$\agentA$} for $\refinement^{\kStateT, \aStateT[\agentC, \gamma]}$ there exists $\kStateV \in \kSuccessorsA{\kStateT}$ such that $(\kStateV, (\kStateU, \aStateU[\agentC, \gamma])) \in \refinement^{\kStateT, \aStateT[\agentC, \gamma]} \subseteq \refinement$.

Consider $(\kStateT, \kStateTP) \in \refinement^{\kStateT, \aStateT[\agentC, \gamma]} \subseteq \refinement$ where $\agentC \in \agentsC$, $\gamma \in \Gamma_\agentC$, $\aStateT[\agentC, \gamma] \in \aStatesT[\agentC, \gamma]$, $\kStateT \in \kStates$, and $\kPModel{\kStateT} \entails \aPrecondition(\aStateT[\agentC, \gamma])$.
Let $\kStateUP \in \kSuccessorsPA{\kStateTP}$.
By {\bf back-$\agentA$} for $\refinement^{\kStateT, \aStateT[\agentC, \gamma]}$ there exists $\kStateU \in \kSuccessorsA{\kStateT}$ such that $(\kStateU, \kStateUP) \in \refinement^{\kStateT, \aStateT[\agentC, \gamma]} \subseteq \refinement$.

Therefore $\refinement$ is a $\agentsB$-refinement and $\kPModel{\kStateS} \simulatesBs \kPModel{\kStateS} \exec \aPModel{\aStateTest}$.

We next show that $\entails \actionA{\aPModel{\aStateTest}} \phi$, in several parts.
We note that from the definition of the cover operator:
$$
\phi = \pi \land \bigwedge_{\agentC \in \agentsC} (\necessaryC \bigvee_{\gamma \in \Gamma_\agentC} \gamma \land \bigwedge_{\gamma \in \Gamma_\agentC} \possibleC \gamma)
$$
Thus we will show individually that:
\begin{enumerate}
\item $\entails \actionA{\aPModel{\aStateTest}} \pi$
\item $\entails \actionA{\aPModel{\aStateTest}} \necessaryC \bigvee_{\gamma \in \Gamma_\agentC} \gamma$, for every $\agentC \in \agentsC$
\item $\entails \actionA{\aPModel{\aStateTest}} \bigwedge_{\gamma \in \Gamma_\agentC} \possibleC \gamma$, for every $\agentC \in \agentsC$
\end{enumerate}

First we show that $\entails \actionA{\aPModel{\aStateTest}} \pi$.
\begin{eqnarray}
    && \entails \phi \implies \pi \label{aaml-covers-1}\\
    && \entails \lnot \pi \implies \lnot \phi \label{aaml-covers-2}\\
    && \entails  \allactsBs (\lnot \pi \implies \lnot \phi) \label{aaml-covers-3}\\
    && \entails  \allactsBs \lnot \pi \implies \allactsBs \lnot \phi \label{aaml-covers-4}\\
    && \entails  \someactsBs \phi \implies \someactsBs \pi \label{aaml-covers-5}\\
    && \entails  \aPrecondition(\aStateTest) \implies \pi \label{aaml-covers-6}\\
    && \entails  \actionA{\aPModel{\aStateTest}} \pi \label{aaml-covers-7}
\end{eqnarray}
(\ref{aaml-covers-1}) and
(\ref{aaml-covers-2}) follow from propositional reasoning;
(\ref{aaml-covers-3}) follows from \axiomAamlK{} rule {\bf NecR};
(\ref{aaml-covers-4}) follows from \axiomAamlK{} axiom {\bf R};
(\ref{aaml-covers-5}) follows from the definition of $\someactsBs$;
(\ref{aaml-covers-6}) follows from the construction of $\aPModel{\aStateTest}$; and
(\ref{aaml-covers-7}) follows from \axiomAamlK{} axiom {\bf AP}.

Next we show that $\entails \actionA{\aPModel{\aStateTest}} \necessaryC \bigvee_{\gamma \in \Gamma_\agentC} \gamma$, for every $\agentC \in \agentsC$.
Let $\agentC \in \agentsC$.
\begin{eqnarray}
    && \entails \actionA{\aPModel[\gamma]{\aStatesT[\gamma]}} \gamma \text{ for every } \gamma \in \Gamma_\agentC \label{aaml-covers-8}\\
    && \entails \actionA{\aPModel{\aStatesT[\gamma]}} \gamma \text{ for every } \gamma \in \Gamma_\agentC \label{aaml-covers-9}\\
    && \entails \bigwedge_{\aStateT \in \aStatesT[\gamma]} \actionA{\aPModel[\gamma]{\aStateT}} \gamma \text{ for every } \gamma \in \Gamma_\agentC \label{aaml-covers-10}\\
    && \entails \necessaryC \bigwedge_{\aStateT \in \aStatesT[\gamma]} \actionA{\aPModel[\gamma]{\aStateT}} \gamma \text{ for every } \gamma \in \Gamma_\agentC \label{aaml-covers-11}\\
    && \entails \bigwedge_{\gamma \in \Gamma_\agentC} \necessaryC \bigwedge_{\aStateT \in \aStatesT[\gamma]} \actionA{\aPModel[\gamma]{\aStateT}} \gamma \label{aaml-covers-12}\\
    && \entails \bigwedge_{\gamma \in \Gamma_\agentC} \bigwedge_{\aStateT \in \aStatesT[\gamma]} \necessaryC \actionA{\aPModel[\gamma]{\aStateT}} \gamma \label{aaml-covers-13}\\
    && \entails \bigwedge_{\gamma \in \Gamma_\agentC} \bigwedge_{\aStateT \in \aStatesT[\gamma]} \necessaryC \actionA{\aPModel[\gamma]{\aStateT}} \bigvee_{\gamma' \in \Gamma_\agentC} \gamma' \label{aaml-covers-14}\\
    && \entails \bigwedge_{\aStateT \in \aSuccessorsC{\aStateTest}} \necessaryC \actionA{\aPModel[\gamma]{\aStateT}} \bigvee_{\gamma \in \Gamma_\agentC} \gamma \label{aaml-covers-15}\\
    && \entails \aPrecondition(\aStateTest) \implies \bigwedge_{\aStateT \in \aSuccessorsC{\aStateTest}} \necessaryC \actionA{\aPModel{\aStateT}} \bigvee_{\gamma \in \Gamma_\agentC} \gamma \label{aaml-covers-16}\\
    && \entails \actionA{\aPModel{\aStateTest}} \necessaryC \bigvee_{\gamma \in \Gamma_\agentC} \gamma \label{aaml-covers-17}
\end{eqnarray}
(\ref{aaml-covers-8}) follows from hypothesis;
(\ref{aaml-covers-9}) follows from the above note that $\aModel[\gamma]{\aStateT[\gamma]} \bisimilar \aPModel{\aStateT[\gamma]}$;
(\ref{aaml-covers-10}) follows from \axiomAamlK{} axiom {\bf AU};
(\ref{aaml-covers-14}) follows from propositional disjunction introduction;
(\ref{aaml-covers-15}) follows from the construction of $\aModel$; and
(\ref{aaml-covers-17}) follows from \axiomAamlK{} axiom {\bf AK}.

Finally we show that $\entails \actionA{\aPModel{\aStateTest}} \bigwedge_{\gamma \in \Gamma_\agentC} \possibleC \gamma$, for every $\agentC \in \agentsC$.
Let $\agentC \in \agentsC$.

Suppose that $\agentC \in \agentsB$.
Then:
\begin{eqnarray}
    &&\entails \someactsBs \phi \implies \someactsBs \coversC \Gamma_\agentC \label{aaml-covers-18}\\
    &&\entails \someactsBs \phi \implies \bigwedge_{\gamma \in \Gamma_\agentC} \possibleC \someactsBs \gamma \label{aaml-covers-19}
\end{eqnarray}
(\ref{aaml-covers-18}) follows from \axiomAamlK{} axiom {\bf R} and rule {\bf NecR}; and
(\ref{aaml-covers-19}) follows from \axiomAamlK{} axiom {\bf RK}.

Suppose that $\agentC \notin \agentsB$.
Then:
\begin{eqnarray}
    &&\entails \someactsBs \phi \implies \someactsBs \coversC \Gamma_\agentC \label{aaml-covers-20}\\
    &&\entails \someactsBs \phi \implies \coversC \{\someactsBs \gamma \mid \gamma \in \Gamma_\agentC \}\label{aaml-covers-21}\\
    &&\entails \someactsBs \phi \implies \bigwedge_{\gamma \in \Gamma_\agentC} \possibleC \someactsBs \gamma \label{aaml-covers-22}
\end{eqnarray}
(\ref{aaml-covers-20}) follows from \axiomAamlK{} axiom {\bf R} and rule {\bf NecR};
(\ref{aaml-covers-21}) follows from \axiomAamlK{} axiom {\bf RComm}; and
(\ref{aaml-covers-22}) follows from the definition of the cover operator.

So we have that $\entails \someactsBs \phi \implies \bigwedge_{\gamma \in \Gamma_\agentC} \possibleC \gamma$.
Then:
\begin{eqnarray}
    && \someactsBs \phi \implies \bigwedge_{\gamma \in \Gamma_\agentC} \possibleC \someactsBs \gamma \label{aaml-covers-23}\\
    && \someactsBs \phi \implies \bigwedge_{\gamma \in \Gamma_\agentC} \possibleC \actionE{\aPModel[\gamma]{\aStatesT[\gamma]}} \gamma \label{aaml-covers-24}\\
    && \someactsBs \phi \implies \bigwedge_{\gamma \in \Gamma_\agentC} \possibleC \actionE{\aPModel{\aStatesT[\gamma]}} \gamma \label{aaml-covers-25}\\
    && \someactsBs \phi \implies \bigwedge_{\gamma \in \Gamma_\agentC} \possibleC \bigvee_{\aStateT \in \aStatesT[\gamma]} \actionE{\aPModel{\aStateT}} \gamma \label{aaml-covers-26}\\
    && \someactsBs \phi \implies \bigwedge_{\gamma \in \Gamma_\agentC} \possibleC \bigvee_{\aStateT \in \aSuccessorsC{\aStateTest}} \actionE{\aPModel{\aStateT}} \gamma \label{aaml-covers-27}\\
    && \bigwedge_{\gamma \in \Gamma_\agentC} \left(\someactsBs \phi \implies \possibleC \bigvee_{\aStateT \in \aSuccessorsC{\aStateTest}} \actionE{\aPModel{\aStateT}} \gamma\right) \label{aaml-covers-28}\\
    && \bigwedge_{\gamma \in \Gamma_\agentC} \actionA{\aPModel{\aStateTest}} \possibleC \gamma \label{aaml-covers-29}
\end{eqnarray}
(\ref{aaml-covers-23}) follows from above;
(\ref{aaml-covers-24}) follows from hypothesis;
(\ref{aaml-covers-25}) follows from the above note that $\aModel[\gamma]{\aStateT[\gamma]} \bisimilar \aPModel{\aStateT[\gamma]}$;
(\ref{aaml-covers-26}) follows from \axiomAamlK{} axiom {\bf AU};
(\ref{aaml-covers-27}) follows from the construction of $\aModel$ and propositional disjunction introduction;
(\ref{aaml-covers-28}) follows from propositional reasoning; and
(\ref{aaml-covers-29}) follows from \axiomAamlK{} axiom {\bf AK}.

Therefore $\entails \actionA{\aPModel{\aStateTest}} \phi$.

Finally we show that $\entails \actionE{\aPModel{\aStateTest}} \phi \iff \someactsBs \phi$. 
This is straight-forward, given what we have shown above.
\begin{eqnarray}
    && \entails \actionE{\aPModel{\aStateTest}} \phi \iff (\aPrecondition(\aStateTest) \land \actionA{\aPModel{\aStateTest}} \phi) \label{aaml-covers-30}\\
    && \entails \actionE{\aPModel{\aStateTest}} \phi \iff \aPrecondition(\aStateTest) \label{aaml-covers-31}\\
    && \entails \actionE{\aPModel{\aStateTest}} \phi \iff \someactsBs \phi \label{aaml-covers-32}
\end{eqnarray}
(\ref{aaml-covers-30}) follows from the definition of $\actionE{}$;
(\ref{aaml-covers-31}) follows from $\entails \actionA{\aPModel{\aStateTest}} \phi$ above;
(\ref{aaml-covers-32}) follows from the construction of $\aModel$.
\end{proof}

\begin{theorem}\label{aaml-k-synthesis}
Let $\agentsB \subseteq \agents$ and let $\phi \in \langAaml$.
There exists an action model $\aPModelAndTuple{\aStatesT} \in \classAmK$ such that 
$\entails \actionA{\aPModel{\aStatesT}} \phi$,
$\entails \actionE{\aPModel{\aStatesT}} \phi \iff \someactsBs \phi$, and
for every $\aStateT \in \aStatesT$, $\kPModel{\kStateS} \in \classK$ if $\kPModel{\kStateS} \entails \aPrecondition(\aStateT)$ then $\kPModel{\kStateS} \simulatesBs \kPModel{\kStateS} \exec \aPModel{\aStateT}$.
\end{theorem}

\begin{proof}
Without loss of generality, by Corollary~\ref{aaml-k-expressive-equivalence} we may assume that $\phi \in \langMl$ and by Lemma~\ref{dnf-equivalent} we may further assume that $\phi$ is in disjunctive normal form.
Then we proceed by induction on the structure of $\phi$.  
Suppose that $\phi = \pi \land \bigwedge_{\agentC \in \agentsC} \Gamma_\agentC$ where $\pi \in \langPl$, $\agentsC \subseteq \agents$ and for every $\agentC \in \agentsC$, $\Gamma_\agentC \subseteq \langMl$ is a finite set of modal formulas.
We note that the base case for the induction occurs when for every $\agentC \in \agentsC$, $\Gamma_\agentC = \emptyset$.
By the induction hypothesis for every $\agentC \in \agentsC$, $\gamma \in \Gamma_\agentC$ there exists 
a $\agentsB$-action model $\aPModel[\gamma]{\aStatesT[\gamma]} \in \classAmK$ such that
$\entails \actionA{\aPModel[\gamma]{\aStatesT[\gamma]}} \gamma$, and 
$\entails \actionE{\aPModel[\gamma]{\aStatesT[\gamma]}} \gamma \iff \someactsBs \gamma$.
By Lemma~\ref{aaml-k-covers} there exists a $\agentsB$-action model $\aPModel{\aStatesT} \in \classAmK$ such that 
$\entails \actionA{\aPModel{\aStatesT}} \phi$, and 
$\entails \actionE{\aPModel{\aStatesT}} \phi \iff \someactsBs \phi$.

Suppose that $\phi = \alpha \lor \beta$ where $\alpha, \beta \in \langMl$.
By the induction hypothesis there exists $\agentsB$-action models
$\aPModel[\alpha]{\aStatesT[\alpha]} \in \classAmK$ and $\aPModel[\beta]{\aStatesT[\beta]} \in \classAmK$ such that 
$\entails \actionA{\aPModel[\alpha]{\aStatesT[\alpha]}} \alpha$, 
$\entails \actionE{\aPModel[\alpha]{\aStatesT[\alpha]}} \alpha \iff \someactsBs \alpha$, 
$\entails \actionA{\aPModel[\beta]{\aStatesT[\beta]}} \beta$, and 
$\entails \actionE{\aPModel[\beta]{\aStatesT[\beta]}} \beta \iff \someactsBs \beta$.
By Lemma~\ref{aaml-k-choice} there exists a $\agentsB$-action model $\aPModel{\aStatesT} \in \classAmK$ such that 
$\entails \actionA{\aPModel{\aStateS}} \phi$, and 
$\entails \actionE{\aPModel{\aStateS}} \phi \iff \someactsBs \phi$.
\end{proof}

\begin{corollary}
The semantics of \logicAamlK{} as defined in Definition~\ref{aaml-semantics} and the alternative semantics of Definition~\ref{aaml-semantics-alt} are equivalent.
\end{corollary}

\begin{proof}
For convenience we will use the notation $\kPModel{\kStateS} \entails_\exec \phi$ to denote that $\kPModel{\kStateS} \entails \phi$ according to the semantics of Definition~\ref{aaml-semantics} and the notation $\kPModel{\kStateS} \entails_{\simulates} \phi$ to denote that $\kPModel{\kStateS} \entails \phi$ according to the semantics of Definition~\ref{aaml-semantics-alt}.

Let $\phi \in \langAaml$.
We show by induction on the structure of $\phi$ that for every $\kPModel{\kStateS} \in \classK$, $\kPModel{\kStateS} \entails_\exec \phi$ if and only $\kPModel{\kStateS} \entails_{\simulates} \phi$.
The cases where $\phi = \atomP$, $\phi = \lnot \psi$,  $\phi = \psi \land \chi$, $\phi = \necessaryA \psi$ or $\phi = \actionA{\aPModel{\aStateS}} \psi$ where $\atomP \in \atoms$ and $\psi, \chi \in \langAaml$ are straight-forward as these cases are handled identically for the two definitions of the semantics.

Suppose that $\phi = \someactsBs \psi$ where $\psi \in \langAaml$.

Suppose that $\kPModel{\kStateS} \entails_\exec \someactsBs \psi$.
Then there exists a $\agentsB$-action model $\aPModel{\aStateS} \in \aSignatureFamily_\agentsB$ such that $\kPModel{\kStateS} \entails_\exec \actionE{\aPModel{\aStateS}} \psi$ 
As $\kPModel{\kStateS} \entails_\exec \actionE{\aPModel{\aStateS}} \psi$ then we have that $\kPModel{\kStateS} \entails_\exec \aPrecondition(\aStateS)$ and 
$\kPModel{\kStateS} \exec \aPModel{\aStateS} \entails_\exec \psi$.
By the induction hypothesis we have $\kPModel{\kStateS} \entails_{\simulates} \aPrecondition(\aStateS)$ and $\kPModel{\kStateS} \exec \aPModel{\aStateS} \entails_{\simulates} \psi$.
As $\aPModel{\aStateS}$ is a $\agentsB$-action model then $\kPModel \simulatesBs \kPModel{\kStateS} \exec \aPModel{\aStateS}$.
Therefore $\kPModel{\kStateS} \entails_{\simulates} \someactsBs \psi$.

Suppose that $\kPModel{\kStateS} \entails_{\simulates} \someactsBs \psi$.
From Theorem~\ref{aaml-k-synthesis} there exists a $\agentsB$-action model $\aPModel{\aStatesT} \in \classAmK$ such that $\entails_{\simulates} \actionE{\aPModel{\aStatesT}} \psi \iff \someactsBs \psi$.
Without loss of generality, by Corollary~\ref{aaml-k-expressive-equivalence} we assume that $\aPModel{\aStatesT}$ has preconditions defined on \langMl{}.
Therefore $\kPModel{\kStateS} \entails_{\simulates} \actionE{\aPModel{\aStatesT}} \psi$ so $\kPModel{\kStateS} \entails_{\simulates} \aPrecondition(\aStatesT)$ and $\kPModel{\kStateS} \exec \aPModel{\aStateS} \entails_{\simulates} \psi$.
As $\aPrecondition(\aStatesT) \in \langMl$ then $\kPModel{\kStateS} \entails_\exec \aPrecondition(\aStatesT)$ and by the induction hypothesis $\kPModel{\kStateS} \exec \aPModel{\aStateS} \entails_{\simulates} \psi$, so $\kPModel{\kStateS} \entails_\exec \actionE{\aPModel{\aStatesT}} \psi$. 
Therefore $\kPModel{\kStateS} \entails_\exec \someactsBs \psi$.
\end{proof}
