\chapter{Introduction}\label{introduction}

Epistemic logic is the logic of knowledge, used to reason about the knowledge a collection of agents hold regarding the truth of propositional atoms and each other's knowledge.
As a modal logic, situations involving knowledge are represented by relational structures known as Kripke models, and are reasoned about using modal operators that denote that an agent knows that a statement is true.
Epistemic logic only considers static situations involving knowledge, where knowledge and the truth of the propositional atoms that the knowledge is about do not change.
However many practical situations involving knowledge are not static, and many natural questions about knowledge directly concern changes in knowledge.
For example, your knowledge may change as a result of reading this dissertation, and you might ask ``What will I know after reading this dissertation?'', ``Will I know about epistemic updates after reading this dissertation?'' or ``How can I learn about epistemic updates?''.

Dynamic epistemic logics are logics of change of knowledge, used to reason about how knowledge changes in response to epistemic updates, events that provide agents with additional information.
Examples of epistemic updates include the direct observation of information by an agent, communication of information between agents, and epistemic protocols formed by composing simpler epistemic updates sequentially, concurrently, or conditionally.
For our purposes, when we discuss epistemic updates we assume that they are purely informative in nature, so they may cause knowledge to change, but not the truth of the propositional atoms that the knowledge is about.
We also assume that epistemic updates increase certainty about information monotonically, so they may not cause agents to forget or revise information that they were previously certain of. 

Previous work in dynamic epistemic logic has considered how knowledge changes in response to specific epistemic updates.
Notable works include the public announcement logic of Plaza~\cite{plaza:1989} and Gerbrandy and Groenveld~\cite{gerbrandy:1997}, and the action model logic of Baltag, Moss and Solecki~\cite{baltag:1998,baltag:2004}, which each introduce models for epistemic updates and logics for reasoning about the effects of specific epistemic updates using these models.
These logics extend epistemic logic with operators that denote that a specific epistemic update results in a statement becoming true, allowing us to answer questions such as ``Will I know about quantifying over epistemic updates after reading this dissertation?''.
Both logics represent changes in knowledge as operations that take a Kripke model representing a situation involving knowledge, and a model representing an epistemic update, and produces a new Kripke model, representing the result of the epistemic update.
These operations for performing epistemic updates on Kripke models give a powerful method for modelling and reasoning about the full effects of a specific epistemic update in a specific situation, allowing us to answer questions such as ``What will I know after reading this dissertation?''.
However many natural questions about changes in knowledge are not questions about specific epistemic updates, such as ``{\em How} can I learn about quantifying over epistemic updates?''.

More recent work in dynamic epistemic logic has considered how knowledge changes in response to arbitrary epistemic updates, by quantifying over epistemic updates.
Notable works include the arbitrary public announcement logic (\logicApal{}) of Balbiani, et al.~\cite{balbiani:2007} and the group announcement logic (GAL) of {\AA}gotnes, et al.~\cite{agotnes:2008,agotnes:2010}, which each introduce logics for quantifying over epistemic updates.
These logics extend public announcement logic with quantifiers that denote either that every epistemic update results in a statement becoming true, or dually, that some epistemic update results in a statement becoming true, allowing us to answer questions such as ``Can I learn about quantifying over epistemic updates (through some epistemic update)?''.
Supposing that the answer is in the affirmative we might subsequently ask ``How can I learn about quantifying over epistemic updates?'', expecting an example of a specific epistemic update that will result in the desired change in knowledge.
In principle, such questions may be answered by model-checking, satisfiability and synthesis procedures for these logics.
Although \logicApal{} and \logicGal{} both have model-checking procedures~\cite{agotnes:2010}, the satisfiability problems for these logics are undecidable~\cite{agotnes:2014}.
In the present work we introduce several decidable logics for quantifying over epistemic updates, in the same style as \logicApal{} and GAL, but quantifying over different models for epistemic updates: refinement modal logic, arbitrary action model logic.

The refinement modal logic (\logicRml{}) is an extension of epistemic logic that introduces quantifiers over {\em refinements} of Kripke models.
Refinements correspond to the results of a very general notion of epistemic updates, in accordance with our informal understanding of epistemic updates as purely informative and monotonically increasing certainty of information.
The refinements of a Kripke model partially correspond to the results of action models, but are more general.
Unlike public announcements or action models, refinements in general are not backed by a model or operation for epistemic updates that produces the results.
\logicRml{} was introduced by van Ditmarsch and French~\cite{vanditmarsch:2009}, and initial results were given by van Ditmarsch, French and Pinchinat~\cite{vanditmarsch:2010} in the setting of single-agent \classK{}.
In the present work we consider \logicRml{} in a variety of modal settings, including multi-agent \classK{}, \classKF{}, \classKFF{}, \classKD{} and \classS{}.
In the settings of \classK{}, \classKFF{}, \classKD{} and \classS{} we provide sound and complete axiomatisations for \logicRml{}, we show that \logicRml{} is compact and decidable, and that \logicRml{} is expressively equivalent to modal logic, via a provably correct translation from the language of \logicRml{} to the language of the underlying modal logic.
In the setting of \classKF{} we show that \logicRml{} is decidable, and that its expressivity lies strictly between that of modal logic and the modal $\mu$-calculus.

Arbitrary action model logic (\logicAaml{}) is an extension of action model logic that introduces quantifiers over action models.
\logicAaml{} was proposed by Balbiani, et al.~\cite{balbiani:2007} as a possible generalisation for \logicApal{}, and the semantics of \logicAaml{} and \logicApal{} are accordingly very similar.
Like \logicRml{}, we consider \logicAaml{} in a variety of settings, including multi-agent \classK{}, \classKFF{} and \classS{}.
In these settings we provide sound and complete axiomatisations, we show that \logicAaml{} is decidable and that \logicAaml{} is expressively equivalent to modal logic.
Moreover we show that action model quantification is equivalent to refinement quantification and we provide a synthesis procedure that constructs a specific action model that results in a desired change in knowledge whenever such an action model exists.

In Chapter~\ref{literature} we provide an overview of literature in epistemic logic and dynamic epistemic logic, giving context and motivation to the present work.
In Chapter~\ref{technical} we recall technical definitions and results used in the following chapters.
In Chapter~\ref{rml} we introduce the notion of refinements and the syntax and semantics of the refinement modal logic, providing results about refinements and semantic results about the refinement modal logic that apply to several modal settings.
In Chapter~\ref{rml-k}, Chapter~\ref{rml-kd45}, and Chapter~\ref{rml-s5} we consider in greater detail the refinement modal logic in the setting of multi-agent \classK{}, \classKFF{} and \classKD{}, and \classS{}, respectively, providing sound and complete axiomatisations, provably correct translations from \langRml{} to \langMl{}, and expressive equivalence, compactness and decidability results.
In Chapter~\ref{rml-k4} we consider in greater detail the refinement modal logic in the setting of \classKF{}, showing that it is decidable and that is expressivity lies strictly between that of modal logic and the modal $\mu$-calculus.
In Chapter~\ref{aaml} we introduce the arbitrary action model logic in various settings, providing sound and complete axiomatisations, expressivity and decidability results, methods for action model synthesis, and we show that action model quantifiers are equivalent to refinement quantifiers.
Finally in Chapter~\ref{conclusion} we summarise our results, and outline on-going work and open questions.
