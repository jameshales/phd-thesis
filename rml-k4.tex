\chapter{Refinement modal logic \classKF{}}\label{rml-k4}

In this chapter we consider in greater detail the logic \logicRmlKF{} in the setting of \classKF{}.
Unlike previous chapters we do not present a sound and complete axiomatisation or a provably correct translation.
In the previous chapters we provided sound and complete axiomatisations for the logics \logicRmlK{}, \logicRmlKFF{}, \logicRmlKD{}, and \logicRmlS{}, showing the completeness of these axiomatisations via a provably correct translation from each refinement modal logic to the respective underlying modal logic.
As we show in this chapter the logic \logicRmlKF{} is not expressively equivalent to the underlying modal logic \logicKF{}, so a provably correct translation is not possible, and a different strategy is necessary to show the completeness of a candidate axiomatisation.
Work on this problem is on-going.
The main results of this chapter are expressivity results.
We show that the logic \logicRmlKF{} is strictly more expressive than the underlying modal logic \logicKF{}, and strictly less expressive than the bisimulation quantified modal logic \logicBqmlKF{}, the modal $\mu$-calculus \logicMuKF{}, and the tangled modal logic \logicTangleKF{}.
A corollary of the latter results are that \logicRmlKF{} is decidable, via a semantic translation from \langRml{} to \langMu{}.

In the following sections we provide expressivity results for \logicRmlKF{} with respect to various logics.
In Section~\ref{rml-k4-ml} we show that \logicRmlKF{} is strictly more expressive than the underlying modal logic \logicKF{} by demonstrating a semantic property that can be expressed as a \langRml{} formula but not as a \langMl{} formula.
In Section~\ref{rml-k4-bqml} we show that \logicRmlKF{} is (non-strictly) less expressive than the \logicBqmlKF{} by demonstrating a semantic translation from \langRml{} to \langBqml{}.
As \logicBqmlKF{} is expressively equivalent to \logicMuKF{} we have as a corollary that \logicRmlKF{} is less expressive than \logicMuKF{}.
In Section~\ref{rml-k4-tangle} we show that \logicRmlKF{} is strictly less expressive than the tangled modal logic \logicTangleKF{} by demonstrating a semantic property that can be expressed as a \langTangle{} formula but not as a \langRml{} formula.
As \logicTangle{} is less expressive than \logicMuKF{} we have as a corollary that \logicRmlKF{} is strictly less expressive than \logicMuKF{}, and as \logicMuKF{} is expressively equivalent to \logicBqmlKF{} we have as a corollary that \logicRmlKF{} is strictly less expressive than \logicBqmlKF{}.

\section{Expressivity: modal logic}\label{rml-k4-ml}

In this section we show that \logicRmlKF{} is strictly more expressive than the underlying modal logic \logicKF{}.
That \logicRmlKF{} is at least as expressive as \logicKF{} is obvious, as the syntax and semantics of \logicRmlKF{} generalise those of \logicKF{}.
To show that \logicRmlKF{} is strictly more expressive than \logicKF{} we demonstrate a semantic property that can be expressed as a \langRml{} formula under the semantics of \logicRmlKF{}, but that cannot be expressed as any \langMl{} formula under the semantics of \logicKF{}.

Specifically, we will show that \logicRmlKF{} can express the semantic property that there exists an infinite path of states starting from a given pointed Kripke model, whereas \logicKF{} cannot express this property.
We show that there is a \langRml{} formula that corresponds to this semantic property under the semantics of \logicRmlKF{}.
We then show that no \langMl{} formula corresponds to this semantic property under the semantics of \logicKF{} by demonstrating a \classKF{} Kripke model with two states that are distinguishable by this \langRml{} formula, but that are indistinguishable by any \langMl{} formula.

We first show that there is a \langRml{} formula that corresponds to the semantic property that there exists an infinite path of states starting from a given pointed Kripke model, under the semantics of \logicRmlKF{}.
We will later show that no \langMl{} formula corresponds to this semantic property under the semantics of \logicKF{}, so clearly we must rely on refinement quantifiers.
However we will demonstrate a \langMl{} formula that corresponds to a stronger semantic property under the semantics of \logicKF{}, which we use as the basis for our \langRml{} formula.

\begin{lemma}\label{rml-k4-all-infinite-paths}
Let $\kPModelAndTuple{\kStateS} \in \classKF$ be a pointed Kripke model.
If $\kPModel{\kStateS} \entails \possible \top \land \necessary \possible \top$ then there exists an infinite path of states in $\kModel$ starting from the state $\kStateS$.
\end{lemma}

\begin{proof}
Suppose that $\kPModel{\kStateS} \entails \possible \top \land \necessary \possible \top$.
We inductively construct an infinite sequence of states $(\kStateT_0, \kStateT_1, \dots) \in \kStates^\naturals$ such that $\kStateT_0 = \kStateS$ and for every $i \in \naturals$: $\kStateT_{i + 1} \in \kSuccessors{}{\kStateT_i}$ and $\kStateT_{i + 1} \in \kSuccessors{}{\kStateS}$.
Let $\kStateT_0 = \kStateS$.
As $\kPModel{\kStateS} \entails \possible \top$ then there exists $\kStateU \in \kSuccessors{}{\kStateS}$, so we let $\kStateT_1 = \kStateU$.
For every $i \in \naturals$ such that $i \geq 1$, as $\kStateT_i \in \kSuccessors{}{\kStateS}$ and $\kPModel{\kStateS} \entails \necessary \possible \top$ then $\kPModel{\kStateT_i} \entails \possible \top$, so there exists $\kStateU \in \kSuccessors{}{\kStateT_i}$, so we let $\kStateT_{i + 1} = \kStateU$, and note by the transitivity of $\kModel$ that $\kStateT_{i + 1} \in \kSuccessors{}{\kStateS}$.
Therefore there exists an infinite path of states in $\kModel$ starting from the state $\kStateS$.
\end{proof}

Here we have shown a \langMl{} formula that under the semantics of \logicKF{} implies that there exists an infinite path of states starting from a given pointed Kripke model.
However not every state with an infinite path necessarily satisfies this formula.
For example, we may have a pointed Kripke model that has an infinite path, as well as a successor with no successors.
Such a Kripke model would clearly not satisfy $\necessary \possible \top$.
However we 

\begin{lemma}\label{rml-k4-pruning}
Let $\kPModelAndTuple{\kStateS} \in \classKF$ be a pointed Kripke model such that there exists an infinite path of states in $\kModel$ starting from the state $\kStateS$.
Then $\kPModel{\kStateS} \entails \somerefs (\possible \top \land \necessary \possible \top)$.
\end{lemma}

\begin{proof}
Let $\kPModelP{\kStateS} = ((\kStateS, \kAccessibilityP{}, \kValuation), \kStateS)$ be a pointed Kripke model where $\kAccessibilityP{} \subseteq \kAccessibility{}$ and $(\kStateT, \kStateV) \in \kAccessibilityP{}$ if and only if there exists an infinite path of states in $\kModel$ starting from the state $\kStateV$.
By Proposition~\ref{subrelations-refinement} as $\kModelP$ is formed by taking subrelations of $\kModel$ then we have $\kPModel{\kStateS} \simulates \kPModelP{\kStateS}$.
As there exists an infinite path of states in $\kModel$ starting from the state $\kStateS$ there exists $\kStateT \in \kSuccessors{}{\kStateS}$ such that there exists an infinite path of states in $\kModel$ starting from the state $\kStateT$.
Therefore $\kStateT \in \kSuccessorsP{}{\kStateS}$ and $\kPModelP{\kStateS} \entails \possible \top$.
Let $\kStateT \in \kSuccessorsP{}{\kStateS}$.
Then there exists an infinite path of states in $\kModel$ starting from the state $\kStateT$ so there exists $\kStateU \in \kSuccessors{}{\kStateU}$ such that there exists an infinite path of states in $\kModel$ starting from the state $\kStateU$.
Then $\kStateU \in \kSuccessorsP{}{\kStateT}$ so $\kPModelP{\kStateT} \entails \possible \top$.
Therefore $\kPModelP{\kStateS} \entails \necessary \possible \top$.
Therefore $\kPModelP{\kStateS} \entails \possible \top \land \necessary \possible \top$ and $\kPModel{\kStateS} \entails \somerefs (\possible \top \land \necessary \possible \top)$.
\end{proof}

To show the converse we first show that if a refinement of a pointed Kripke model has an infinite path then the original pointed Kripke model also has an infinite path.

\begin{lemma}\label{rml-k4-refinements-infinite-paths}
Let $\kPModelAndTuple{\kStateS}, \kPModelAndTupleP{\kStateSP}$ be pointed Kripke models such that $\kPModel{\kStateS} \simulates \kPModelP{\kStateSP}$.
If there exists an infinite path of states in $\kModelP$ starting from the state $\kStateSP$ then there exists an infinite path of states in $\kModel$ starting from the state $\kStateS$.
\end{lemma}
 
\begin{proof}
Suppose that there exists an infinite path of states in $\kModelP$ starting from the state $\kStateSP$.
Then there exists infinite sequence of states $(\kStateTP_0, \kStateTP_1, \dots) \in \kStatesP^\naturals$ such that $\kStateTP_0 = \kStateSP$ and for every $i \in \naturals$: $\kStateTP_{i + 1} \in \kSuccessorsP{}{\kStateTP_i}$.
As $\kPModel{\kStateS} \simulates \kPModelP{\kStateSP}$ then there exists a refinement relation $\refinement \subseteq \kStates \times \kStatesP$ such that $(\kStateS, \kStateSP) \in \refinement$.
We inductively construct an infinite sequence of states $(\kStateT_0, \kStateT_1, \dots) \in \kStates^\naturals$ such that $\kStateT_0 = \kStateS$ and for every $i \in \naturals$: $\kStateT_{i + 1} \in \kSuccessors{}{\kStateT_{i + 1}}$ and $(\kStateT_i, \kStateTP_i) \in \refinement$.
Let $\kStateT_0 = \kStateS$ and for every $i \in \naturals$, as $(\kStateT_i, \kStateTP_i) \in \refinement$ and $\kStateTP_{i + 1} \in \kSuccessorsP{}{\kStateTP_{i + 1}}$ then from {\bf back} there exists $\kStateU \in \kSuccessors{}{\kStateT_i}$ such that $(\kStateU, \kStateTP_{i + 1}) \in \refinement$, so we let $\kStateTP_{i + 1} = \kStateU$.
Therefore there exists an infinite path of states in $\kModel$ starting from the state $\kStateS$.
\end{proof}

Then the converse follows trivially.

\begin{lemma}\label{rml-k4-pruning-converse}
Let $\kPModel{\kStateS} \in \classKF$ be a pointed Kripke model such that $\kPModel{\kStateS} \entails \somerefs (\possible \top \land \necessary \possible \top)$.
If $\kPModel{\kStateS} \entails \somerefs \possible \top \land \necessary \possible \top$ then there exists an infinite path of states in $\kModel$ starting from the state $\kStateS$.
\end{lemma}

\begin{proof}
Suppose that $\kPModel{\kStateS} \entails \somerefs \possible \top \land \necessary \possible \top$.
Then there exists $\kPModelP{\kStateSP} \in \classKF$ such that $\kPModel{\kStateS} \simulates \kPModelP{\kStateSP}$ such that $\kPModelP{\kStateSP} \entails \possible \top \land \necessary \possible \top$.
By Lemma~\ref{rml-k4-all-infinite-paths} there exists an infinite path of states in $\kModelP$ starting from the state $\kStateSP$.
By Lemma~\ref{rml-k4-refinements-infinite-paths} there exists an infinite path of states in $\kModel$ starting from the state $\kStateS$.
\end{proof}

Finally we get that the formula $\somerefs (\possible \top \land \necessary \possible \top)$ corresponds to the semantic property that there exists an infinite path of states starting from a given pointed Kripke model, under the semantics of \logicRmlKF{}.

\begin{lemma}\label{refinements-infinite-paths-formula}
Let $\kPModel{\kStateS} \in \classKF$ be a pointed Kripke model.
Then $\kPModel{\kStateS} \entails \somerefs(\possible \top \land \necessary \possible \top)$ if and only if there exists an infinite path in $\kModel$ starting from $\kStateS$.
\end{lemma}

\begin{proof}
The left-to-right direction follows from Lemma~\ref{rml-k4-pruning-converse} and the right-to-left direction follows from Lemma~\ref{rml-k4-pruning}.
\end{proof}

We finally show that no \langMl{} formula corresponds to this semantic property under the semantics of \logicKF{} by demonstrating a \classKF{} Kripke model with two states that are distinguishable by this \langRml{} formula, but that are indistinguishable by any \langMl{} formula.

\begin{theorem}
The logic \logicRmlKF{} is strictly more expressive than \logicKF{}.
\end{theorem}

\begin{proof}
Let $\kModelAndTuple$ be a Kripke model where:
\begin{eqnarray*}
\kStateS &=& \naturals \cup \{\omega, \omega'\}\\
\kAccessibility{} &=& \{(m, n), (\omega, n), (\omega', n) \mid m, n \in \naturals, m > n\} \cup \{(\omega', \omega')\}\\
\kValuation(\atomP) &=& \emptyset
\end{eqnarray*}

\begin{figure}
    \centering
    \begin{tikzpicture}[>=stealth',shorten >=1pt,auto,node distance=7em,thick]

      \node (0) {0};
      \node (1) [right of=0] {1};
      \node (2) [right of=1] {2};
      \node (3) [right of=2] {\ldots};
      \node (omega) [above right of=3] {$\omega$};
      \node (omega') [below right of=3] {$\omega'$};

      \path[every node/.style={font=\sffamily\small},->]
        (1) edge node {} (0)
        (2) edge node {} (1)
        (3) edge node {} (2)
        (omega) edge node {} (3)
        (omega') edge node {} (3)
            edge [loop right] node {} (omega');
    \end{tikzpicture}
    \caption{The model $\kModel$, omitting implied transitive edges.}\label{model-naturals}
\end{figure}

The model $\kModel$ is represented in Figure~\ref{model-naturals}.
We note that $\kModel \in \classKFF$.

We claim that $\kPModel{\omega}$ is modally indistinguishable from $\kPModel{\omega'}$.
To show this we show the following intermediate results:
\begin{enumerate}
    \item $\kPModel{i} \bisimilar[n] \kPModel{j}$ for $i, j, n \in \naturals$ where $i, j \geq n$.

        We proceed by induction on $n \in \naturals$.

        Suppose that $n = 0$. Then we trivially have that $\kPModel{i} \bisimilar[0] \kPModel{j}$.

        Suppose that $n > 0$.

        \paragraph{atoms} Trivial.

        \paragraph{forth} Let $k \in \kSuccessors{}{i}$.
        Suppose that $k \geq n - 1$. Then from the induction hypothesis $\kPModel{k} \bisimilar [n-1] \kPModel{j - 1}$.
        Suppose that $k < n - 1$. Then $k < n - 1 < j$ and $k \in \kSuccessors{}{j}$, so we trivially have that $\kPModel{k} \bisimilar \kPModel{k}$.

        \paragraph{back} Symmetrical reasoning to {\bf forth}.

    \item $\kPModel{n} \bisimilar[n] \kPModel{\omega'}$ for $n \in \naturals$

        We proceed by induction on $n \in \naturals$.

        Suppose that $n = 0$. Then we trivially have that $\kPModel{0} \bisimilar[0] \kPModel{\omega'}$.

        Suppose that $n > 0$.

        \paragraph{atoms} Trivial.

        \paragraph{forth} Let $k \in \kSuccessors{}{n}$. Then $k \in \kSuccessors{}{\omega'}$ and we trivially have that $\kPModel{k} \bisimilar \kPModel{k}$.

        \paragraph{back} Let $k \in \kSuccessors{}{\omega'}$. 
        Suppose that $k = \omega'$. Then $n - 1 \in \kSuccessors{}{n}$ and by the induction hypothesis $\kPModel{n - 1} \bisimilar[n - 1] \kPModel{\omega'}$. 
        Suppose that $k \neq \omega'$ and $k < n$. Then $k \in \kSuccessors{}{n}$ and we trivially have that $\kPModel{k} \bisimilar \kPModel{k}$.
        Suppose that $k \neq \omega'$ and $k \geq n$. Then $n - 1 \in \kSuccessors{}{n}$ and from above $\kPModel{n - 1} \bisimilar[n - 1] \kPModel{k}$.

    \item $\kPModel{\omega} \bisimilar[n] \kPModel{\omega'}$ for $n \in \naturals$

        We proceed by induction on $n \in \naturals$.

        Suppose that $n = 0$. Then we trivially have that $\kPModel{\omega} \bisimilar[0] \kPModel{\omega'}$.

        Suppose that $n > 0$.

        \paragraph{atoms} Trivial.

        \paragraph{forth} Let $k \in \kSuccessors{}{\omega}$. Then $k \in \kSuccessors{}{\omega'}$ and from above we have that $\kPModel{k} \bisimilar \kPModel{k}$.

        \paragraph{back} Let $k \in \kSuccessors{}{\omega'}$.
        Suppose that $k = \omega'$. Then $n - 1 \in \kSuccessors{}{\omega}$ and from above $\kPModel{n - 1} \bisimilar[n - 1] \kPModel{\omega'}$.
        Suppose that $k \neq \omega'$. Then $k \in \kSuccessors{}{\omega}$ and we trivially have that $\kPModel{k} \bisimilar \kPModel{k}$.
\end{enumerate}

Therefore $\kPModel{\omega} \bisimilar[n] \kPModel{\omega'}$ for every $n \in \naturals$.

Let $\phi \in \langMl$ and suppose that $d(\phi) = n$ where $n \in \naturals$.
From above $\kPModel{\omega} \bisimilar[n] \kPModel{\omega'}$ so $\kPModel{\omega} \entails \phi$ if and only if $\kPModel{\omega'} \entails \phi$.
Therefore $\kPModel{\omega}$ is modally indistinguishable from $\kPModel{\omega'}$.

However $\kPModel{\omega}$ is refinement modally distinguishable from $\kPModel{\omega'}$ through the \langRml{} formula $\somerefs (\possible \top \land \necessary \possible \top)$. 
As $\omega'$ has a reflexive edge, then there is an infinite path in $\kModel$ starting from $\omega'$.
By Lemma~\ref{refinements-infinite-paths-formula} this implies that $\kPModel{\omega'} \entails \somerefs (\possible \top \land \necessary \possible \top)$.
As any successor $n \in \naturals$ of $\omega$ has a maximum of $n$ steps until the terminal state $0$ then there is no infinite path in $\kModel$ starting from $\omega$.
By Lemma~\ref{refinements-infinite-paths-formula} this implies that $\kPModel{\omega} \nentails \somerefs (\possible \top \land \necessary \possible \top)$.
\end{proof}

\section{Expressivity: bisimulation quantified modal logic}\label{rml-k4-bqml}

\section{Expressivity: tangled modal logic}\label{rml-k4-tangle}

\begin{definition}[$n$-mutual refinements]
Let $\kPModelAndTuple{\kStateS} \in \classK$ and $\kPModelAndTupleP{\kStateSP} \in \classK$ be pointed Kripke models.
We say that $\kPModel{\kStateS}$ and $\kPModelP{\kStateSP}$ are {\em $0$-mutual refinements}
and we write $\kPModel{\kStateS} \bisimref[0] \kPModelP{\kStateSP}$
if and only if for every $\emptyset \subset \agentsB \subseteq \agents$: 
$\kPModel{\kStateS} \refinesBs \kPModelP{\kStateSP}$ 
and $\kPModel{\kStateS} \simulatesBs \kPModelP{\kStateSP}$.
We say that $\kPModel{\kStateS}$ and $\kPModelP{\kStateSP}$ are {\em $n$-mutual refinements}
for some $n \in \naturals$
and we write $\kPModel{\kStateS} \bisimref[n] \kPModelP{\kStateSP}$
if and only if for every $\agentA \in \agents$
the following, {\bf mutual refinements}, {\bf forth-$\agentA$} and {\bf back-$\agentA$} hold:

\paragraph{mutual refinements} 
For every $\emptyset \subset \agentsB \subseteq \agents$:
$\kPModel{\kStateS} \refinesBs \kPModelP{\kStateSP}$ 
and $\kPModel{\kStateS} \simulatesBs \kPModelP{\kStateSP}$ 

\paragraph{forth-$a$} For every $\kStateT \in \kStateS \kAccessibility{\agentA}$ 
there exists $\kStateTP \in \kStateSP \kAccessibilityP{\agentA}$
such that $\kPModel{\kStateT} \bisimref[n-1] \kPModelP{\kStateTP}$.

\paragraph{back-$a$} For every $\kStateTP \in \kStateSP \kAccessibilityP{\agentA}$
there exists $\kStateT \in \kStateS \kAccessibility{\agentA}$ 
such that $\kPModel{\kStateT} \bisimref[n-1] \kPModelP{\kStateTP}$.
\end{definition}

\begin{lemma}
Let $\classC$ be a class of Kripke frames,
let $n \in \naturals$,
let $\phi \in \langRml$ such that $d(\phi) \leq n$
and let $\kPModelAndTuple{\kStateS}, \kPModelAndTupleP{\kStateSP} \in \classC$ be pointed Kripke models
such that $\kPModel{\kStateS} \bisimref[n] \kPModelP{\kStateSP}$.
Then $\kPModel{\kStateS} \entails_\logicRmlC \phi$ if and only if $\kPModelP{\kStateSP} \entails_\logicRmlC \phi$.
\end{lemma}

\begin{proof}
We proceed by induction on the modal depth and structure of $\phi$.

Suppose that $\phi = \atomP$ where $\atomP \in \atoms$. 
As $\kPModel{\kStateS}$ and $\kPModelP{\kStateSP}$ are $n$-mutual refinements then $\kPModel{\kStateS} \refines \kPModelP{\kStateSP}$ and from {\bf atoms-$\atomP$} we have that $\kStateS \in \kValuation(\atomP)$ if and only if $\kStateSP \in \kValuationP(\atomP)$ and therefore $\kPModel{\kStateS} \entails \atomP$ if and only if $\kPModelP{\kStateSP} \entails \atomP$.

Suppose that $\phi = \neg \psi$ or $\phi = \psi \land \chi$.
These cases follow directly from the induction hypothesis.

Suppose that $\phi = \necessaryA \psi$ and $\kPModel{\kStateS} \entails \necessaryA \psi$.
Then for every $\kStateT \in \kStateS \kAccessibility{\agentA}$ we have $\kPModel{\kStateT} \entails \psi$.
Let $\kStateTP \in \kStateSP \kAccessibility{\agentA}$.
By {\bf back-$\agentA$} there exists $\kStateT \in \kStateS \kAccessibility{\agentA}$ such that $\kPModel{\kStateT}$ is $(n-1)$-mutual refinements to $\kPModelP{\kStateTP}$.
As $\kPModel{\kStateT} \entails \psi$ and $d(\psi) \leq n - 1$ then by the induction hypothesis we have that $\kPModelP{\kStateTP} \entails \psi$.
So for every $\kStateTP \in \kSuccessorsA{\kStateSP}$ we have $\kPModelP{\kStateTP} \entails \psi$.
Therefore $\kPModelP{\kStateSP} \entails \necessaryA \psi$.
The converse follows from symmetrical reasoning.

Suppose that $\phi = \allrefsBs \psi$ and $\kPModel{\kStateS} \entails \allrefsBs \psi$.
Then there exists $\kPModelPP{\kStateSPP} \in \classC$ such that $\kPModel{\kStateS} \simulatesBs \kPModelPP{\kStateSPP}$ and $\kPModelPP{\kStateS} \entails \psi$.
As $\kPModelP{\kStateSP} \simulatesBs \kPModel{\kStateS}$ then by Lemma~\ref{refinements-preorder} we have that $\kPModelP{\kStateSP} \simulatesBs \kPModelPP{\kStateSPP}$.
Therefore $\kPModelP{\kStateSP} \entails \allrefsBs \psi$.
The converse follows from symmetrical reasoning.
\end{proof}

\begin{theorem}
The logic \logicRmlKF{} is strictly less expressive than \logicTangleKF{}.
\end{theorem}

\begin{proof}
Let $\kModelAndTuple$ be a Kripke model where:
\begin{eqnarray*}
    \kStates &=& \{n, n^+, n^- \mid n \in \naturals\} \cup \{\omega, \omega'\}\\
    \kAccessibility{} &=& \{(n, m), (n, m^+), (n, m^-), (n, n^+), \\&&\quad(n, n^-), (n^+, n^+), (n^-, n^-) \mid n, m \in \naturals, n > m\} \\&&
        \cup \{0, 0^+, 0^-\} \times \{0^+, 0^-\}
        \cup \{(0, 0)\} \\&&
        \cup \{\omega, \omega'\} \times \{n, n^+, n^- \mid n \in \naturals\}
        \cup \{(\omega', \omega')\}\\
    \kValuation(\atomP) &=& \{n, n^+ \mid n \in \naturals\} \cup \{\omega, \omega'\}
\end{eqnarray*}
The model $\kModel$ is represented in Figure~\ref{model-caterpillar}.

\begin{figure}
    \centering
    \begin{tikzpicture}[>=stealth',shorten >=1pt,auto,node distance=7em,thick]

      \node (0) {$0$};
      \node (0+) [above left of=0] {$0^+$};
      \node (0-) [below left of=0] {$0^-$};
      \node (1) [right of=0] {$1$};
      \node (1+) [above left of=1] {$1^+$};
      \node (1-) [below left of=1] {$1^-$};
      \node (2) [right of=1] {$2$};
      \node (2+) [above left of=2] {$2^+$};
      \node (2-) [below left of=2] {$2^-$};
      \node (3) [right of=2] {\ldots};
      \node (omega) [above right of=3] {$\omega$};
      \node (omega') [below right of=3] {$\omega'$};

      \path[every node/.style={font=\sffamily\small},->]
        (0) edge [loop left] node {} (0)
            edge node {} (0+)
            edge node {} (0-)
        (0+) edge [loop above] node  {} (0+)
            edge [<->,bend right] node {} (0-)
        (0-) edge [loop below] node  {} (0-)
        (1) edge node {} (0)
            edge node {} (1+)
            edge node {} (1-)
        (1+) edge [loop above] node {} (1+)
        (1-) edge [loop below] node {} (1-)
        (2) edge node {} (1)
            edge node {} (2+)
            edge node {} (2-)
        (2+) edge [loop above] node {} (2+)
        (2-) edge [loop below] node {} (2-)
        (3) edge node {} (2)
        (omega) edge node {} (3)
        (omega') edge node {} (3)
            edge [loop right] node {} (omega');
    \end{tikzpicture}
    \caption{The model $\kModel$, omitting implied transitive edges.}\label{model-caterpillar}
\end{figure}

To show that $\kPModel{\omega}$ and $\kPModel{\omega'}$ are indistinguishable by the refinement modal logic we show that $\kPModel{\omega} \bisimref[n] \kPModel{\omega'}$ for all $n \in \naturals$. To do so we first show that $\kPModel{\omega}$ and $\kPModel{\omega'}$ are mutual refinements.

That $\kPModel{\omega} \refines \kPModel{\omega'}$ is trivial, as $\omega$ is the same as $\omega'$ except for the reflexive edge, so we need only show that $\kPModel{\omega'} \refines \kPModel{\omega}$.

Let $\refinement \subseteq \kStates \times \kStates$ be defined as follows:
$$
\refinement = \{(\kStateS, \kStateS) \mid \kStateS \in \kStates\} \cup
\{(0, n), (n, 0), (0^+, n^+), (0^-, n^-), (\omega, n), (\omega', n) \mid n \in \naturals\} \cup
\{(0, \omega), (0, \omega'), (\omega, \omega')\}
$$

% TODO
We show that $\refinement$ satisfies {\bf atoms-$\atomP$} and {\bf back} for every $(\kStateS, \kStateSP) \in \refinement$.

\paragraph{atoms}
Trivial.

\paragraph{back-$a$}
Consider $(\kStateS, \kStateS) \in \refinement$ and let $\kStateT \in \kSuccessors{}{\kStateS}$. Then $(\kStateT, \kStateT) \in \refinement$.

Consider $(0, n) \in \refinement$ and let $\kStateT \in \kSuccessors{}{n}$. 
Suppose that $\kStateT = m$ for $m \in \naturals$. 
Then $0 \in \kSuccessors{}{0}$ and $(0, m) \in \refinement$.
Suppose that $\kStateT = m^+$ for $m \in \naturals$.
Then $0^+ \in \kSuccessors{}{0}$ and $(0^+, m^+) \in \refinement$.
Suppose that $\kStateT = m^-$ for $m \in \naturals$.
Then $0^- \in \kSuccessors{}{0}$ and $(0^-, m^-) \in \refinement$.

Consider $(n, 0) \in \refinement$ and let $\kStateT \in \kSuccessors{}{0}$.
Then $\kStateT \in \kSuccessors{}{n}$ and $(\kStateT, \kStateT) \in \refinement$.

Consider $(0^+, n^+) \in \refinement$ and let $\kStateT \in \kSuccessors{}{n^+}$. 
Then $t = n^+$, $0^+ \in \kSuccessors{}{0^+}$ and $(0^+, n^+) \in \refinement$.

Consider $(0^-, n^-) \in \refinement$ and let $\kStateT \in \kSuccessors{}{n^-}$. 
Then $t = n^-$, $0^- \in \kSuccessors{}{0^-}$ and $(0^-, n^-) \in \refinement$.

Consider $(0, \omega) \in \refinement$ and let $\kStateT \in \kSuccessors{}{\omega}$. 
This case follows the same reasoning as for the case for $(0, n) \in \refinement$.

Consider $(0, \omega') \in \refinement$ and let $\kStateT \in \kSuccessors{}{\omega}$. 
Suppose that $t \neq \omega'$. 
This case follows the same reasoning as for the case for $(0, n) \in \refinement$. 
Suppose that $t = \omega'$. 
Then $0 \in \kSuccessors{}{0}$ and $(0, \omega') \in \refinement$.

Consider $(\omega, \omega') \in \refinement$ and let $\kStateT \in \kSuccessors{}{\omega'}$. 
Suppose that $t \neq \omega'$. 
Then $t \in \kSuccessors{}{\omega}$ and $(t, t) \in \refinement$. 
Suppose that $t = \omega'$. 
Then $0 \in \kSuccessors{}{\omega}$ and $(0, \omega') \in \refinement$.

Consider $(\omega, n) \in \refinement$ and let $\kStateT \in \kSuccessors{}{n}$.
Then $\kStateT \in \kSuccessors{}{\omega}$ and $(t, t) \in \refinement$.

Consider $(\omega', n) \in \refinement$ and let $\kStateT \in \kSuccessors{}{n}$.
Then $\kStateT \in \kSuccessors{}{\omega'}$ and $(t, t) \in \refinement$.

We show that $\kPModel{\omega} \bisimref[n] \kPModel{\omega'}$ for every $n \in \naturals$. 
To do so we show the following intermediate results:

\begin{enumerate}
    \item $\kPModel{i} \bisimref[n] \kPModel{j}$ for $i, j, n \in \naturals$ where $i, j \geq n$.

    By induction on $n \in \naturals$.

    Suppose that $n = 0$. 
    From above, $\kPModel{i} \refines \kPModel{0} \refines \kPModel{j}$ and $\kPModel{i} \simulates \kPModel{0} \simulates \kPModel{j}$ so we have that $\kPModel{i} \bisimref[0] \kPModel{j}$.

    Suppose that $n > 0$.

    \paragraph{mutual refinements}

    From above, $\kPModel{i} \refines \kPModel{0} \refines \kPModel{j}$
    and $\kPModel{i} \simulates \kPModel{0} \simulates \kPModel{j}$.

    \paragraph{forth}

    Let $k^* \in \kSuccessors{}{i}$.
    Suppose that $k^* = k$ where $k \in \naturals$ and $k \geq n - 1$.
    Then from the induction hypothesis $\kPModel{k} \bisimilar [n-1] \kPModel{j - 1}$.
    Suppose that $k^* = k$ where $k \in \naturals$ and $k < n - 1$.
    Then $k < n - 1 < j$ so $k \in \kSuccessors{}{j}$ and we trivially have that $\kPModel{k} \bisimilar \kPModel{k}$.
    Suppose that $k^* = 0^+$.
    Then $0^+ \in \kSuccessors{}{j}$ and we trivially have that $\kPModel{0^+} \bisimilar \kPModel{0^+}$.
    Suppose that $k^* = k^+$ where $k \in \naturals$ and $k > 0$.
    Then $j^+ \in \kSuccessors{}{j}$ and as $j \geq n > 0$ we trivially have that $\kPModel{k^+} \bisimilar \kPModel{k^+}$.
    Suppose that $k^* = k^-$ for $k \in \naturals$. This follows from similar reasoning to the case where $k^* = k^+$.

    \paragraph{back}

    Symmetrical reasoning to {\bf forth}.

    \item $\kPModel{n} \bisimref[n] \kPModel{\omega'}$ for $n \in \naturals$.

    By induction on $n$.

    Suppose that $n = 0$. 
    From above, $\kPModel{n} \refines \kPModel{0} \refines \kPModel{\omega'}$ and $\kPModel{n} \simulates \kPModel{\omega'}$ so we have that $\kPModel{0} \bisimref[0] \kPModel{\omega'}$.

    Suppose that $n > 0$.

    \paragraph{mutual refinements}

    From above, $\kPModel{n} \refines \kPModel{0} \refines \kPModel{\omega'}$ and $\kPModel{n} \simulates \kPModel{\omega'}$.

    \paragraph{forth}

    Let $k \in \kSuccessors{}{n}$. 
    Then $k \in \kSuccessors{}{\omega'}$ and we trivially have that $\kPModel{k} \bisimilar \kPModel{k}$.

    \paragraph{back}

    Let $k \in \kSuccessors{}{\omega'}$. 
    Suppose that $k = \omega'$. 
    Then $n - 1 \in \kSuccessors{}{n}$ and by the induction hypothesis $\kPModel{n - 1} \bisimilar{n - 1} \kPModel{\omega'}$.
    Suppose that $k \neq \omega'$ and $k < n$. Then $k \in \kSuccessors{}{n}$ and we trivially have that $\kPModel{k} \bisimilar \kPModel{k}$.
    Suppose that $k \geq n$. Then $n - 1 \in \kSuccessors{}{n}$ and from above we have that $\kPModel{k} \bisimref[n - 1] \kPModel{n - 1}$.
    
    \item $\kPModel{\omega} \bisimref[n] \kPModel{\omega'}$ for $n \in \naturals$.

    By induction on $n$.

    Suppose that $n = 0$. 
    From above $\kPModel{\omega} \refines \kPModel{\omega'}$ and  $\kPModel{\omega} \simulates \kPModel{\omega'}$ so we have that $\kPModel{\omega} \bisimref[0] \kPModel{\omega'}$.

    Suppose that $n > 0$.

    \paragraph{mutual refinements}

    From above $\kPModel{\omega} \refines \kPModel{\omega'}$ and  $\kPModel{\omega} \simulates \kPModel{\omega'}$.

    \paragraph{forth}

    Let $k \in \kSuccessors{}{\omega}$.
    Then $k \in \kSuccessors{}{\omega'}$ and we trivially have that $\kPModel{k} \bisimilar \kPModel{k}$.

    \paragraph{back}

    Let $k \in \kSuccessors{}{\omega'}$.
    Suppose that $k = \omega'$.
    Then $n - 1 \in \kSuccessors{}{\omega}$ and from above we have that $\kPModel{n - 1} \bisimref[n - 1] \kPModel{\omega'}$.
    Suppose that $k \neq \omega'$.
    Then $k \in \kSuccessors{}{\omega}$ and we trivially have that $\kPModel{k} \bisimilar \kPModel{k}$.
\end{enumerate}

Therefore $\kPModel{\omega} \bisimref[n] \kPModel{\omega'}$ for every $n \in \naturals$ and so $\kPModel{\omega}$ and $\kPModel{\omega'}$ are indistinguishable to refinement modal logic.

We next show that the states $\kPModel{\omega}$ and $\kPModel{\omega'}$ are distinguishable by the tangled modal logic formula $\tangle \{\possible \necessary \atomP, \possible \necessary \neg \atomP\}$.
We first note that in the modal $\mu$-calculus we have that $\tangle \{\possible \necessary \atomP, \possible \necessary \neg \atomP\} \equiv \gfp{\varX} (\possible (\varX \land \possible \necessary \atomP) \land \possible (\varX \land \possible \necessary \neg \atomP))$.
We proceed with model checking using the modal $\mu$-calculus.

For any assignment $\kAssignment$ we have the following:
\begin{eqnarray*}
    \interpretation[\kAssignment]{\atomP} &=& \{n, n^+ \mid n \in \naturals\} \cup \{\omega, \omega'\}\\
    \interpretation[\kAssignment]{\necessary \atomP} &=& \{n^+ \mid n \in \naturals, n > 0\}\\
    \interpretation[\kAssignment]{\possible \necessary \atomP} &=& \{n, n^+ \mid n \in \naturals, n > 0\} \cup \{\omega, \omega'\}\\
    \interpretation[\kAssignment]{\neg \atomP} &=& \{n^- \mid n \in \naturals\}\\
    \interpretation[\kAssignment]{\necessary \neg \atomP} &=& \{n^- \mid n \in \naturals, n > 0\}\\
    \interpretation[\kAssignment]{\possible \necessary \neg \atomP} &=& \{n, n^- \mid n \in \naturals, n > 0\} \cup \{\omega, \omega'\}
\end{eqnarray*}

For any assignment $\kAssignment$ where $\kAssignment(\varX) = \kStates$ we have the following:
\begin{eqnarray*}
    \interpretation[\kAssignment]{\varX} &=& \kStates\\
    \interpretation[\kAssignment]{\varX \land \possible \necessary \atomP} &=& \{n, n^+ \mid n \in \naturals, n > 0\} \cup \{\omega, \omega'\}\\
    \interpretation[\kAssignment]{\possible (\varX \land \possible \necessary \atomP)} &=& \{n, n^+ \mid n \in \naturals, n > 0\} \cup \{\omega, \omega'\}\\
    \interpretation[\kAssignment]{\varX \land \possible \necessary \neg \atomP} &=& \{n, n^- \mid n \in \naturals, n > 0\} \cup \{\omega, \omega'\}\\
    \interpretation[\kAssignment]{\possible (\varX \land \possible \necessary \neg \atomP)} &=& \{n, n^- \mid n \in \naturals, n > 0\} \cup \{\omega, \omega'\}\\
    \interpretation[\kAssignment]{\possible (\varX \land \possible \necessary \atomP) \land \possible (\varX \land \possible \necessary \neg \atomP)} &=& \{n \mid n \in \naturals, n > 0\} \cup \{\omega, \omega'\}
\end{eqnarray*}

For any assignment $\kAssignment$ where $\kAssignment(\varX) = \{n \mid n \in \naturals, n > m\} \cup \{\omega, \omega'\}$ for some $m \in \naturals$ we have the following:
\begin{eqnarray*}
    \interpretation[\kAssignment]{\varX} &=& \{n \mid n \in \naturals, n > m\} \cup \{\omega, \omega'\}\\
    \interpretation[\kAssignment]{\varX \land \possible \necessary \atomP} &=& \{n \mid n \in \naturals, n > m\} \cup \{\omega, \omega'\}\\
    \interpretation[\kAssignment]{\possible (\varX \land \possible \necessary \atomP)} &=& \{n \mid n \in \naturals, n > m + 1\} \cup \{\omega, \omega'\}\\
    \interpretation[\kAssignment]{\varX \land \possible \necessary \neg \atomP} &=& \{n \mid n \in \naturals, n > m\} \cup \{\omega, \omega'\}\\
    \interpretation[\kAssignment]{\possible (\varX \land \possible \necessary \neg \atomP)} &=& \{n \mid n \in \naturals, n > m + 1\} \cup \{\omega, \omega'\}\\
    \interpretation[\kAssignment]{\possible (\varX \land \possible \necessary \atomP) \land \possible (\varX \land \possible \necessary \neg \atomP)} &=& \{n \mid n \in \naturals, n > m + 1\} \cup \{\omega, \omega'\}
\end{eqnarray*}

For any assignment $\kAssignment$ where $\kAssignment(\varX) = \{\omega, \omega'\}$ for some $m \in \naturals$ we have the following:
\begin{eqnarray*}
    \interpretation[\kAssignment]{\varX} &=& \{\omega, \omega'\}\\
    \interpretation[\kAssignment]{\varX \land \possible \necessary \atomP} &=& \{\omega, \omega'\}\\
    \interpretation[\kAssignment]{\possible (\varX \land \possible \necessary \atomP)} &=& \{\omega'\}\\
    \interpretation[\kAssignment]{\varX \land \possible \necessary \neg \atomP} &=& \{\omega, \omega'\}\\
    \interpretation[\kAssignment]{\possible (\varX \land \possible \necessary \neg \atomP)} &=& \{\omega'\}\\
    \interpretation[\kAssignment]{\possible (\varX \land \possible \necessary \atomP) \land \possible (\varX \land \possible \necessary \neg \atomP)} &=& \{\omega'\}
\end{eqnarray*}

Therefore for any assignment $\kAssignment$ we have that: $$\interpretation[\kAssignment]{\gfp{\varX} (\possible (\varX \land \possible \necessary \atomP) \land \possible (\varX \land \possible \necessary \neg \atomP))} = \{\omega'\}$$

Therefore $\kPModel{\omega'} \entails \gfp{\varX} (\possible (\varX \land \possible \necessary \atomP) \land \possible (\varX \land \possible \necessary \neg \atomP))$, 
but $\kPModel{\omega} \nentails \gfp{\varX} (\possible (\varX \land \possible \necessary \atomP) \land \possible (\varX \land \possible \necessary \neg \atomP))$
and so $\kPModel{\omega}$ and $\kPModel{\omega'}$ are distinguishable by the tangled modal logic.

\end{proof}

\begin{corollary}
The logic \logicRmlKF{} is strictly less expressive than \logicMuKF{}.
\end{corollary}
