\chapter{Literature review}\label{literature}

The field of dynamic epistemic logic considers the changes of knowledge that
result from the communication of information through events known as informative
updates. Often the effects of informative updates on knowledge can be
unintuitive or surprising: sometimes announcing a true statement makes it
become false, as in Fitch's knowability paradox~\cite{fitch:1963}; sometimes
repeating a statement can provide different information each time it is
repeated, as in the muddy children puzzle~\cite{barwise:1981, vanditmarsch:2007}.
Being able to reason about knowledge and changes in knowledge has applications
in a number of areas, including in economics and game theory, where we want to
reason about games with imperfect knowledge, in areas such as artificial
intelligence, where we want to represent and update knowledge bases, and in
areas such as computer security, where we want to reason about unintended
side-effects of information that is communicated through a system. Sometimes it
is useful to know the effects of a specific informative
update~\cite{plaza:1989,gerbrandy:1997,baltag:1998}, such as when a robot updates a
knowledge base with new sensor information, while at other times it is useful to
know about the effects of arbitrary informative
updates~\cite{balbiani:2007,agotnes:2010,vanditmarsch:2009}, such as when we want
to guarantee that a computer system will not leak sensitive information to
unauthorised parties no matter what communication it takes part in. Formal
logics of knowledge have existed for many
decades~\cite{vonwright:1951,hintikka:1957,hintikka:1961,hintikka:1962},
whilst logics for reasoning about the effects of specific informative updates
have only arisen relatively recently~\cite{plaza:1989,gerbrandy:1997,baltag:1998}.
Much more recently logics for reasoning about arbitrary informative updates have
been considered~\cite{balbiani:2007,vanditmarsch:2009,agotnes:2010}, and these are
the focus of our research. This literature review covers the development of
logics of knowledge, logics of specific informative updates and finally logics
of arbitrary informative updates.

\section{Logics of knowledge}

Epistemic logic is the modal logic in which discussions about knowledge take
place in dynamic epistemic logic. Modal logics extend propositional logic with
modal operators that qualify the truth of statements in the logic. In epistemic
logic, the modal operators allow us to qualify the truth of a statement by
saying that a particular agent knows that the statement is true. For example, we
can qualify the proposition ``The coin has landed heads up'' by saying that
``Alice knows that the coin has landed heads up''. Modal operators may also be
nested, allowing us to make statements about an agent's knowledge about its own
or another agent's knowledge. For example, we could say that ``Alice knows that
Alice knows that the coin has landed heads up'', or ``Bob doesn't know that
Alice knows that the coin has landed heads up''.

The semantics for many modal logics, including epistemic logic, are based in
relational structures known as Kripke models~\cite{kripke:1963,blackburn:2001}. A
Kripke model consists of a set of nodes called worlds, where each world has a
different propositional valuation. Each agent has an accessibility relation
defined over the worlds. At each world, the worlds that are accessible for an
agent from that world are considered to be the worlds that the agent considers
possible. An agent is said to know that a statement is true in a given world if
that statement is true on each of the worlds that the agent considers possible.

\subsection{Modal and epistemic logics}

Lewis and Langford are widely acknowledged as the progenitors of early modal
logic, with the earliest symbolic treatment of modal logic dating back to work
by Lewis in 1912, and leading to a book with Langford~\cite{langford:1959} in
1959. Early work in modal logic was mostly syntactic, lacking any formal
semantics. Carnap~\cite{carnap:1946, carnap:1947} first considered the notion of
possible worlds to represent the semantics of modal logics, and other authors,
amongst them Hintikka~\cite{hintikka:1957, hintikka:1961} and
Kripke~\cite{kripke:1959} further developed these semantics, resulting in the
final form by Kripke~\cite{kripke:1963}, the namesake of Kripke models and Kripke
semantics for modal logics. von Wright~\cite{vonwright:1951} was responsible for
the first logical analysis of knowledge in terms of modal logic in 1951, and
this was further developed by Hintikka~\cite{hintikka:1957,hintikka:1961}
culminating in the first book-length treatment of the subject by
Hintikka~\cite{hintikka:1962} in 1962.

\subsection{Common knowledge}

The first work on the topic of common knowledge was by Lewis~\cite{lewis:1969}
and later work was by McCarthy, Sato, Hayashi and Igarishi~\cite{mccarthy:1979}.
Common knowledge is described by McCarthy as what ``any fool knows''; for a
statement to be common knowledge, it is required that everyone knows that the
statement is true, that every agent knows that every agent knows that the
statement is true, and so on. Whereas the definition of common knowledge of
Lewis and McCarthy was in terms of modal logics, Aumann~\cite{aumann:1976} gave
an alternative definition for common knowledge using Aumann structures and the
meet of structures rather than Kripke models and modal formulas. Common
knowledge is of interest to economics and game theory, where common knowledge of
rules or outcomes is assumed in order to reason about games.
Aumann~\cite{aumann:1976} discusses common knowledge with a focus towards
discussing economics and game theory.  Lehmann~\cite{lehmann:1984} and Halpern
and Moses~\cite{halpern:1985} considered common knowledge in depth, and the book
by Fagin, Halpern, Moses and Vardi~\cite{fagin:1995} gives a survey of much of
their work in this area.

\subsection{Alternative logics of knowledge}

Non-modal logics of knowledge have been considered.  Aumann~\cite{aumann:1976}
proposed an event-based approach using Aumann structures, which represents
knowledge as an operator on events rather than reasoning about knowledge using
logical formulas. There is in fact a one-to-one correspondence between epistemic
Kripke models and Aumann structures~\cite{fagin:1995}. A number of authors, among
them van Emde Boas, Groenendijk, and Stokhof~\cite{vanemdeboas:1980}, Fagin and
Vardi~\cite{fagin:1985}, Mertens and Zamir~\cite{mertens:1985} and Fagin, Halpern
and Vardi~\cite{fagin:1991} have considered modeling knowledge and belief using
an infinite hierarchy of sets representing the relative strength or plausibility
of each piece of knowledge. This representation lends itself easily to the
concept of belief revision, discussed in the next section. Fagin, Halpern and
Vardi~\cite{fagin:1991} discussed the relationship between this representation
of knowledge and belief with modal logic.

%\subsection{Applications of logics of knowledge}

%\begin{enumerate}
    %\item McCarthy 1979 - Knowledge and belief for machines~\cite{mccarthy:1979b}
    %\item Aumann [5~\cite{aumann:1976},6~\cite{aumann:1999},7~\cite{aumann:1995}] -
        %Economics and Game theory

    %\item Orlowska 1989 - Knowledge as operator on events~\cite{orlowska:1989}
    %\item Brandenburger, et al. 1993, etc. - "Direct" representation of
        %knowledge~\cite{brandenburger:1993}
    %\item Fagin, Halpern, Vardi 1991 - Relationship of Brandenburger knowledge
        %to ML~\cite{fagin:1991}
    %\item Levesque 1984a - KD45 as knowledge~\cite{levesque:1984a}
    %\item Fagin and Halpern 1988a, Levesque 1984b - KD45 as
        %belief~\cite{fagin:1987, levesque:1984b}
    %\item Friedman and Halpern 1977, Kraus and Lehmann 1988, Moses and Shoham
        %1933, Voorbraak 1992 - System with knowledge and belief, interaction
    %between knowledge and belief~\cite{friedman:1997, kraus:1988, moses:1993,
    %voorbraak:1992}
    %% TODO AI, economics, game theory
%\end{enumerate}

\section{Logics of specific informative updates}

In dynamic epistemic logic, information is communicated through informative
updates, such as messages or announcements. Informative updates may communicate
information about the state of the world, or about the knowledge of agents in
the world, but cannot change the state of the world itself. For example, after
flipping a coin, if Alice were to tell Bob that the coin has landed heads up,
this would be a valid informative update; it communicates both information about
the state of the world (that the coin has landed heads up) as well as
information about the knowledge of an agent (that Alice knows that the coin has
landed heads up) and its only effect is to change the knowledge of Alice and
Bob. However the act of Alice flipping the coin would not be a valid informative
update, as it changes the state of the world itself (the truth value of whether
the coin has landed heads up or not).

Branches of dynamic epistemic logic consider logics for reasoning about the
effects of specific informative updates on the knowledge of agents. These logics
allow one to make statements such as ``After the informative update $\alpha$,
then $\phi$ becomes true''. For example, we can say that ``After Alice tells Bob
that the coin has landed heads up, then Bob knows that the coin has landed heads
up.'' Logics of this type have been considered for a number of different models
for informative updates. Often the effects of performing an informative update
are modelled as an operation of the informative update on a Kripke model, but
there are examples where this is not the case.  

% TODO Muddy children

Notable logics for reasoning about specific informative updates include the
logics of belief revision of Alchourr{\'o}n, G{\"a}rdenfors and
Makinson~\cite{alchourron:1985}, the logics of public announcements of
Plaza~\cite{plaza:1989} and Gerbrandy and Groenvald~\cite{gerbrandy:1997}, the
logics of epistemic actions of van Ditmarsch~\cite{vanditmarsch:1999,
vanditmarsch:2000, vanditmarsch:2002} and the logics of action models of Baltag,
Moss and Solecki~\cite{baltag:1999, baltag:2004}. A survey of these logics and
related areas is given in the book by van Ditmarsch, van der Hoek and
Kooi~\cite{vanditmarsch:2007}.

\subsection{Public announcement logic}

The public announcement logic was introduced by Plaza~\cite{plaza:1989}, and
Gerbrandy and Groenvald~\cite{gerbrandy:1997}. A public announcement is a simple
informative update where a true statement is announced to all agents in a system
at once. The public announcement logic extends epistemic logic with an
operator that means ``After publicly announcing $\alpha$, then $\phi$ is
true.'' The result of publicly announcing a statement is often that the
statement becomes common knowledge amongst the agents. As the statement is
announced to all agents, obviously the statement becomes known by all agents,
but as the statement is announced {\em publicly} this fact also becomes known
by all agents, so that every agent knows that every agent knows that the
statement is true and so on. For example, Alice could announce publicly to Bob
and Carol that the coin has landed heads up. As a result, not only does Bob now
know that the coin has landed heads up, Bob also knows that Carol knows that
the coin has landed up, because Bob witnessed Carol receive the same
information.  Public announcements are modelled as modal formulas, and the
effect of a public announcement can be modelled as an operation on a Kripke
model by restricting the worlds of the Kripke model to those worlds where the
formula is true, removing those worlds where the formula is false.

Plaza~\cite{plaza:1989} formulated and axiomatised a multi-agent public
announcement logic with common knowledge operators, but without introspection of
knowledge, i.e. agents cannot reason about their own knowledge.  Gerbrandy and
Groenvald~\cite{gerbrandy:1997} formulated and axiomatised a multi-agent public
announcement logic without common knowledge operators, but with introspection of
knowledge. Baltag, Moss and Solecki~\cite{baltag:1998,baltag:2004} provided a sound
and complete axiomatisation of the public announcement logic with common
knowledge operators and introspection of knowledge as a special case of their
action model logic with common knowledge.

Public announcements are a very limited form of informative update, as the
information communicated by a public announcement must be communicated publicly
to all agents. Public announcements cannot model informative updates that
provides information to only some of the agents in the system, or that provides
different information to each agent. Public announcements are however suited to
some interesting problems; for example, Fitch's knowability
paradox~\cite{fitch:1963} can be adequately modelled with the public announcement
logic, as can the muddy children puzzle~\cite{barwise:1981, vanditmarsch:2007}.

% TODO Discuss Fitch/muddy children in more depth (?)

\subsection{Action model logic}

Action models capture a more general notion of informative updates than public
announcements. Compared to public announcements, action models are able to
represent informative updates that provide information privately to some of the
agents in the system, providing different information to each agent in the
system. When considering informative updates that communicate information privately to
some agents, there are a number of ways in which the other agents in the system
can interpret that informative update. 

For example, suppose that Alice and Bob have made a bet on the flip of a coin.
Alice flips the coin and catches it in her hand so that neither person can see
what side it has landed on initially. Suppose that in reality the coin has
landed heads up. Alice then opens her hand and looks at the coin without letting
Bob see the coin. The act of Alice looking at the coin is a private informative
update, the result of which is that Alice now knows that the coin has landed
heads up, whilst Bob still doesn't know either way. Suppose that Bob saw Alice
look at the coin. Then Bob witnessed an informative update take place, but is
uncertain as to which update actually occurred. From the point of view of Bob,
Alice may have learned that the coin has landed heads up, or she may have
learned that the coin has landed tails up. Bob cannot distinguish between these
two possibilities, however he does know that whichever update took place, the
result is that Alice now knows whether the coin landed heads up or not. We can
compare this to a situation where Bob didn't see Alice look at the coin, but
considers it possible that she took a peek. Then from the point of view of Bob,
Alice may have learned that the coin has landed heads up, she may have learned
that the coin has landed tails up, or she might not have learned anything at
all. Compared to the previous situation, in this situation Bob doesn't know that
Alice learned which side the coin landed on, because from his point of view it's
possible she didn't learn anything at all.

Action models are relational structures similar to Kripke models, but instead of
each node, called an action point, having its propositional valuation, action
points instead have an epistemic formula called a precondition. The effect of an
action model on a Kripke model is modelled as an operation of the action model
on the Kripke model that results in another Kripke model representing the
resulting knowledge state after performing the informative update that the
action model represents. Roughly speaking the operation consists of a set
product between the set of worlds from the Kripke model and the set of action
points from the action model, followed by a restriction of the resulting pairs
of worlds and action points to those pairs where the world of the Kripke model
satisfies the precondition of the action point. The action points and their
preconditions can be seen as representing a possible ``action'', representing an
informative update that communicates the information in the precondition in some
way. The accessibility relation in the action model is used to represent
uncertainty from the point of view of the agents as to which informative update
actually took place. Similar to the concept of possible worlds in a Kripke
model, in an action model there is a concept of possible actions. For example,
when Alice looks at the coin after flipping it, she only considers one
informative update to have been possible (where she learns that the coin has
landed heads up), whereas Bob considers several informative updates to have been
possible (where Alice learns either that the coin has landed heads up or the
coin has landed tails up). This uncertainty can be represented in an action
model by having separate action points for each possible action (for Alice
learning that the coin has landed heads up and for Alice learning that the coin
has landed tails up) and then giving Alice and Bob different accessibility
relations over the action points.

The action model logic was introduced by Baltag, Solecki and
Moss~\cite{baltag:1998, baltag:1999}. The action model logic extends epistemic
logic with an operator that means ``After performing the action model $M$, then
$\phi$ is true.'' Baltag, Solecki and Moss~\cite{baltag:1998} provided a sound
and complete axiomatisation for the logic with and without common knowledge
operators. Later work by Baltag and Moss~\cite{baltag:2004} emphasised the
generality of the action model approach, providing many examples of action
models representing various kinds of informative updates. Baltag and
Moss~\cite{baltag:2004} introduced the notion of an action signature,
representing a class of action models that have the same relational structure
but which have different formulas as preconditions. They show that sublanguages
of the action model logic can be defined by restricting the possible action
models to those corresponding to sets of action signatures, and that the
resulting sublanguages have a sound and complete axiomatisation. This gives for
example a sound and complete axiomatisation for the public announcement logic of
Gerbrandy and Groenvald~\cite{gerbrandy:1997}, the logic of completely private
announcements to groups and the logic of common knowledge of alternatives.

Although the notion of information change that the action model logic captures
is intuitively explained in a setting of knowledge, the formulation that Baltag,
Moss and Solecki~\cite{baltag:1998} provide is in a more general modal setting
that can be applied not only to epistemic logic, but to other modal logics, such
as doxastic logic, the logic of belief. Whereas public announcements can only
represent true informative updates, where the information that is communicated
must actually be true, there is no such restriction for action models. It is
possible in a setting of doxastic logic for an action model to represent
informative updates containing false information, leading agents to believe that
false statements are true. Baltag and Moss~\cite{baltag:2004} refer to the
informative updates that action models represent as {\em justifiable changes in
belief}, meaning that it is not assumed that action models communicate true
information, only that they communicate information that is assumed to be
trustworthy. It is possible for action models to represent intentionally
deceptive informative updates, such as if Alice knows that the coin landed heads
up, but tells Bob that the coin landed tails up. It is also possible for action
models to represent unintentionally false informative updates, such as if Bob
believes that the coin landed tails up, when it in fact did not, but then tells
Carol that the coin landed tails up.  However, action models are not capable of
{\em revising} beliefs. That is, after Bob has been lead to believe that the
coin has landed tails up, it is not possible to convince him otherwise using an
action model. Action models represent in some sense a monotonic change of
knowledge or belief.

\subsection{Belief revision}

In contrast to the true informative updates of public announcements, and the
justifiable, monotonic informative updates of action models, methods for belief
revision consider ways in which agents can revise their beliefs in the light of
new information. The system of truth maintenance of Doyle~\cite{doyle:1979} is an
early approach to belief revision in the setting of artificial intelligence,
which models a ``knowledge base'' of beliefs along with the reasons for those
beliefs, which are used to revise those beliefs when contradicting information
is discovered.  Levi~\cite{levi:1983} and Harper~\cite{harper:1976} provided a
model of rational belief change which models beliefs and belief revision using
Bayesian probability.

More recent developments in belief revision are heavily influenced by the AGM
approach to belief revision, named for Alchourr{\'o}n, G{\"a}rdenfors
and Makinson~\cite{alchourron:1985}. The AGM approach models a single agent's
beliefs with a belief set, consisting of a set of propositional formulas. An
informative update is represented by an operation on the belief set called a
revision, which consists of adding a new formula to the belief set, and then
removing contradicting formulas from the belief set until the resulting belief
set is consistent. Often there are multiple ways to remove formulas from the
belief set that will result in a consistent belief set, and so the AGM approach
uses a model of entrenchment, representing how strongly certain beliefs are
held, in order to determine which formulas should be removed in favour of
others. Alchourr{\'o}n, G{\"a}rdenfors and Makinson do not provide a logical
framework for reasoning about their method of belief revision, and their
approach is limited in the sense that it only deals with propositional beliefs,
and therefore cannot represent introspective beliefs (beliefs about the agent's
own beliefs) or beliefs about other agents' beliefs. 

van Benthem~\cite{vanbenthem:1989, vanbenthem:1994, vanbenthem:1996},
Jaspars~\cite{jaspars:1994} and de Rijke~\cite{derijke:1994} applied dynamic modal
logic to doxastic logic to model information change, taking influences from the
AGM approach to belief revision. This provided a logical framework for reasoning
about belief revision, however the results still did not allow introspection of
beliefs.  Subsequent work by Lindstr{\"o}m and Rabinowicz~\cite{lindstrom:1999a,
lindstrom:1999b} and Segerberg~\cite{segerberg:1999a, segerberg:1999b} developed a
full dynamic doxastic logic, allowing reasoning about belief revision with
introspective beliefs. These logics introduce operators allowing one to make
statements such as ``After revising the agent's beliefs with $\alpha$, then
$\phi$ becomes true''.

%\subsection{Other logics}

%\begin{enumerate}
    %\item Groenendijk and Stokhot [81]~\cite{groenendijk:1991} - Philosophy of information change and
        %linguistics
    %\item Harel and Kozen and Tiuryu [93]~\cite{harel:1983},
        %Pratt~\cite{pratt:1980}, Halpern [17~\cite{benari:1982}, 90~\cite{halpern:1983},
        %28~\cite{berman:1982}], Parikh [161~\cite{parikh:1978}, Goldblatt
            %[79~\cite{goldblatt:1992}] - Dynamic modal
        %logic
    %\item Halpern and Moses [88~\cite{halpern:1985}] - Common knowledge is hard to achieve
    %\item Parikh and Ramanujan [164~\cite{parikh:1985}] - History-based semantics for change of
        %knowledge
    %\item Chandy and Misra [35~\cite{chandy:1986}] - Minimum information flow
    %\item Veltman~\cite{veltman:1996} - Update semantics
%\end{enumerate}
%\begin{enumerate}
    %\item van Ditmarsch [42~\cite{vanditmarsch:1999}, 43~\cite{vanditmarsch:2000},
            %44~\cite{vanditmarsch:2002}] - epistemic actions
    %\item van Benthem and van Eijck and Kooi [26~\cite{vanbenthem:2006}] - Factual change
    %\item Aucher [4~\cite{aucher:2005}], van Ditmarsch
        %[48~\cite{vanditmarsch:2005}], van Benthem and Liu
        %[27~\cite{vanbenthem:2007}]  -
        %Preference-based belief revision
    %\item Leveaque, Lakemeyer, Demolombe [125~\cite{lakemeyer:2000}, 41~\cite{demolombe:2003}] - AI-flavoured semantics
%\end{enumerate}

\section{Logics of arbitrary informative updates}

A more recent development in the field of dynamic epistemic logic concerns
logics for reasoning about arbitrary informative updates. These logics allow one
to make statements such as ``There exists an informative update after which
$\phi$ is true'' or ``After any informative update $\phi$ is true''. For
example, in computer security we may want to guarantee that a computer system
will not leak sensitive information to unauthorised parties as a result of
communication it takes part; we could say ``After any informative update, no
unauthorised parties will know the sensitive information''. In the development
of a communication protocol, we may want to assert that several parties can
reach certain knowledge amongst themselves through a series of messages; we
could say ``There exists an informative update after which the parties have
reached the knowledge goal''. 

A closely related problem is that of synthesising
informative updates that achieve a particular post-condition. That is, if we
know that there exists an informative update after which $\phi$ is true, then we
would like to produce a specific example of such an informative update. For
example in the development of communication protocols, if there exists an
informative update that achieves a desired knowledge goal, then a method for
synthesising informative updates can be applied to give us such an informative
update. In computer security, if we attempt to prove that a computer system will
not leak sensitive information to unauthorised parties, but find that this is
not the case, then a method for synthesising informative updates can be applied
to give us a specific informative update (a counter-example) where information
is leaked, which may be useful in understanding the leak and securing the
computer system.

\subsection{Arbitrary public announcement logic}

Early considerations of arbitrary informative updates were in relation to the
concept of knowability. A true epistemic formula is knowable by an agent if it
is possible for the agent to know that it is true. An example of an unknowable
formula (for an agent $a$) was given by Moore (see
Hintikka~\cite{hintikka:1962}), which can be described as ``$\phi$ is true and
agent $a$ doesn't know that $\phi$ is true''. For example, it may be true that
``the coin has landed heads up and Bob doesn't know that the coin has landed
heads up'', but in conventional treatments of knowledge it is not possible for
Bob to know that this statement is true. If Bob knew that the statement were
true, then Bob would know that the coin has landed heads up -- but this
contradicts the statement that says that Bob doesn't know that the coin has
landed heads up, and therefore invalidates his knowledge of the statement.
Knowability was considered by Fitch~\cite{fitch:1963} in relation to the
verification principle, which says that ``every true statement is knowable''. Fitch
shows that if every true statement is knowable then every true statement must be
known; this is known as Fitch's knowability paradox. It shows that if we accept
the verification principle then the notions of truth and knowledge become
equivalent, and therefore that the notion of knowledge is redundant in such a
setting. van Benthem~\cite{vanbenthem:2004} considers knowability in the setting
of dynamic epistemic logic and dismisses a number of logical treatments of
knowledge that attempt to accept the verification principle by weakening the
rules for knowledge. van Benthem~\cite{vanbenthem:2004} also considers the notion
of a successful formula, which is a true epistemic formula that is known by an
agent after it is announced to that agent. For example, if Bob were to be told
that the coin has landed heads up then he will know that the coin has landed
heads up, and so ``the coin has landed heads up'' is a successful statement
(formula). All successful formulas are knowable, and so the previous example of
an unknowable formula is also an example of an unsuccessful update; after
telling Bob that ``the coin has landed heads up and Bob doesn't know that the
coin has landed heads up'', Bob does not know that this statement is true
because its truth has been invalidated by telling it to Bob. These treatments of
knowable and successful formula introduce an informal syntactic notion of ``what
can be known'' that bears some similarity to quantifiers over informative
updates.

Fitch's knowability paradox motivated the work by Balbiani et
al.~\cite{balbiani:2007} on the arbitrary public announcement logic. The
arbitrary public announcement logic extends the public announcement logic with
an operator for quantifying over public announcements, allowing one to make
statements such as ``There exists a public announcement after which $\phi$ is
true'' or ``After any public announcement $\phi$ is true''. Balbiani et
al.~\cite{balbiani:2007} provided a number of semantic results for the arbitrary
public announcement logic, along with a sound and complete axiomatisation,
however the logic was shown to be undecidable in the setting of multiple agents
by French and van Ditmarsch~\cite{french:2008}. Balbiani et
al.~\cite{balbiani:2007} also suggested a generalisation of the arbitrary public
announcement logic to quantify over more general classes of informative updates,
such as action models.

%\begin{enumerate}
    %\item Moore (see Hintikka 1962) - $K(\phi \land \neg K \phi)$~\cite{hintikka:1962}
    %\item Fine~\cite{fine:1970} - Propositional quantifiers
    %\item van Benthem~\cite{vanbenthem:2004} - knowable/successful formulas
    %\item Balbiani et al.~\cite{balbiani:2007} - arbitrary public announcement logic
    %\item van Ditmarsch and French~\cite{vanditmarsch:2008} - undecidability
    %\item Balbiani et al.~\cite{balbiani:2008} - ???
    %\item van Ditmarsch, van der Hoek and Iliev~\cite{vanditmarsch:2011} - ???
%\end{enumerate}

\subsection{Group announcement and coalition announcement logic}

Two logics related to the arbitrary public announcement logic are the group
announcement and coalition announcement logics of {\AA}gotnes and van
Ditmarsch~\cite{agotnes:2008,agotnes:2010}.  Compared to the quantifier of the
arbitrary public announcement logic, the group announcement logic restricts the
public announcements that are quantified over.  A group announcement consists of
a set of public announcements made concurrently by a group of agents. Each agent
in the group may only announce statements that the agent knows to be true. The
group announcement logic extends the public announcement logic with a quantifier
over group announcements, similar to the quantifier introduced in the arbitrary
public announcement logic.  This allows one to make statements such as ``There
exists public announcements that the group of agents $G$ can make after which
$\phi$ is true''. Coalition announcements differ from group announcements in
that the agents outside of the coalition are also able to make public
announcements that may sabotage whatever the coalition of agents is attempting
to achieve through its announcements. The coalition announcement logic allows
one to make statements such as ``There exists public announcements that the
coalition of agents $C$ can make such that no matter what public announcements
the agents outside of $C$ make, $\phi$ is true afterwards''.  {\AA}gotnes et
al.~\cite{agotnes:2010} provide a sound and complete axiomatisation of the group
announcement logic, along with expressivity results and a complexity result for
model checking. It is yet unknown whether the group announcement logic and
coalition announcement logic are expressively equivalent.

%\begin{enumerate}
    %\item {\AA}gotnes et al.~\cite{agotnes:2010} - group announcement logic
    %\item Pauly~\cite{pauly:2001} - GAL embeds coalition logic
    %\item de Lima~\cite{delima:2011} - Alternating time temporal announcement logic
    %\item {\AA}gotnes and van Ditmarsch~\cite{aagotnes:2008b} - Coalitions and announcements ???
%\end{enumerate}

\subsection{Refinement modal logic}

Compared to the arbitrary public announcement logic, whereas the group
announcement and coalition announcement logics of {\AA}gotnes and van
Ditmarsch~\cite{agotnes:2008} restricted the informative updates that are
quantified over, the refinement modal logic of van Ditmarsch and
French~\cite{vanditmarsch:2009} introduces an operator that quantifies over a
much more general class of informative updates. The refinement modal logic is
related to the bisimulation quantified modal logic of French~\cite{french:2006},
which introduces an operator for quantifying over the pointed Kripke models that
are bisimilar to the pointed Kripke model currently being considered, except for
the value of a propositional atom that is allowed to vary. Bisimulations are an
important concept in the semantics of modal logics, that correspond to a notion
of equivalence of Kripke models: if two pointed Kripke models are bisimilar then
they are indistinguishable to any modal formula. For two Kripke models to be
considered bisimilar there must exist a bisimulation relation that satisfies the
three conditions known as {\bf atoms}, {\bf forth} and {\bf back}. Refinements
are related to bisimulations: for a Kripke model to be a refinement of another
Kripke model, there must exist a refinement relation that satisfies {\bf atoms}
and {\bf forth}. Refinements can be seen as one direction of a bisimulation;
whereas bisimulation corresponds to an equivalence relation, refinement
corresponds to a partial ordering. The refinement modal logic of van Ditmarsch
and French~\cite{vanditmarsch:2009} (also known as the future event logic)
quantifies over the pointed Kripke models that are refinements of the pointed
Kripke model currently being considered. Whereas the quantifier in the
bisimulation quantified modal logic of French~\cite{french:2006} binds a
propositional atom as a variable, the quantifier in the refinement 
modal logic binds no variables. van Ditmarsch and French~\cite{vanditmarsch:2009}
provide several semantic results to justify that refinements correspond to a
very general notion of informative updates. In particular, the result of
executing any action model on a Kripke model is a refinement of the original
Kripke model, and any refinement of a finite Kripke model corresponds to the
result of executing some action model~\cite{vanditmarsch:2009}. van Ditmarsch and
French~\cite{vanditmarsch:2009} also compare their refinement modal logic to the
arbitrary action model logic suggested by Balbiani et al.~\cite{balbiani:2007},
showing that adding the operator from the action model logic to the refinement
yields a logic equivalent to the arbitrary action model logic.

The refinement modal logic is considered in further detail by van Ditmarsch,
French and Pinchinat~\cite{vanditmarsch:2010}, who provide a sound and complete
axiomatisation for the single agent refinement modal logic over the class
\classK{} of all Kripke models. These results were generalised to single-agent
\classKD{} (doxastic logic) and single-agent \classS{} (epistemic logic) by
Hales, French and Davies~\cite{hales:2011a} and then to the multi-agent case of
these logics by Hales, French, Davies, et al.~\cite{hales:2011b, hales:2012,
bozzelli:2014b}. Each of the sound and complete axiomatisations given for these
refinement modal logics are given in the form reduction axioms that allow
formulas containing refinement quantifiers to be translated into basic modal
formulas containing no refinement quantifiers. This allows one to show that each
of the considered refinement modal logics are expressively equivalent to their
underlying modal, doxastic or epistemic logic, and that each logic is decidable.

In addition to axiomatisations, decidability and expressivity results, van
Ditmarsch, French and Pinchinat~\cite{vanditmarsch:2010} have provided results
for the refinement modal $\mu$-calculus (which adds a refinement quantifier to
the modal $\mu$-calculus), and Bozzelli, van Ditmarsch and
Pinchinat~\cite{bozzelli:2014a} have shown complexity and succinctness results
for the single-agent refinement modal logic over the \classK{} of all Kripke
models.  Recently Hales~\cite{hales:2013} has extended the multi-agent
refinement modal logic over the class \classK{} of all Kripke models with
action model operators, forming the arbitrary action model logic.
Hales~\cite{hales:2013} provided a method for synthesising action models that
satisfy a given post-condition, showed that the arbitrary action model logic is
expressively equivalent to its underlying modal logic and that it is decidable.

Yet to be considered are the complexity and succinctness of refinement modal
logic in the settings of \classKD{} and \classS{}, the addition of common
knowledge operators to the refinement modal logic, the extension of these
results to the setting of transitive Kripke models such as \classKF{} and
\classSF{} and their extension to the arbitrary action model logic, and the
extension of action model synthesis results to other modal settings such as
\classKD{} and \classS{}.
