\chapter{Action formula logic}

\section{Syntax and semantics}

\begin{definition}[Language of action formula logic]
The language \langAfl{} of action formula logic is inductively defined as:
$$
\phi ::= \atomP \mid 
       \neg \phi \mid
       (\phi \land \phi) \mid
       \necessaryA \phi \mid
       \actionA{\alpha} \phi
$$
where $\atomP \in \atoms$, $\agentA \in \agents$ and $\alpha \in \langAflAct{}$, and where the language \langAflAct{} of action formulas is inductively as:
$$
\alpha ::= \test{\phi} \mid
       \alpha \choice \alpha \mid
       \alpha \compose \alpha \mid
       \learns \{\alpha_\agentA\}_{\agentA \in \agents}
$$
where $\phi \in \langAfl{}$, and $\{\alpha_\agentA\}_{\agentA \in \agents}$ is a set of action formulas indexed by agents $\agents$.
\end{definition}

We use all of the standard abbreviations from modal logic, in addition to the abbreviations $\actionE{\alpha} \phi ::= \lnot \actionA{\alpha} \lnot \phi$.

\begin{definition}[Semantics of action formula logic]
Let \classC{} be a class of Kripke frames, let $\tau : \langAflAct \to \classAmK$ be a function from action formulas to multi-pointed action models with preconditions defined on \langAfl{}, let $\phi \in \langAfl$, and let $\kPModelAndTuple{\kStateS} \in \classC$ be a pointed Kripke model.
The interpetation of the formula $\phi$ in the logic \logicAflC{} on the pointed Kripke model $\kPModel{\kStateS}$ is the same as its interpretation in modal logic, defined in Definition~\ref{ml-semantics}, with the additional inductive cases:
$$
\begin{array}{lcl}
    \kPModel{\kStateS} \entails \actionA{\alpha} \phi & \text { iff } & \kPModel{\kStateS} \entails \aPrecondition(\aStateS) \text{ implies } \kPModel{\kStateS} \exec \aPModel{\aStateS} \entails \phi\\
    \kPModel{\kStateS} \entails \actionA{\beta} \phi & \text { iff } & \text{for every } \aStateSP \in \aStatesTP: \kPModel{\kStateSP} \entails \actionA{\aPModelP{\aStateSP}} \phi
\end{array}
$$
where $\tau(\alpha) = \aPModel{\aStateS}$ is a single-pointed action model, $\tau(\beta) = \aPModelP{\aStatesTP}$ is a multi-pointed action model, and action model execution $\exec$ is as defined in Definition~\ref{aml-semantics}.
\end{definition}

\begin{lemma}\label{afl-aml}
Let \classC{} be a class of Kripke models, let \aSignatureFamily{} be a non-empty, countable set of action signatures, let $\tau : \langAflAct \to \classAmK$ be a function from action formulas to action models such that for every $\alpha \in \langAflAct$ we have $\tau(\alpha) \in \aSignatureFamily$.
For every $\phi \in \langAfl$ there exists $\phi' \in \langAml(\aSignatureFamily)$ such that for every $\kPModel{\kStateS} \in \classC$, $\kPModel{\kStateS} \entails_\logicAflC \phi$ if and only if $\kPModel{\kStateS} \entails_\logicAmlC \phi'$.
\end{lemma}

\begin{proof}
We proceed by induction on the structure of $\phi$.

Suppose that $\phi = \atomP$ where $\atomP \in \atoms$ and let $\kPModel{\kStateS} \in \classC$.
Then $\kPModel{\kStateS} \entails_\logicAflC \atomP$ if and only if $\kStateS \in \kValuation(\atomP)$ if and only if $\kPModel{\kStateS} \entails_\logicAmlC \atomP$.

Suppose that $\phi = \lnot \psi$ or $\phi = \psi \land \chi$ where $\psi, \chi \in \langAfl$.
These cases follow directly from the induction hypothesis.

Suppose that $\phi = \necessaryA \psi$ where $\psi \in \langAfl$ and let $\kPModel{\kStateS} \in \classC$.
Then $\kPModel{\kStateS} \entails_\logicAflC \necessaryA \psi$ if and only if for every $\kStateT \in \kSuccessorsA{\kStateS}$ we have $\kPModel{\kStateT} \entails_\logicAflC \psi$.
By the induction hypothesis there exists $\psi' \in \langAml$ such that for every $\kPModelP{\kStateSP} \in \classC$ we have $\kPModelP{\kStateSP} \entails_\logicAflC \psi$ if and only if $\kPModelP{\kStateSP} \entails_\logicAmlC \psi'$.
Therefore for every $\kStateT \in \kSuccessorsA{\kStateS}$ we have that $\kPModel{\kStateT} \entails_\logicAflC \psi$ if and only if $\kPModel{\kStateT} \entails_\logicAmlC \psi'$.
Therefore $\kPModel{\kStateS} \entails_\logicAflC \necessaryA \psi$ if and only if $\kPModel{\kStateS} \entails_\logicAmlC \necessaryA \psi'$.

Suppose that $\phi = \actionA{\alpha} \psi$ where $\alpha \in \langAflAct$ and $\psi \in \langAfl$, and let $\kPModel{\kStateS} \in \classC$.
Suppose that $\tau(\alpha) = \aPModelAndTuple{\aStateS}$ is a single-pointed action model with preconditions defined on \langAfl{}.
Then $\kPModel{\kStateS} \entails_\logicAflC \actionA{\alpha} \psi$ if and only if $\kPModel{\kStateS} \entails_\logicAflC \aPrecondition(\aStateS)$ implies $\kPModel{\kStateS} \exec \aPModel{\aStateS} \entails_\logicAflC \psi$.
By the induction hypothesis there exists $\psi' \in \langAml$ such that for every $\kPModelP{\kStateSP} \in \classC$ we have $\kPModelP{\kStateSP} \entails_\logicAflC \psi$ if and only if $\kPModelP{\kStateSP} \entails_\logicAmlC \psi'$.
Likewise by the induction hypothesis, for every $\aStateT \in \aStates$ there exists $\chi' \in \langAml$ such that for every $\kPModelP{\kStateSP} \in \classC$ we have $\kPModelP{\kStateSP} \entails_\logicAflC \aPrecondition(\aStateT)$ if and only if $\kPModelP{\kStateSP} \entails_\logicAmlC \chi'$.
Let $\aPModelP{\aStateS} = ((\aStates, \aAccessibility{}, \aPreconditionP), \aStateS)$ be an action model with preconditions defined on \langAml{} where for every $\aStateT \in \aStates$ we define $\aPreconditionP(\aStateT)$ to be a \langAml{} formula equivalent to $\aPrecondition(\aStateT)$.
Then $\kPModel{\kStateS} \entails_\logicAflC \aPrecondition(\aStateS)$ if and only if $\kPModel{\kStateS} \entails_\logicAmlC \aPreconditionP(\aStateS)$, $\kPModel{\kStateS} \exec \aPModel{\aStateS} = \kPModel{\kStateS} \exec \aPModelP{\kStateS}$, and $\kPModel{\kStateS} \exec \aPModel{\aStateS} \entails_\logicAflC \psi$ if and only if $\kPModel{\kStateS} \exec \aPModelP{\aStateS} \entails_\logicAmlC \psi'$.
Therefore $\kPModel{\kStateS} \entails_\logicAflC \actionA{\alpha} \psi$ if and only if $\kPModel{\kStateS} \entails_\logicAmlC \actionA{\aPModel{\aStateSP}} \psi'$.
Suppose that $\tau(\alpha) = \aPModel{\aStatesT}$ is a multi-pointed action model with preconditions defined on \langAfl{}.
Then we proceed as above, except that the constructed equivalent action model is multi-pointed instead of single-pointed.
\end{proof}

\begin{definition}[Non-deterministic choice]\label{afl-choice}
Let $\alpha, \beta \in \langAflAct$ where $\tau(\alpha) = \aPModel[\alpha]{\aStatesT[\alpha]} = \aPModelTuple[\alpha]{\aStatesT[\alpha]}$ and $\tau(\beta) = \aPModel[\beta]{\aStatesT[\beta]} = \aPModelTuple[\beta]{\aStatesT[\beta]}$ such that $\aStates[\alpha]$ and $\aStates[\beta]$ are disjoint.
We define $\tau(\alpha \choice \beta) = \aPModel{\aStatesT} = \aPModelTuple{\aStatesT}$ where:
\begin{eqnarray*}
    \aStates &=& \aStates[\alpha] \cup \aStates[\beta]\\
    \aAccessibility{\agentA} &=& \aAccessibility[\alpha]{\agentA} \cup \aAccessibility[\beta]{\agentA} \text{ for } \agentA \in \agents\\
    \aPrecondition &=& \aPrecondition[\alpha] \cup \aPrecondition[\beta]\\
    \aStatesT &=& \aStatesT[\alpha] \cup \aStatesT[\beta]
\end{eqnarray*}
\end{definition}

\begin{definition}[Sequential execution]\label{afl-sequential}
Let $\alpha, \beta \in \langAflAct$ where $\tau(\alpha) = \aPModel[\alpha]{\aStatesT[\alpha]} = \aPModelTuple[\alpha]{\aStatesT[\alpha]}$ and $\tau(\beta) = \aPModel[\beta]{\aStatesT[\beta]} = \aPModelTuple[\beta]{\aStatesT[\beta]}$.
We define $\tau(\alpha \compose \beta) = \aPModel[\alpha]{\aStatesT[\alpha]} \exec \aPModel[\beta]{\aStatesT[\beta]}$.
\end{definition}

\begin{lemma}\label{afl-choice-sequential-validities}
Let \classC{} be a class of Kripke frames, let $\alpha, \beta, \gamma \in \langAflAct$ and let $\phi \in \langAfl$. Then the following are valid in \logicAflC{}.
$$
\begin{array}{l}
    \entails \actionA{\alpha \choice \beta} \phi \iff (\actionA{\alpha} \phi \land \actionA{\beta} \phi) \label{afl-axiom-choice}\\
    \entails \actionA{\alpha \choice \alpha} \phi \iff \actionA{\alpha} \phi\\
    \entails \actionA{\alpha \choice \beta} \phi \iff \actionA{\beta \choice \alpha} \phi\\
    \entails \actionA{(\alpha \choice \beta) \choice \gamma} \phi \iff \actionA{\alpha \choice (\beta \choice \gamma)} \phi\\
    \entails \actionA{\alpha \compose \beta} \phi \iff \actionA{\alpha} \actionA{\beta} \phi \label{afl-axiom-sequential}\\
    \entails \actionA{(\alpha \compose \beta) \compose \gamma} \phi \iff \actionA{\alpha \compose (\beta \compose \gamma)} \phi\\
    \entails \actionA{(\alpha \choice \beta) \compose \gamma} \phi \iff \actionA{(\alpha \compose \gamma) \choice (\beta \compose \gamma)} \phi
\end{array}
$$
\end{lemma}

\section{K}

\subsection{Semantics}

\begin{definition}[Test]\label{afl-k-test}
Let $\phi \in \langAfl$. 
We define $\tau(\test{\phi}) = \aPModel{\aStatesT} = \aPModelTuple{\aStatesT}$ where:
\begin{eqnarray*}
    \aStates &=& \{\aStateTest, \aStateSkip\}\\
    \aAccessibility{\agentA} &=& \{(\aStateTest, \aStateSkip), (\aStateSkip, \aStateSkip)\} \text{ for } \agentA \in \agents\\
    \aPrecondition &=& \{(\aStateTest, \phi), (\aStateSkip, \top)\}\\
    \aStatesT &=& \{\aStateTest\}
\end{eqnarray*}
\end{definition}

\begin{definition}[Learning]\label{afl-k-learning}
Let $\{\alpha_\agentA\}_{\agentA \in \agents} : \agents \to \langAflAct$ be a set of action formulas indexed on agents $\agents$, and for every $\agentA \in \agents$ let $\tau(\alpha_\agentA) = \aPModel[\agentA]{\aStatesT[\agentA]} = \aPModelTuple[\agentA]{\aStatesT[\agentA]}$.
Let $\aStateLearn$ be a new state not appearing in $\aStates[\agentA]$ for any $\agentA \in \agentsB$.
We define $\tau(\learns \{\alpha_\agentA\}_{\agentA \in \agents}) = \aPModel{\aStatesT} = \aPModelTuple{\aStatesT}$ where:
\begin{eqnarray*}
    \aStates &=& \bigcup_{\agentA \in \agents} \aStates[\agentA] \cup \{\aStateLearn\}\\
    \aAccessibility{\agentA} &=& \bigcup_{\agentB \in \agentsB} \aAccessibility[\agentB]{\agentA} \cup \{(\aStateLearn, \aStateT[\agentA]) \mid \aStateT[\agentA] \in \aStatesT[\agentA]\} \text{ for } \agentA \in \agents\\
    \aPrecondition &=& \bigcup_{\agentA \in \agents} \aPrecondition[\agentA] \cup \{(\aStateLearn, \top)\}\\
    \aStatesT &=& \{\aStateLearn\}
\end{eqnarray*}
\end{definition}

\begin{lemma}
Let $\alpha \in \langAflAct$. Then $\tau(\alpha)$ is a finite action model.
\end{lemma}

\begin{proposition}
The logic \logicAflK{} is expressively equivalent to the logic \logicAmlK{}. 
\end{proposition}

\begin{proof}
Let $\phi \in \langAfl$.
By Lemma~\ref{afl-aml} there exists $\phi' \in \langAml(\classK)$ such that for every $\kPModel{\kStateS} \in \classK$, $\kPModel{\kStateS} \entails_\logicAflK \phi$ if and only if $\kPModel{\kStateS} \entails_\logicAmlK \phi'$.

Let $\phi \in \langAml(\classK)$.
By Lemma~\ref{aml-expressively-equivalent} there exists $\phi' \in \langMl$ such that $\kPModel{\kStateS} \in \classK$, $\kPModel{\kStateS} \entails_\logicAmlK \phi$ if and only if $\kPModel{\kStateS} \entails_\logicK \phi'$.
Then $\phi' \in \langAfl$ and $\kPModel{\kStateS} \entails_\logicK \phi'$ if and only if $\kPModel{\kStateS} \entails_\logicAflK \phi'$.
Therefore $\kPModel{\kStateS} \entails_\logicAmlK \phi$ if and only if $\kPModel{\kStateS} \entails_\logicAflK \phi'$.
\end{proof}

\begin{corollary}
The logic \logicAflK{} is expressively equivalent to the logic \logicK{}.
\end{corollary}

\subsection{Axiomatisation}

\begin{definition}[Axiomatisation \axiomAflK{}]\label{afl-k-axioms}
The axiomatisation \axiomAflK{} is a substitution schema consisting of the rules and axioms of \axiomK{} along with the axioms:
$$
\begin{array}{rl}
    {\bf LT} & \proves \actionA{\test{\phi}} \psi \iff (\phi \implies \psi) \text{ for } \psi \in \lang\\
    {\bf LU} & \proves \actionA{\alpha \choice \beta} \phi \iff (\actionA{\alpha} \phi \land \actionA{\beta} \phi)\\
    {\bf LS} & \proves \actionA{\alpha \compose \beta} \phi \iff \actionA{\alpha} \actionA{\beta} \phi\\
    {\bf LP} & \proves \actionA{\learns \{\alpha_\agentA\}_{\agentA \in \agents}} \atomP \iff \atomP\\
    {\bf LN} & \proves \actionA{\learns \{\alpha_\agentA\}_{\agentA \in \agents}} \neg \phi \iff \neg \actionA{\learns \{\alpha_\agentA\}_{\agentA \in \agents}} \phi\\
    {\bf LC} & \proves \actionA{\learns \{\alpha_\agentA\}_{\agentA \in \agents}} (\phi \land \psi) \iff (\actionA{\learns \{\alpha_\agentA\}_{\agentA \in \agents}} \phi \land \actionA{\learns \{\alpha_\agentA\}_{\agentA \in \agents}} \psi)\\
    {\bf LK} & \proves \actionA{\learns \{\alpha_\agentA\}_{\agentA \in \agents}} \necessaryA \phi \iff \necessaryB \actionA{\alpha_\agentB} \phi \text{ for } \agentB \in \agents\\
\end{array}
$$
and the rule:
$$
\begin{array}{rl}
    {\bf NecL} & \text{From $\proves \phi$ infer $\proves \actionA{\alpha} \phi$}
\end{array}
$$
\end{definition}

\begin{lemma}\label{afl-k-axioms-soundness}
The axiomatisation \axiomAflK{} is sound in the logic \logicAmlK{}.
\end{lemma}

\begin{proof}
{\bf LT} follows from applying the reduction axioms of \axiomAmlK{} inductively to $\actionA{\test{\phi}} \psi$.

{\bf LU} and {\bf LS} follow from Lemma~\ref{afl-choice-sequential-validities}.

Let $\tau(\learns \{\alpha_\agentB\}_{\agentB \in \agentsB}) = \aPModel{\aStateS} = \aPModelTuple{\aStateS}$.
{\bf LP}, {\bf LN} and {\bf LC} follow trivially from the \axiomAmlK{} axioms {\bf AP}, {\bf AN} and {\bf AC} respectively, noting from Definition~\ref{afl-k-learning} that $\aPrecondition(\aStateS) = \top$.
{\bf LK} follows trivially from the \axiomAmlK{} axiom {\bf AK}, noting from Definition~\ref{afl-k-learning} that $\aPModel{\aStateS \aAccessibility{\agentB}} \bisimilar \tau(\alpha_\agentB)$.
{\bf NecL} follows trivially from the \axiomAmlK{} rule {\bf NecA}.
\end{proof}

\begin{theorem}\label{afl-k-axioms-completeness}
The axiomatisation \axiomAflK{} is sound and complete for the logic \logicAflK{}.
\end{theorem}

We note that the axiomatisation \axiomAflK{} forms a set of reduction axioms that gives a provably correct translation from \langAfl{} to \lang{}.

\begin{example}\label{grant-example-derivation}
We give an example derivation that the action formula $\alpha$ given in Example~\ref{grant-example-formula} does indeed satisfy (part of) the epistemic goal stated in Example~\ref{grant-example}.
\begin{eqnarray}
    &\proves& \actionA{\test{\atomP}} \atomP \iff (\atomP \implies \atomP)\label{grant-example-derivation-1}\\
    &\proves& \actionA{\test{\atomP}} \atomP\label{grant-example-derivation-2}\\
    &\proves& \necessary[Ed] \actionA{\test{\atomP}} \atomP\label{grant-example-derivation-3}\\
    &\proves& \actionA{\learns[Ed] \test{\atomP}} \necessary[Ed] \atomP\label{grant-example-derivation-4}
\end{eqnarray}
(\ref{grant-example-derivation-1}) follows from {\bf LT},
(\ref{grant-example-derivation-3}) follows from {\bf NecK} and
(\ref{grant-example-derivation-4}) follows from {\bf LK1}.

Similarly we have:
\begin{eqnarray*}
    &\proves& \actionA{\learns[Ed] \test{\neg \atomP}} \necessary[Ed] \neg \atomP\\
    &\proves& \actionA{\learns[Tim] \test{\atomP}} \necessary[Tim] \atomP\\
    &\proves& \actionA{\learns[Tim] \test{\neg \atomP}} \necessary[Tim] \neg \atomP
\end{eqnarray*}

Let $\phi = \necessary[Ed] \atomP \lor  \necessary[Ed] \neg \atomP \lor \necessary[Tim] \atomP \lor \necessary[Tim] \neg \atomP$. Then:
\begin{eqnarray}
    &\proves& \actionA{\learns[Ed] \test{\atomP} \choice \learns[Ed] \test{\neg \atomP} \choice \learns[Tim] \test{\atomP} \choice \learns[Tim] \test{\neg \atomP}} \phi\label{grant-example-derivation-5}\\
    &\proves& \necessary[James] \actionA{\learns[Ed] \test{\atomP} \choice \learns[Ed] \test{\neg \atomP} \choice \learns[Tim] \test{\atomP} \choice \learns[Tim] \test{\neg \atomP}} \phi\label{grant-example-derivation-6}\\
    &\proves& \actionA{\learns[James] (\learns[Ed] \test{\atomP} \choice \learns[Ed] \test{\neg \atomP} \choice \learns[Tim] \test{\atomP} \choice \learns[Tim] \test{\neg \atomP})} \necessary[James] \phi\label{grant-example-derivation-7}\\
    &\proves& \actionA{\alpha} \necessary[James] \phi\label{grant-example-derivation-8}
\end{eqnarray}
(\ref{grant-example-derivation-5}) follows from {\bf LU},
(\ref{grant-example-derivation-6}) follows from {\bf NecK} and
(\ref{grant-example-derivation-7}) follows from {\bf LK1}.
(\ref{grant-example-derivation-8}) follows from {\bf LS} and {\bf LK2}.

Therefore a consequence of successfully executing $\alpha$ is that James learns that Ed or Tim knows whether the grant application was successful.
\end{example}

\subsection{Correspondence}

\begin{lemma}\label{afl-k-construction-test}
Let $\aPModelAndTuple{\aStateS} \in \classAmK$ be a single-pointed action model, let $\phi \in \langAfl$ and let $\aPModelAndTupleP{\aStateSP}$ be an action model where:
\begin{eqnarray*}
    \aStatesP &=& \aStates \cup \{\aStateSP\}\\
    \aAccessibilityPA &=& \aAccessibilityA \cup \{(\aStateSP, \aStateT) \mid \aStateT \in \aSuccessorsA{\aStateS}\}\\
    \aPreconditionP &=& \aPrecondition \cup \{(\aStateSP, \phi \land \aPrecondition(\aStateS)\}
\end{eqnarray*}
Then $\tau(\test{\phi}) \exec \aPModel{\aStateS} \bisimilar \aPModelP{\aStateSP}$.
\end{lemma}

\begin{proof}
Without loss of generality, by the expressive equivalence of \logicAmlK{} and \logicK{} (Proposition~\ref{aml-expressive-equivalence}) we assume that $\aPModel{\aStateS}$ has preconditions defined on \langMl{}.

Let $\test{\phi} = \aPModelAndTuplePP{\aStateTestPP}$ as defined in Definition~\ref{afl-k-test}, and let $\test{\phi} \exec \aPModel{\aStateS} = \aPModelAndTuplePPP{(\aStateTestPP, \aStateS)}$ as defined in Definition~\ref{aml-composition}.
Let $\bisimulation \subseteq \aStatesPPP \times \aStatesP$ where:
$$
\bisimulation = \{((\aStateTestPP, \aStateS), \aStateSP)\} \cup \{((\aStateSkipPP, \aStateT), \aStateT) \mid \aStateT \in \aStates\}
$$
We claim that $\bisimulation$ is a bisimulation between $\aPModelPPP{(\aStateTestPP, \aStateS)}$ and $\aPModelP{\aStateSP}$.

\paragraph{pre}
Consider $((\aStateTestPP, \aStateS), \aStateSP) \in \bisimulation$.
Then $\aPreconditionPPP((\aStateTestPP, \aStateS)) = \aPreconditionPP(\aStateTestPP) \land \actionA{\aPModelPP{\aStateTestPP}} \aPrecondition(\aStateS)$ and $\aPreconditionP(\aStateSP) = \phi \land \aPrecondition(\aStateS)$.
From axiom {\bf LP} from \axiomAflK{} we have that $\entails \actionA{\test{\phi}} \aPrecondition(\aStateS) \iff (\phi \implies \aPrecondition(\aStateS))$ and so $\actionA{\aPModelPP{\aStateTestPP}} \aPrecondition(\aStateS) \iff (\phi \implies \aPrecondition(\aStateS))$.
Therefore $\entails \aPreconditionPP(\aStateTestPP) \land \actionA{\aPModelPP{\aStateTestPP}} \aPrecondition(\aStateS) \iff \phi \land \aPrecondition(\aStateS)$.

Consider $((\aStateSkipPP, \aStateT), \aStateT) \in \bisimulation$ where $\aStateT \in \aStates$.
Then $\aPreconditionPPP((\aStateSkipPP, \aStateS)) = \aPreconditionPP(\aStateSkipPP) \land \actionA{\aPModelPP{\aStateSkipPP}} \aPrecondition(\aStateT)$.
We note that $\aPModelPP{\aStateSkipPP} \bisimilar \tau(\test(\phi))$ and so $\entails \actionA{\aPModelPP{\aStateSkipPP}} \aPrecondition(\aStateT) \iff \aPrecondition(\aStateT)$.

\paragraph{forth-$\agentA$}
Consider $((\aStateTestPP, \aStateS), \aStateSP) \in \bisimulation$.
Let $(\aStateTPP, \aStateT) \in \aSuccessorsPPPA{(\aStateTestPP, \aStateS)}$.
By construction $\aStateTPP \in \aSuccessorsPPA{\aStateTestPP} = \{\aStateSkipPP\}$ and $\aStateT \in \aSuccessorsA{\aStateS} \subseteq \aStates$, so $\aStateTPP = \aStateSkipPP$ and $\aStateT \in \aStates$.
By construction $((\aStateSkipPP, \aStateT), \aStateT) \in \bisimulation$.

Consider $((\aStateSkipPP, \aStateT), \aStateT) \in \bisimulation$ where $\aStateT \in \aStates$.
Let $(\aStateUPP, \aStateU) \in \aSuccessorsPPPA{(\aStateSkipPP, \aStateT)}$.
By construction $\aStateUPP \in \aSuccessorsPPA{\aStateSkipPP} = \{\aStateSkipPP\}$ and $\aStateU \in \aSuccessorsA{\aStateT} \subseteq \aStates$, so $\aStateUPP = \aStateSkipPP$ and $\aStateU \in \aStates$.
By construction $((\aStateSkipPP, \aStateU), \aStateU) \in \bisimulation$.

\paragraph{back-$\agentA$}
Consider $((\aStateTestPP, \aStateS), \aStateSP) \in \bisimulation$.
Let $\aStateT \in \aSuccessorsPA{\aStateSP} = \aSuccessorsA{\aStateS} \subseteq \aStates$.
By construction $\aStateSkipPP \in \aSuccessorsPPA{\aStateTestPP}$ and so $(\aStateSkipPP, \aStateT) \in \aSuccessorsPPA{(\aStateTestPP, \aStateS)}$.
By construction $((\aStateSkipPP, \aStateT), \aStateT) \in \bisimulation$.

Consider $((\aStateSkipPP, \aStateT), \aStateT) \in \bisimulation$ where $\aStateT \in \aStates$.
Let $\aStateU \in \aSuccessorsPA{\aStateT} = \aSuccessorsA{\aStateS} \subseteq \aStates$.
By construction $\aStateSkipPP \in \aSuccessorsPPA{\aStateSkipPP}$ and so $(\aStateSkipPP, \aStateU) \in \aSuccessorsPPA{(\aStateSkipPP, \aStateT)}$.
By construction $((\aStateSkipPP, \aStateU), \aStateU) \in \bisimulation$.

Therefore $\tau(\test{\phi}) \exec \aPModel{\aStateS} \bisimilar \aPModelP{\aStateSP}$.
\end{proof}

\begin{theorem}\label{afl-k-correspondence}
Let $\aPModel{\aStateS} \in \classAmK$ and let $n \in \mathbb{N}$. 
Then there exists $\alpha \in \langAflAct$ such that $\aPModel{\aStateS} \bisimilar[n] \tau(\alpha)$.
\end{theorem}

\begin{proof}
Without loss of generality, by the expressive equivalence of \logicAmlK{} and \logicK{} (Proposition~\ref{aml-expressive-equivalence}) we assume that $\aPModel{\aStateS}$ has preconditions defined on \langMl{}.

By induction on $n$.

Suppose that $n = 0$.
Let $\alpha = \test{\aPrecondition(\aStateS)}$ and $\tau(\alpha) = \aPModelP{\aStateSP} = \aPModelTupleP{\aStateSP}$.
From Definition~\ref{afl-k-test} we have that $\aPrecondition(\aStateS) = \aPreconditionP(\aStateSP)$, so $(\aPModel{\aStateS}, \aPModelP{\aStateSP})$ satisfies {\bf pre} and therefore $\aPModel{\aStateS} \bisimilar[0] \aPModelP{\aStateSP}$.

Suppose that $n > 0$. 
By the induction hypothesis, for every $\agentA \in \agents$, $\aStateT \in \aStateS \aAccessibility{\agentA}$ there exists $\alpha^{\agentA,\aStateT} \in \langAflAct$ such that $\aPModel{\aStateT} \bisimilar[(n - 1)] \tau(\alpha^{\agentA,\aStateT})$, where $\tau(\alpha^{\agentA,\aStateT}) \bisimilar \aPModel[\agentA,\aStateT]{\aStateS[\agentA,\aStateT]} = \aPModelTuple[\agentA,\aStateT]{\aStateS[\agentA,\aStateT]}$.

Let $\alpha = \test{\aPrecondition(\aStateS)} \compose \learns \{\bigchoice_{\aStateT \in \aStateS \aAccessibility{\agentA}} \alpha^{\agentA,\aStateT} \mid \agentA \in \agents\}$.
Then by Lemma~\ref{afl-k-construction-test} we have that $\tau(\alpha) \bisimilar \aPModelP{\aStateSP} = \aPModelTupleP{\aStateSP}$ where:
\begin{eqnarray*}
    \aStatesP &=& \bigcup_{\agentA \in \agents, \aStateT \in \aStateS \aAccessibility{\agentA}} \aStates[\agentA,\aStateT] \cup \{\aStateSP\}\\
    \aAccessibilityP{\agentA} &=& \bigcup_{\agentB \in \agents, \aStateT \in \aStateS \aAccessibility{\agentB}} \aAccessibility[\agentB,\aStateT]{\agentA} \cup \{(\aStateSP, \aStateS[\agentA,\aStateT]) \mid \aStateT \in \aStateS \aAccessibility{\agentA}\} \text{ for } \agentA \in \agents\\
    \aPreconditionP &=& \bigcup_{\agentA \in \agents, \aStateT \in \aStateS \aAccessibility{\agentA}} \aPrecondition[\agentA,\aStateT] \cup \{(\aStateSP, \aPrecondition(\aStateS))\}
\end{eqnarray*}

We note for every $\agentA \in \agents$, $\aStateT \in \aStateS \aAccessibility{\agentA}$ that $\aPModelP{\aStateS[\agentA,\aStateT]} \bisimilar \aPModel[\agentA,\aStateT]{\aStateS[\agentA,\aStateT]}$ as for every $\agentA \in \agents$, $\aStateU \in \aStates[\agentA,\aStateT]$ we have $\aStateU \aAccessibilityP{\agentA} = \aStateU \aAccessibility[\agentA,\aStateT]{\agentA}$.

We show that $(\aPModel{\aStateS}, \aPModelP{\aStateSP})$ satisfies {\bf pre}, {\bf forth-$n$-$\agentA$} and {\bf back-$n$-$\agentA$} for every $\agentA \in \agents$.

\paragraph{pre} By construction
$\aPreconditionP(\aStateSP) = \aPrecondition(\aStateS)$.

\paragraph{forth-$n$-$\agentA$} 
Let $\aStateT \in \aStateS \aAccessibility{\agentA}$.
By construction $\aStateS[\agentA,\aStateT] \in \aStateSP \aAccessibilityP{\agentA}$, by the induction hypothesis $\aPModel{\aStateT} \bisimilar[(n - 1)] \aPModel[\agentA,\aStateT]{\aStateS[\agentA,\aStateT]}$ and from above $\aPModel[\agentA,\aStateT]{\aStateS[\agentA,\aStateT]} \bisimilar \aPModelP{\aStateS[\agentA,\aStateT]}$.
Therefore by transitivity $\aPModel{\aStateT} \bisimilar[(n - 1)] \aPModelP{\aStateS[\agentA,\aStateT]}$.

\paragraph{back-$n$-$\agentA$} 
Let $\aStateS[\agentA,\aStateT] \in \aSuccessorsPA{\aStateSP}$.
By hypothesis $\aStateT \in \aSuccessorsA{\aStateS}$, by the induction hypothesis $\aPModel{\aStateT} \bisimilar[(n-1)] \aPModel[\agentA,\aStateT]{\aStateS[\agentA,\aStateT]}$, and from above $\aPModelP{\aStateS[\agentA,\aStateT]} \bisimilar[(n - 1)] \aPModel[\agentA,\aStateT]{\aStateS[\agentA,\aStateT]}$.
Therefore by transitivity $\aPModel{\aStateT} \bisimilar[(n - 1)] \aPModelP{\aStateS[\agentA,\aStateT]}$.

Therefore $\aPModel{\aStateS} \bisimilar[n] \tau(\alpha)$.
\end{proof}

\begin{corollary}\label{afl-k-correspondence-aml-allacts}
Let $\aPModel{\aStateS} \in \classAmK$.
Then for every $\phi \in \langAml$ there exists $\alpha \in \langAflAct$ such that $\entails_\logicAmlK{} \actionA{\aPModel{\aStateS}} \phi \iff \actionA{\tau(\alpha)} \phi$.
\end{corollary}

\begin{proof}
Suppose that $d(\phi) = n$.
From Theorem~\ref{afl-k-correspondence} there exists $\alpha \in \langAflAct$ such that $\aPModel{\aStateS} \bisimilar[n] \tau(\alpha)$.
Therefore for every $\kPModel{\aStateS} \in \classK$ we have $\kPModel{\aStateS} \exec \aPModel{\aStateS} \bisimilar[n] \kPModel{\aStateS} \exec \tau(\alpha)$ and so $\kPModel{\aStateS} \exec \aPModel{\aStateS} \entails_\logicAmlK{} \phi$ if and only if $\kPModel{\aStateS} \exec \tau(\alpha) \entails_\logicAmlK{} \phi$.
Therefore $\kPModel{\aStateS} \entails_\logicAmlK{} \actionA{\aPModel{\aStateS}}  \phi$ if and only if $\kPModel{\aStateS} \entails_\logicAmlK{} \actionA{\tau(\alpha)}  \phi$.
\end{proof}

\section{K45}

\subsection{Semantics}

\begin{definition}[Test]\label{afl-kff-test}
Let $\phi \in \langAfl$.
We define $\tau(\test{\phi})$ as in Definition~\ref{afl-k-test} for \classK{}.
\end{definition}

\begin{definition}[Learning]\label{afl-kff-learning}
Let $\{\alpha_\agentA\}_{\agentA \in \agents} : \agents \to \langAflAct$ be a set of action formulas indexed on agents $\agents$, and for every $\agentA \in \agents$ let $\tau(\alpha_\agentA) = \aPModel[\agentA]{\aStatesT[\agentA]} = \aPModelTuple[\agentA]{\aStatesT[\agentA]}$.
Let $\aStateLearn$ be a new state not appearing in $\aStates[\agentA]$ for any $\agentA \in \agents$.
For every $\agentA \in \agents$, $\aStateT[\agentA] \in \aStatesT[\agentA]$ let $\aPStateT[\agentA]$ be a new state not appearing in $\aStates[\agentC]$ for any $\agentC \in \agents$.
We call each $\aPStateT[\agentA]$ a {\em proxy state} for $\aStateT[\agentA]$.
We define $\tau(\learns \{\alpha_\agentA\}_{\agentA \in \agents}) = \aPModel{\aStatesT} = \aPModelTuple{\aStatesT}$ where:
\begin{eqnarray*}
    \aStates &=& \bigcup_{\agentA \in \agents} \aStates[\agentA] \cup \{\aStateLearn\} \cup \{\aPStateT[\agentA] \mid \aStateT[\agentA] \in \aStatesT[\agentA], \agentA \in \agents\}\\
    \aAccessibility{\agentA} &=& \bigcup_{\agentB \in \agentsB} \aAccessibility[\agentB]{\agentA} \cup \{(\aStateLearn, \aPStateT[\agentA]) \mid \aStateT[\agentA] \in \aStatesT[\agentA]\} \cup \{(\aPStateT[\agentA], \aPStateU[\agentA]) \mid \aStateT[\agentA], \aStateU[\agentA] \in \aStatesT[\agentA]\} \\&& \quad \cup \{(\aPStateT[\agentB], \aStateU[\agentB]) \mid \aStateU[\agentB] \in \aSuccessorsA[\agentB]{\aStateT[\agentB]}, \agentB \in \agents \setminus \{\agentA\}\} \text{ for } \agentA \in \agents\\
    \aPrecondition &=& \bigcup_{\agentB \in \agentsB} \aPrecondition[\agentB] \cup \{(\aStateLearn, \top)\} \cup \{(\aPStateT[\agentB], \aPrecondition[\agentB](\aStateT[\agentB])) \mid \aStateT[\agentB] \in \aStatesT[\agentB], \agentB \in \agentsB\}\\
    \aStatesT &=& \{\aStateLearn\}
\end{eqnarray*}
\end{definition}

\begin{lemma}\label{afl-kff-structure}
Let $\alpha \in \langAflAct$. Then $\tau(\alpha) \in \classAmKFF$ and $\tau(\alpha)$ is finite.
\end{lemma}

\begin{lemma}\label{afl-kff-exec}
Let $\alpha \in \langAflAct$ and let $\kPModel{\kStatesT} \in \classKFF$.
Then $\kPModel{\kStatesT} \exec \tau(\alpha) \in \classKFF$.
\end{lemma}

\begin{proposition}
The logic \logicAflKFF{} is expressively equivalent to the logic \logicAmlKFF{}.
\end{proposition}

\begin{corollary}
The logic \logicAflKFF{} is expressively equivalent to the logic \logicKFF{}.
\end{corollary}

\subsection{Axiomatisation}

\begin{definition}[Axiomatisation \axiomAflKFF{}]\label{afl-kff-axioms}
The axiomatisation \axiomAflKFF{} is a substitution schema consisting of the rules and axioms of \axiomKFF{} along with the rules and axioms of \axiomAflK{}, but substituting the \axiomAflK{} axiom {\bf LK1} for the axiom:
$$
\begin{array}{rl}
    {\bf LK1} & \proves \actionA{\learns \{\alpha_\agentB\}_{\agentB \in \agentsB}} \necessaryA \chi \iff \necessaryA \actionA{\alpha_\agentA} \chi \text{ for } \agentA \in \agentsB\\
\end{array}
$$
and the rule:
$$
\begin{array}{rl}
    {\bf NecL} & \text{From $\proves \phi$ infer $\proves \actionA{\alpha} \phi$}
\end{array}
$$
where $\chi$ is a $(\agents \setminus \{\agentA\})$-restricted formula.
\end{definition}

\begin{lemma}\label{afl-kff-axioms-soundness}
The axiomatisation \axiomAflKFF{} is sound in the logic \logicAmlKFF{}.
\end{lemma}

\begin{proof}
Soundness of {\bf LT}, {\bf LU}, {\bf LS}, {\bf LP}, {\bf LN}, {\bf LC}, {\bf LK2} and {\bf NecL} follow from the same reasoning as in the proof of Lemma~\ref{afl-k-axioms-soundness}.

{\bf LK1} follows from the \axiomAmlKFF{} axiom {\bf AK}.
We note that as $\agentA \in \agentsB$, from Definition~\ref{afl-kff-learning} we have $\aPModel{\aStateS \aAccessibility{\agentA}} \bisimilar[(\agents \setminus \{\agentA\})] \tau(\alpha_\agentA)$, and as $\chi$ is $(\agents \setminus \{\agentA\})$-restricted formula then $\entails \actionA{\aPModel{\aStateS \aAccessibility{\agentA}}} \chi \iff \actionA{\tau(\alpha_\agentA)} \chi$.
\end{proof}

\begin{theorem}\label{afl-kff-axioms-completeness}
The axiomatisation \axiomAflKFF{} is sound and complete for the logic \logicAflKFF{}.
\end{theorem}

We note that the axiomatisation \axiomAflKFF{} forms a set of reduction axioms that gives a provably correct translation from \langAfl{} to \lang{}.
To translate a subformula $\actionA{\alpha} \phi$, where $\phi \in \lang$, we must first translate $\phi$ to the alternating disjunctive normal form of \cite{hales:2012}, which gives the property that for every subformula $\necessaryA \psi$, the formula $\psi$ is $(\agents \setminus \{\agentA\})$-restricted, and therefore {\bf LK1} is applicable.

\subsection{Correspondence}

\begin{theorem}\label{afl-kff-correspondence}
Let $\aPModel{\aStateS} \in \classAmKFF$ and let $n \in \mathbb{N}$. 
Then there exists $\alpha \in \langAflAct$ such that $\aPModel{\aStateS} \bisimilar[n] \tau(\alpha)$.
\end{theorem}

\begin{proof}
Without loss of generality, by the expressive equivalence of \logicAmlKFF{} and \logicKFF{} (Proposition~\ref{aml-expressive-equivalence}) we assume that $\aPModel{\aStateS}$ has preconditions defined on \langMl{}.

By induction on $n$.

Suppose that $n = 0$. 
Let $\alpha = \test{\aPrecondition(\aStateS)}$ and $\tau(\alpha) = \aPModelP{\aStateSP} = \aPModelTupleP{\aStateSP}$.
From Definition~\ref{afl-kff-test} we have that $\aPrecondition(\aStateS) = \aPreconditionP(\aStateSP)$, so $(\aPModel{\aStateS}, \aPModelP{\aStateSP})$ satisfies {\bf pre} and therefore $\aPModel{\aStateS} \bisimilar[0] \aPModelP{\aStateSP}$.

Suppose that $n > 0$. 
By the induction hypothesis, for every $\agentA \in \agents$, $\aStateT \in \aStateS \aAccessibility{\agentA}$ there exists $\alpha^{\agentA,\aStateT} \in \langAflAct$ such that $\aPModel{\aStateT} \bisimilar[(n - 1)] \tau(\alpha^{\agentA,\aStateT})$.
For every $\agentA \in \agents$, $\aStateT \in \aStateS \aAccessibility{\agentA}$ let $\tau(\alpha^{\agentA,\aStateT}) = \aPModel[\agentA,\aStateT]{\aStateS[\agentA,\aStateT]} = \aPModelTuple[\agentA,\aStateT]{\aStateS[\agentA,\aStateT]}$.

Let $\alpha = \test{\aPrecondition(\aStateS)} \compose \learns \{\bigchoice_{\aStateT \in \aStateS \aAccessibility{\agentA}} \alpha^{\agentA,\aStateT} \mid \agentA \in \agents\}$. 
Then by Lemma~\ref{afl-k-construction-test} we have that $\tau(\alpha) \bisimilar \aPModelP{\aStateSP} = \aPModelTupleP{\aStateSP}$ where:
\begin{eqnarray*}
    \aStatesP &=& \bigcup_{\agentA \in \agents, \aStateT \in \aStateS \aAccessibility{\agentA}} \aStates[\agentA,\aStateT] \cup \{\aPStateS[\agentA,\aStateT] \mid \agentA \in \agents, \aStateT \in \aStateS \aAccessibility{\agentA}\} \cup \{\aStateSP\}\\
    \aAccessibilityP{\agentA} &=& \bigcup_{\agentB \in \agents, \aStateT \in \aStateS \aAccessibility{\agentB}} \aAccessibility[\agentB,\aStateT]{\agentA} \cup \{(\aStateSP, \aPStateS[\agentA, \aStateT]) \mid \aStateT \in \aStateS \aAccessibility{\agentA}\} \cup \{(\aPStateS[\agentA, \aStateT], \aPStateS[\agentA, \aStateU]) \mid \aStateT, \aStateU \in \aStateS \aAccessibility{\agentA}\} \cup\\
                                               &&\hspace{45pt}\{(\aPStateS[\agentB, \aStateT], \aStateU) \mid \agentB \in \agents \setminus \{\agentA\}, \aStateT \in \aStateS \aAccessibility{\agentB}, \aStateU \in \aStateS[\agentB,\aStateT] \aAccessibility[\agentB,\aStateT]{\agentA}\} \text{ for } \agentA \in \agents\\
    \aPreconditionP &=& \bigcup_{\agentA \in \agents, \aStateT \in \aStateS \aAccessibility{\agentA}} \aPrecondition[\agentA,\aStateT] \cup \{(\aPStateS[\agentA, \aStateT], \aPrecondition[\agentA,\aStateT](\aStateS[\agentA,\aStateT])) \mid \agentA \in \agents, \aStateT \in \aStateS \aAccessibility{\agentA}\} \cup \{(\aStateSP, \aPrecondition(\aStateS))\}
\end{eqnarray*}

As in the proof of Theorem~\ref{afl-k-correspondence}, we note for every $\agentA \in \agents$, $\aStateT \in \aStateS \aAccessibility{\agentA}$ that $\aPModelP{\aStateS[\agentA,\aStateT]} \bisimilar \aPModel[\agentA,\aStateT]{\aStateS[\agentA,\aStateT]}$.

We need to show that $(\aPModel{\aStateS}, \aPModelP{\aStateSP})$ satisfies {\bf pre}, {\bf forth-$n$-$\agentA$} and {\bf back-$n$-$\agentA$} for every $\agentA \in \agents$.
We use reasoning similar to the proof of Theorem~\ref{afl-k-correspondence}, however noting that the successors of $\aStateSP$ in $\aModelP$ are not the same as in the construction used previously.
We claim that each $\aPStateS[\agentA,\aStateT]$ state is $(n-1)$-bisimilar to the corresponding $\aStateS[\agentA,\aStateT]$ state.
We show this by showing for every $0 \leq i \leq n - 1$, $\agentA \in \agents$, $\aStateT \in \aStateS \aAccessibility{\agentA}$ that $\aPModelP{\aPStateS[\agentA,\aStateT]} \bisimilar[i] \aPModelP{\aStateS[\agentA,\aStateT]}$.
We proceed by induction on $i$.

\paragraph{pre} By construction $\aPreconditionP(\aPStateS[\agentA, \aStateT]) = \aPreconditionP(\aStateS[\agentA,\aStateT])$.

\paragraph{forth-$i$-$\agentB$} Suppose that $0 < i \leq n - 1$. Let $\aStateU \in \aPStateS[\agentA,\aStateT] \aAccessibilityP{\agentB}$. 

Suppose that $\agentB = \agentA$. 
By construction there exists $\aStateV \in \aStateS \aAccessibility{\agentA}$ such that $\aStateU = \aPStateS[\agentA,\aStateV]$.
From above $\aPModelP{\aStateS[\agentA,\aStateT]} \bisimilar \aPModel[\agentA,\aStateT]{\aStateS[\agentA,\aStateT]}$ and $\aPModelP{\aStateS[\agentA,\aStateV]} \bisimilar \aPModel[\agentA,\aStateV]{\aStateS[\agentA,\aStateV]}$.
By the outer induction hypothesis $\aPModel[\agentA,\aStateT]{\aStateS[\agentA,\aStateT]} \bisimilar[(n - 1)] \aPModel{\aStateT}$ and $\aPModel[\agentA,\aStateV]{\aStateS[\agentA,\aStateV]} \bisimilar[(n - 1)] \aPModel{\aStateV}$.
By transitivity $\aPModelP{\aStateS[\agentA,\aStateT]} \bisimilar[(n - 1)] \aPModel{\aStateT}$ and $\aPModelP{\aStateS[\agentA,\aStateV]} \bisimilar[(n - 1)] \aPModel{\aStateV}$.
As $\aStateV \in \aStateT \aAccessibility{\agentA}$ from {\bf back-$(n - 1)$-$\agentA$} there exists $\aStateW \in \aStateS[\agentA,\aStateT] \aAccessibilityP{\agentA}$ such that $\aPModelP{\aStateW} \bisimilar[(n - 2)] \aPModel{\aStateV}$.
By transitivity $\aPModelP{\aStateW} \bisimilar[(n - 2)] \aPModelP{\aStateS[\agentA,\aStateV]}$.
By the induction hypothesis $\aPModelP{\aPStateS[\agentA,\aStateV]} \bisimilar[(i - 1)] \aPModelP{\aStateS[\agentA,\aStateV]}$.
Therefore by transitivity $\aPModelP{\aPStateS[\agentA,\aStateV]} \bisimilar[(i - 1)] \aPModelP{\aStateW}$.

Suppose that $\agentB \neq \agentA$.
By construction $\aPStateS[\agentA,\aStateT] \aAccessibilityP{\agentB} = \aStateS[\agentA,\aStateT] \aAccessibilityP{\agentB}$, so $\aStateU \in \aStateS[\agentA,\aStateT] \aAccessibilityP{\agentB}$ and we trivially have that $\aPModelP{\aStateU} \bisimilar \aPModelP{\aStateU}$.

\paragraph{back-$i$-$\agentB$} Follows similar reasoning to {\bf forth-$i$-$\agentB$}.

Therefore for every $\agentA \in \agents$, $\aStateT \in \aStateS \aAccessibility{\agentA}$ we have that $\aPModelP{\aPStateS[\agentA,\aStateT]} \bisimilar[(n - 1)] \aPModelP{\aStateS[\agentA,\aStateT]}$.

We can now show that $\aPModel{\aStateS} \bisimilar[n] \aPModelP{\aStateSP}$ by using the same reasoning as the proof for Theorem~\ref{afl-k-correspondence}, using the $(n - 1)$-bisimilar $\aPModelP{\aPStateS[\agentA,\aStateT]}$ states in place of corresponding $\aPModelP{\aStateS[\agentA,\aStateT]}$ states.

Therefore $\aPModel{\aStateS} \bisimilar[n] \tau(\alpha)$.
\end{proof}

\begin{corollary}
Let $\aPModel{\aStateS} \in \classAmKFF$.
Then for every $\phi \in \langAml$ there exists $\alpha \in \langAflAct$ such that $\entails_\logicAmlKFF{} \actionA{\aPModel{\aStateS}} \phi \iff \actionA{\tau(\alpha)} \phi$.
\end{corollary}

\section{S5}

\subsection{Semantics}

\begin{definition}[Test]\label{afl-s-test}
Let $\phi \in \langAfl$. 
We define $\tau(\test{\phi}) = \aPModel{\aStatesT} = \aPModelTuple{\aStatesT}$ where:
\begin{eqnarray*}
    \aStates &=& \{\aStateTest, \aStateSkip\}\\
    \aAccessibility{\agentA} &=& \aStates^2 \text{ for } \agentA \in \agents\\
    \aPrecondition &=& \{(\aStateTest, \phi), (\aStateSkip, \top)\}\\
    \aStatesT &=& \{\aStateTest\}
\end{eqnarray*}
\end{definition}

\begin{definition}[Learning]\label{afl-s-learning}
Let $\{\alpha_\agentA\}_{\agentA \in \agents} : \agents \to \langAflAct$ be a set of action formulas indexed on agents $\agents$, and for every $\agentA \in \agents$ let $\tau(\alpha_\agentA) = \aPModel[\agentA]{\aStatesT[\agentA]} = \aPModelTuple[\agentA]{\aStatesT[\agentA]}$.
Let $\aStateLearn$ be a new state not appearing in $\aStates[\agentA]$ for any $\agentA \in \agents$.
We define $\tau(\learns \{\alpha_\agentA\}_{\agentA \in \agents}) = \aPModel{\aStatesT} = \aPModelTuple{\aStatesT}$ where:
For every $\agentA \in \agents$, $\aStateT[\agentA] \in \aStatesT[\agentA]$ let $\aPStateT[\agentA]$ be a new state not appearing in $\aStates[\agentC]$ for any $\agentC \in \agents$.
We define $\tau(\learns \{\alpha_\agentA\}_{\agentA \in \agents}) = \aPModel{\aStatesT} = \aPModelTuple{\aStatesT}$ where:
\begin{eqnarray*}
    \aStates &=& \bigcup_{\agentA \in \agents} \aStates[\agentA] \cup \{\aStateLearn\} \cup \{\aPStateT[\agentA] \mid \aStateT[\agentA] \in \aStatesT[\agentA], \agentA \in \agents\}\\
    \aAccessibilityA &=& \bigcup_{\agentB \in \agents} \aAccessibilityA[\agentB] \cup (\{\aStateLearn\} \cup \{\aPStateT[\agentA] \mid \aStateT[\agentA] \in \aStatesT[\agentA]\})^2 \cup \bigcup_{\agentB \in \agents \setminus \{\agentA\}} (\{\aPStateT[\agentB]\} \cup \aSuccessorsA[\agentB]{\aStateT[\agentB]})^2 \text{ for } \agentA \in \agents\\
    \aPrecondition &=& \bigcup_{\agentA \in \agents} \aPrecondition[\agentA] \cup \{(\aStateLearn, \top)\} \cup \{(\aPStateT[\agentA], \aPrecondition[\agentA](\aStateT[\agentA])) \mid \aStateT[\agentA] \in \aStatesT[\agentA], \agentA \in \agents\}
\end{eqnarray*}
\end{definition}

\begin{lemma}\label{afl-s-structure}
Let $\alpha \in \langAflAct$. Then $\tau(\alpha) \in \classAmS$.
\end{lemma}

\begin{lemma}\label{afl-s-exec}
Let $\alpha \in \langAflAct$ and 
let $\kPModel{\kStatesT} \in \classS$.
Then $\kPModel{\kStatesT} \exec \tau(\alpha) \in \classS$.
\end{lemma}

\begin{proposition}
The logic \logicAflS{} is expressively equivalent to the logic \logicAmlS{}.
\end{proposition}

\begin{corollary}
The logic \logicAflS{} is expressively equivalent to the logic \logicS{}.
\end{corollary}

\subsection{Correspondence}

\begin{theorem}\label{afl-s-correspondence}
Let $\aPModel{\aStateS} \in \classAmS$ and let $n \in \mathbb{N}$. 
Then there exists $\alpha \in \langAflAct$ such that $\aPModel{\aStateS} \bisimilar[n] \tau(\alpha)$.
\end{theorem}

\begin{proof}
Without loss of generality, by the expressive equivalence of \logicAmlS{} and \logicS{} (Proposition~\ref{aml-expressive-equivalence}) we assume that $\aPModel{\aStateS}$ has preconditions defined on \langMl{}.

By induction on $n$.

Suppose that $n = 0$. 
Let $\alpha = \test{\aPrecondition(\aStateS)}$ and $\tau(\alpha) = \aPModelP{\aStateSP} = \aPModelTupleP{\aStateSP}$. 
From Definition~\ref{afl-s-test} we have that $\aPrecondition(\aStateS) = \aPreconditionP(\aStateSP)$, so $(\aPModel{\aStateS}, \aPModelP{\aStateSP})$ satisfies {\bf pre} and therefore $\aPModel{\aStateS} \bisimilar[0] \aPModelP{\aStateSP}$.

Suppose that $n > 0$. 
By the induction hypothesis, for every $\agentA \in \agents$, $\aStateT \in \aStateS \aAccessibility{\agentA}$ there exists $\alpha^{\agentA,\aStateT} \in \langAflAct$ such that 
$\aPModel{\aStateT} \bisimilar[(n - 1)] \tau(\alpha^{\agentA,\aStateT})$. 
For every $\agentA \in \agents$, $\aStateT \in \aStateS \aAccessibility{\agentA}$ let $\tau(\alpha^{\agentA,\aStateT}) = \aPModel[\agentA,\aStateT]{\aStateS[\agentA,\aStateT]} = \aPModelTuple[\agentA,\aStateT]{\aStateS[\agentA,\aStateT]}$.

Let $\alpha = \test{\aPrecondition(\aStateS)} \compose \learnsA \{\bigchoice_{\aStateT \in \aStateS \aAccessibility{\agentA}} \alpha^{\agentA,\aStateT} \mid \agentA \in \agents\}$.
Then from $\tau(\alpha) = \aPModelP{\aStateSP} =\aPModelTupleP{\aStateSP}$ where:
\begin{eqnarray*}
    \aStatesP &=& \bigcup_{\agentA \in \agents, \aStateT \in \aStateS \aAccessibility{\agentA}} \aStates[\agentA,\aStateT] \cup \{\aPStateS[\agentA,\aStateT] \mid \agentA \in \agents, \aStateT \in \aStateS \aAccessibility{\agentA}\} \cup \{\aStateSP\}\\
    \aAccessibilityP{\agentA} &=& \bigcup_{\agentB \in \agents, \aStateT \in \aStateS \aAccessibility{\agentB}} \aAccessibility[\agentB,\aStateT]{\agentA} \cup (\{\aStateSP\} \cup \{\aPStateS[\agentA,\aStateT] \mid \aStateT \in \aStateS \aAccessibility{\agentA}\})^2 \cup\\&&\quad\bigcup_{\agentB \in \agents \setminus \{\agentA\}, \aStateT \in \aAccessibility{\agentB}} (\{\aPStateS[\agentB,\aStateT]\} \cup \aStateS[\agentB,\aStateT] \aAccessibility[\agentB,\aStateT]{\agentA})^2 \text{ for } \agentA \in \agents\\
    \aPreconditionP &=& \bigcup_{\agentA \in \agents, \aStateT \in \aStateS \aAccessibility{\agentA}} \aPrecondition[\agentA,\aStateT] \cup \{(\aPStateS[\agentA,\aStateT], \aPrecondition[\agentA,\aStateT](\aStateS[\agentA,\aStateT])) \mid \agentA \in \agents, \aStateT \in \aStateS \aAccessibility{\agentA}\} \cup \{(\aStateSP, \aPrecondition(\aStateS))\}
\end{eqnarray*}

We note that unlike the constructions used for Theorems~\ref{afl-k-correspondence} and \ref{afl-kff-correspondence}, this construction does not have $\aPModelP{\aStateU} \bisimilar \aPModel[\agentA,\aStateT]{\aStateU}$, as we do not have that 
$\aStateS[\agentA,\aStateT] \aAccessibilityP{\agentA} = \aStateS[\agentA,\aStateT] \aAccessibility[\agentA,\aStateT]{\agentA}$.
Similar to the proof of Theorem~\ref{afl-kff-correspondence} we claim that each $\aPStateS[\agentA,\aStateT]$ state is $(n-1)$-bisimilar to the corresponding $\aStateS[\aStateT]$ state. 
However in lieu of bisimilarity of $\aStates[\agentA,\aStateT]$ states we need another result for these states.
We also need to consider the additional state $\aStateSP$, which due to reflexivity is also a successor of itself.

We need to show for every $0 \leq i \leq n - 1$: 
\begin{enumerate}
    \item For every $\agentA \in \agents$: $\aPModelP{\aStateSP} \bisimilar[i] \aPModelP{\aPStateS[\agentA,\aStateS]}$.
    \item For every $\agentA \in \agents$, $\aStateT \in \aStateS \aAccessibility{\agentA}$: $\aPModelP{\aPStateS[\agentA,\aStateT]} \bisimilar[i] \aPModelP{\aStateS[\agentA,\aStateT]}$.
    \item For every $\agentA \in \agents$, $\aStateT \in \aStateS \aAccessibility{\agentA}$, $\aStateU \in \aStates[\agentA,\aStateT]$, $\aStateV \in \aStates$: if $\aPModel[\agentA,\aStateT]{\aStateU} \bisimilar[i] \aPModel{\aStateV}$ then $\aPModelP{\aStateU} \bisimilar[i] \aPModel{\aStateV}$.
\end{enumerate}

We proceed by induction on $i$.

\begin{enumerate}
    \item 
        For every $\agentA \in \agents$: $\aPModelP{\aStateSP} \bisimilar[i] \aPModelP{\aPStateS[\agentA,\aStateS]}$.

        \paragraph{pre}

        By the outer induction hypothesis $\aPModel[\agentA,\aStateS]{\aStateS[\agentA,\aStateS]} \bisimilar[(n - 1)] \aPModel{\aStateS}$ and so $\proves \aPrecondition[\agentA,\aStateS](\aStateS[\agentA,\aStateS]) \iff \aPrecondition(\aStateS)$.
        By construction $\aPreconditionP(\aStateSP) = \aPrecondition(\aStateS)$ and $\aPreconditionP(\aPStateS[\agentA,\aStateS]) = \aPrecondition[\agentA,\aStateS](\aStateS[\agentA,\aStateS])$ and therefore $\proves \aPreconditionP(\aStateSP) \iff \aPrecondition[\agentA,\aStateS](\aPStateS[\agentA,\aStateS])$.

        \paragraph{forth-$i$-$\agentB$} Suppose that $0 < i \leq n - 1$. Let $\aStateU \in \aStateSP \aAccessibilityP{\agentA}$. 

        Suppose that $\agentB = \agentA$.
        By construction $\aStateSP \aAccessibilityP{\agentA} = \aPStateS[\agentA,\aStateS] \aAccessibilityP{\agentA}$ and we trivially have that $\aPModelP{\aStateU} \bisimilar \aPModelP{\aStateU}$.

        Suppose that $\agentB \neq \agentA$.
        By construction $\aStateSP \aAccessibilityP{\agentB} = \{\aPStateS[\agentB,\aStateT] \mid \aStateT \in \aStateS \aAccessibility{\agentB}\} \cup \{\aStateSP\}$ and $\aPStateS[\agentA,\aStateS] \aAccessibilityP{\agentB} = \aStateS[\agentA,\aStateS] \aAccessibility[\agentA,\aStateS]{\agentB} \cup \{\aPStateS[\agentA,\aStateS]\}$. 
        Suppose that $\aStateU = \aStateSP$. 
        Then by the induction hypothesis $\aPModelP{\aStateSP} \bisimilar[(i-1)] \aPModelP{\aPStateS[\agentA,\aStateS]}$.
        Suppose that $\aStateU \in \{\aPStateS[\agentB,\aStateT] \mid \aStateT \in \aStateS \aAccessibility{\agentB}\}$. 
        Then there exists $\aStateT \in \aStateS \aAccessibility{\agentB}$ such that $\aStateU = \aPStateS[\agentB,\aStateT]$.
        By the outer induction hypothesis $\aPModel[\agentA,\aStateS]{\aStateS[\agentA,\aStateS]} \bisimilar[(n - 1)] \aPModel{\aStateS}$.
        As $\aStateT \in \aStateS \aAccessibility{\agentB}$ then by {\bf back-$(n-1)$-$\agentB$} there exists $\aStateV \in \aStateS[\agentA,\aStateS] \aAccessibility[\agentA,\aStateS]{\agentB} \subseteq \aPStateS[\agentA,\aStateS] \aAccessibilityP{\agentB}$ such that $\aPModel[\agentA,\aStateS]{\aStateV} \bisimilar[(n - 2)] \aPModel{\aStateT}$.
        Then by the inner induction hypothesis this implies $\aPModelP{\aStateV} \bisimilar[(i - 1)] \aPModel{\aStateT}$.
        By the inner induction hypothesis $\aPModelP{\aPStateS[\agentB,\aStateT]} \bisimilar[(i - 1)] \aPModelP{\aStateS[\agentB,\aStateT]} \bisimilar[(i - 1)] \aPModel[\agentB,\aStateT]{\aStateS[\agentB,\aStateT]}$ and by the outer induction hypothesis $\aPModel[\agentB,\aStateT]{\aStateS[\agentB,\aStateT]} \bisimilar[(n - 1)] \aPModel{\aStateT}$ so by transitivity $\aPModelP{\aPStateS[\agentB,\aStateT]} \bisimilar[(i - 1)] \aPModel{\aStateT}$.
        Therefore by transitivity we have that $\aPModelP{\aPStateS[\agentB,\aStateT]} \bisimilar[(i - 1)] \aPModelP{\aStateV}$.

        \paragraph{back-$i$-$\agentB$} Follows similar reasoning to {\bf forth-$i$-$\agentB$}.

    \item 
        For every $\agentA \in \agents$, $\aStateT \in \aStateS \aAccessibility{\agentA}$: $\aPModelP{\aPStateS[\agentA,\aStateT]} \bisimilar[i] \aPModelP{\aStateS[\agentA,\aStateT]}$.

        \paragraph{pre} By construction $\aPreconditionP(\aPStateS[\agentA,\aStateT]) = \aPreconditionP(\aStateS[\agentA,\aStateT])$.

        \paragraph{forth-$i$-$\agentB$} Suppose that $0 < i \leq n - 1$. Let $\aStateU \in \aPStateS[\agentA,\aStateT] \aAccessibilityP{\agentA}$. 

        Suppose that $\agentB = \agentA$.
        By construction $\aPStateS[\agentA,\aStateT] \aAccessibilityP{\agentA} = \{\aPStateS[\agentA,\aStateV] \mid \aStateV \in \aStateT \aAccessibility{\agentA}\} \cup \{\aStateSP\}$. 
        Suppose that $\aStateU \in \{\aPStateS[\agentA,\aStateV] \mid \aStateV \in \aStateT \aAccessibility{\agentA}\}$.
        Then there exists $\aStateV \in \aStateT \aAccessibility{\agentA}$ such that $\aStateU = \aPStateS[\agentA,\aStateV]$.
        By the outer induction hypothesis $\aPModel[\agentA,\aStateT]{\aStateS[\agentA,\aStateT]} \bisimilar[(n - 1)] \aPModel{\aStateT}$.
        As $\aStateV \in \aStateT \aAccessibility{\agentA}$ then by {\bf back-$(n-1)$-$\agentA$} there exists $\aStateW \in \aStateS[\agentA,\aStateT] \aAccessibility[\agentA,\aStateT]{\agentA} \subseteq \aStateS[\agentA,\aStateT] \aAccessibilityP{\agentA}$ such that $\aPModel[\agentA,\aStateT]{\aStateW} \bisimilar[(n - 2)] \aPModel{\aStateV}$.
        Then by the inner induction hypothesis this implies $\aPModelP{\aStateW} \bisimilar[(i - 1)] \aPModel{\aStateV}$.
        By the inner and outer induction hypothesis $\aPModelP{\aPStateS[\agentA,\aStateV]} \bisimilar[(i - 1)] \aPModel{\aStateV}$.
        Therefore by transitivity we have that $\aPModelP{\aPStateS[\agentA,\aStateV]} \bisimilar[(i - 1)] \aPModelP{\aStateW}$.
        Suppose that $\aStateU = \aStateSP$. 
        Then from the inner induction hypothesis $\aPModelP{\aStateSP} \bisimilar[(i - 1)] \aPModelP{\aPStateS[\agentA,\aStateS]}$ and we can proceed using the same reasoning as in the case where $\aStateU = \aPStateS[\agentA,\aStateS] \in \{\aPStateS[\agentA,\aStateV] \mid \aStateV \in \aStateT \aAccessibility{\agentA}\}$.

        Suppose that $\agentB \neq \agentA$.
        By construction $\aPStateS[\agentB,\aStateT] \aAccessibilityP{\agentB} = \aStateS[\agentA,\aStateT] \aAccessibility[\agentA,\aStateT]{\agentB} \cup \{\aPStateS[\agentB,\aStateT]\}$. 
        Suppose that $\aStateU = \aPStateS[\agentB,\aStateT]$.
        By construction $\aStateS[\agentA,\aStateT] \in \aStateS[\agentA,\aStateT] \aAccessibilityP{\agentB}$ and by the induction hypothesis $\aPModelP{\aPStateS[\agentB,\aStateT]} \bisimilar[(i - 1)] \aPModelP{\aStateS[\agentA,\aStateT]}$.
        Suppose that $\aStateU \in \aStateS[\agentA,\aStateT] \aAccessibility[\agentA,\aStateT]{\agentB} \subseteq \aStateS[\agentA,\aStateT] \aAccessibilityP{\agentB}$.
        Then we trivially have that $\aPModelP{\aStateU} \bisimilar \aPModelP{\aStateU}$.

        \paragraph{back-$i$-$\agentB$} Follows similar reasoning to {\bf forth-$i$-$\agentB$}.

    \item 
        For every $\agentA \in \agents$, $\aStateT \in \aStateS \aAccessibility{\agentA}$, $\aStateU \in \aStates[\agentA,\aStateT]$, $\aStateV \in \aStates$: if $\aPModel[\agentA,\aStateT]{\aStateU} \bisimilar[i] \aPModel{\aStateV}$ then $\aPModelP{\aStateU} \bisimilar[i] \aPModel{\aStateV}$.

        Suppose that $\aPModel[\agentA,\aStateT]{\aStateU} \bisimilar[i] \aPModel{\aStateV}$. 

        \paragraph{pre}

        As $\aPModel[\agentA,\aStateT]{\aStateU} \bisimilar[i] \aPModel{\aStateV}$ then $\proves \aPrecondition[\agentA,\aStateT](\aStateU) \iff \aPrecondition(\aStateV)$. 
        By construction $\aPreconditionP(\aStateU) = \aPrecondition[\agentA,\aStateT](\aStateU)$ and therefore $\proves \aPreconditionP(\aStateU) \iff \aPrecondition(\aStateV)$.

        \paragraph{forth-$i$-$\agentB$}

        Suppose that $0 < i \leq n - 1$.
        Let $\aStateW \in \aStateU \aAccessibilityP{\agentB}$.

        Suppose that $\aStateU \neq \aStateS[\agentA,\aStateT]$ or $\agentB = \agentA$.
        By construction $\aStateU \aAccessibilityP{\agentA} = \aStateU \aAccessibility[\agentA,\aStateT]{\agentA}$ and so $\aStateW \in \aStateU \aAccessibility[\agentA,\aStateT]{\agentA}$. 
        As $\aStateW \in \aStateU \aAccessibility[\agentA,\aStateT]{\agentA}$ then by {\bf forth-$i$-$\agentB$} there exists $\aStateX \in \aStateV \aAccessibility{\agentB}$ such that $\aPModel[\agentA,\aStateT]{\aStateW} \bisimilar[(i - 1)] \aPModel{\aStateX}$.
        By the induction hypothesis $\aPModelP{\aStateW} \bisimilar[(i - 1)] \aPModel{\aStateX}$.

        Suppose that $\aStateU = \aStateS[\agentA,\aStateT]$ and $\agentB \neq \agentA$. 
        By construction $\aStateS[\agentA,\aStateT] \aAccessibilityP{\agentA} = \aStateS[\agentA,\aStateT] \aAccessibility[\agentA,\aStateT]{\agentA} \cup \{\aPStateS[\agentA,\aStateT]\}$. 
        Suppose that $\aStateW \in \aStateS[\agentA,\aStateT] \aAccessibility[\agentA,\aStateT]{\agentA}$. 
        We proceed using the same reasoning as above, where $\aStateW \in \aStateU \aAccessibility[\agentA,\aStateT]{\agentA}$. 
        Suppose that $\aStateW = \aPStateS[\agentA,\aStateT]$.
        By the induction hypothesis $\aPModelP{\aPStateS[\agentA,\aStateT]} \bisimilar[(i - 1)] \aPModelP{\aStateS[\agentA,\aStateT]}$ and we proceed using the same reasoning above, where $\aStateW = \aStateS[\agentA,\aStateT] \in \aStateS[\agentA,\aStateT] \aAccessibility[\agentA,\aStateT]{\agentA}$.

        \paragraph{back-$i$-$\agentB$} Follows similar reasoning to {\bf forth-$i$-$\agentB$}.
\end{enumerate}

Therefore for every $\agentA \in \agents$, $\aStateT \in \aStateS \aAccessibility{\agentA}$ we have that $\aPModelP{\aStateSP} \bisimilar[(n - 1)] \aPModel{\aStateS}$ and $\aPModelP{\aPStateS[\agentA,\aStateT]} \bisimilar[(n - 1)] \aPModel{\aStateT}$.
We can now show that $\aPModel{\aStateSP} \bisimilar[n] \aPModel{\aStateS}$ by using the same reasoning as the proof for Theorem~\ref{afl-k-correspondence}, using the $(n-1)$-bisimilar $\aPModelP{\aPStateS[\agentA,\aStateT]}$ in place of corresponding $\aPModelP{\aStateS[\aStateT]}$ states.
\end{proof}

\begin{corollary}
Let $\aPModel{\aStateS} \in \classAmS$.
Then for every $\phi \in \langAml$ there exists $\alpha \in \langAflAct$ such that $\entails_\logicAmlS{} \actionA{\aPModel{\aStateS}} \phi \iff \actionA{\tau(\alpha)} \phi$.
\end{corollary}
